\documentclass[11pt]{article}
\usepackage{c0}
\usepackage{datetime}

% begin palatino.sty
% copied from palatino.sty, except left ttdefault as CMTT
\renewcommand{\rmdefault}{ppl}
\renewcommand{\sfdefault}{phv}
% \renewcommand{\ttdefault}{pcr}
% end palatino.sty

\usepackage[breaklinks=true,
  colorlinks=true,
  citecolor=blue,
  linkcolor=blue,
  urlcolor=blue]{hyperref}

\newcommand{\rev}{590}

\title{Additional C0 Libraries}
\author{Informatik 2 \\ Torsten Grust}
% \date{May 2012\\Compiler revision 71 (Wed May 23 2012
%   \hyperlink{sec:updates}{updates})}
\date{\today\\Compiler revision \rev}

\usepackage{fancyhdr}
\pagestyle{fancyplain}
\setlength{\headheight}{14pt}
\addtolength{\oddsidemargin}{30pt}
\addtolength{\evensidemargin}{-22pt}

\lhead[\fancyplain{}{\bfseries C0.\thepage}]%
      {\fancyplain{}{\bfseries Libraries}}
\chead[]{}
\rhead[\fancyplain{}{\bfseries Library}]%
      {\fancyplain{}{\bfseries C0.\thepage}}
\lfoot[{\small\scshape Informatik 2 (Sommersemester 2012)}]{{\small\scshape Informatik 2 (Sommersemester 2012)}}
\cfoot[]{}
\rfoot[{\small\scshape\today}]{{\small\scshape\today}}


\begin{document}

\maketitle

\section{Introduction}

This document describes additional libraries available for use in the
C0 language implementation. The additional libraries complement the
standard libraries shipped with the C0 implementation. All libraries
can be used with the \lstinline'cc0' compiler and \lstinline'coin'.  Please
consult the \href{http://c0.typesafety.net/tutorial}{C0 Tutorial} and
\href{http://c0.typesafety.net/doc/c0-reference.pdf}{C0 Language
  Reference} for more information on C0. The standard libraries are
described in the \href{http://c0.typesafety.net/tutorial}{C0 Library
  Reference}.

Libraries can be included on the command line using \lstinline'-l lib' or in
files with \lstinline'#use <lib>'.  The latter is recommended to make
source files less dependent on context.  The implementation ensures
that each library will be loaded at most once.

\section{\tt clock}
\label{sec:tt-clock}

The \lstinline'clock' library contains functions for timing programs and
provides access to the system time.

\begin{lstlisting}
/* Return CPU time used since process started
 * (measured in milliseconds) */
int millisecs();

/* Return time elapsed since midight, Jan 1, 1970 (UTC)
 * (measured in seconds) */
int now();
\end{lstlisting}

\end{document}
