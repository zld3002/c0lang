\documentclass[11pt]{article}
\usepackage{c0}
\usepackage{datetime}
\usepackage{makeidx}

\makeindex
\renewcommand{\indexname}{List of C0 Library Functions}
\newcommand{\indextt}[1]{\index{#1@\texttt{#1}}}

% begin palatino.sty
% copied from palatino.sty, except left ttdefault as CMTT
\renewcommand{\rmdefault}{ppl}
\renewcommand{\sfdefault}{phv}
% \renewcommand{\ttdefault}{pcr}
% end palatino.sty

\usepackage[breaklinks=true,
  colorlinks=true,
  citecolor=blue,
  linkcolor=blue,
  urlcolor=blue]{hyperref}

\newcommand{\rev}{590}
\smalllistings

\title{C0 Libraries}
\author{15-122: Principles of Imperative Computation \\ Frank Pfenning}
\date{\today\\Compiler revision \rev\\
  (\hyperlink{sec:updates}{updates} since September 2011)}

\usepackage{fancyhdr}
\pagestyle{fancyplain}
\setlength{\headheight}{14pt}
\addtolength{\oddsidemargin}{30pt}
\addtolength{\evensidemargin}{-22pt}

\lhead[\fancyplain{}{\bfseries C0.\thepage}]%
      {\fancyplain{}{\bfseries Libraries}}
\chead[]{}
\rhead[\fancyplain{}{\bfseries Library}]%
      {\fancyplain{}{\bfseries C0.\thepage}}
%\lfoot[{\small\scshape 15-122 Fall 2011}]{{\small\scshape 15-122 Fall }}
\lfoot[]{}
\cfoot[]{}
\rfoot[{\small\scshape\today}]{{\small\scshape\today}}

\newcommand{\tint}{\texttt{int}}
\newcommand{\tbool}{\texttt{bool}}
\newcommand{\vtrue}{\texttt{true}}
\newcommand{\vfalse}{\texttt{false}}
\newcommand{\tstring}{\texttt{string}}
\newcommand{\tchar}{\texttt{char}}
\newcommand{\tarray}{\texttt{[\,]}}
\newcommand{\tstar}{\texttt{*}}
\newcommand{\vnull}{\texttt{NULL}}
\newcommand{\tstruct}{\texttt{struct}}
\newcommand{\tvoid}{\texttt{void}}

\begin{document}

\maketitle
\tableofcontents

\newpage
\section{Introduction}

This document describes the standard libraries available for use
in the C0 language implementation.  All libraries can be used
with the \lstinline'cc0' compiler, and most can also be used with
\lstinline'coin'.   Please consult the
\href{http://c0.typesafety.net/tutorial}{C0 Tutorial} and
\href{http://c0.typesafety.net/doc/c0-reference.pdf}{C0 Language
Reference} for more information on C0.

Libraries can be include on the command line using \lstinline'-l lib' or in
files with \lstinline'#use <lib>'.  The latter is recommended to make
source files less dependent on context.  Each library will be loaded
at most once.

\section{Input/Output}

\subsection{\tt conio}

The \lstinline'conio' library contains functions for performing basic
console input and output.  Note that output may be \emph{buffered},
which means output does not necessarily appear on the console
until a newline character is printed or \lstinline'flush()' is called.

\begin{lstlisting}
void print(string s);       // print s to standard output[*\indextt{print}*]
void println(string s);     // print s with trailing newline[*\indextt{println}*]
void printint(int i);       // print i to standard output[*\indextt{printint}*]
void printbool(bool b);     // print b to standard output[*\indextt{printbool}*]
void printchar(char c);     // print c to standard output[*\indextt{printchar}*]
void flush();               // flush standard output[*\indextt{flush}*]

bool eof();                 // test end-of-file on standard input[*\indextt{eof}*]

string readline()           // read a line from standard input[*\indextt{readline}*]
  /*@requires !eof(); @*/ ; // do not include the trailing newline
\end{lstlisting}


\newpage
\subsection{\tt file}

The \lstinline'file' library contains functions for reading files, line by
line.  File handles are represented with the \lstinline'file_t' type.  The
handle contains an internal position which ranges from 0 to the
logical size of the file in byes.  File handles should be closed
explicitly when they are no longer needed to release system resources.

\begin{lstlisting}
// typedef ______* file_t; /* file handle or NULL */[*\indextt{file\_t}*]

/* Test whether the given file has been closed */
bool file_closed(file_t f)[*\indextt{file\_closed}*]
  /*@requires f != NULL; @*/ ;

/* Create a handle for reading from the file given by the specified
 * path, NULL if the file cannot be opened for reading. */
file_t file_read(string path)[*\indextt{file\_read}*]
  /*@ensures \result == NULL || !file_closed(\result); @*/ ;

/* Release any resources associated with the file handle.  This
 * function should not be invoked twice on the same handle. */
void file_close(file_t f)[*\indextt{file\_close}*]
  /*@requires f != NULL;       @*/
  /*@requires !file_closed(f); @*/
  /*@ensures file_closed(f);   @*/ ;

/* Test if we have read the whole file. */
bool file_eof(file_t f)[*\indextt{file\_eof}*]
  /*@requires f != NULL;       @*/
  /*@requires !file_closed(f); @*/ ;

/* Read a line from the given file (without the trailing newline)
 * and advance the handle's internal position by one line.  The
 * contract requires that the handle is not at the end-of-file,
 * so this must be checked before (with file_eof). */
string file_readline(file_t f)[*\indextt{file\_readline}*]
  /*@requires f != NULL;       @*/
  /*@requires !file_closed(f); @*/
  /*@requires !file_eof(f);    @*/ ;
\end{lstlisting}


\clearpage
\subsection{\tt args}

The \lstinline'args' library provides function for basic parsing of command
line arguments provided to the executable provided by the \lstinline'cc0'
compiler.  The \lstinline'args' library is not supported by \lstinline'coin'
since it never produces an executable.

\begin{lstlisting}
/* Add a flag with the given name. During parsing if that flag is
 * set (with -name on the command line), it writes the value true
 * to the location given by ptr. */
void args_flag(string name, bool *ptr)[*\indextt{args\_flag}*]
  /*@requires ptr != NULL; @*/ ;

/* Add an integer option with the given name. During parsing if
 * that option is given (with -name <int>) it attempts to parse
 * it as an integer and write it to the location given by ptr. */
void args_int(string name, int *ptr)[*\indextt{args\_int}*]
  /*@requires ptr != NULL; @*/ ;

/* Add an string option with the given name. During parsing if
 * that option is given (with -name <string>) it write it to the
 * location given by ptr. */
void args_string(string name, string *ptr)[*\indextt{args\_string}*]
  /*@requires ptr != NULL; @*/ ;

struct args {
    int argc;
    string[] argv;
};
typedef struct args* args_t;[*\indextt{args\_t}*]

/* Parse the program's arguments according to any flags and options
 * set up by previous calls. Any unrecognized arguments are returned
 * in order in \result->argv.
 *
 * If there is an error, args_parse will return NULL. */
args_t args_parse()[*\indextt{args\_parse}*]
  /*@ensures \result == NULL
          || \result->argc == \length(\result->argv); @*/ ;
\end{lstlisting}


\section{Strings}

\subsection{\tt parse}

The \lstinline'parse' library provides functions to convert strings to
booleans or integers and for whitespace-tokenizing a string.  This is
useful, for example, if you want to convert strings read from a file
into integers. The \lstinline'parse_ints' function is aimed more at unit
tests than user input.

\begin{lstlisting}
/* Attempt to parse "true" or "false from s and return a pointer to
 * the result or NULL if not of that form. */
bool* parse_bool(string s);[*\indextt{parse\_bool}*]

/* Attempt to parse an integer from the given string.
 * For base > 10, the letters A-Z (case insignificant) are used as
 * digits.  Return NULL if s cannot be completely parsed to an int
 * in the given base. */
int* parse_int(string s, int base)[*\indextt{parse\_int}*]
  /*@requires 2 <= base && base <= 36; @*/ ;

/* Returns the number of whitespace-delimited tokens in a string */
int num_tokens(string s);[*\indextt{num\_tokens}*]

/* Returns true iff the string contains only whitespace separated
 *  ints */
bool int_tokens(string s, int base)[*\indextt{int\_tokens}*]
  /*@requires 2 <= base && base <= 36; @*/ ;

/* Parses a string as a list of whitespace-delimited tokens */
string[] parse_tokens(string s)[*\indextt{parse\_tokens}*]
  /*@ensures \length(\result) == num_tokens(s); @*/ ;

/* Parses a string as list of whitespace-delimited integers */
int[] parse_ints(string s, int base)[*\indextt{parse\_ints}*]
  /*@requires int_tokens(s, base);              @*/
  /*@ensures \length(\result) == num_tokens(s); @*/ ;
\end{lstlisting}

\clearpage
\subsection{\tt string}

The \lstinline'string' library contains a few basic functions for working
with strings consisting of ASCII characters.  For most nontrivial
tasks, it is best to convert back and forth between characters arrays
and strings, using the functions \lstinline'string_to_chararray' and
\lstinline'string_from_chararray'.  The two are identified in C,
but distinguished in C0 to allow a type-safe and memory-safe
implementation.  Note that character arrays should contain a trailing
\lstinline"'\0'" character not present in the corresponding strings.

\begin{lstlisting}
/* Return length of s, in characters.
 * May be an O(n) operation. */
int string_length(string s);[*\indextt{string\_length}*]

/* Return s[idx] and abort if the idx is out of bound.
 * May be an O(n) operation. */
char string_charat(string s, int idx)[*\indextt{string\_charat}*]
  /*@requires 0 <= idx && idx < string_length(s); @*/ ;

/* Return a new string that is the result of concatenating a
 * and b. */
string string_join(string a, string b)[*\indextt{string\_join}*]
  /*@ensures string_length(\result)
               == string_length(a) + string_length(b); @*/ ;

/* Returns the substring composed of the characters of s beginning
 * at the index given by start and continuing up to but not
 * including the index given by end.  If end == start, the empty
 * string is returned. Aborts if start or end are out of bounds,
 * or end < start. */
string string_sub(string a, int start, int end)[*\indextt{string\_sub}*]
  /*@requires 0 <= start && start <= end && end <= string_length(a); @*/
  /*@ensures string_length(\result) == end - start; @*/ ;

/* Compare strings lexicographically */
bool string_equal(string a, string b);[*\indextt{string\_equal}*]
int string_compare(string a, string b)[*\indextt{string\_compare}*]
  /*@ensures -1 <= \result && \result <= 1; @*/ ;

/* Create strings from other basic values */
string string_fromint(int i);[*\indextt{string\_fromint}*]
string string_frombool(bool b);[*\indextt{string\_frombool}*]
string string_fromchar(char c)[*\indextt{string\_fromchar}*]
  /*@requires c != '\0';                     @*/
  /*@ensures string_length(\result) == 1;    @*/
  /*@ensures string_charat(\result, 0) == c; @*/ ;

/* Convert every uppercase character A-Z to lowercase a-z */
string string_tolower(string s);[*\indextt{string\_tolower}*]

/* Check if character array is properly \0-terminated */
bool string_terminated(char[] A, int n)[*\indextt{string\_terminated}*]
  /*@requires 0 <= n && n <= \length(A); @*/ ;

/* Construct a '\0'-terminated character array from s */
char[] string_to_chararray(string s)[*\indextt{string\_to\_chararray}*]
  /*@ensures \length(\result) >= string_length(s) + 1;         @*/
  /*@ensures string_terminated(\result, string_length(s) + 1); @*/ ;

/* Construct a string from the the array A up to (but not
 * including) the terminating '\0' */
string string_from_chararray(char[] A)[*\indextt{string\_from\_chararray}*]
  /*@requires string_terminated(A, \length(A));        @*/
  /*@ensures string_length(\result) + 1 <= \length(A); @*/ ;

/* Convert between characters and their ASCII value */
int char_ord(char c)[*\indextt{char\_ord}*]
  /*@ensures 0 <= \result && \result <= 127; @*/ ;
char char_chr(int n)[*\indextt{char\_chr}*]
  /*@requires 0 <= n && n <= 127; @*/ ;
\end{lstlisting}


\section{Images}

\subsection{\tt img}

The \lstinline'img' library defines a type for two-dimensional images
represented with 4-channel color---alpha, red, green and blue---
packed into one \lstinline'int'  It defines and image type \lstinline'image_t'.
Like file handles, images must be explicitly destroyed when they
are no longer needed; the garbage collector will not be able to
free them.

\begin{lstlisting}
// typedef ______* image_t; /* image handle or NULL */[*\indextt{image\_t}*]

/* Retrieves the width of the given image */
int image_width(image_t image)[*\indextt{image\_width}*]
  /*@requires image != NULL; @*/
  /*@ensures \result > 0;    @*/ ;

/* Retrieves the height of the given image */
int image_height(image_t image)[*\indextt{image\_height}*]
  /*@requires image != NULL; @*/
  /*@ensures \result > 0;    @*/ ;

/* Creates an ARGB image with dimensions width * height */
image_t image_create(int width, int height)[*\indextt{image\_create}*]
  /*@requires 0 < width && 0 < height;        @*/
  /*@ensures \result != NULL;                 @*/
  /*@ensures image_width(\result) == width;   @*/
  /*@ensures image_height(\result) == height; @*/ ;

/* Copies an existing image */
image_t image_clone(image_t image)[*\indextt{image\_clone}*]
  /*@requires image != NULL;                               @*/
  /*@ensures image_width(\result) == image_width(image);   @*/
  /*@ensures image_height(\result) == image_height(image); @*/ ;

/* Returns a copy of a subrectangle of the given image. The new
 * image has dimensions width * height. If part of the given
 * rectangle is not contained within the given image, those pixels
 * are assumed to be transparent black.  */
image_t image_subimage(image_t image, int x, int y, int width, int height)[*\indextt{image\_subimage}*]
  /*@requires image != NULL;                  @*/
  /*@ensures image_width(\result) == width;   @*/
  /*@ensures image_height(\result) == height; @*/ ;

/* Loads an image from the given path and convert it if need be
 * to an ARGB image.  If the file does not exist, it returns NULL.
 * Aborts if the file has the wrong format. */
image_t image_load(string path);[*\indextt{image\_load}*]

/* Saves the given image to disk, inferring the file type from the
 * file extension given in the path. */
void image_save(image_t image, string path)[*\indextt{image\_save}*]
  /*@requires image != NULL; @*/ ;

/* Returns an array of pixels representing the image. The pixels
 * are given line by line so a pixel at (x,y) would be located at
 * y*image_width(image) + x. Any writes to the array will be
 * reflected in calls to image_save, image_clone and image_subimage.
 * The channels are encoded as 0xAARRGGBB. */
int[] image_data(image_t image)[*\indextt{image\_data}*]
  /*@requires image != NULL; @*/
  /*@ensures \length(\result)
              == image_width(image) * image_height(image); @*/ ;
\end{lstlisting}


\section{C0-implemented libraries}

A few libraries are included as part of the default C0 distribution
but are implemented in C0, not in C. Both the interface (\lstinline'.h0')
and the implementation (\lstinline'.c0') of these libraries are provided in
the \lstinline'lib' directory.

\subsection{\tt rand}

\begin{lstlisting}
// typedef _____* rand_t;[*\indextt{rand\_t}*]
rand_t init_rand (int seed)[*\indextt{init\_rand}*]
  /*@requires seed != 0;     @*/
  /*@ensures \result != NULL @*/ ;

int rand(rand_t gen)[*\indextt{rand}*]
  /*@requires gen != NULL; @*/ ;
\end{lstlisting}

\subsection{\tt util}

\begin{lstlisting}
/* C0 constants */
int int_size() /*@ensures \result == 4;           @*/ ;[*\indextt{int\_size}*]
int int_max()  /*@ensures \result == 2147483647;  @*/ ;[*\indextt{int\_max}*]
int int_min()  /*@ensures \result == -2147483648; @*/ ;[*\indextt{int\_min}*]

/* Absolute value */
int abs(int x)[*\indextt{abs}*]
  /*@requires x > int_min();              @*/
  /*@ensures \result >= 0;                @*/
  /*@ensures \result == (x < 0 ? -x : x); @*/ ;

/* Maximum of two integers */
int max(int x, int y)[*\indextt{max}*]
  /*@ensures \result == x || \result == y; @*/
  /*@ensures \result >= x && \result >= y; @*/ ;

/* Minimum of two integers */
int min(int x, int y)[*\indextt{min}*]
  /*@ensures \result == x || \result == y; @*/
  /*@ensures \result <= x && \result <= y; @*/ ;

/* Returns the hexidecimal representation of the given integer
 * as a string (without the leading 0x) */
string int2hex(int x);[*\indextt{int2hex}*]
\end{lstlisting}


\section{Updates}
\label{sec:updates}
\hypertarget{sec:updates}{}

\begin{description}
\item[Rev. C0.0011, Jan 06, 2012.]%
  Image Library.  \lstinline'image_destroy' has been deprecated, since
  images are now garbage collected.  Contracts on image functions have
  been sharpened to require strictly positive height and width.
  \lstinline'image_load' returns \lstinline'NULL' if the file does not
  exist or is not readable.
\item[Rev. C0.0113, Sep 27, 2012.]%
  Conio Library.  Function \lstinline'eof' has been added to test
  end-of-file along standard input.  This fixes a problem with
  \lstinline'^D' during \lstinline'readline' and end of stream for
  shell redirect to stdin.  \lstinline'readline' now requires
  \lstinline'!eof()'.
\item[Rev. C0.0167, Dec 8, 2012.]%
  Conio Library.  Function \lstinline'error' has been removed from the
  conio library, since it now exists as a language primitive.
\item[Rev. C0.0273, Feb 1, 2013.]%
  Add \lstinline'util' and \lstinline'rand' libraries.
\item[Rev. C0.0438, January 8, 2015.]%
  Remove deprecated \lstinline'image_destroy' function, add
  \lstinline'max' and \lstinline'min' to \lstinline'util' and add
  \lstinline'num_tokens', \lstinline'parse_tokens',
  \lstinline'int_tokens', and \lstinline'parse_ints' to the
  \lstinline'parse' library. Formatting changes.
\item[Rev. C0.0590, March 30, 2018.]%
  Reformatted documentation to use the \lstinline'listings' Latex
  package.
\item[Rev. C0.0590, December 20, 2019.]%
  Added the table of contents and  index.
\end{description}


\newpage
\addcontentsline{toc}{section}{\indexname}
\printindex
\end{document}
