Designing a new language is not something one does lightly. There are hundreds
of existing languages to choose from with many different properties so it seems
reasonable to assume that there must exist some language suitable for teaching
introductory data structures. Many of these langauges were not designed with
pedagogical purposes in mind. They may have historical or designer quirks and
features that no longer (or never did) make sense. They are not usually designed
with novice programmers in mind but to accomplish some particular task or set of
tasks. In the case of Carnegie Mellon's introductory data structures course, a
suitable language would need to be easily accessible by programmers with
minimal experience and it must be able to adequately express the data
structures used in the course.

\section{Motivation}

Carnegie Mellon is revising its early undergraduate curriculum. In particular,
they are bringing Jeanette Wing's \cite{Henderson2007Computational} idea of
computational thinking to both majors and nonmajors, increase focus on software
reliability and integrate parallelism into the core curriculum. These changes
involve restructuring the course sequence and content.

The first course, 15-110, in the sequence is intended for nonmajors and majors
with no programming experience. It is intended to be an introductory computer
science course containing the basic principles including an emphasis on
computational thinking.  The next course, 15-122, focuses on writing imperative
programs with fundamental data structures and algorithms and reasoning about
them. Students also take a functional programming course, 15-150, which focuses
on working with immutable data and the concepts of modularity and abstraction.
Following 15-122 and 15-150, students take 15-210 where they continue to learn
more advanced data structures as in 15-122 though with an additional emphasis
on parallel programming. 15-213 remains an introductory systems course and
15-214 focuses on designing large-scale systems and includes content on
object-oriented programming and distributed concurrency.

For this thesis, we focus on 15-122 and the choice of programming language for
its assignments.

\section{Language Requirements}

To formulate the requirements, we must first examine the types of programs that
students will be assigned to write for the course. The assignments are intended
to assess and educate students according to the learning goals of the course.

The majority of students taking 15-122 will have scored well on the AP test
which currently uses Java. Those that have not will have taken a semester-long
basic programming course taught using Python as the primary language for
assignments.  Though students should be able to write basic programs upon entry
to this course, we do not assume that they will have had much experience in
debugging or using the intermediate or advanced features of either language.
This course is intended to instruct students in the basic data structures and
algorithms used in computer science. This includes:

\begin{itemize}

\item Stacks and Queues

\item Binary Search

\item Sorting

\item Priority queues

\item Graph traversal / shortest path

\item Balanced binary trees

\item Game tree search

\item String searching

\end{itemize}

By the end of the course, students should be able to:

\begin{itemize}

\item For the aforementioned data structures, explain the basic performance
characteristics for various operations and write imperative code that implements
these data structures.

\item Using their knowledge of the basic performance characteristics, choose
appropriate data structures to solve common computer science problems.

\item Be able to reason about their programs using invariants, assertions,
preconditions and postconditions.

\end{itemize}

Based on the learning goals of the course and its place in the course sequence,
I formulated the following design requirements for the language and
compiler/interpreter:

\begin{description}

\item[Simple as possible] The languages used in industry are aimed at
professionals and problems in software engineering, not necessarily computer
science. Many of these languages impose design requirements which, though often
valuable for software engineering purposes, are unnecessary cognitive overhead
for students to learn and comprehend. Explaining these features takes time away
from teaching the intended topics of the course.

\item[Friendly error messages]
Compilers are also not known for giving helpful or useful
error messages to novices. Most SML and C++ compilers are notorious in the
academic and industrial worlds for giving cryptic error messages that only
begin to make sense once a deep understanding of the language is obtained.

\item[Similar enough to languages used in industry] Students often want to
search for internships in the summer following their first year as an
undergraduate. It does not help them in their search if they have learned an
obscure or irrelevant programming language in this course. Thus our choice of
language must be or be similar enough to an existing language that is used in
industry so that students can legitimately claim, based on their coursework,
that they have a marketable skill.

\end{description}

\subsection{Simplicity}
\begin{quote}
{\it "Il semble que la perfection soit atteinte non quand il n'y a plus
rien \`a ajouter, mais quand il n'y a plus rien \`a retrancher" - Antoine de Saint
Exup\'ery}
\end{quote}

Some languages have low cognitive overhead because they can be taught in
layers, starting with simple concepts and introducing new features that build
upon these basics.  Algebra is often taught this way to high school students by
starting with the concepts of variables and the basic axioms and gradually
moving onto harder problems.  Carnegie Mellon's sophomore-level Principles of
Programming course teaches Standard ML this way: the first assignment has
students writing simple values and expressions in the language and then moves
on to functions and more complicated data structures like lists and
trees\footnote{Also see {\tt tryhaskell.org} for a compressed approach with
Haskell}.  Though these sorts of languages are often amenable to including an
interpreter, it is by no means a requirement.

Though languages like Java are quite popularly used in introductory data
structures courses, that does not mean they are a good choice.  Programming
even the tiniest bit in Java requires using classes, objects. For basic IO,
packages are needed.  Take the example of writing a linked list type and
calculating the length of a one element instance. The Java\cite{JavaSpec}
version (~\ref{javall}) requires 3 classes and the explanation of member
variables and methods. Compare this to the conceptual size of comparable code
in C\cite{ISO:C99} (~\ref{cll}) and Standard ML\cite{mthm97} (~\ref{smlll})
which just require understanding functions (and in ML's case, a recursively
defined one - more on this later).  Notice in particular how the Java
definition requires a separate Node and List class because methods such as
\langtext{length()} cannot be invoked on null objects\footnote{In C++ you can
get away with having a single class in most implementations since non-virtual
methods don't require a non-null \langtext{this}}.

\langsample{ll.java}{Java}{Simple linked list written in Java}{javall}
\langsample{ll.c}{C}{Simple linked list written in C}{cll}
\langsample{ll.sml}{ML}{Simple linked list written in SML}{smlll}

Looking at the course content, all that is needed to express the mathematical
ideas is relatively simple data structures and functions to modify and query
them. Furthermore, object-oriented programming is covered in a subsequent
couse. Thus our desired language need not support objects and should not
require them.

Functional programming languages like SML, Haskell and F\# are generally still
problematic for students new to programming. The majority (if not all) of
students taking 15-122 will have been taught a language like Python or Java and
likely have no experience working in a functional programming language. Often
the basic programming tasks in these functional languages require defining
simple functions recursively as in ~\ref{smlll} or using functions as first
class values which are difficult concepts for novices to understand. Though
understanding these concepts is important for their overall education, they are
not key towards understanding the material in the course. The functional
programming course, 15-150, will cover this material and functional programming
is also heavily used in the subsequent parallel data structures and algorithms
course, 15-210.

It is also often the case that many algorithms are given in terms of an
imperative pseudo-code with mutable variables and loops. These features are
desirable to have in our language as well to minimize the student's burden of
transferring their knowledge into code.

\subsection{Friendly Error Messages}

Error messages are a frequent source of frustration for introductory computer
science students. Even professionals find them unhelpful or even misleading.
Though some work has been done recently in production compilers
\footnote{http://clang.llvm.org/diagnostics.html}, many compilers produce
technically worded errors which are sometimes not even local to the actual
problem! Thus the compiler for 15-122 should produce friendly error messages
whenever possible. They should point to the precise source of the problem and
offer suggestions to fixes. Note that we don't want to automatically fix these
errors - the compiler may not always be able to choose the right fix to apply
and fixing problems in the source is an important skill for students to have.
For some languages, this can be rather difficult due to extensive elaboration or
other transformations of the source prior to typechecking.

\subsection{Similarity to a Real Industrial Language}

Carnegie Mellon undergraduates often take summer internships and end up taking a
fulltime job in the software industry upon graduation. While one could argue
that the purpose of a university education is not to be job training, it is
important that motivated freshmen should be able to obtain an internship after
their first year. It helps tremendously to be able to claim experience with a
well-known language used in industry. Though ideally the language used for the
assignments would be an industrial language, for previously mentioned reasons
most, if not all, of the popular languages in industry are so far unsuitable.
Going down the path of picking an unpopular language requires an eventual
transition to a real industrial language. The more similarities between our
language and some existing one allows for a easier transition for students. The
difficulty lies in balancing this requirement with the aforementioned ones, in
particular simplicity.

Carnegie Mellon's systems courses heavily use C for their assignments. Given
that 15-122 is a prerequisite for these courses and C's heavy use in industry,
it is the natural choice. Using a C-like language early in the curriculum may
have the added benefit of increasing students' proficiency with the language due
to the extra time they spend using the language.

\vspace{3em}

This thesis is organized as follows: Chapter 2 presents the language with
concrete syntax and a prose description of the semantics. Chapter 3 introduces
the abstract syntax and presents the progress and preservation properties.
Chapter 4 covers the implementation details of our reference compiler and
runtime. Chapter 5 introduces the standard libraries used by assignments in the
course.  Chapter 6 covers the differences between \langname{} and C and presents
ideas for future work.
