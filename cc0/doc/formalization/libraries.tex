These are the standard libraries as of this publication.

\newcommand{\describefunction}[2]{
\begin{tabbing}
\hspace{3em} \= \\
\> {\tt\small \lstinline|#1| } \\
\end{tabbing}
#2
}

\newcommand{\beginstruct}{
\begin{tabbing}
\hspace{3em} \= \hspace{9em} \= \\
}

\newcommand{\structfield}[2]{
\> {\tt\small \lstinline|#1|} \> #2 \\
}

\newcommand{\structend}{
\end{tabbing}
}

\section{Input/Output}

\subsection{\tt conio}

The {\tt conio} library contains functions for performing basic
console input and output.
\describefunction{void print(string s)}{
Prints {\tt s} to standard output.
}

\describefunction{void println(string s)}{
Prints {\tt s} to standard output along with a trailing newline \langtext{\\n}.
}
\describefunction{void printint(int i)}{
Prints the integer {\tt i} to standard output.
}
\describefunction{void printbool(bool b)}{
Prints the boolean {\tt b} to standard output.
}
\describefunction{void printchar(char c)}{
Prints the char c to standard output.
}
\describefunction{string readline()}{
Reads a sequence of characters from standard input followed by a newline
(either \langtext{\\n} or \langtext{\\r\\n}) and returns the sequence as a
string.  The trailing newline is not returned.
}

\describefunction{void error(string s)}{
Prints s to standard error and aborts the program.
}

\subsection{\tt file}

The {\tt file} library contains functions for reading lines out of files.
File handles are represented by the {\tt file\_t} type.  The handle contains
an internal position which ranges from 0 to the logical size of the file in
bytes.  File handles should be closed when they are no longer needed.  The
program must close them explicitly---garbage collection of a file handle
will not close it.

\describefunction{file\_t file\_read(string path)}{
Creates a handle for reading from the file given by the specified path. If
the file cannot be opened for reading, the program aborts.
}

\describefunction{void file\_close(file\_t f)}{
Releases any resources associated with the file handle. This function should
not be invoked twice on the same handle.
}

\describefunction{bool file\_eof(file\_t f)}{
Returns true if the internal position of the handle is the size of the file.
}

\describefunction{string file\_readline(file\_t f)}{
Reads a sequence of characters from the given file followed by a newline
(either \langtext{\\n} or \langtext{\\r\\n}) and returns the sequence as a
string, advancing the handle's internal position by the number of
characters in the returned string plus the trailing newline sequence.  The
trailing newline is not returned.
}

\subsection{\tt args}

The {\tt args} library provides functions for basic argument parsing. There are
several functions that set up the description of the argument schema and then a
single function (\langtext{args\_parse}) which performs the parsing.

\describefunction{void args\_flag(string name, bool *ptr)}{
Describes a simple boolean flag. If \langtext{name} is present on the
command line, \langtext{args\_parse} sets \langtext{*ptr} to
\langtext{true}.
}
\describefunction{void args\_int(string name, int *ptr)}{
Describes a switch expecting an integer of the form accepted by
\langtext{parse\_int} with \langtext{base = 10}. If \langtext{name} is
present on the command line, \langtext{args\_parse} sets \langtext{*ptr}
to the value parsed from the argument following \langtext{name}. If the
value could not be parsed, it is not set.
}
\describefunction{void args\_string(string name, string *ptr)}{
Describes a switch expecting some additional argument. If \langtext{name}
is present on the command line, \langtext{args\_parse} sets \langtext{*ptr}
to the argument following \langtext{name}.
}

The \langtext{args\_parse} function returns a pointer to a
\langtext{struct args}, which has the following members:
\beginstruct
    \structfield{int argc}{The count of unparsed arguments}
    \structfield{string[] argv}{An array containing the unparsed arguments}
\structend
By invariant, the length of the array \langtext{argv} is always
\langtext{argc}.

\describefunction{struct args *args\_parse()}{
Attempts to parse the command line arguments given to the program by the
operating system according to the argument schema described by calls to
the functions above.  Arguments that indicate a switch consume the next
argument.  Arguments that are not matched to switches or flags are
considered positional arguments and are returned in a pointer to an
\texttt{args} struct.  The result \texttt{args} struct does not contain
the name of the program itself.
}

\section{Data manipulation}

\subsection{\tt parse}

The {\tt parse} library provides two functions to parse integers and
booleans, returning pointers to the resulting data to indicate the
possibility of failure..

% These functions return pointers to integers
% and \langtext{struct parsed\_int}. \\\langtext{struct parsed\_bool} is has the
% following members:
% \beginstruct
% \structfield{bool parsed}{Indicates if the string was successfully parsed}
% \structfield{bool value}{If the {\tt parsed} field is true, holds the parsed
% value}
% \structend
% \langtext{struct parsed\_int} has the same members except that \langtext{value}
% is of type \langtext{int}.

\describefunction{bool *parse\_bool(string s)}{
Attempts to parse {\tt s} into a value of type {\tt bool}. Accepts {\tt
"true"} and {\tt "false"} as valid strings, and returns \langtext{NULL}
if {\tt s} is neither.
}
\describefunction{int *parse\_int(string s, int b)}{
Attempts to parse {\tt s} as a number written in base {\tt b}. Supported bases
range from \langtext{2} to \langtext{36}, with the letters {\tt A} through
{\tt Z} representing the digits above {\tt 9} in bases greater than {\tt 10}.
% include 8, 10 and 16. If {\tt b} is 0, the base of the number is inferred from
% the leading digits. \langtext{0x} indicates that the base is 16, otherwise
% \langtext{0} indicates base 8 and any other digit indicates base 10.
Returns \langtext{NULL} if {\tt s} cannot be completely parsed to an
\langtext{int}, or if its value would be too large to be represented as an
\langtext{int}.
}

\subsection{\tt string}

The {\tt string} library contains a few basic routines for working with strings
and characters.

\describefunction{int string\_length(string s)}{
Returns the number of characters in {\tt s}.
}
\describefunction{char string\_charat(string s, int n)}{
Returns the {\tt n}th character in {\tt s}. If {\tt n} is less than zero or
greater than the length of the string, the program aborts.
}
\describefunction{string string\_join(string a, string b)}{
Returns a string containing the contents of \langtext{b} appended to the
contents of \langtext{a}.  The result string has a length equal to the sum
of the lengths of \langtext{a} and \langtext{b}.
}

% force pagebreak -wjl
\newpage
\describefunction{string string\_sub(string s, int start, int end)}{
Returns the substring composed of the characters of {\tt s} beginning at index
given by {\tt start} and continuing up to but not including the index given
by end.
If \langtext{end == start}, the empty string is returned.
If \langtext{start} and \langtext{end} do not represent valid substring
indices, the program aborts.
}
\describefunction{bool string\_equal(string a, string b)}{
Returns \langtext{true} if the contents of \langtext{a} and \langtext{b} are
equal and \langtext{false} otherwise.
}
\describefunction{int string\_compare(string a, string b)}{
Compares \langtext{a} and \langtext{b} lexicographically. If \langtext{a} comes
before \langtext{b}, then the return value is \langtext{-1}. If
\langtext{string\_equal(a,b)} is \langtext{true}, the return value is
\langtext{0}.  Otherwise \langtext{a} comes after \langtext{b} and the
return value is \langtext{1}.
}

\describefunction{string string\_frombool(bool b)}{
Returns a canonical representation of \langtext{b} as a string. The
returned value will always be parsed by \langtext{parse\_bool} into a value
equal to {\tt b}.
}
\describefunction{string string\_fromint(int i)}{
Returns a canonical representation of {\tt i} as a string. The returned value
will always be parsed by \langtext{parse\_int} into a value equal to {\tt i}.
}
\describefunction{string string\_fromchar(char c)}{
Returns a string of length one containing the character {\tt c}.
}

\describefunction{string string\_tolower(string s)}{
Returns a string containing the same character sequence as {\tt s} but with
each uppercase character replaced by its lowercase version.
}

\describefunction{char[] string\_to\_chararray(string s)}{
Returns the characters of {\tt s}.  The length of the result array is
at least one more than the length of {\tt s}, and the end of the string is
indicated by a \langtext{'\\0'} character.
}

\describefunction{string string\_from\_chararray(char[] A)}{
Returns a string containing the characters from {\tt A} up to a terminating
\langtext{'\\0'} character.  If {\tt A} does not contain a \langtext{'\\0'}
character, the program will abort.
}

\describefunction{int char\_ord(char c)}{
Returns an integer representing the ASCII encoding of {\tt c}.
}

\describefunction{char char\_chr(int n)}{
Decodes {\tt n} as a 7-bit ASCII character and returns the result.
If {\tt n} cannot be decoded as 7-bit ASCII, the program aborts.
}

\section{Images}

\subsection{\tt img}

The {\tt img} library defines a type for two dimensional images represented as
pixels with 4 color channels---alpha, red, green and blue---packed into one {\tt
int}. It defines an image type {\tt image\_t}. Images must be explicitly
destroyed when they are no longer needed with the {\tt image\_destroy} function.

\describefunction{image\_t image\_create(int width, int height)}{
Creates an image with the given width and height. The default pixel color is
transparent black. {\tt width} and {\tt height} must be positive.
}

\describefunction{image\_t image\_clone(image\_t image)}{
Creates a copy of the image.
}

\describefunction{void image\_destroy(image\_t image)}{
Releases any internal resources associated with {\tt image}. The array returned
by a previous {\tt image\_data} call will remain valid however any subsequent calls using {\tt
image} will cause the program to abort.
}

\describefunction{image\_t image\_subimage(image\_t image, int x, int y, int w, int h)}{
Creates a partial copy of {\tt image} using the rectangle as the source
coordinates in {\tt image}. Any parts of the given rectangle that are not contained in
{\tt image} are treated as transparent black.
}

\describefunction{image\_t image\_load(string path)}{
Loads an image from the file given by {\tt path} and converts it if need be to
an ARGB image. If the file cannot be found, the program aborts.
}

\describefunction{void image\_save(image\_t image, string path)}{
Saves {\tt image} to the file given by {\tt path}. If the file cannot be
written, the program aborts.
}

\describefunction{int image\_width(image\_t image)}{
Returns the width in pixels of {\tt image}.
}

\describefunction{int image\_height(image\_t image)}{
Returns the height in pixels of {\tt image}.
}

\describefunction{int[] image\_data(image\_t image)}{
Returns an array of pixels representing the image. The pixels are given line by
line so a pixel at (x,y) would be located at \langtext{y*image\_width(image) +
x}. Any writes to the array will be reflected in calls to {\tt image\_save},
{\tt image\_clone} and {\tt image\_subimage}. The channels are encoded as
\langtext{0xAARRGGBB}.
}

