For 15-122, Principles of Imperative Computation, our implementation consists of
a compiler and three interchangable runtimes with different characteristics. Due
to the simplicity of the language and the design of the compiler, there is very
little that the runtime needs to do.

\section{Compiler}

The compiler, \langtext{cc0}, performs typechecking and optional elaboration of
dynamic specifications. It produces a C program which is then compiled with a
runtime using \langtext{gcc}. Since the C compiler need not follow the same
semantics required by \langname{}, \langtext{cc0} converts the program into A-normal
form and uses externally defined functions for certain operations such as
division and bit shifts. The optimization level for \langtext{gcc} is set very
low because some of its optimizations assume C's semantics (ex: integer overflow
is undefined in C) which can lead to incorrect program execution.

\langtext{cc0} uses the following representations for \langname{}'s types
\begin{tabular}{l|l}
\langname\ Type & C/C++ Type \\
\hline
\langtext{bool} & \langtext{bool} \\
\langtext{char} & \langtext{char} \\
\langtext{int} & \langtext{int32\_t} \\
\langtext{T*} & \langtext{T*} \\
\langtext{T[]} & \langtext{c0\_array*} \\
\langtext{struct S} & \langtext{struct S} \\
\langtext{string} & \langtext{c0\_string} \\
\end{tabular}

To ease interoperability with native code, structures are laid out in memory
using the same rules as C for member alignment and structure size. Each field
of a struct is aligned according to its C/C++ type. Array elements are also
aligned as in C.

Functions use the standard {\tt cdecl} calling convention for easy interaction
with C. To avoid collisions between C functions and \langname{} functions,
\langname{} function names and references are mangled to produce a unique but
fairly human-readable name. The compiler simply takes the function/global
variable name as written in \langname{} and prepends \langtext{\_c0\_} to them.
References to functions declared in libraries are not mangled.

\section{Runtimes}

We've developed three runtimes for \langname{}: \langtext{bare},
\langtext{c0rt}, and \langtext{unsafe}. Although our runtimes have only been
tested for x86 and x86-64, the language was designed to be purposely abstract
enough to be easily ported to other popular architectures.

The exact representation of arrays is provided by the runtime that the program
is linked against. Access to members and the array's length is provided via
runtime functions. Runtimes provide the interface declared in Appendix
\ref{c0runtimeinterface} for use by the generated code and any libraries:

Each runtime was built to suit a particular use case.

\begin{itemize}
\item
The \langtext{bare} runtime is an extremely basic runtime which provides bounds
checking but no garbage collection. Its primary purpose was to allow
\langname{} programs to be compiled and run on systems which aren't supported by
the garbage collector library used by the other runtimes. Since many of the
assignments and test programs allocate only a little memory, the lack of garbage
collection is not very noticeable. Arrays are represented as contiguous
allocation as in C but with a header that includes the size of the array for
bounds checking.

\item
The \langtext{c0rt} runtime performs bounds checking but also garbage collects
allocations using the Boehm-Demers-Weiser conservative collector \cite{BoehmGC}.
It uses the same representation for arrays as \langtext{bare}.

\item
The \langtext{unsafe} runtime garbage collects allocations using the same
conservative collector but performs no bounds checking. Arrays are represented
exactly as in C as a raw pointer to a contiguous block of cells. Since
\langtext{cc0} compiles to C, the header for the runtime can redefine the
macros used for array types and element access so that the efficient direct
access to array elements is used. If \langtext{gcc}'s optimizer could be
trusted, this would enable \langname{} programs compiled against the
\langtext{unsafe} runtime to better match C for execution speed.

\end{itemize}

Since strings are represented opaquely in the language, the runtime is free to
choose any representation for strings it wants. For efficient interoperability
with native code, all three runtimes represent strings as in C as an array of
ASCII characters with a null terminator however other representations (ex: ropes
\cite{Boehm95ropes}) are possible.
