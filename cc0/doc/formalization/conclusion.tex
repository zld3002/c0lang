\section{Related work}
Projects like SAFECode \cite{SAFECode} and CCured \cite{Necula02ccured} are also
attempts to produce safe C-like imperative languages. Unlike \langname{}, they
are aimed more at verifying or porting existing C programs. Another attempt at
an imperative systems language is Google's Go Language \cite{GoLang} which
breaks away from C but it does contain additional features which are only useful
for writing larger programs than those in 15-122. Go also does not have
assertions or exception built into the language, preferring instead to return
error codes and propagate or handle them explicitly without assistance from the
compiler. The designers' goal is to force programmers to think about proper error
handling whereas the intended focus of 15-122 is to encourage students to reason
about the correctness of their code.

Processing.org \cite{Processing} is another take on providing an easy and simple
programming environment for novice programmers. It specializes in programming
graphical applications such as games or visualizations. Though it simplifies
away some of its Java roots by eliminating the need for a driver class, it is by
its own admission "just Java, but with a new graphics and utility API along with
some simplifications."

\section{Future work and speculation}
The work presented in this thesis is intended to be only the start of a larger
project. Due to the time constraints of needing the language and compiler to be
finished in time for the course in the Fall of 2010, several interesting
projects and ideas have been given little attention. These include a static
prover for assertions, a module system for the language to replace the one found
in C, and using a precise garbage collector in the runtime as well as other more
advanced garbage collection techniques.

\subsection{Static prover for assertions}
\langname{} is likely simple enough to do some strong analysis of programs to
prove invariants and assertions will hold at runtime. The lack of an address-of
and cast operators limits aliasing and the local variables can be hoisted to the
heap by rewriting the program without changing any discernible semantics. As a
project for a graduate program analysis class, Karl Naden, Anvesh Komuravelli
and Sven Stork built a small static verification tool for programs.  They
annotated programs with assertions and invariants using JML-style notation.
The work at the end of the course showed promise towards being to be able to
prove some of the invariants relevant to the data structures used in the course.

\subsection{Module system \& Polymorphism}
C's module system has well known faults and both C and \langname{} lack a decent
means of supporting polymorphism. I did some preliminary work towards defining a
module system that was simple enough to be used by the course's students but
powerful enough to allow for abstract polymorphism. The intention was to allow
students to write generic data structures. Combined with the aforementioned
static verification tool, students would be able to write assertions and
independently check that their generic data structure or algorithm was correct.

\subsection{Precise collector}
Though the conservative collector in the runtime is adequate for most purposes
of the course, it is not ideal. Since there is no casting or internal aliasing,
it's possible to construct a more precise collector that can analyze the stack
layout and walk the heap efficiently. Using a precise collector would also
necessitate a change to the runtime interface to allow the garbage collector to
walk library-defined types and provide a rooting API.

\subsection{Dynamically check explicit deallocation}

Tools like Valgrind are often used on C and C++ programs to check for common
mistakes such as double frees and use-after-free bugs. \langname{} could be
augmented with the ability to explicitly free memory and at runtime (if not
statically) assert that the program does not use memory after freeing it or
free memory twice. This would bring the language closer to C without
compromising on safety.

\subsection{Convenient \langtext{printf} support}

One of the notable features absent from C in \langname{} is support for
variadic functions like \langtext{printf}. Supporting these functions while
maintaining type safety is rather non-trivial. Attempt to support it in
languages like SML have been tried by reformulating the format string as higher
order functions but that approach does not suit \langname{}.

\subsection{Formatting/style checker}

Good code hygiene is an often overlooked aspect of computer science education.
Though there is no single style for writing good C code, there are several
popular styles and techniques for all languages. Some upper level courses such
as Operating Systems include a code review as part of the grading mechanism.
Though quite time consuming, it provides generally helpful feedback to students
in ways that an automatic grader usually does not. Some languages such as Go
\cite{GoLang} include a formatting program which automatically
rewrites source files to a pre-defined style. The former is time consuming and
prone to divergent styles/feedback and the latter does not encourage/require
students to learn proper styling. A program which could consistently
verify/grade the style of student code and generate useful human readable
feedback, would be of enormous value to both the students and the course staff.

\subsection{Interactive debugger}

Asking students to run \langtext{gdb} on the compiled binaries is asking too
much of them. We can produce a debugger for \langname{} programs that can
provide a friendlier interface than GDB and enable additional features such as
reversible debugging. Some work on this was started before the course by Rob
Simmons.

\subsection{Feedback on the course and its use of \langnameb{}}

At the time of this publication, it is too early to know what impact choice of
\langname{} will have. Getting feedback from the students over the course of
their time at Carnegie Mellon will provide valuable data. Already from the first
few weeks, suggestions for incremental changes to the language and compiler have
been made from both students and staff.
