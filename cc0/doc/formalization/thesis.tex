%for a more compact document, add the option openany to avoid
%starting all chapters on odd numbered pages
\documentclass[12pt]{cmuthesis}

% This is a template for a CMU thesis.  It is 18 pages without any content :-)
% The source for this is pulled from a variety of sources and people.
% Here's a partial list of people who may or may have not contributed:
%
%        bnoble   = Brian Noble
%        caruana  = Rich Caruana
%        colohan  = Chris Colohan
%        jab      = Justin Boyan
%        josullvn = Joseph O'Sullivan
%        jrs      = Jonathan Shewchuk
%        kosak    = Corey Kosak
%        mjz      = Matt Zekauskas (mattz@cs)
%        pdinda   = Peter Dinda
%        pfr      = Patrick Riley
%        dkoes = David Koes (me)

% My main contribution is putting everything into a single class files and small
% template since I prefer this to some complicated sprawling directory tree with
% makefiles.

% some useful packages
\usepackage{times}
\usepackage{fullpage}
\usepackage{graphicx}
\usepackage{amsmath}
\usepackage{amsfonts}
\usepackage[numbers,sort]{natbib}
\usepackage[backref,pageanchor=true,plainpages=false, pdfpagelabels, bookmarks,bookmarksnumbered,
%pdfborder=0 0 0,  %removes outlines around hyper links in online display
]{hyperref}
\usepackage{subfigure}
\usepackage{appendix}
\usepackage{mathpartir}
\usepackage{listings}
\usepackage{multicol}

% Approximately 1" margins, more space on binding side
%\usepackage[letterpaper,twoside,vscale=.8,hscale=.75,nomarginpar]{geometry}
%for general printing (not binding)
\usepackage[letterpaper,twoside,vscale=.8,hscale=.75,nomarginpar,hmarginratio=1:1]{geometry}

% Provides a draft mark at the top of the document. 
\draftstamp{\today}{DRAFT}

\begin {document} 
\frontmatter

%initialize page style, so contents come out right (see bot) -mjz
\pagestyle{empty}

\newcommand{\langname}{${\rm C}_0$}
\newcommand{\langnameb}{${\bf C}_0$}

\title{
{\bf \langnameb{}, an Imperative Programming Language for Novice Computer Scientists}}
\author{Rob Arnold}
\date{October 2010}
\Year{2010}
\trnumber{}

\committee{
Frank Pfenning, Chair \\
Mark Stehlik \\
}

\support{This work was partially supported by the Computational Thinking Center
at Carnegie Mellon University, sponsored by Microsoft Research.}
\disclaimer{}

\keywords{c, programming language, education}

\maketitle

%\begin{dedication}
%For my dog
%\end{dedication}

\pagestyle{plain} % for toc, was empty

\begin{abstract}
The primary goal of a computer science course on data structures and algorithms
is to educate students on the fundamental abstract concepts and expose them to
concrete examples using programming assignments. For these assignments, the
programming language should not dominate the assignment but instead complement
it so that students spend most of their time engaged in learning the course
material. Choosing an appropriate programming language requires consideration
not just of the course material and the programs students will write but also
their prior programming experience and the position of the course in the course
sequence for their degree program.

We have developed a new programming language, \langname{}, designed for
Carnegie Mellon's new Principles of Imperative Programming course. \langname{}
is almost a subset of C, eschewing complex semantics and undefined or
unspecified behaviors in favor of simple semantics and formally specified
behavior. It provides features such as garbage collection and array bounds
checks to aid students in developing and reasoning about their programs. In
addition to the compiler and runtime, we have also developed a small set of
libraries for students to use in their assignments.


\end{abstract}

\begin{acknowledgments}
There are many people who supported me in my work.

First, I would like to thank my advisor Frank Pfenning for his guidance,
feedback, and funding this past year. I would also like to thank Mark Stehlik
for serving on my thesis committee.

Rob Simmons, William Lovas, and Roger Wolff were willing at all times to discuss
all ideas and implementation ideas. Michael Sullivan, Max Buevich, and Dan
Schafer also gave me good feedback during the development of \langname{}. Their
feedback was invaluable in helping me in this endeavor.

I want to extend a special thanks to Deborah Cavlovich for helping me sort out
all the administrative details and issues.

Finally I want to thank my parents, sister and grandmother for their support of
my education over the years.

\end{acknowledgments}



\tableofcontents
%\listoffigures
%\listoftables

\mainmatter

%% Double space document for easy review:
%\renewcommand{\baselinestretch}{1.66}\normalsize

% The other requirements Catherine has:
%
%  - avoid large margins.  She wants the thesis to use fewer pages, 
%    especially if it requires colour printing.
%
%  - The thesis should be formatted for double-sided printing.  This
%    means that all chapters, acknowledgements, table of contents, etc.
%    should start on odd numbered (right facing) pages.
%
%  - You need to use the department standard tech report title page.  I
%    have tried to ensure that the title page here conforms to this
%    standard.
%
%  - Use a nice serif font, such as Times Roman.  Sans serif looks bad.
%
% Other than that, just make it look good...
\newcommand{\todo}[1]{{\bf TODO: #1}}
\newcommand{\langtext}[1]{{\tt\small \lstinline|#1|}}
\newcommand{\langtextish}[1]{{\tt #1}}
\newcommand{\abslang}[1]{{\tt #1}}

\lstset{language=C++,showstringspaces=false,morekeywords=string}

% text - caption - reference name
\newcommand{\langsamplewhere}[5]{
  \begin{figure}[#5]
  \lstinputlisting[language=#2,
                   basicstyle=\small\tt,
                   title=#3,
                   label=#4]{#1}
  \end{figure}
}
\newcommand{\langsampleinline}[1]{
  \begin{figure}[here]
  \langtext{#1}
  \end{figure}
}
\newcommand{\langsnippet}[1]{
  \lstinputlisting[language=C,
                   morekeywords={bool,string},
                   basicstyle=\small\tt]{#1}
}
\newcommand{\langsample}[4]{\langsamplewhere{#1}{#2}{#3}{#4}{p}}
\newcommand{\footer}[1]{{\bf Footnote:} #1}

% formal language definition required commands
\newcommand{\G}{\Gamma}
\newcommand{\entails}{\vdash}
\newcommand{\startdef}[2]{\ensuremath{#1} & ::= & \ensuremath{#2} \\}
\newcommand{\contdef}[1]{ &$\vert$ & \ensuremath{#1} \\}
\newcommand{\OR}{\ensuremath{\ | \ }}

\newcommand{\kw}[1]{{\bf #1}}


% Include all the formal language definitions
\input{l5gen.tex}

% Include the grammar definitions
\newcommand{\nt}[1]{{\normalsize \em #1}}
\newcommand{\lit}[1]{{\normalsize \tt #1}}
\newcommand{\declprod}[1]{\nt{#1:}\\}
\newcommand{\proddef}[1]{\> #1\\}
\newcommand{\maybe}[1]{\ensuremath{$#1$_{opt}}}
\newcommand{\kstar}[1]{#1*}

\newcommand{\grammarbegin}{
\begin{tabbing}
\hspace{3em} \= \\
}
\newcommand{\grammarend}{\end{tabbing}}

\newcommand{\grmboolconst}{
\declprod{bool-constant}
\proddef{\lit{true}}
\proddef{\lit{false}}
}

\newcommand{\grmexprcommon}{
\declprod{expression}
\proddef{\nt{constant}}
\proddef{\nt{variable-name}}
\proddef{\nt{expression} \nt{binop} \nt{expression}}
\proddef{\nt{monop} \nt{expression}}
}

\newcommand{\grmexprarray}{
\proddef{\nt{expression} \lit{[} \nt{expression} \lit{]}}
}

\newcommand{\grmcallexpr}{
\declprod{call-expression}
\proddef{\nt{lhs} \lit{(} \maybe{\nt{argument-list}} \lit{)}}
}

\newcommand{\grmcallstm}{
\proddef{\maybe{\nt{call-expression}}}
}

\newcommand{\grmselectionstatement}{
\declprod{selection-statement}
\proddef{\lit{if} \lit{(} \nt{expression} \lit{)} \nt{statement}}
\proddef{\lit{if} \lit{(} \nt{expression} \lit{)} \nt{statement} \lit{else} \nt{statement}}
}

\newcommand{\grmwhilestatement}{
\proddef{\lit{while} \lit{(} \nt{expression} \lit{)} \nt{statement}}
}

\newcommand{\grmforstatement}{
\proddef{\lit{for} \lit{(} \nt{initializer} \lit{;} \maybe{\nt{expression}} \lit{;} \nt{simple-statement} \lit{)} \nt{statement}}
}

\newcommand{\grmbreakstatement}{
\proddef{\lit{break} \lit{;}}
}

\newcommand{\grmcontinuestatement}{
\proddef{\lit{continue} \lit{;}}
}

\newcommand{\grmaebasic}{
\declprod{atomic-expression}
\proddef{\nt{variable-name}}
}
\newcommand{\grmaeptr}{
\proddef{\lit{(} \lit{*} \nt{lhs} \lit{)}}
}
\newcommand{\grmaearr}{
\proddef{\nt{lhs} \lit{[} \nt{expression} \lit{]}}
}
\newcommand{\grmaestr}{
\proddef{\nt{lhs} \lit{->} \nt{field-name}}
\proddef{\nt{lhs} \lit{.} \nt{field-name}}
}

\newcommand{\grmlhsbasic}{
\declprod{lhs}
\proddef{\nt{atomic-expression}}
}

\newcommand{\grmlhsptr}{
\proddef{\lit{*} \nt{lhs}}
}

\newcommand{\grmdeclstatement}{
\declprod{declaration}
\proddef{\nt{ty} \nt{variable-name}}
}

\newcommand{\grmdeclexpstatement}{
\proddef{\nt{ty} \nt{variable-name} \lit{=} \nt{expression}}
}

\newcommand{\grmassignstatement}{
\declprod{simple-statement}
\proddef{\nt{lhs} \nt{assign-op} \nt{exp}}
}

\newcommand{\grminitializerstatement}{
\declprod{initializer}
\proddef{\nt{declaration}}
\proddef{\nt{simple-statement}}
}

\newcommand{\grmcompoundstatement}{
\declprod{compound-statement}
\proddef{\lit{\{} \kstar{\nt{statement}} \lit{\}}}
}

\newcommand{\grmfundecl}{
\declprod{function-decl}
\proddef{\nt{ty} \nt{function-name} \lit{(} \maybe{\nt{function-params}} \lit{)}}
}

\newcommand{\grmfundef}{
\declprod{function-def}
\proddef{\nt{function-decl} \nt{compound-statement}}
}


\chapter{Introduction}
Designing a new language is not something one does lightly. There are hundreds
of existing languages to choose from with many different properties so it seems
reasonable to assume that there must exist some language suitable for teaching
introductory data structures. Many of these langauges were not designed with
pedagogical purposes in mind. They may have historical or designer quirks and
features that no longer (or never did) make sense. They are not usually designed
with novice programmers in mind but to accomplish some particular task or set of
tasks. In the case of Carnegie Mellon's introductory data structures course, a
suitable language would need to be easily accessible by programmers with
minimal experience and it must be able to adequately express the data
structures used in the course.

\section{Motivation}

Carnegie Mellon is revising its early undergraduate curriculum. In particular,
they are bringing Jeanette Wing's \cite{Henderson2007Computational} idea of
computational thinking to both majors and nonmajors, increase focus on software
reliability and integrate parallelism into the core curriculum. These changes
involve restructuring the course sequence and content.

The first course, 15-110, in the sequence is intended for nonmajors and majors
with no programming experience. It is intended to be an introductory computer
science course containing the basic principles including an emphasis on
computational thinking.  The next course, 15-122, focuses on writing imperative
programs with fundamental data structures and algorithms and reasoning about
them. Students also take a functional programming course, 15-150, which focuses
on working with immutable data and the concepts of modularity and abstraction.
Following 15-122 and 15-150, students take 15-210 where they continue to learn
more advanced data structures as in 15-122 though with an additional emphasis
on parallel programming. 15-213 remains an introductory systems course and
15-214 focuses on designing large-scale systems and includes content on
object-oriented programming and distributed concurrency.

For this thesis, we focus on 15-122 and the choice of programming language for
its assignments.

\section{Language Requirements}

To formulate the requirements, we must first examine the types of programs that
students will be assigned to write for the course. The assignments are intended
to assess and educate students according to the learning goals of the course.

The majority of students taking 15-122 will have scored well on the AP test
which currently uses Java. Those that have not will have taken a semester-long
basic programming course taught using Python as the primary language for
assignments.  Though students should be able to write basic programs upon entry
to this course, we do not assume that they will have had much experience in
debugging or using the intermediate or advanced features of either language.
This course is intended to instruct students in the basic data structures and
algorithms used in computer science. This includes:

\begin{itemize}

\item Stacks and Queues

\item Binary Search

\item Sorting

\item Priority queues

\item Graph traversal / shortest path

\item Balanced binary trees

\item Game tree search

\item String searching

\end{itemize}

By the end of the course, students should be able to:

\begin{itemize}

\item For the aforementioned data structures, explain the basic performance
characteristics for various operations and write imperative code that implements
these data structures.

\item Using their knowledge of the basic performance characteristics, choose
appropriate data structures to solve common computer science problems.

\item Be able to reason about their programs using invariants, assertions,
preconditions and postconditions.

\end{itemize}

Based on the learning goals of the course and its place in the course sequence,
I formulated the following design requirements for the language and
compiler/interpreter:

\begin{description}

\item[Simple as possible] The languages used in industry are aimed at
professionals and problems in software engineering, not necessarily computer
science. Many of these languages impose design requirements which, though often
valuable for software engineering purposes, are unnecessary cognitive overhead
for students to learn and comprehend. Explaining these features takes time away
from teaching the intended topics of the course.

\item[Friendly error messages]
Compilers are also not known for giving helpful or useful
error messages to novices. Most SML and C++ compilers are notorious in the
academic and industrial worlds for giving cryptic error messages that only
begin to make sense once a deep understanding of the language is obtained.

\item[Similar enough to languages used in industry] Students often want to
search for internships in the summer following their first year as an
undergraduate. It does not help them in their search if they have learned an
obscure or irrelevant programming language in this course. Thus our choice of
language must be or be similar enough to an existing language that is used in
industry so that students can legitimately claim, based on their coursework,
that they have a marketable skill.

\end{description}

\subsection{Simplicity}
\begin{quote}
{\it "Il semble que la perfection soit atteinte non quand il n'y a plus
rien \`a ajouter, mais quand il n'y a plus rien \`a retrancher" - Antoine de Saint
Exup\'ery}
\end{quote}

Some languages have low cognitive overhead because they can be taught in
layers, starting with simple concepts and introducing new features that build
upon these basics.  Algebra is often taught this way to high school students by
starting with the concepts of variables and the basic axioms and gradually
moving onto harder problems.  Carnegie Mellon's sophomore-level Principles of
Programming course teaches Standard ML this way: the first assignment has
students writing simple values and expressions in the language and then moves
on to functions and more complicated data structures like lists and
trees\footnote{Also see {\tt tryhaskell.org} for a compressed approach with
Haskell}.  Though these sorts of languages are often amenable to including an
interpreter, it is by no means a requirement.

Though languages like Java are quite popularly used in introductory data
structures courses, that does not mean they are a good choice.  Programming
even the tiniest bit in Java requires using classes, objects. For basic IO,
packages are needed.  Take the example of writing a linked list type and
calculating the length of a one element instance. The Java\cite{JavaSpec}
version (~\ref{javall}) requires 3 classes and the explanation of member
variables and methods. Compare this to the conceptual size of comparable code
in C\cite{ISO:C99} (~\ref{cll}) and Standard ML\cite{mthm97} (~\ref{smlll})
which just require understanding functions (and in ML's case, a recursively
defined one - more on this later).  Notice in particular how the Java
definition requires a separate Node and List class because methods such as
\langtext{length()} cannot be invoked on null objects\footnote{In C++ you can
get away with having a single class in most implementations since non-virtual
methods don't require a non-null \langtext{this}}.

\langsample{ll.java}{Java}{Simple linked list written in Java}{javall}
\langsample{ll.c}{C}{Simple linked list written in C}{cll}
\langsample{ll.sml}{ML}{Simple linked list written in SML}{smlll}

Looking at the course content, all that is needed to express the mathematical
ideas is relatively simple data structures and functions to modify and query
them. Furthermore, object-oriented programming is covered in a subsequent
couse. Thus our desired language need not support objects and should not
require them.

Functional programming languages like SML, Haskell and F\# are generally still
problematic for students new to programming. The majority (if not all) of
students taking 15-122 will have been taught a language like Python or Java and
likely have no experience working in a functional programming language. Often
the basic programming tasks in these functional languages require defining
simple functions recursively as in ~\ref{smlll} or using functions as first
class values which are difficult concepts for novices to understand. Though
understanding these concepts is important for their overall education, they are
not key towards understanding the material in the course. The functional
programming course, 15-150, will cover this material and functional programming
is also heavily used in the subsequent parallel data structures and algorithms
course, 15-210.

It is also often the case that many algorithms are given in terms of an
imperative pseudo-code with mutable variables and loops. These features are
desirable to have in our language as well to minimize the student's burden of
transferring their knowledge into code.

\subsection{Friendly Error Messages}

Error messages are a frequent source of frustration for introductory computer
science students. Even professionals find them unhelpful or even misleading.
Though some work has been done recently in production compilers
\footnote{http://clang.llvm.org/diagnostics.html}, many compilers produce
technically worded errors which are sometimes not even local to the actual
problem! Thus the compiler for 15-122 should produce friendly error messages
whenever possible. They should point to the precise source of the problem and
offer suggestions to fixes. Note that we don't want to automatically fix these
errors - the compiler may not always be able to choose the right fix to apply
and fixing problems in the source is an important skill for students to have.
For some languages, this can be rather difficult due to extensive elaboration or
other transformations of the source prior to typechecking.

\subsection{Similarity to a Real Industrial Language}

Carnegie Mellon undergraduates often take summer internships and end up taking a
fulltime job in the software industry upon graduation. While one could argue
that the purpose of a university education is not to be job training, it is
important that motivated freshmen should be able to obtain an internship after
their first year. It helps tremendously to be able to claim experience with a
well-known language used in industry. Though ideally the language used for the
assignments would be an industrial language, for previously mentioned reasons
most, if not all, of the popular languages in industry are so far unsuitable.
Going down the path of picking an unpopular language requires an eventual
transition to a real industrial language. The more similarities between our
language and some existing one allows for a easier transition for students. The
difficulty lies in balancing this requirement with the aforementioned ones, in
particular simplicity.

Carnegie Mellon's systems courses heavily use C for their assignments. Given
that 15-122 is a prerequisite for these courses and C's heavy use in industry,
it is the natural choice. Using a C-like language early in the curriculum may
have the added benefit of increasing students' proficiency with the language due
to the extra time they spend using the language.

\vspace{3em}

This thesis is organized as follows: Chapter 2 presents the language with
concrete syntax and a prose description of the semantics. Chapter 3 introduces
the abstract syntax and presents the progress and preservation properties.
Chapter 4 covers the implementation details of our reference compiler and
runtime. Chapter 5 introduces the standard libraries used by assignments in the
course.  Chapter 6 covers the differences between \langname{} and C and presents
ideas for future work.

\chapter{Language Description}
\langname{} is a statically typed imperative programming language. It strongly
resembles C in syntax and semantics. In designing it, we tried to stay close to
C's syntax and semantics in the hopes that students could simply include a
header file in each \langname{} program and it would be accepted by a C
compiler.  Unfortunately, some of \langname's semantics differ from those of C
and in these cases, using C's syntax would be a confusing choice for students
when it came time for them to learn C. Thus we changed the syntax for these
constructs to the syntax of modern languages with similar semantics. This made
it impossible for there to be a header that will allow \langname{} programs to
be compiled as C.  However, the transformation remains a simple operation with a
trivial {\tt sed} script or doing it by hand for small programs.

\section{Basic data types and expressions}

Expressions are evaluated in left-to-right order (that is, a post-order
traversal of the abstract syntax tree). Expressions come in many forms. The
common ones are:

\grammarbegin
\grmexprcommon
\grammarend

Expressions in \langname{} are not pure; they may have side effects such as
modifying memory or raising exceptions.

\subsection{Booleans}
\langname{} has a few basic builtin types. The simplest of these is the boolean
type. Similar to C++, Java and C\#, we write the type as \langtext{bool} and its
two values are \langtext{true} and \langtext{false}.

\grammarbegin
\grmboolconst

\declprod{constant}
\proddef{\nt{bool-constant}}

\declprod{ty}
\proddef{\lit{bool}}
\grammarend

\langsampleinline{bool b = true \|\| false \&\& !(x == y);}

The usual set of basic boolean operations are provided for logical and
(\langtext{\&\&}), logical or (\langtext{\|\|}), logical not (\langtext{!}), and
equality/inequality (\langtext{==},\langtext{!=}). As in many languages, these
operators can "short circuit" their evaluation when the value from evaluating
the left hand side determines the expression's value.

There is also a conditional expression as in C and Java: \langtext{e ? a : b}
where \langtext{e} is an expression that yields a boolean. If it evaluates to
\langtext{true} then \langtext{a} is evaluated to be the result of the ternary
expression. Otherwise \langtext{b} is evaluted to be the result of the ternary
expression.

\subsection{Integers}

\langname{} has a single numerical type \langtext{int} which has a range of
$[-2^{31},2^{31})$. It is represented as a signed 32 bit two's complement
integer. We felt that it would be more helpful to specify the exact size,
numerical range and representation of the integer type.  In addition, having a
single numerical type greatly simplifies the language and cognitive requirements
for students. It is an unfortunate limitation that the numerical type is
integral as there are a number of interesting data structures and projects that
would be much easier to represent as some non-integral type like IEEE 754
floating point or .NET's decimal type.  The problem with these data types
however is that they are only approximations of real numbers so students will
need to be made aware of their underlying implementation and properties.  These
properties are non-trivial and the cognitive overhead required for students to
understand them is not worth the time and effort when there are already many
interesting data structures within the scope of the course that work well with
integers. Furthermore, these numerical types and their representations are
covered in the subsequent introductory systems course, 15-213.

Integer literals are written as a sequence of digits, either hexadecimal digits
with an \langtext{0x} prefix as in C or decimal digits with an optional
\langtext{-} prefix to indicate negative numbers. Basic arithmetic operations
include modular addition, subtraction, negation, multiplication, division and
remainder operations (\langtext{+ - * / \%}). For the division and remainder
operations, if the divisor is 0 or additionally for
negation if the result does not fall into the range of the \langtext{int} type,
an exception is triggered which results in the termination of the program. The
division and remainder operations round towards 0.

\grammarbegin
\declprod{ty}
\proddef{\lit{int}}

\declprod{constant}
\proddef{\nt{integer-constant}}
\grammarend

\langsampleinline{int x = 0x123 + -456 * 20;}

Basic bitwise operations are also supported: bitwise and (\langtext{\&}),
bitwise or (\langtext{\|}), bitwise xor ({\tt \^{}}), bitwise inversion ({\tt \~{}}).
Arithmetic shift operations are also supported with the \langtext{<<} and
\langtext{>>} binary operators. Only the least significant 5 bits of the
shift's two's complement binary representation are used to determine the number
of bits that are shifted. These 5 bits are treated as an unsigned integer. For
example, the expression \langtext{\~19 << 2} evaluates to \langtext{-80}.

\subsection{Strings and Characters}

\langname{} provides a basic opaque type \langtext{char} for representing
characters. The exact representation is not defined however they may be
specified by a printable ASCII character enclosed in single quotes such as
\langtext{'a'} and \langtext{'9'}. To accommodate some useful but not printable
ASCII characters, the following escape sequences are also permitted:
\langtextish{\textbackslash{}t} (tab), \langtextish{\textbackslash{}r} (return),
\langtextish{\textbackslash{}b} (backspace), \langtextish{\textbackslash{}n}
(newline), \langtextish{\textbackslash{}'} (single quote),
\langtextish{\textbackslash{}"} (double quote) and
\langtextish{\textbackslash{}0} (null).

\langname{} also defines an opaque immutable string type \langtext{string} which
contains a sequence of characters. There is no direct conversion between strings
and characters however several library functions such as
\langtext{string_charat} and \langtext{string_to_chararray} provide access to
the characters. String literals are written as a sequence of ASCII characters or
permitted \langtext{char} escape sequences (except \langtextish{\textbackslash
0}) enclosed in double quotes. Strings may be compared using the library
functions \langtext{string_equal} and \langtext{string_compare}.

\langtext{string s = "Hello world";}

\grammarbegin
\declprod{ty}
\proddef{\lit{char}}
\proddef{\lit{string}}

\declprod{constant}
\proddef{\nt{character-constant}}
\proddef{\nt{string-constant}}
\grammarend

\subsection{Comparisons}

Owing to its C heritage, \langname{} provides equality and inequality operators
as \langtext{==} and \langtext{!=} respectively for expressions of the same
type. These operators are defined for the \langtext{bool}, \langtext{int} and
\langtext{char} types as well as pointer types. Pointer types are checked only
for reference equality.  Comparisons on integers are also provided with the
standard and self-explanatory concrete syntax from C (\langtext{< <= > >=}).

\section{Variables \& Statements}

For now we will consider only three kinds of statements: blocks, variable
declarations and assignments. As in C, blocks are defined as a possibly empty
sequence of statements enclosed in curly braces (\langtext{\{ \}}). They are
used for grouping related statements and determining variable scope. A block
statement is executed by executing each of its statements in sequence.

\grammarbegin
\grmcompoundstatement
\grammarend

Variables are automatically managed memory locations that hold values.
Variables are identified using a sequence of alphanumeric characters (though
the identifier must start with a letter). Variable declarations such as
\langtext{int x;} introduce a new variable that can be used by expressions and
assignments. As in C99, a variable's scope is defined as the statement
following the variable declaration to the end of the innermost block the
declaration occured in.

\grammarbegin
\grmdeclstatement
\grammarend

Variable declaration restrictions:
\begin{enumerate}

\item A variable may only be referenced by expressions and statements inside
its scope.

\item A variable may not be used in an expression if it cannot be statically
verified that it has been assigned prior to use.  That is, an assignment
statement must dominate all uses in the control flow graph of the block
containing the variable's scope. Loops are assumed to potentially never run for
the purpose of this analysis.

\item Two variables of the same name may not have overlapping scopes.

\item The variable's type must be \langtext{bool}, \langtext{int}, a pointer
type or an array type.

\end{enumerate}

Ex:

\langsnippet{assignmentsample.c}

Assignment statements are used to update values stored in variables. The variable on the left hand
side must be in scope.

\grammarbegin
\grmassignstatement

\grmaebasic

\grmlhsbasic

\grammarend

For convenience, there is also hybrid class of assignment statements that
combine a binary operator with assignment for the common case where the left
hand side of a binary operation is the left hand side of the assignment as well.
For the assignment to variables, \langtext{x op=e} is simple syntactic sugar
for \langtext{x = x op e}. Note that \langtext{ *p op= e } cannot
simply desugar to \langtext{ *p = *p op e} when the evaluation of \langtext{p}
may have side effects. For this reason, the left hand side is
evaluated before the right hand side and only once.

There is a hybrid statement that is formed by combining a variable declaration
with an assignment:

\grammarbegin
\declprod{declaration}
\grmdeclexpstatement
\grammarend

This is merely syntactic sugar: $\langtext{T x = e;} \equiv \langtext{T x; x =
e;}$

\section{Functions} Functions are the primary mechanism for abstracting
computations and reusing code. A function declares parameters which behave much
like variable declarations at the top of the function's body except that they
are guaranteed to be assigned some value upon entry to the body of the function.
A function's body is executed when the function is applied to some arguments.
The arguments are evaluated to values in the conventional left-to-right
ordering. These values are then bound to the function's parameters in the same
order and the body of the function is then executed in a new scope where only
those parameters are bound. This means that functions cannot refer to variables
declared in other functions. Thus each function can be typechecked independently
from the definitions (but not declarations) of other functions.

\grammarbegin
\grmcompoundstatement

\grmfundecl

\grmfundef
\grammarend

Function application may occur as a top-level statement (that is, for its
side-effects only) or as part of an expression, including arguments to other
function applications. As an expression, function application must evaluate to
some value. To indicate this value, a function uses the \langtext{return}
statement. The \langtext{return} statement evaluates an expression to a value
and then aborts execution of the callee's body, yielding the value as the result
of the function application in the caller. To indicate what type of value is
returned, functions declare a return type. All functions must return a value and
must only return values of the declared return type. As with checking for
assignment before use for variables, a conservative analysis is used so loops
are not considered to necessarily be run, even in the trivial cases.

\grammarbegin
\grmcallexpr

\declprod{simple-statement}
\grmcallstm
\grammarend

For example, suppose we have:

\langsnippet{simplefunction.c}

which is a function named \langtext{triarea} that returns a value of type
\langtext{int} and has two parameters of type \langtext{int}, \langtext{b} and
\langtext{h}. \langtext{triarea(3,4)} is an application of \langtext{triarea} to
the expressions \langtext{3} and \langtext{4}. To evaluate this application, the
arguments are evaluated into values which they already are, then they are bound
in a new scope to the variables \langtext{b} and \langtext{h}. The return
statement then evaluates \langtext{b*h/2} to a value (the integer represented by
\langtext{6}) which then becomes the result of the original application.

Since the parameters are bound to values, a callee cannot affect the variable
bindings in its caller's scope\footnote{It can effect a change through the heap
with more advanced types which refer to the heap such as pointers and arrays.
The heap is discussed in section ~\ref{heapsection}}.

Functions may be named with any identifier using the same rules as for
variables. Function and variable names may not conflict. Like C, a function
needs to be declared before it can be used in another function.

\langsnippet{callbyvalue.c}

\section{Conditionals}

\langname{} offers a subset of the control flow statements in C that are found in
other popular languages. The most basic one is \langtext{if} statement. The
\langtext{if} statement allows execution the execution of a statement
conditional upon some provided condition. It has two forms:

\grammarbegin
\grmselectionstatement
\grammarend

Each time the \langtext{if} statement is executed, the condition is evaluated
and the corresponding inner statement (if there is one) is executed. To
repeatedly execute a statement until a condition no longer holds, the
\langtext{while} statement can be used. It behaves just like the first form of
an \langtext{if} statement that is re-executed until the condition is false.

\grammarbegin
\declprod{iteration-statement}
\grmwhilestatement
\grammarend

\langsnippet{whilesample.c}

There are two additional statement types that are available for use within a
loop: \langtext{break} and \langtext{continue}. Whenever a \langtext{continue}
statement is executed, execution jumps to the top of the current innermost loop
as if the body were done executing. Whenever a \langtext{break} is executed
within a loop, the current innermost loop is "broken" and execution resumes at
whatever statement or condition follows that loop, as if the condition were
\langtext{false}.

\grammarbegin
\declprod{jump-statement}
\grmbreakstatement
\grmcontinuestatement
\grammarend

There is an additional form of loop, the \langtext{for} loop. It behaves much
like C's: there is an initial statement that is executed before the loop, a
condition that is checked before every iteration, much like the while loop, and
a post-loop-body statement which is executed after the body of the loop.

\grammarbegin
\grminitializerstatement{}

\declprod{iteration-statement}
\grmforstatement
\grammarend

Either of the two statements may be omitted and if the condition is omitted, it
is assumed to be \langtext{true}. The initial statement may be a declaration
but the post statement may not. To give the same behavior as C99/C++, the for
loop is wrapped in an implicit block so the scope of a declaration in the
initial statement is limited to the condition, body, and post statement.

As with \langtext{while} statements, the body is repeatedly executed while the
condition is true. After the execution of the body, the post statement is run.
The \langtext{continue} statement behaves slightly differently in a
\langtext{for} loop; it jumps to the end of the body so that the post-loop-body
statement is executed. The \langtext{break} statement behaviors exactly the
same as in the \langtext{while} loop - it does not execute the post statement.

\section{The Heap \& Pointers} \label{heapsection}

The heap is a separate persistent memory store, accessible from any function.
Access to values in the heap is done only through references. References to
values in the heap (addresses) are stored as pointer values.  These pointer
values may be copied around and passed to functions just as any other value is.
They have a type dependent upon the type value they reference.  This type is
written as \langtext{T*} where \langtext{T} is the type of the value pointed
to.

As with all local variables, pointers must be initialized before use. A pointer
may be initialized to either a new value on the heap, the address of an existing
value in the heap, or a special value called \langtext{NULL}. Attempts to
reference the heap using \langtext{NULL} will abort execution. It exists solely
to indicate an invalid reference.

New values on the heap are created with the \langtext{alloc} keyword like so:

\grammarbegin
\declprod{expression}
\proddef{\lit{alloc} \lit{(} \nt{ty} \lit{)}}
\grammarend

Because it is difficult to determine if a value in the heap is initialized
before being used, all values in the heap are initialized to type-dependent
default values as follows:

\begin{tabular}{r|l}

\langtext{bool} & \langtext{false} \\

\langtext{int} & \langtext{0} \\

\langtext{char} & \langtext{'\0'} \\

\langtext{string} & \langtext{""} \\

\langtext{T*} & \langtext{NULL} \\

\langtext{T[]} & An array of dimension 0 \\

\langtext{struct { ... } } & Each field initialized according to its type.  \\

\end{tabular}

Heap values may be read by de-referencing a pointer value. Because pointers
always point to the heap, there is no explicit de-allocation of heap memory (no
\langtext{free}, no casting of pointer types and no pointer arithmetic in the
language, a pointer value is either \langtext{NULL} or a valid address in the
heap. There is also no means to obtain a pointer to a variable. The reasons for
this are discussed in section \ref{noaddressof}.

\grammarbegin
\declprod{monop}
\proddef{\lit{*}}

\declprod{lhs}
\grmlhsptr
\grammarend

\langsnippet{heapsample.c}

\section{Arrays}

\langname{} supports variable-sized mono-typed-element arrays much like those
found in Java or C\#. Array access is checked against the bound of the array,
yielding a runtime error if the bounds are exceeded. Arrays are passed around by
reference; their values are stored in the heap.

Arrays have their element type declared as in Java: \langtext{int[] x;} This is
a departure from the syntax of C but the semantics are also different. The
semantics of \langname{} arrays are identical (modulo bounds checking) to those
of pointers of the same base type in C.

An array is created by the \langtext{alloc\_array} which takes the array's
element type and the array dimensions. If the dimensions are less than 0, a
runtime error is issued and program execution is aborted. Arrays of dimension 0
are permitted but rather useless. All elements of the new array are initialized
as described in section \ref{heapsection}.

Array access uses the familiar C syntax:

\grammarbegin
\declprod{expression}
\grmexprarray

\declprod{atomic-expression}
\grmaearr
\grammarend

\langsnippet{arraysample.c}

\section{Structures}

\langname{} includes structure types like those found in C - a nominally-typed
set of name/value pairs (fields). Because structures allow for recursive types,
there are some additional complications so that compatibility with C is
maintained. Structure types may not be declared in functions - they are instead
declared at the global scope along with functions.

\grammarbegin
\declprod{ty}
\proddef{\nt{struct-decl}}

\declprod{struct-decl}
\proddef{\lit{struct} \nt{struct-name}}

\declprod{struct-def}
\proddef{\nt{struct-decl} \lit{\{} \maybe{\nt{struct-fields}} \lit{\}}}

\declprod{struct-fields}
\proddef{\nt{field-decl}}
\proddef{\nt{struct-fields} \nt{field-decl}}

\declprod{field-decl}
\proddef{\nt{ty} \nt{field-name} \lit{;}}

\declprod{gdecl}
\proddef{\nt{struct-decl} \lit{;}}
\proddef{\nt{struct-def} \lit{;}}

\grammarend

As with functions, \langname{} makes a distinction for types between declaring
and defining.  A type must be defined in order to use it with a local variable
declaration or structure field. The primitive intrinsic types \langtext{void},
\langtext{bool}, \langtext{int}, \langtext{char}, \langtext{string} are always
defined. Any type that is defined is also considered to be declared. Pointer
and array types are considered defined if their base type is declared.

A structure may not be defined twice but it may be declared more than once.

A structure definition may reference itself.

A type must be defined before it can be used with the \langtext{alloc} or
\langtext{alloc\_array} expressions.  In keeping with the limitations of C, its
definition must be parsed by the time the compiler parses the allocation
expression.

There are no equality or assignment operators provided for structures. The
motivation for their omission is to force students to think about how their
mutable data structures ought to be copied and compared. Due to the lack of
intrinsic support for these operations, structure types may not be used for
parameters, or function return types. They are also prohibited from being used
with local variables due to the lack of an equivalent to C's address-of operator
coupled with aforementioned restrictions. As such, all structures are allocated
on the heap and pointers to structure types are how structures are intended to
be used. Though this is not promoting good practice when writing programs in C,
we deemed the additional complications from allowing safe access to structures
on the stack to be unacceptable.

\section{Typedef}

Since the name of a type may not always be the best description of its purpose,
\langname{} provides the ability to alias types by providing new names.

\grammarbegin
\declprod{ty}
\proddef{\nt{type-name}}

\declprod{type-def}
\proddef{\lit{typedef} \nt{ty} \nt{type-name} \lit{;}}

\declprod{gdecl}
\proddef{\nt{type-def}}
\grammarend

The type must be declared before it can be used with the \langtext{typedef}
construct. Any program text after the \langtext{typedef} may use the alias in
place of the full type.

\section{Reasoning}

Due to its simplicity and safety properties, \langname{} allows the programmer
to reason about their programs via annotations. These annotations are not part
of the core language; they are specified via a special syntax that is embedded
in the comments of the program much like JML \cite{Leavens-Cheon05}, D
\cite{DLang} and Eiffel \cite{ecma2006ead}. Annotations allow the programmer to
declare pre- and post- conditions for functions, loop invariants, and general
assertions.  Library functions declarations can also be annotated.

These assertions about programs are checked at compile time when possible and
at runtime when not. If they fail at runtime, an exception is raised and the
program terminates.

Annotations are designed to encourage students to reason about their code
however they can only check what the programmer has intended to check and may
even contain bugs. The following example is an implementation of binary search
which contains at least one bug that is not caught by the assertions:
\lstinputlisting[language=C,
                 morekeywords={bool,string},
                 firstline=40,
                 lastline=67,
                 basicstyle=\small\tt]{binsearch.c0}

\section{Comparison with C}

\langname{} is almost but not quite a subset of C. Though we strived to stay as
close to C as possible, the array semantics proved impossible to resolve.
However, it is still quite easy to convert a \langname{} program to C such that
the meaning of the \langname{} program is one of the possible meanings of its C
version. The conversion process requires 2 steps:

\begin{enumerate}

\item Convert array declarations by replacing all occurrences of \langtext{[]}
with \langtext{*}.

\item Insert \langtext{\#include "c0defs.h"} at the top of each

\langname{} file.  \langtext{c0defs.h} defines the macros for the
\langtext{alloc} and \langtext{alloc\_array} expressions using
\langtext{calloc} to allocate the memory and \langtext{sizeof} to determine the
bytes. It would also need to define the \langtext{bool} type as well as
\langtext{true} and \langtext{false} since those are not defined in C. Any
library headers used would also need to be converted and included. A sample
\langtext{c0defs.h} is included in the appendix.

\end{enumerate}

\subsection{Differences}

Differences between \langname{} and C fall into two categories: omissions from
C and changes from C. There are many examples of the former:

\begin{itemize}
\item No unions
\item No casting
\item No pointer arithmetic
\item No sizeof operator
\item No address-of operator (\&)
\item No storage modifiers (static, volatile, register, extern)
\item No labels, goto, switch statements or do...while loops
\item No assignments in expressions
\item No floating point datatypes or literals
\item No complex types
\item No unsigned or other integral types besides int
\item Structure types may not be declared as local variables or used as return types for functions
\item No comma separated expressions
\item No explicit memory deallocation
\item Allocation is done using types not a size in bytes.
\item No fixed size arrays
\item No stack allocated arrays
\end{itemize}

As previously noted, the array semantics are different in \langname{} than C,
matching the pointer index semantics instead.

\subsection{Union Support}

It is unfortunate that \langname{} does not contain any good constructs for
disjoint types. Though C has the \langtext{union} construct which supports them in an
unsafe way, getting a type-safe version of them while maintaining the design
goals of the language is rather difficult. The straightforward approach would be
to layout the union in memory as in C and keep a hidden tag that is updated
whenever a field in the union is written and checked when read. The major
drawback to this approach is that there is no way to inspect the tag without
attempting to read a value. One convention in C is to nest unions in a struct
and use a separate struct field to indicate the tag. Either this separate field
must be kept in sync with the hidden tag or else new syntax/semantics must be
invented to tie the tag field to the union. The former is an undue burden on the
programmer and the latter would deviate even more from C.

The issue is deeper however. In C and other languages, unions keep the tag and
data separate which leads to issues such as the classic variant record update
problem in Pascal, later encountered in Cyclone, another strongly typed
imperative C-like language, \cite{GrossmanHJM04}. The problem is that if you
allow aliases to the data stored in a union, you can break type safety because
users of those aliases won't know to typecheck their writes and reads.  In the
following example, neither function is obviously incorrect until their
definitions are used together.

\langsnippet{unionsample.c}

The real fix is to always tie the tag and the data together. In the ML family of
languages, this is accomplished with constructors which take the appropriately
typed data and add the right tag. Examining the data stored in a union requires
a case analysis of the tag which allows access to the data. Even with the
addition of constructors and case analysis, aliasing is still an issue in C's
memory model if access to the data is given by reference. Consider the following
example:

\langsnippet{cctor.c}

A solution here would to copy the value and provide that to the case body
instead of a reference. Cyclone supports this solution as well as forbidding
assignment \cite{CycloneUnions}.

\subsection{Lack of \& for local variables} \label{noaddressof}

In C, it is a common idiom to return multiple values from a function by
requiring the caller pass a pointer to a location where the callee can write
them. Since the callee typically never stores the pointer anywhere, the caller
can expect to safely pass the address of a stack variable instead of allocating
a variable on the heap. For \langname{}, we'd like to permit this in a typesafe
manner. We could do so by introducing an unescaping pointer type which cannot
be stored into the heap. Having two types of pointers is rather strange so it
would be nice to allow regular pointers to be treated as unescaping pointer but
this necessitates the introduction of subtyping which so far has been avoided.
It would also entail that most pointers in function declarations would be
rewritten to use the unescaping pointer type instead of the regular pointer type
since there is no conversion from a potentially escaping pointer to one that
does not. This puts more distance between \langname{} and C in a fairly
intrusive way.

\section{Analysis}

\langname{} strikes a balance between being too limited and too complex. Though
clearly unsuitable for writing large programs, it provides sufficient syntax and
semantics for expressing the data structures and algorithms used in the course.
In comparison to C, it removes a number of dangerous features to gain type
safety while maintaining the core syntax and semantics of the language. Though
not a subset, it is close enough so that

\begin{enumerate}

\item A single lecture would be enough to instruct students in proper usage of
C for writing the equivalent programs.

\item The meaning of a well-formed (terminates without exceptions) \langname{}
program is one of the possible meanings of its transliterated C version.

\end{enumerate}

\chapter{Properties}
The prior section specified the behavior of \langname{} via examples and prose
description. To make the specifications and expected behavior clear, we must
define not only the concrete syntax, but also an abstract one whose semantics
are unambiguously defined. The abstract syntax follows the concrete syntax
fairly closely but omits some features from the concrete language such as
strings, characters and annotations because they are not interesting or
essential to the formal model.

\section{Concrete syntax}

The grammar specification below does not give definitions for the terminals
\nt{identifier}, \nt{integer-constant}, \nt{character-constant} and
\nt{string-constant}. These terminals are defined as follows:

\begin{description}
\item[\nt{identifier}]
  An identifier is a sequence of letters, digits, and underscores. The first
  character must be a letter. Identifiers are case sensitive.
\item[\nt{integer-constant}]
  Integer literals come in two forms. A decimal literal is a sequence of digits
  (\langtext{0} through \langtext{9}) where the first digit is not 0.
  Hexadecimal literals begin with the two character sequence 0x and are followed
  by one or more hexadecimal digits (\langtext{0} through \langtext{9},
  \langtext{a} through \langtext{f}). Hexadecimal literals are case insensitive.
\item[\nt{character-constant}]
  A character constant is a single ASCII character or one of the permitted
  escape sequences enclosed in single quotes (\langtext{'x'}). The character
  \langtext{'} must be escaped. The following escape sequences are
  permitted:

  \begin{tabular}{ll|ll}
  \langtextish{\textbackslash{}t} & tab &
  \langtextish{\textbackslash{}b} & backspace \\
  \langtextish{\textbackslash{}r} & return &
  \langtextish{\textbackslash{}n} & newline \\
  \langtextish{\textbackslash{}'} & single quote &
  \langtextish{\textbackslash{}"} & double quote \\
  \langtextish{\textbackslash{}0} & null & \ & \ \\
  \end{tabular}
\item[\nt{string-constant}]
  A string constant (or literal) is a sequence of characters and escape
  sequences enclosed in double quotes (\langtext{"Hello"}). The set of
  permitted escape sequences for strings is the same as for characters.
\end{description}

\grammarbegin
\grmboolconst

\declprod{pointer-constant}
\proddef{\lit{NULL}}

\declprod{constant}
\proddef{\nt{bool-constant}}
\proddef{\nt{integer-constant}}
\proddef{\nt{character-constant}}
\proddef{\nt{string-constant}}
\proddef{\nt{pointer-constant}}

\declprod{variable-name}
\proddef{\nt{identifier}}

\declprod{function-name}
\proddef{\nt{identifier}}

\declprod{type-name}
\proddef{\nt{identifier}}

\declprod{field-name}
\proddef{\nt{identifier}}

\declprod{monop}
\proddef{any of \lit{* ! - \~{\ }}}

\declprod{binop}
\proddef{any of \lit{+ - * / \% \& | \^{\ } \&\& |\hspace{-1em}| => << >> < > <= >= == !=}}

\declprod{assign-op}
\proddef{any of \lit{= += -= *= /= \%= \&= |= \^{}= <<= >>=}}

\declprod{post-op}
\proddef{any of \lit{++ --}}

\grmexprcommon
\grmexprarray
\proddef{\nt{function-name}}
\proddef{\nt{call-expression}}
\proddef{\nt{expression} \lit{->} \nt{field-name}}
\proddef{\nt{expression} \lit{.} \nt{field-name}}
\proddef{\nt{expression} \lit{?} \nt{expression} \lit{:} \nt{expression}}
\proddef{\lit{alloc} \lit{(} \nt{ty} \lit{)}}
\proddef{\lit{alloc\_array} \lit{(} \nt{ty} \lit{,} \nt{expression} \lit{)}}
\proddef{\lit{(} \nt{expression} \lit{)}}

\grmcallexpr

\declprod{argument-list}
\proddef{\nt{expression}}
\proddef{\nt{argument-list} \lit{,} \nt{expression}}

\grmaebasic
\grmaeptr
\grmaearr
\grmaestr

\grmlhsbasic
\grmlhsptr

\grmdeclstatement
\grmdeclexpstatement

\grmassignstatement
\grmcallstm
\proddef{\nt{atomic-expression} \nt{post-op}}

\grminitializerstatement{}

\declprod{jump-statement}
\proddef{\lit{return} \maybe{\nt{expression}} \lit{;}}
\grmbreakstatement
\grmcontinuestatement

\grmselectionstatement

\declprod{iteration-statement}
\grmwhilestatement
\grmforstatement

\grmcompoundstatement

\declprod{statement}
\proddef{\nt{initializer} \lit{;}}
\proddef{\nt{jump-statement}}
\proddef{\nt{selection-statement}}
\proddef{\nt{iteration-statement}}
\proddef{\nt{compound-statement}}

\declprod{struct-decl}
\proddef{\lit{struct} \nt{struct-name}}

\declprod{ty}
\proddef{\nt{type-name}}
\proddef{\lit{void}}
\proddef{\lit{bool}}
\proddef{\lit{int}}
\proddef{\nt{ty}\lit{*}}
\proddef{\nt{ty}\lit{[]}}
\proddef{\nt{struct-decl}}

\declprod{struct-def}
\proddef{\nt{struct-decl} \lit{\{} \maybe{\nt{struct-fields}} \lit{\}}}

\declprod{struct-fields}
\proddef{\nt{field-decl}}
\proddef{\nt{struct-fields} \nt{field-decl}}

\declprod{field-decl}
\proddef{\nt{ty} \nt{field-name} \lit{;}}

\grmfundecl

\grmfundef

\declprod{function-param}
\proddef{\nt{ty} \nt{variable-name}}

\declprod{function-params}
\proddef{\nt{function-param}}
\proddef{\nt{function-params} \lit{,} \nt{function-param}}

\declprod{type-def}
\proddef{\lit{typedef} \nt{ty} \nt{type-name} \lit{;}}

\declprod{gdecl}
\proddef{\nt{struct-decl} \lit{;}}
\proddef{\nt{struct-def} \lit{;}}
\proddef{\nt{type-def}}
\proddef{\nt{function-def}}
\proddef{\nt{function-decl} \lit{;}}

\declprod{program}
\proddef{\nt{gdecl}}
\proddef{\nt{program} \nt{gdecl}}

\grammarend

The binary and unary operators have the following precedence and associativity in order from highest to least:\\
\begin{tabular}{l|l}
Operators & Associativity \\
\hline
\langtext{()} \langtext{[]} \langtext{->} \langtext{.} & Left \\
\langtext{!} \langtext{\~{\ }} \langtext{-} \langtext{*} & Right \\
\langtext{*} \langtext{/} \langtext{\%} & Left \\
\langtext{+} \langtext{-} & Left \\
\langtext{<<} \langtext{>>} & Left \\
\langtext{<} \langtext{<=} \langtext{>} \langtext{>=} & Left \\
\langtext{==} \langtext{!=} & Left \\
\langtext{\&} & Left \\
\langtextish{\^{}} & Left \\
\langtextish{|} & Left \\
\langtext{&&} & Left \\
\langtext{||} & Left \\
\langtext{=>} & Left \\
\langtext{?} \langtext{:} & Right \\
\langtext{=} \langtext{op=} & Right \\
\end{tabular}

There is one ambiguity in the grammar which is resolved as follows:
\begin{description}
\item[Matching \langtextish{else} branches in nested \langtextish{if} statements] \ \\
As in C, in the case of \langsnippet{ifelseambiguity.c} the \langtext{else}
branch is bound to the innermost \langtext{if} statement.
\end{description}

\subsection{Abstract syntax}

We start with the abstract syntax's types $\tau$.  There are two basic types:
\bool{} and \integer.  Given a type $\tau$ we can construct pointer and array
types $\pointer{\tau}$ and $\arraytype{\tau}$.  Function types are denoted as
\function{\tau'}{\tau} Structure types are represented as \struct{p} where $p$
describes the fields.  $p$ is an ordered, possibly empty, mapping of field
names to types. The definition of $p$ is written as follows:

\classp

Two other types are used internally in the abstract syntax - they cannot be
writte in the concrete language. The first is the command type \commandty{}
which represents the type of statements such as loops or declarations. The other
is a tuple type \tuple{\tau_1, \ldots, \tau_n} which is composed of 0 or more
types. These are used in the elaboration of the concrete \langtext{void} type
and as the type for function arguments. The full definition of $\tau$ is written
as follows:

\classtau

The heap is modeled as a mapping of addresses $a$ to values or functions. The
addressing model in the abstract syntax is more expressive than that of the
concrete syntax. In particular it allows addresses into arrays and structure
fields. A basic address is just a label that identifies some value or function
in the heap. These are obtained from evaluating \langtext{alloc}. An address may
also be formed by combining a label and an integer offset. These are used in
intermediate expressions involving arrays such as \langtext{a[i] += 9;} where a
temporary pointer needs to be created to point to the \langtext{i}th index of
\langtext{a}. Similarly, structure fields can be addressed by combining an
address with a field name. Structure fields addresses need to use an arbitrary
address instead of a label because structures can nest inside each other and can
be used as the element type of an array. Thus the definition of addresses is
writen as follows:

\classa

Values are rather straightforward. \true{} and \false{} are inhabitants of the
\bool{} type. \intliteral{n} represents an integer $n$. \address{a} is a pointer
with address $a$ with type $\pointer{\tau}$ where the value at address $a$ has
type $\tau$. Similarly, \arrayval{a}{n} is a reference to an array with base
address $a$ of length $n$. \structval{a} is a reference to a structure with base
address $a$. \func{x_1, \ldots, x_n, e} represents a function with bound
variables $x_1, \ldots x_n$ and function body $e$. Type \commandty{} has one
value \nop{} which represents a command that does nothing. There are no values
of type \tuple{\ldots} however the empty tuple \tuple{} is treated as a value
for the purposes of returning from functions.

\classv

Unlike most other languages, expressions are permitted to have both local and
non-local side effects. They encompass both statements and expressions in the
concrete language. The simplest expressions are variables $x$ and values $v$.
Binary and unary operations are represented with \binop{op}{e_1}{e_2} and
\monop{op}{e}. Tuple expressions look just like their concrete counterpart.
Function calls are represented as \call{e_f, e} where $e_f$ is an expression for
a pointer to the function to be called and $e$ is the expression for the
argument tuple. \langtext{if} statements and the conditional operator are
represented by the \ifexp{e_c}{e_t}{e_f} structure. Variable declarations
\decl{x}{\tau}{e} restrict the declared variable's scope to $e$. \assign{x}{e}
represents assignments to local variables. \return{e} represents the
\langtext{return} statement from the concrete language. \loopstm{e_c}{e}
represents a basic \langtext{while} loop with condition $e_c$ and body $e$.
\langtext{break} and \langtext{continue} are straightforward. \allocexp{\tau} is
an expression whose evaluation allocates a value of type $\tau$ on the heap.
Similarly, \allocarrayexp{\tau}{e} is an expression whose evaluation allocates
an array of type \arraytype{\tau} with length given by the value of $e$.

Noticeably missing from these expressions are memory reads/writes and sequencing
of expressions. These operations are represented as binary and unary operations.
The \opseq{} binary operator sequences two statements. The \opwrite{} binary
operator takes an address and a value and writes the value to the address in
memory (or generates an exception). Similarly, the \opread{} unary operator
takes an address and produces the value at that address (or an exception). The
unary \opfield{f} and binary \oparrayindex{} provide a means for "pointer
arithmetic" so that addresses to fields of structures and elements of arrays
can be produced. Finally, the \opign{} unary operator allows for the evaluation
of an expression for its side effects to be treated as a command.

\classe

\classop

To statically check if \breakstm{} and \continue{} are used in loops, this
simple structure is used:

\classL

When evaluating a function, it is necessary to store some information such as
pending expressions and the values of local variables. This information is
stored in function frames $F$. A function frame is an ordered list of the
following: variables that are declared but not assigned (\framevar{F}{x}{\tau}),
the values of assigned variables (\frameval{F}{x}{\tau_x}{v}), expressions that are yet
to be evaluated pending the evaluation of their subexpressions
(\frameexp{F}{e}), and a construct to indicate a pending loop
(\frameloop{F}{e_c}{e}). The latter is used for the execution of the \breakstm{}
and \continue{} expressions.

\classF

A stack $K$ is an ordered list of function frames:

\classK

The top of the stack $t$ is represented explicitly. It is either an expression
or an exception.

\classt

There is also a direction associated with the top to indicate if it is needs
potentially further evaluation ($\rhd$) or if it is ready to be used in
evaluating the rest of the call stack ($\lhd$).

\classbowtie

Evaluation involves two primary components:

\begin{itemize}

\item An abstract heap $\mu$ with signature $\Sigma$ containing values $v$ at
addresses $a$.

\item An abstract stack $K$ with top $t$ and direction $\bowtie$ yielding a value of type $\tau$.

\end{itemize}

The initial heap contains entries for the functions and globals defined by the
program and the initial stack is empty and the initial top is a call to the
"main" function of the program with no arguments.

Evaluation terminates when either a value or exception is on top and the rest of
the stack is empty.

\section{Elaboration to Abstract Syntax}

In order to formalize the language definition, we must provide an elaboration from the concrete syntax to the abstract syntax. Though mostly straightforward, there are some non-trivial portions of the elaboration:
\begin{description}
\item[Declarations]
In the concrete syntax, declarations extend to the end of the block. In the abstract syntax, there is no notion of "block." Instead declaration statements explicitly contain the scope of the declaration. When elaborating a declaration in a block, the rest of the block is first elaborated into some abstract statement $s$ and then the declaration can then wrap $s$.
\item[Short-circuit boolean operators]
The abstract syntax has no notion of the short-circuiting logical boolean operators. Both of them can be desugared into a use of the ternary operator as follows:
\begin{tabular}{lll}
\langtext{a \&\& b} & $\Rightarrow$ & \langtext{a ? b : false} \\
\langtext{a \|\| b} & $\Rightarrow$ & \langtext{a ? true : b} \\
\end{tabular}

The ternary operator \langtext{a ? b : c} elaborates to \ifexp{a}{b}{c}.
\item[Empty statements]
These elaborated to \monop{\opign}{\false}
\item[For-loops]
The \langtext{for} loop can be transformed into a \langtext{while} loop to elaborate it.
The post-body statement is inserted into the loop body at the end and at every location with a non-nested \langtext{continue}. Call this \langtext{body'}.
The loop \langtext{for (init; cond; post) body}

is rewritten as \langtext{ init; while(cond) body' }.

\item[Array indices]
\langtext{a[i]} is elaborated as \monop{\opread}{\binop{\oparrayindex}{a}{i}} where $a$ and $i$ are the elaborations of \langtext{a} and \langtext{i} respectively.

\item[Structure field projection]
Field projection comes in two forms. The first \langtext{e->f} elaborates to \monop{\opread}{\monop{\opfield{f}}{e}}. The second \langtext{e.f} is trickier to handle since the type of $e$ in a well-typed program should be a structure type, not a pointer type. By induction on the elaboration of \langtext{e} we know that the elaboration of \langtext{e} can be canonicalized the form \monop{\opread}{e'}. Intuitively this makes sense since all structures are allocated on the heap. Thus, \langtext{e.f} elaborates to \monop{\opread}{\monop{\opfield{f}}{e'}}.

\item[Compound assignment]
For compound assignment, the elaboration depends on the left hand side.
\begin{description}
\item[\langtextish{x op= e}] This elaborates to \assign{x}{\binop{op}{x}{e}}.
\item[\langtextish{*p op= e}] The left hand side elaborates to \monop{\opread}{p} where $p$ is the elaboration of \langtext{p}. Thus the elaboration becomes \\\decl{p'}{\tau*}{\binop{\opseq}{\assign{p'}{p}}{\monop{\opign}{\binop{\opwrite}{p'}{s}}}} where $p'$ is a fresh variable and $s = \binop{op}{\monop{\opread}{p'}}{e}$.
\item[Arrays and structures]
Since array index expressions and structure field projections elaborate to the same form as pointers, the procedure is the same as with pointers.
\end{description}

\item[Functions with \langtextish{void} return type]
The \langtext{void} type is elaborated to \abslang{\tuple{}} and
empty return statements are simply elaborated to \abslang{\return{\tupleexp{}}}.
Since these functions are not required to have a return statement to terminate
them, one is implicitly added to the end of the function body.

\item[Structure type elaboration]
Because the abstract syntax does not have nominal typing, structural types in
the concrete syntax are elaborated by using the name of the structure to rename
the fields of the structure. References in code to these fields must also be
renamed. Structure types without any fields must remain distinct in the
elaboration so a dummy field must be added to them prior to elaboration
otherwise \langtext{struct F \{\}} and \langtext{struct G \{\}} would both
elaborate to \struct{\cdot}. Here are some examples:

\begin{tabular}{ll}
Concrete & Abstract \\
\langtext{struct F \{\};}
  & \struct{\varctx{\cdot}{F\$\$dummy}{\bool}} \\
\langtext{struct G \{ bool dummy; \}}
  & \struct{\varctx{\cdot}{G\$dummy}{\bool}} \\
\langtext{struct Point \{ int x; int y; \}}
  & \struct{\varctx{\varctx{\cdot}{Point\$x}{\integer}}{Point\$y}{\integer}} \\
\langtext{struct Size \{ int x; int y; \}}
  & \struct{\varctx{\varctx{\cdot}{Size\$x}{\integer}}{Size\$y}{\integer}} \\
%\langtext{struct Rect \{ struct Point p; struct Size s; \}} &
\end{tabular}

Since \$ is not permitted in field names or structure names in \langname{}, it
is safe to use as a delimeter.

\item[Order of declarations for functions/types and typedefs]
The abstract syntax does not take the order of declarations into account. The
elaborator can ignore the declarations and elaborate the definitions. Typedefs
are resolved to their underlying structural type during elaboration.

\end{description}

\newcommand{\abstractstep}{
$\prgstate{\mu}{K}{\bowtie}{t} \rightarrow \prgstate{\mu'}{K'}{\bowtie'}{t'}$
}
\newcommand{\progresstheorem}{ If $\mu : \Sigma$ and
\rulecheckstatedefaultconclusion{} then either
\rulefinalstatedefaultconclusion{} or \abstractstep{} for some
$\mu'$, $K'$, $\bowtie'$, and $t'$. }

\newcommand{\preservationtheorem}{
If $\mu : \Sigma$ and \rulecheckstate{\Sigma}{K}{\tau}{\bowtie}{t} and \abstractstep{} then $\exists \Sigma'. \mu' : \Sigma'$ and \rulecheckstate{\Sigma'}{K'}{\tau}{\bowtie}{t'} and $\Sigma \le \Sigma'$.

}

\section{Proof of safety}

The proof of safety for this abstract language is given by two theorems called progress and preservation. They are defined in \ref{progressproof} and \ref{preservationproof}.

\begin{description}
\item[Progress] \ \\
\progresstheorem{}

\item[Preservation] \ \\
\preservationtheorem{}

\end{description}

\chapter{Implementation}
For 15-122, Principles of Imperative Computation, our implementation consists of
a compiler and three interchangable runtimes with different characteristics. Due
to the simplicity of the language and the design of the compiler, there is very
little that the runtime needs to do.

\section{Compiler}

The compiler, \langtext{cc0}, performs typechecking and optional elaboration of
dynamic specifications. It produces a C program which is then compiled with a
runtime using \langtext{gcc}. Since the C compiler need not follow the same
semantics required by \langname{}, \langtext{cc0} converts the program into A-normal
form and uses externally defined functions for certain operations such as
division and bit shifts. The optimization level for \langtext{gcc} is set very
low because some of its optimizations assume C's semantics (ex: integer overflow
is undefined in C) which can lead to incorrect program execution.

\langtext{cc0} uses the following representations for \langname{}'s types
\begin{tabular}{l|l}
\langname\ Type & C/C++ Type \\
\hline
\langtext{bool} & \langtext{bool} \\
\langtext{char} & \langtext{char} \\
\langtext{int} & \langtext{int32\_t} \\
\langtext{T*} & \langtext{T*} \\
\langtext{T[]} & \langtext{c0\_array*} \\
\langtext{struct S} & \langtext{struct S} \\
\langtext{string} & \langtext{c0\_string} \\
\end{tabular}

To ease interoperability with native code, structures are laid out in memory
using the same rules as C for member alignment and structure size. Each field
of a struct is aligned according to its C/C++ type. Array elements are also
aligned as in C.

Functions use the standard {\tt cdecl} calling convention for easy interaction
with C. To avoid collisions between C functions and \langname{} functions,
\langname{} function names and references are mangled to produce a unique but
fairly human-readable name. The compiler simply takes the function/global
variable name as written in \langname{} and prepends \langtext{\_c0\_} to them.
References to functions declared in libraries are not mangled.

\section{Runtimes}

We've developed three runtimes for \langname{}: \langtext{bare},
\langtext{c0rt}, and \langtext{unsafe}. Although our runtimes have only been
tested for x86 and x86-64, the language was designed to be purposely abstract
enough to be easily ported to other popular architectures.

The exact representation of arrays is provided by the runtime that the program
is linked against. Access to members and the array's length is provided via
runtime functions. Runtimes provide the interface declared in Appendix
\ref{c0runtimeinterface} for use by the generated code and any libraries:

Each runtime was built to suit a particular use case.

\begin{itemize}
\item
The \langtext{bare} runtime is an extremely basic runtime which provides bounds
checking but no garbage collection. Its primary purpose was to allow
\langname{} programs to be compiled and run on systems which aren't supported by
the garbage collector library used by the other runtimes. Since many of the
assignments and test programs allocate only a little memory, the lack of garbage
collection is not very noticeable. Arrays are represented as contiguous
allocation as in C but with a header that includes the size of the array for
bounds checking.

\item
The \langtext{c0rt} runtime performs bounds checking but also garbage collects
allocations using the Boehm-Demers-Weiser conservative collector \cite{BoehmGC}.
It uses the same representation for arrays as \langtext{bare}.

\item
The \langtext{unsafe} runtime garbage collects allocations using the same
conservative collector but performs no bounds checking. Arrays are represented
exactly as in C as a raw pointer to a contiguous block of cells. Since
\langtext{cc0} compiles to C, the header for the runtime can redefine the
macros used for array types and element access so that the efficient direct
access to array elements is used. If \langtext{gcc}'s optimizer could be
trusted, this would enable \langname{} programs compiled against the
\langtext{unsafe} runtime to better match C for execution speed.

\end{itemize}

Since strings are represented opaquely in the language, the runtime is free to
choose any representation for strings it wants. For efficient interoperability
with native code, all three runtimes represent strings as in C as an array of
ASCII characters with a null terminator however other representations (ex: ropes
\cite{Boehm95ropes}) are possible.

\chapter{Standard Libraries}
One of the best ways to get students to learn is to get them excited about
projects and motivated to complete them. These interesting projects often
require supporting code such as graphics or file system access which is provided
as a library to students. The goal of the standard libraries is to provide just
enough support to students so that they can do these interesting projects. One
nice side effect is that students are briefly exposed to different fields of
computer science through these libraries and this can inform their choices of
electives later on in the program.  \footnote{The standard library is also a
good place to show students how to design a good API. The C standard library has
some poor choices in modularity and an unfortunate distaste for function names
longer than 8 characters.  Unfortunately, \langname{} does not provide any
additional features to aid modularity so the best we can do is adopt a prefix
for the different libraries}

\section{Design Principles}

The standard libraries were developed at the same time as the projects for the
course. This allowed us to tailor each library exactly to the requirements of
the assignments. The standard libraries so far just cover just a few basic
areas: images, string manipulation, and basic file/console IO. Though these are
the basics for the projects in the course, they are also sufficient for
students to explore projects beyond the scope of the course. By the end of the
course, they will know enough C (primarily from using \langname{}) to write
their own libraries.

Though allocations within the language are garbage collected, the native
libraries upon which the standard libraries are built do not use garbage
collection. As in popular modern garbage collected languages like C\#, manual
resource management is used for external resources.

\section{Design}

Carnegie Mellon provides several clusters of computers that run Linux for
students to use so our primary platform for the initial offering of the course
is these RedHat Linux-based machines. Students may have little or no experience
with Linux though and will probably prefer, at least initially, to develop on
their own machines. Since the compiler and runtimes work on at least two major
desktop platforms (OS X, Linux) and can support the third (Windows) with some
effort, this is feasible so long as the standard library also works on these
platforms.  In considering which libraries to base the standard library upon, we
tried to pick only libraries that would work with minimal hassle both now and to
support in the future.

There are several broad areas for which we would like to have libraries
support. Though some areas such as windowing are not currently needed, we must
consider which libraries to use for their implementation when considering
related areas such as graphics and images. The libraries' APIs were designed to
be as minimal as possible and we looked at libraries used in other introductory
computer science courses such as Princeton's \cite{PrincetonJavaLibs} and
Processing \cite{Processing}.

\subsection{Input/Output}

We expect that most output in the course will be for debugging purposes and not
a major focus of assignments. The console IO library (\langtext{conio}) contains
just 4 functions to handle the basic task of printing data to the screen and
the occasional user input. The following example demonstrates the basic usage:

\langsnippet{basicio.c}

The lack of a \langtext{printf}-like function adds to the verbosity but it is not a
trivial feature to add to this language. It would require the addition of C's
varargs typing to the concrete syntax and typechecker as well as the formal
semantics. The type of the function would have to be inferred from its format
string which require the compiler to understand the format string. This would
break the abstraction boundary between the compiler and libraries. An
alternative is to introduce \langtext{printf} as an intrinsic part of the
language where the complicated type checking would not need to be exposed to
programmers.

A simple file library was added to allow programs to read lines from a file as
{\tt string}s. File handles are one of the first manually managed resources
with which students interact.

\subsection{Data Manipulation}

The {\tt string} library contains basic routines for manipulating and inspecting
strings and characters. It allows strings to be joined, split, converted to and
from null-terminated character arrays, compared, generated from \langtext{int}s
and \langtext{bool}s and queried for their length.  It also allows
\langtext{char}s to be compared and converted to and from their ASCII
representation.

\subsection{Images}

The first assignment for 15-122 had students write code to manipulate
two-dimensional images. The API allows programs to load, save and manipulate
the image as a \langname{} array of integers representing the pixel data
packed as ARGB. Like file handles, images also require manual resource
management.

\subsection{Windowing}

The projects in 15-122 should require only the most basic windowing support.  It
defines a window as an abstraction of a keyboard, a mouse, and a screen - it
receives input via mouse and keyboard and displays an image. There are no child
widgets for buttons or editboxes - students can write those themselves if
desired but the text console ought to be sufficient for functionality required
the projects in the course. Furthermore, the text interface is much easier to
write automated tests for than a graphical one.

\subsection{Graphics}

For graphics support, we want to expose a minimal but powerful API. For
inspiration, we looked at Java's AWT graphics objects, HTML's {\tt <canvas>} 2D
context API, and Cairo's API. Students are able to draw and manipulate images
as well as draw shapes at arbitrary positions and scales and rotations.
Students are able to construct 2D geometric paths and perform basic queries on
them.  They can also construct transformations from primitives like
translation, rotation and scale. Paths and transforms are used with a graphics
context which is obtained from an image or a window. With a graphics context,
basic operations such as drawing/filling a path with a transformation and
clipping region allow great flexibility with a minimal API.

\section{Implementation}

Defining libraries is a rather simple process. A library named \langtext{foo}
must provide a header file \langtext{foo.h0} containing the \langname{} types
and function declarations for the library and a shared library with the same
name base name as the library (ex: \langtext{libfoo.so},
\langtext{libfoo.dylib}, \langtext{foo.dll}). The compiler is instructed to use
the library \langtext{foo} by passing \langtext{-lfoo}. It will process the
declarations and definitions in the library's header and link the executable
against the shared library.

\subsection{Native Dependencies}

Many of the basic libraries can be implemented using the standard C library
since it is available on all three major desktop platforms.

For windowing there are both many and few choices. Each platform has its own
native libraries for basic window operations but supporting separate branches
that work well on moving platforms such as OS X or diverse platforms such as
Linux is not a good long-term approach so we decided to look at cross-platform
abstractions. There are many such libraries but the popular ones are Mozilla's
XUL/Gecko library, GTK, wxWidgets, FLTK, Qt.

The size of XUL/Gecko both in terms of size and scope is simply too great for
our needs. Likewise, GTK requires quite a bit of overhead on non-Linux
platforms (particularly Windows) and does not integrate with Microsoft's
compiler and toolchain, something that students may want to do when learning C
on Windows. wxWidgets has a fairly full featured set of tools that go beyond
just windowing. It does not abstract the details of the underlying platform
that well so students may see different behavior than their grader and likewise
the graders may not be able to see problems that the students are having. FLTK
is a minimal windowing toolkit that does not attempt to look native. It
provides basic graphics support as well. Like wxWidgets, Qt provides quite a
bit of non-GUI support and though it doesn't have a native look and feel, it
comes quite close.

The choices of graphics library is closely related to that of the windowing
library. FLTK, wxWidgets and Qt provide their own graphics libraries. Out of
these three, only wxWidgets does not abstract out the details of the underlying
platform since wxWidgets relies upon the native platform for much of its
drawing. There are also dedicated graphics libraries such as Skia and Cairo
which could potentially work with whatever windowing library was chosen.

Though there are many individual libraries that would be suitable for the
implementation of the major components of the standard library, Qt has the
advantage of providing APIs for all the major requirements. Moreover, the
interaction between these separate components is already written and tested. Qt
is used by a number of major corporations and projects around the world such as
Adobe\footnote{http://qt.nokia.com/qt-in-use/story/app/adobe-photoshop-album},
KDE, Mathematica, Opera and VLC. It is thus expected that Qt will remain a
stable and supported platform upon which the standard library can be used by
students with a minimal number of bugs and support required by the course staff.

\chapter{Related and Future Work}
\section{Related work}
Projects like SAFECode \cite{SAFECode} and CCured \cite{Necula02ccured} are also
attempts to produce safe C-like imperative languages. Unlike \langname{}, they
are aimed more at verifying or porting existing C programs. Another attempt at
an imperative systems language is Google's Go Language \cite{GoLang} which
breaks away from C but it does contain additional features which are only useful
for writing larger programs than those in 15-122. Go also does not have
assertions or exception built into the language, preferring instead to return
error codes and propagate or handle them explicitly without assistance from the
compiler. The designers' goal is to force programmers to think about proper error
handling whereas the intended focus of 15-122 is to encourage students to reason
about the correctness of their code.

Processing.org \cite{Processing} is another take on providing an easy and simple
programming environment for novice programmers. It specializes in programming
graphical applications such as games or visualizations. Though it simplifies
away some of its Java roots by eliminating the need for a driver class, it is by
its own admission "just Java, but with a new graphics and utility API along with
some simplifications."

\section{Future work and speculation}
The work presented in this thesis is intended to be only the start of a larger
project. Due to the time constraints of needing the language and compiler to be
finished in time for the course in the Fall of 2010, several interesting
projects and ideas have been given little attention. These include a static
prover for assertions, a module system for the language to replace the one found
in C, and using a precise garbage collector in the runtime as well as other more
advanced garbage collection techniques.

\subsection{Static prover for assertions}
\langname{} is likely simple enough to do some strong analysis of programs to
prove invariants and assertions will hold at runtime. The lack of an address-of
and cast operators limits aliasing and the local variables can be hoisted to the
heap by rewriting the program without changing any discernible semantics. As a
project for a graduate program analysis class, Karl Naden, Anvesh Komuravelli
and Sven Stork built a small static verification tool for programs.  They
annotated programs with assertions and invariants using JML-style notation.
The work at the end of the course showed promise towards being to be able to
prove some of the invariants relevant to the data structures used in the course.

\subsection{Module system \& Polymorphism}
C's module system has well known faults and both C and \langname{} lack a decent
means of supporting polymorphism. I did some preliminary work towards defining a
module system that was simple enough to be used by the course's students but
powerful enough to allow for abstract polymorphism. The intention was to allow
students to write generic data structures. Combined with the aforementioned
static verification tool, students would be able to write assertions and
independently check that their generic data structure or algorithm was correct.

\subsection{Precise collector}
Though the conservative collector in the runtime is adequate for most purposes
of the course, it is not ideal. Since there is no casting or internal aliasing,
it's possible to construct a more precise collector that can analyze the stack
layout and walk the heap efficiently. Using a precise collector would also
necessitate a change to the runtime interface to allow the garbage collector to
walk library-defined types and provide a rooting API.

\subsection{Dynamically check explicit deallocation}

Tools like Valgrind are often used on C and C++ programs to check for common
mistakes such as double frees and use-after-free bugs. \langname{} could be
augmented with the ability to explicitly free memory and at runtime (if not
statically) assert that the program does not use memory after freeing it or
free memory twice. This would bring the language closer to C without
compromising on safety.

\subsection{Convenient \langtext{printf} support}

One of the notable features absent from C in \langname{} is support for
variadic functions like \langtext{printf}. Supporting these functions while
maintaining type safety is rather non-trivial. Attempt to support it in
languages like SML have been tried by reformulating the format string as higher
order functions but that approach does not suit \langname{}.

\subsection{Formatting/style checker}

Good code hygiene is an often overlooked aspect of computer science education.
Though there is no single style for writing good C code, there are several
popular styles and techniques for all languages. Some upper level courses such
as Operating Systems include a code review as part of the grading mechanism.
Though quite time consuming, it provides generally helpful feedback to students
in ways that an automatic grader usually does not. Some languages such as Go
\cite{GoLang} include a formatting program which automatically
rewrites source files to a pre-defined style. The former is time consuming and
prone to divergent styles/feedback and the latter does not encourage/require
students to learn proper styling. A program which could consistently
verify/grade the style of student code and generate useful human readable
feedback, would be of enormous value to both the students and the course staff.

\subsection{Interactive debugger}

Asking students to run \langtext{gdb} on the compiled binaries is asking too
much of them. We can produce a debugger for \langname{} programs that can
provide a friendlier interface than GDB and enable additional features such as
reversible debugging. Some work on this was started before the course by Rob
Simmons.

\subsection{Feedback on the course and its use of \langnameb{}}

At the time of this publication, it is too early to know what impact choice of
\langname{} will have. Getting feedback from the students over the course of
their time at Carnegie Mellon will provide valuable data. Already from the first
few weeks, suggestions for incremental changes to the language and compiler have
been made from both students and staff.

%\newcommand{\boxed}[1]{\fbox{\ensuremath{#1}}}

\begin{appendices}
\chapter{Language Definition}
\section{Syntax}

\newcommand{\rulefmt}[1]{{\small\tt\sc #1}}
\begin{tabular}{lrl}
Struct fields \classpdata

Types \classtaudata

Address \classadata

Values \classvdata

Operators \classopdata

\end{tabular}

\begin{tabular}{lrl}
Expression \classedata

Loop states \classLdata

Call frames \classFdata

Stack frames \classKdata

Directions \classbowtiedata

Foci \classtdata

Variable contexts \startdef{\Gamma}{\emptyparams \OR \varctx{\Gamma}{x}{\tau}}

Memory signatures $\Sigma(a)$ & = & $\tau$ \\

Memory stores $\mu(a)$ & = & $v$ \\

\end{tabular}

First we begin with the judgment \ruletausmalldefaultconclusion{} which determines if a type is permitted to be used as the type of a declaration, parameter of a function, or return type of a function. It specifically excludes structures, functions, non-degenerate tuples and commands.

\begin{mathpar}
\rulecheckvaldefaultconclusion{}
\end{mathpar}

Given a value and a memory signature, this judgment determines a type for the value.  Booleans and integers can be typed without the memory signature but the other types require it so that they do not need to embed any types. The null pointer is allowed to be any pointer type.

\begin{mathpar}
\rulecheckexpdefaultconclusion{}
\end{mathpar}

This judgment determines the type of an expression from a context which describes the variables that are in scope and a signature of the memory. Checking that an expression only uses assigned variables is checked via a separate judgment.

\begin{mathpar}
\rulecanallocdefaultconclusion{}
\end{mathpar}

We must restrict the types which can be allocated to those which are directly expressible in the concrete language. This excludes tuples, commands and functions.

\begin{mathpar}
\rulecheckassigndefaultconclusion{}
\end{mathpar}

Since \langname{} requires variables to be assigned before being used, we use this judgment to determine the set of variables an expression requires to be assigned ($A$) and given that set, which set of variables will be defined after evaluating the expression ($A'$) in a context of declared variables $\Gamma$.

\begin{mathpar}
\rulecheckloopdefaultconclusion{}
\end{mathpar}

This judgment captures the need to restrict expressions such as break and continue to only occur within a loop.

\begin{mathpar}
\ruleonlyreturnsdefaultconclusion{}

\ruledoesreturndefaultconclusion{}

\rulereturnsdefaultconclusion{}
\end{mathpar}

Determining that a function matches its declared return type requires three judgments. The first checks that every possible execution path will, if it returns a value, return a value of the correct type. The second judgment ensures that every possible execution path which reaches the end of the function will return. The third judgment combines the first two and requires that the return type be a small type to be consistent with \langname{}.

\begin{mathpar}
\rulecheckbinopdefaultconclusion{}

\rulecheckmonopdefaultconclusion{}
\end{mathpar}

These two judgments determine the types of the binary and unary operators in the language.

\begin{mathpar}
\rulecheckframetypedefaultconclusion{}
\end{mathpar}

Given a memory signature and a set of assigned variables, we want to know if a (partial) call frame is well formed. This judgment captures the type that the top framelet expects and the type that the call frame will eventually returns.

\begin{mathpar}
\rulegetframecontextdefaultconclusion{}

\rulegetframeassigneddefaultconclusion{}

\rulegetloopcontextdefaultconclusion{}

\rulegetinnerloopdefaultconclusion{}
\end{mathpar}

These four judgments define recursive functions which extract information embedded in the stack. The first determines the typing context of the frame. The second determines which variables have been assigned. The third determines if the current call frame is in the middle of a loop or not. If the call frame is in a loop, the fourth function will return the partial stack frame which indicates the innermost loop.

\begin{mathpar}
\ruleispendingdefaultconclusion{}
\end{mathpar}

Expressions on the stack are always waiting for some value so that they can continue with their evaluation. This judgment captures the type of the expected value ($\tau$), the type that the expression will return once evaluation is complete ($\tau'$), and the return type which the expression will return, if it does return ($\tau''$).

\begin{mathpar}
\ruledirokdefaultconclusion{}
\end{mathpar}

This judgment simply checks if an expression is permitted to be returned to the topmost framelet of the top call frame. Only values and empty tuples are permitted to be returned.

\begin{mathpar}
\rulecheckstackdefaultconclusion{}
\end{mathpar}

We also must check the rest of the stack. Given a value of type $\tau$, evaluation of the call frames will yield a value of type $\tau'$.

\begin{mathpar}
\rulecheckstatedefaultconclusion{}
\end{mathpar}

Typing all these judgments together, we can check that the entire stack $K$ and the focus $t$ are well formed, yielding a final value of type $\tau$. The focus of a stack is an expression (often a subexpression of a function) that is being evalutated, a value that is being used to evaluate a pending expression on the topmost call frame, or an exception which is propagating up the stack.

\begin{mathpar}
\rulefinalstatedefaultconclusion{}
\end{mathpar}

Evaluation is finished when the stack is in a final state. There are no call frames and the focus $t$ is either an exception or a value.

\begin{mathpar}
\ruleevalbinopdefaultconclusion{}

\ruleevalmonopdefaultconclusion{}
\end{mathpar}

These two judgments define the behavior of the binary and unary operators respectively. Binary operators such as \opwrite{} may modify memory and unary operators such as \opread{} may depend on memory but do not modify it. Evaluating an operator is expected to result in a value of the appropriate type but may result in an exception (ex: trying to write a value to \anull{}).

\begin{mathpar}
\ruleallocvaldefaultconclusion{}

\rulemakearraydefaultconclusion{}
\end{mathpar}

These two judgments define functions which perform the allocation of values. Given a heap, an address not in the heap and a type (and a length for arrays), these functions return a new memory containing the appropriate initialized allocations.

\begin{mathpar}
\rulestepdefaultconclusion{}
\end{mathpar}

This judgment defines the transitions between well-typed program states. Evaluation consists of iterating these transitions until a final state is reached.

\begin{mathpar}
\mu : \Sigma

\Sigma \le \Sigma'

\end{mathpar}

The first of these judgments checks that a memory matches the given signature. The second defines a partial relation between signatures which is used to check that memory operations do not destroy or alter any existing allocations.

\section{Rules}

\boxed{\ruletausmalldefaultconclusion}
\begin{mathpar}
\ruletausmallboolnamed

\ruletausmallintnamed

\ruletausmallptrnamed

\ruletausmallarraynamed

\end{mathpar}

\boxed{\rulecheckvaldefaultconclusion}
\begin{mathpar}
\rulecheckvalnopnamed

\rulecheckvaltruenamed

\rulecheckvalfalsenamed

\rulecheckvalintnamed

\rulecheckvaladdressnamed

\rulecheckvalnullnamed

\rulecheckvalarraynamed

\rulecheckvalstructnamed

\rulecheckvalfuncnamed

\end{mathpar}

\boxed{\rulecheckexpdefaultconclusion}
\begin{mathpar}
\rulecheckexpvarnamed

\rulecheckexpvaluenamed

\rulecheckexpbinopnamed

\rulecheckexpmonopnamed

\rulecheckexptuplenamed

\rulecheckexpcallnamed

\rulecheckexpifnamed

\rulecheckexpdeclnamed

\rulecheckexpassignnamed

\rulecheckexpreturnnamed

\rulecheckexploopnamed

\rulecheckexpbreaknamed

\rulecheckexpcontinuenamed

\rulecheckexpallocnamed

\rulecheckexpallocarraynamed

\end{mathpar}

\boxed{\rulecanallocdefaultconclusion}
\begin{mathpar}
\rulecanallocboolnamed

\rulecanallocintnamed

\rulecanallocptrnamed

\rulecanallocarraynamed

\rulecanallocstructnamed

\end{mathpar}
\boxed{\rulecheckassigndefaultconclusion}
\begin{mathpar}
\rulecheckassignvarnamed

\rulecheckassignholenamed

\rulecheckassignvaluenamed

\rulecheckassignbinopnamed

\rulecheckassignmonopnamed

\rulecheckassigntuplenamed

\rulecheckassigncallnamed

\rulecheckassignifnamed

\rulecheckassigndeclnamed

\rulecheckassignassignnamed

\rulecheckassignreturnnamed

\end{mathpar}\begin{mathpar}
\rulecheckassignloopnamed

\rulecheckassignbreaknamed

\rulecheckassigncontinuenamed

\rulecheckassignnopnamed

\rulecheckassignallocnamed

\rulecheckassignallocarraynamed

\end{mathpar}

\boxed{\rulecheckloopdefaultconclusion}
\begin{mathpar}
\rulecheckloopvarnamed

\rulecheckloopvaluenamed

\rulecheckloopbinopnamed

\rulecheckloopmonopnamed

\rulechecklooptuplenamed

\rulecheckloopcallnamed

\rulecheckloopifnamed

\rulecheckloopdeclnamed

\rulecheckloopassignnamed

\rulecheckloopreturnnamed

\rulechecklooploopnamed

\rulecheckloopbreaknamed

\rulecheckloopcontinuenamed

\rulecheckloopnopnamed

\end{mathpar}\begin{mathpar}
\rulecheckloopallocnamed

\rulecheckloopallocarraynamed

\end{mathpar}

\boxed{\ruleonlyreturnsdefaultconclusion}
\begin{mathpar}

\ruleonlyreturnsbinopnamed

\ruleonlyreturnsmonopnamed

\ruleonlyreturnsifnamed

\ruleonlyreturnsdeclnamed

\ruleonlyreturnsassignnamed

\ruleonlyreturnsreturnnamed

\ruleonlyreturnsloopnamed

\ruleonlyreturnsbreaknamed

\ruleonlyreturnscontinuenamed

\ruleonlyreturnsnopnamed

\end{mathpar}
\boxed{\ruledoesreturndefaultconclusion}
\begin{mathpar}
\ruledoesreturnbinoplhsnamed

\ruledoesreturnbinoprhsnamed
\end{mathpar}\begin{mathpar}

\ruledoesreturnifnamed

\ruledoesreturndeclnamed

\ruledoesreturnreturnnamed

\end{mathpar}

\boxed{\rulereturnsdefaultconclusion}
\begin{mathpar}
\rulereturnsonlynamed

\end{mathpar}

\boxed{\rulecheckbinopdefaultconclusion}
\begin{mathpar}
\rulecheckbinopopaddintegernamed

\rulecheckbinopopsubintegernamed

\rulecheckbinopopmulintegernamed

\rulecheckbinopopdivintegernamed

\rulecheckbinopopmodintegernamed

\rulecheckbinopopbitandintegernamed

\rulecheckbinopopbitorintegernamed

\rulecheckbinopopbitxorintegernamed

\rulecheckbinopopshlintegernamed

\rulecheckbinopopshrintegernamed
\end{mathpar}\begin{mathpar}

\rulecheckbinopopcmpgintegernamed

\rulecheckbinopopcmplintegernamed

\rulecheckbinopopcmpgeintegernamed

\rulecheckbinopopcmpleintegernamed

\rulecheckbinopopcmpeqintegernamed

\rulecheckbinopopcmpneintegernamed

\rulecheckbinopopcmpeqboolnamed

\rulecheckbinopopcmpneboolnamed

\rulecheckbinopopcmpeqpointertaunamed

\rulecheckbinopopcmpnepointertaunamed

\rulecheckbinopseqnamed

\rulecheckbinopwritenamed

\rulecheckbinoparrayindexnamed

\end{mathpar}

\boxed{\rulecheckmonopdefaultconclusion}
\begin{mathpar}
\rulecheckmonopopnegintegernamed

\rulecheckmonopoplognotboolnamed

\rulecheckmonopopbitnotintegernamed

\rulecheckmonopignnamed

\rulecheckmonopreadnamed
\end{mathpar}\begin{mathpar}

\rulecheckmonopfieldnamed

\end{mathpar}
\boxed{\rulecheckframetypedefaultconclusion}
\begin{mathpar}
\rulecheckframetypeframeexpnoretnamed

\rulecheckframetypeframeexpretnamed

\rulecheckframetypeframevalnamed

\rulecheckframetypeframevarnamed

\rulecheckframetypeframeloopnamed

\end{mathpar}
\boxed{\rulegetframecontextdefaultconclusion}
\begin{mathpar}
\rulegetframecontextemptynamed

\rulegetframecontextexpnamed

\rulegetframecontextvarnamed

\rulegetframecontextvalnamed

\rulegetframecontextloopnamed

\end{mathpar}
\boxed{\rulegetframeassigneddefaultconclusion}
\begin{mathpar}
\rulegetframeassignedemptynamed

\rulegetframeassignedexpnamed

\rulegetframeassignedvarnamed

\rulegetframeassignedvalnamed

\rulegetframeassignedloopnamed

\end{mathpar}
\boxed{\rulegetloopcontextdefaultconclusion}
\begin{mathpar}
\rulegetloopcontextemptynamed

\rulegetloopcontextdeclnamed

\rulegetloopcontextdefnamed

\rulegetloopcontextexpnamed

\rulegetloopcontextloopnamed

\end{mathpar}
\boxed{\rulegetinnerloopdefaultconclusion}
\begin{mathpar}
\rulegetinnerloopdeclnamed

\rulegetinnerloopdefnamed

\rulegetinnerloopexpnamed

\end{mathpar}\begin{mathpar}
\rulegetinnerlooploopnamed

\end{mathpar}
\boxed{\ruleispendingdefaultconclusion}
\begin{mathpar}
\ruleispendingbinoplnamed

\ruleispendingbinoprnamed

\ruleispendingmonopnamed

\ruleispendingcallfnamed

\ruleispendingcallanamed

\ruleispendingtuplenamed

\ruleispendingifnamed

\ruleispendingassignnamed

\ruleispendingreturnnamed

\ruleispendingallocarraynamed

\end{mathpar}
\boxed{\ruledirokdefaultconclusion}
\begin{mathpar}
\ruledirokpushingnamed

\ruledirokreturningnamed

\rulediroktuplenamed

\end{mathpar}
\boxed{\rulecheckstackdefaultconclusion}
\begin{mathpar}
\rulecheckstackemptynamed

\rulecheckstacknonemptynamed

\end{mathpar}
\boxed{\rulecheckstatedefaultconclusion}
\begin{mathpar}
\rulecheckstateexnnamed

\rulecheckstateemptynamed

\rulecheckstatenormalnamed

\rulecheckstatereturnsnamed

\rulecheckstateloopbrknamed

\rulecheckstateloopcontnamed

\end{mathpar}
\boxed{\rulefinalstatedefaultconclusion}
\begin{mathpar}
\rulefinalstateexnnamed

\rulefinalstatevalnamed

\end{mathpar}

\boxed{\ruleallocvaldefaultconclusion}\\
Here we use the syntax $[\mu | a : v]$ to indicate a new function $\mu'(a') = $ if $a' = a$ then $v$ else $\mu(a')$
\begin{mathpar}
\ruleallocvalboolnamed

\ruleallocvalintnamed

\ruleallocvalptrnamed

\ruleallocvalarraynamed

\ruleallocvalemptystructnamed

\ruleallocvalstructnamed

\end{mathpar}
\boxed{\rulemakearraydefaultconclusion}
\begin{mathpar}
\rulemakearrayemptynamed

\rulemakearraynonemptynamed

\end{mathpar}

\boxed{\ruleevalbinopdefaultconclusion}
Most of the rules for this judgment are elided for brevity as they have clear semantics from the prose description in Chapter 2.
\begin{mathpar}
\inferrule*{ }{\ruleevalbinop{\mu}{\opdiv}{\intliteral{n}}{\intliteral{0}}{\mu}{\exn}}

\inferrule*{ }{\ruleevalbinop{\mu}{\opmod}{\intliteral{n}}{\intliteral{0}}{\mu}{\exn}}

\inferrule*{n > m}{\ruleevalbinop{\mu}{\opcmpg}{\intliteral{n}}{\intliteral{m}}{\mu}{\true}}

\inferrule*{n \le m}{\ruleevalbinop{\mu}{\opcmpg}{\intliteral{n}}{\intliteral{m}}{\mu}{\false}}

\inferrule*{ }{\ruleevalbinop{\mu}{\opseq}{\nop}{\nop}{\mu}{\nop}}

\inferrule*{ a \not\in \dom{\mu} }{\ruleevalbinop{\mu}{\opwrite}{\address{a}}{v}{\mu}{\exn}}

\inferrule*{ a \in \dom{\mu} }{\ruleevalbinop{\mu}{\opwrite}{\address{a}}{v}{[\mu{} |\ a:v]}{v}}

\inferrule*{ 0 < i < n }{\ruleevalbinop{\mu}{\oparrayindex}{\arrayval{a}{n}}{\intliteral{i}}{\mu}{\address{\arrayoffset{a}{i}}}}

\inferrule*{ i < 0 }{\ruleevalbinop{\mu}{\oparrayindex}{\arrayval{a}{n}}{\intliteral{i}}{\mu}{\exn}}

\inferrule*{ i \ge n }{\ruleevalbinop{\mu}{\oparrayindex}{\arrayval{a}{n}}{\intliteral{i}}{\mu}{\exn}}

\end{mathpar}
\boxed{\ruleevalmonopdefaultconclusion}
As with the previous judgment, many of the inference rules are omitted.
\begin{mathpar}
\inferrule*{ }{\ruleevalmonop{\mu}{\oplognot}{\false}{\true}}

\inferrule*{ }{\ruleevalmonop{\mu}{\oplognot}{\true}{\false}}

\inferrule*{ }{\ruleevalmonop{\mu}{\opign}{v}{\nop}}

\inferrule*{ a \not\in \dom{\mu}}{\ruleevalmonop{\mu}{\opread}{\address{a}}{\exn}}

\inferrule*{\mu(a) = v}{\ruleevalmonop{\mu}{\opread}{\address{a}}{v}}

\inferrule*{ }{\ruleevalmonop{\mu}{\opfield{f}}{\structval{a}}{\address{\fieldoffset{a}{f}}}}

\end{mathpar}
\boxed{\rulestepdefaultconclusion}
\begin{mathpar}
\rulestepvalnamed

\rulestepvarnamed

\rulesteppushbinopnamed

\rulestepswapbinopnamed

\rulesteppopbinopnamed

\rulesteppushmonopnamed

\rulesteppopmonopnamed

\rulesteppushcallfnnamed

\rulesteppushcallargsnamed

\rulestepfinalizecall{}\\
\text{\rulefmt{\rulestepfinalizecallname}}

\rulestepcallnullnamed

\end{mathpar}\begin{mathpar}

\rulesteppushemptytuplenamed

\rulesteppushtupleelemnamed

\rulestepnexttupleelemnamed

\rulesteplasttupleelemnamed

\rulestepallocnamed

\rulesteppushallocarraynamed

\rulesteppopallocarraynamed

\rulesteppopallocarrayerrnamed

\rulesteppushifnamed

\rulesteppopiftruenamed

\rulesteppopiffalsenamed

\rulesteppushdeclnamed

\rulesteppopdeclnamed

\rulesteppushassignnamed

\rulesteppopassignnamed

\rulesteppopassignfirstnamed

\rulesteppopassignednamed
\end{mathpar}\begin{mathpar}

\rulesteppushretnamed

\rulesteppopretnamed

\rulesteploopnamed

\rulesteplooppopnamed

\rulestepbreaknamed

\rulestepbreakvalnamed

\rulestepbreakvarnamed

\rulestepbreakloopnamed

\rulestepbreakexpnamed

\rulestepcontinuenamed

\rulestepcontinuevalnamed

\rulestepcontinuevarnamed

\end{mathpar}\begin{mathpar}

\rulestepcontinueloopnamed

\rulestepcontinueexpnamed

\rulestepexnpropnamed

\end{mathpar}

\boxed{\mu : \Sigma}
\begin{mathpar}
\memcheck{\mu}{\Sigma}
\end{mathpar}

\boxed{\Sigma \le \Sigma'}
\begin{mathpar}
\sigcheck{\Sigma}{\Sigma'}
\end{mathpar}

\section{Safety}

\newcommand{\ruleref}[1]{Rule \rulefmt{#1}}
\newcommand{\impliesab}[2]{$#1 \Rightarrow #2$}
\newcommand{\impliescmdreturn}[5]{\impliesab{#1 = \commandty}{\ruleonlyreturns{#2}{#3}{#4}{#5}}}
\newcommand{\impliescmdreturndefault}{\impliescmdreturn{\tau'}{\Sigma}{\Gamma}{e}{\tau''}}
\newcommand{\impliescmdreturnsimple}[1]{\impliescmdreturn{\tau'}{\Sigma}{\Gamma}{#1}{\tau''}}
\newcommand{\expandedG}{\varctx{\varctx{\cdot}{x_1}{\tau_1},\ldots}{x_n}{\tau_n}}

\subsection{Progress}

\progresstheorem{}

\subsection{Preservation}

\preservationtheorem{}

\section{Lemmas}

\begin{enumerate}
\item \label{lemmaassigned} If \rulegetframeassigneddefaultconclusion\ and $x \in A$ then $F$ has the form \frameexp{\frameval{F'}{x}{\tau_x}{v}}{\ldots}. \\
Proof is by induction on the derivation of \rulegetframeassigneddefaultconclusion{}.

\item \label{cfl} If \rulecheckvaldefaultconclusion{} then
\begin{enumerate}
\item If $\tau = \commandty$ then $v = \nop$.
\item If $\tau = \bool$ then either $v = \true$ or $v = \false$.
\item If $\tau = \integer$ then $v = \intliteral{n}$ for some $n \in [-2^{31},2^{31})$.
\item If $\tau = \function{\tuple{\tau_1, \ldots, \tau_n}}{\tau}$ then $v = \func{x_1, \ldots, x_n, e}$ where \rulecheckvalfuncpremisea, \rulecheckvalfuncpremiseb, \rulecheckvalfuncpremised, \rulecheckvalfuncpremisee, \rulecheckvalfuncpremisef, \rulecheckvalfuncpremiseg, \rulecheckvalfuncpremiseh, and \ruletausmall{\tau_1} \ldots{} \ruletausmall{\tau_n}.
\item If $\tau = \pointer{\tau}$ then either $v = \address{\anull}$ or $v = \address{a}$ where \rulecheckvaladdresspremisea
\item If $\tau = \arraytype{\tau}$ then $v = \arrayval{a}{n}$ such that $\forall i \in [0,n).\Sigma(a+i) = \tau$.
\item If $\tau = \struct{p}$ then $v = \structval{a}$ where \rulecheckvalstructpremisea.
\end{enumerate}

Proof is by inversion on the rules for \rulecheckval{\Sigma}{v}{\tau}.

\item \label{binopprogress} If $\mu : \Sigma$ and \rulecheckbinop{op}{\tau_1}{\tau_2}{\tau} and \rulecheckval{\Sigma}{v_1}{\tau_1} and \rulecheckval{\Sigma}{v_2}{\tau_2} then \ruleevalbinopdefaultconclusion{}

  Proof by case analysis on the derivation of \rulecheckbinop{op}{\tau_1}{\tau_2}{\tau}.

\item \label{monopprogress} If $\mu : \Sigma$ and \rulecheckmonop{op}{\tau}{\tau'} and $\rulecheckval{\Sigma}{v}{\tau}$ then \ruleevalmonopdefaultconclusion{}

  Proof by case analysis of the derivation of \rulecheckmonop{op}{\tau}{\tau'}.

\item \label{stackvarshape} If \rulegetframecontext{\Sigma}{F}{\Gamma,x:\tau} then either $F$ has the form \frameexp{\framevar{F'}{x}{\tau}}{\ldots} or the form \frameexp{\frameval{F'}{x}{\tau}{v}}{\ldots} where \rulecheckvaldefaultconclusion

  Proof by case analysis of the derivation of \rulegetframecontext{\Sigma}{F}{\Gamma,x:\tau}.

\item \label{allocprogress} If \rulecanalloc{\tau} and $a \not\in \dom{\mu}$ then $\exists \mu' . \ruleallocval{\mu}{a}{\tau}{\mu'}$.

  Proof by induction on $\tau$.

\item \label{allocarrayprogress} If \rulecanalloc{\tau} and $a \not\in \dom{\mu}$ and $n \ge 0$ then $\exists \mu' . \rulemakearray{\mu}{a}{\tau}{n}{\mu'}$.

  Proof by induction on $n$.

\item \label{allocpres} If \rulecanalloc{\tau} and $a \not\in \dom{\mu}$ and \ruleallocval{\mu}{a}{\tau}{\mu'} then $\exists \Sigma'. \mu' : \Sigma'$ and $\Sigma \le \Sigma'$ and $\Sigma'(a) = \tau$.

  Proof by induction on $\tau$.

\item \label{allocarraypres} If \rulecanalloc{\tau} and $a \not\in \dom{\mu}$ and $n \ge 0$ and \rulemakearray{\mu}{a}{\tau}{n}{\mu'} then $\exists \Sigma'.\mu' : \Sigma'$ and $\Sigma \le \Sigma'$ and \rulecheckval{\Sigma'}{\arrayval{a}{n}}{\arraytype{\tau}}.

  Proof by induction on $n$.

\item \label{allocsafety} If \rulecheckstate{\Sigma}{K}{\tau}{\bowtie}{e} and $\Sigma \le \Sigma'$ then \rulecheckstate{\Sigma'}{K}{\tau}{\bowtie}{e} \\

  Proof is by induction on the derivation of \rulecheckstate{\Sigma}{K}{\tau}{\bowtie}{e} with lots of auxiliary induction.

\item \label{ependingcmd}
  If \ruleispending{\Sigma}{\Gamma}{e}{\commandty}{\tau}{\tau''} then $\tau = \commandty{}$ and $e = \binop{\opseq}{e_1}{e_2}$

  Proof by case analysis on the derivation of \ruleispending{\Sigma}{\Gamma}{e}{\commandty}{\tau}{\tau''}

  \begin{description}
  \item[\rulelemmacmdispendingbinopl] \ \\
    \vspace{-1.1em}
    \begin{tabbing}
     $op = \opseq$ and $\tau = \commandty$ \` Inversion on \rulecheckbinop{op}{\commandty}{\tau''}{\tau} \\
    \end{tabbing}
  \item[\rulelemmacmdispendingbinopr] \ \\
    \vspace{-1.1em}
    \begin{tabbing}
     $op = \opseq$ and $\tau = \commandty$ \` Inversion on \rulecheckbinop{op}{\commandty}{\tau''}{\tau} \\
    \end{tabbing}
  \end{description}
  No other cases can occur.

\item \label{innerloopcheck}
  If \rulegetloopcontext{F}{\inloop} and \rulecheckframetype{\Sigma}{A}{F}{\commandty}{\tau}
  then \rulegetinnerloop{F}{\frameloop{F'}{e_c}{e}}, \rulegetframeassigned{F'}{A'} and \rulecheckframetype{\Sigma}{A'}{F'}{\commandty}{\tau} for some $e_c$, $e$, $F'$, and $A'$.

  Proof by induction on the structure of $F$:
  \begin{description}
  \item[$F = \framevar{F_0}{x}{\tau_{x}}$] \ \\
    \vspace{-1.8em}
    \begin{tabbing}
    \rulecheckframetype{\Sigma}{A' - \{x\}}{F_0}{\commandty}{\tau}
      \` Inversion on \rulecheckframetype{\Sigma}{A}{F}{\commandty}{\tau} \\
    \rulegetloopcontext{F_0}{\frameloop{F'}{e_c}{e}} and \rulegetframeassigned{F'}{A'} \\
    and \rulecheckframetype{\Sigma}{A'}{F'}{\commandty}{\tau}
      \` By inductive hypothesis \\
    \rulegetloopcontext{F}{\frameloop{F'}{e_c}{e}}
      \` \ruleref{\rulegetloopcontextdeclname} \\
    \end{tabbing}
  \item[$F = \frameval{F_0}{x}{\tau_x}{v}$] \ \\
    \vspace{-1.8em}
    \begin{tabbing}
    \rulecheckframetype{\Sigma}{A' - \{x\}}{F_0}{\commandty}{\tau}
      \` Inversion on \rulecheckframetype{\Sigma}{A}{F}{\commandty}{\tau} \\
    \rulegetinnerloop{F_0}{\frameloop{F'}{e_c}{e}} and \rulegetframeassigned{F'}{A'} \\
    and \rulecheckframetype{\Sigma}{A'}{F'}{\commandty}{\tau}
      \` By inductive hypothesis \\
    \rulegetloopcontext{F}{\frameloop{F'}{e_c}{e}}
      \` \ruleref{\rulegetloopcontextdefname} \\
    \end{tabbing}
  \item[$F = \frameexp{F_0}{e}$] \ \\
    \vspace{-1.8em}
    \begin{tabbing}
    \rulecheckassign{\Gamma}{A}{e}{A''} and \ruleispending{\Sigma}{\Gamma}{e}{\commandty}{\tau'}{\tau''} \\
      \` By inversion on \rulecheckframetype{\Sigma}{A}{F}{\commandty}{\tau} \\
    $\tau' = \commandty$
      \` By Lemma \ref{ependingcmd} \\
    \rulecheckframetype{\Sigma}{A''}{F}{\commandty}{\tau}
      \` By inversion on \rulecheckframetype{\Sigma}{A}{F}{\commandty}{\tau} \\
    \rulegetloopcontext{F_0}{\inloop} \\
      \` By inversion on the derivation of \rulegetloopcontext{F}{\inloop} \\
    \rulegetinnerloop{F_0}{\frameloop{F'}{e_c}{e}} and \rulegetframeassigned{F'}{A'} \\
    and \rulecheckframetype{\Sigma}{A'}{F'}{\commandty}{\tau}
      \` By inductive hypothesis \\
    \rulegetinnerloop{F}{\frameloop{F'}{e_c}{e}}
      \` \ruleref{\rulegetinnerloopexpname} \\
    \end{tabbing}
  \item[$F = \frameloop{F_0}{e_c}{e}$] \ \\
    \vspace{-1.8em}
    \begin{tabbing}
    \rulegetinnerloop{F}{\frameloop{F_0}{e_c}{e}}
      \` \ruleref{\rulegetinnerlooploopname} \\
    \rulegetframeassigned{F'}{A'} and \rulecheckframetype{\Sigma}{A'}{F}{\commandty}{\tau}
      \` Inversion on \rulecheckframetype{\Sigma}{A}{F}{\commandty}{\tau} \\
    \end{tabbing}
  \end{description}

\item \label{binoppres} If \rulecheckbinopdefaultconclusion{} and $\mu : \Sigma$ and \ruleevalbinopdefaultconclusion{} then $\mu' : \Sigma$ and either $t = \exn{}$ or else $t = v$ and \rulecheckvaldefaultconclusion{}.

  Proof by case analysis on the derivation of \ruleevalbinopdefaultconclusion{}

\item \label{monoppres} If \rulecheckmonopdefaultconclusion{} and $\mu : \Sigma$ and \ruleevalmonopdefaultconclusion{} then either $t = \exn{}$ or else $t = v$ and \rulecheckvaldefaultconclusion{}.

  Proof by case analysis on the derivation of \ruleevalmonopdefaultconclusion{}

\item \label{constructctx} If $\Gamma = \expandedG$ and $\forall i \in [1,n]. \Sigma \entails v_i : \tau_i$ and $F = \frameval{\frameexp{\frameval{\cdot}{x_1}{\tau_1}{v_1}}{\ldots}}{x_n}{\tau_n}{v_n}$ then \rulegetframecontextdefaultconclusion{} \\

  Proof by induction on the structure of $F$.

\item \label{constructA} If $A = \{x_1, \ldots, x_n\}$ and $F = \frameval{\frameexp{\frameval{\cdot}{x_1}{\tau_1}{v_1}}{\ldots}}{x_n}{\tau_n}{v_n}$ then \rulegetframeassigneddefaultconclusion{} \\

  Proof by induction on the structure of $F$.

\item \label{extendAassign}
  If \rulecheckassign{\expandedG}{A_1}{e}{A_2} and $A_1 \subseteq A_1' \subseteq \{x_1,\ldots,x_n\}$
  then \rulecheckassign{\expandedG}{A_1'}{e}{A_2'} where $A_2 \subseteq A_2' \subseteq \{x_1,\ldots,x_n\}$

  Proof by induction on the derivation of \rulecheckassign{\expandedG}{A_1}{e}{A_2}.

\item \label{extendAframetype}
  If \rulecheckframetypedefaultconclusion{} and \rulegetframecontext{\Sigma}{F}{\expandedG} and $A \subseteq A' \subseteq \{x_1,\ldots,x_n\}$
  then \rulecheckframetype{\Sigma}{A'}{F}{\tau}{\tau'}

  Proof by induction on the derivation of \rulecheckframetypedefaultconclusion{}.

\item \label{assignfirstassigned}
  If $F = F_0,x:\tau_x,\ldots$ and \rulecheckval{\Sigma}{v}{\tau_x} and $F' = F_0,x:\tau_x=v,\ldots$ and \rulegetframeassigned{F}{A}
  then \rulegetframeassigned{F'}{A \cup \{x\}}

  Proof by induction on the structure of $F$.
\item \label{assignassigned}
  If $F = F_0,x=v',\ldots$ and \rulecheckval{\Sigma}{v'}{\tau_v} and \rulecheckval{\Sigma}{v}{\tau_v} and $F' = F_0,x:\tau_x=v,\ldots$ and \rulegetframeassigned{F}{A}
  then \rulegetframeassigned{F'}{A}

  Proof by induction on the structure of $F$.

\item \label{assignfirstctx}
  If $F = F_0,x:\tau_x,\ldots$ and \rulecheckval{\Sigma}{v}{\tau_x} and $F' = F_0,x:\tau_x=v,\ldots$ and \rulegetframecontext{\Sigma}{F}{\Gamma}
  then \rulegetframecontext{\Sigma}{F'}{\Gamma}

  Proof by induction on the structure of $F$.
\item \label{assignctx}
  If $F = F_0,x:\tau_x=v',\ldots$ and \rulecheckval{\Sigma}{v'}{\tau_v} and \rulecheckval{\Sigma}{v}{\tau_v} and $F' = F_0,x:\tau_x=v,\ldots$ and \rulegetframecontext{\Sigma}{F}{\Gamma}
  then \rulegetframecontext{\Sigma}{F'}{\Gamma}

  Proof by induction on the structure of $F$.

\item \label{assignfirstframetype}
  If $F = F_0,x:\tau_x,\ldots$ and \rulecheckval{\Sigma}{v}{\tau_x} and $F' = F_0,x:\tau_x=v,\ldots$ and \rulecheckframetype{\Sigma}{A}{F}{\tau}{\tau'}
  then \rulecheckframetype{\Sigma}{A}{F'}{\tau}{\tau'}

  Proof by induction on the structure of $F$.

\item \label{assignframetype}
  If $F = F_0,x:\tau_x=v',\ldots$ and \rulecheckval{\Sigma}{v'}{\tau_x} and \rulecheckval{\Sigma}{v}{\tau_x} and $F' = F_0,x:\tau_x=v,\ldots$ and \rulecheckframetype{\Sigma}{A}{F}{\tau}{\tau'}
  then \rulecheckframetype{\Sigma}{A}{F'}{\tau}{\tau'}

  Proof by induction on the structure of $F$.

\item \label{assignexpand}
  If \rulecheckassign{\Gamma}{A}{e}{A'} then $A \subseteq A'$.

  Proof by induction on the derivation of \rulecheckassign{\Gamma}{A}{e}{A'}.

\item \label{loopsub}
  If \rulecheckloop{L}{e} then \rulecheckloop{\inloop}{e}.

  Proof by induction on the derivation of \rulecheckloop{L}{e}.

\end{enumerate}

\section{Proofs}

\subsection{Progress}
\label{progressproof}

Proof by case analysis on the derivation of \rulecheckstatedefaultconclusion.

\newcommand{\progressruleused}[2]{Case \rulefmt{#1} was used so #2}

\begin{enumerate}

\item \progressruleused{\rulecheckstateexnname}{$t = \exn$}

  Via case analysis of the structure of $K$ either
  \begin{enumerate}
  \item Case $K = \emptystack$
    \begin{tabbing}
    \rulefinalstateexnconclusion{}
      \` \ruleref{\rulefinalstateexnname} \\
    \end{tabbing}

  \item Case $K = \stackframe{K'}{F}$
    \begin{tabbing}
    \ruleKPrimestepexnpropconclusion{}
      \` \ruleref{\ruleKPrimestepexnpropname} \\
    \end{tabbing}

  \end{enumerate}

\item \progressruleused{\rulecheckstateemptyname}{$K = \emptystack$}

  \begin{tabbing}
  \rulefinalstatevalconclusion{}
    \` \ruleref{\rulefinalstatevalname} \\
  \end{tabbing}

\item \progressruleused{\ruleKPrimecheckstatenormalname}{$K = \stackframe{K'}{F}$ and $t = e$ and \ruleKPrimecheckstatenormalpremiseb{}} \ \\
  We now do a case analysis on \ruleKPrimecheckstatenormalpremiseb{}:

  \begin{enumerate}

  \item \progressruleused{\ruleprogressmodifiedtauandgammacheckexpvarname}{$e = x$} \ \\

    \begin{tabbing}
    \rulecheckassignvarpremisea \` Inversion on \rulecheckassign{\Gamma}{A}{x}{A} \\
    $F = \frameexp{\frameval{F'}{x}{\tau_x}{v}}{\ldots}$ \` Lemma \ref{lemmaassigned} \\
    \ruleKPrimeFPrimestepvarconclusion{}
      \` \ruleref{\rulestepvarname} \\
    \end{tabbing}

  \item \progressruleused{\ruleprogressmodifiedtaucheckexpvaluename}{$e = v$} \ \\

    Via case analysis of \ruledirokdefaultconclusion{} we know that either \rulefmt{\ruleeisvdirokpushingname} or \rulefmt{\ruleeisvdirokreturningname} was used. \\

    In the former case, \rulefmt{\rulestepvalname} applies.
    For the latter, we must case analyze \ruleKPrimecheckstatenormalpremiseg:\\

    \begin{enumerate}
    \item \progressruleused{\ruleKPrimecheckframetypeframeloopname}{$F = \frameloop{F'}{e_c}{e'}$ for some $F'$, $e_c$ and $e'$} \ \\

      \begin{tabbing}
      $v = \nop$
        \` Lemma \ref{cfl} \\
      \ruleKPrimeFPrimesteplooppopconclusion{} \\
        \` \ruleref{\rulesteplooppopname} \\
      \end{tabbing}

    \item \progressruleused{\ruleKPrimecheckframetypeframevarname}{$F = \framevar{F'}{x}{\tau_x}$ for some $F'$, $x$ and $\tau_x$} \ \\
      \begin{tabbing}
      $v = \nop$
        \` Lemma \ref{cfl} \\
      \ruleKPrimeFPrimesteppopdeclconclusion{}
        \` \ruleref{\rulesteppopdeclname} \\
      \end{tabbing}

    \item \progressruleused{\ruleKPrimecheckframetypeframevalname}{$F = \frameval{F'}{x}{\tau_x}{v}$ for some $F'$, $x$, $\tau_x$ and $v$} \ \\
      \begin{tabbing}
      $v = \nop$
        \` Lemma \ref{cfl} \\
      \ruleKPrimeFPrimesteppopassignedconclusion{}
        \` \ruleref{\rulesteppopassignedname} \\
      \end{tabbing}

    \item \progressruleused{\ruleKPrimeFPrimecheckframetypeframeexpretname}{$F = \frameexp{F'}{e'}$ for some $F'$ and $e'$} \ \\
      We must now case analysis on the derivation of \ruledoesreturn{e'}:
      \begin{enumerate}

      \item \progressruleused{\ruledoesreturnreturnname}{$e' = \return{e''}$} \ \\
        \begin{tabbing}
        $e'' = \cdot$
          \` Inversion on \ruleispending{\Sigma}{\Gamma}{e'}{\tau'}{\commandty}{\tau''} \\
        \ruleKPrimeFPrimesteppopretconclusion{}
          \` \ruleref{\rulesteppopretname} \\
        \end{tabbing}

      \item \progressruleused{\ruledoesreturnifname}{$e' = \ifexp{e_c}{e_t}{e_f}$} \ \\
        \begin{tabbing}
        \hspace{1em} \= \kill
        $e_c = \cdot$ and $\tau' = \bool$
          \` Inversion on \ruleispending{\Sigma}{\Gamma}{e'}{\tau'}{\commandty}{\tau''} \\
        $v = \true$ or $v = \false$
          \` Lemma \ref{cfl} \\
        If $v = \true$ then \+ \\
          \ruleKPrimeFPrimesteppopiftrueconclusion{} \\
            \` \ruleref{\rulesteppopiftruename} \- \\
        Otherwise $v = \false$ so \+ \\
          \ruleKPrimeFPrimesteppopiffalseconclusion{} \\
            \` \ruleref{\rulesteppopiffalsename} \- \\
        \end{tabbing}

      \item \progressruleused{\ruledoesreturnbinoprhsname}{$e' = \binop{\opseq}{e_1}{e_2}$} \ \\

        \begin{tabbing}
        $e_2 \not= \cdot$
          \` Via case analysis we know there is no derivation of \ruledoesreturn{\cdot} \\
        $e_1 = \cdot$
          \` Inversion on \ruleispending{\Sigma}{\Gamma}{e'}{\tau'}{\commandty}{\tau''} \\
        \ruleKPrimeFPrimestepswapbinopconclusion{} \\
          \` \ruleref{\rulestepswapbinopname} \\
        \end{tabbing}

      \item \progressruleused{\ruledoesreturnbinoplhsname}{$e' = \binop{\opseq}{e_1}{e_2}$}  \ \\

        \begin{tabbing}
        $e_1 \not= \cdot$
          \` Via case analysis we know there is no derivation of \ruledoesreturn{\cdot} \\
        $e_2 = \cdot$ and $e_1 = v'$ for some $v'$ \\
          \` Inversion on \ruleispending{\Sigma}{\Gamma}{e'}{\tau'}{\commandty}{\tau''} \\
        This case does not occur. \\
          \` Via case analysis we know there is no derivation of \ruledoesreturn{v'} \\
        \end{tabbing}

      \item \progressruleused{\ruledoesreturndeclname}{$e' = \decl{x}{\tau_x}{e''}$}

        This case does not occur because via case analysis we know there is no derivation of \ruleispending{\Sigma}{\Gamma}{e'}{\tau'}{\commandty}{\tau''}

      \end{enumerate}

    \item \progressruleused{\ruleKPrimeFPrimecheckframetypeframeexpnoretname}{$F = \frameexp{F'}{e'}$ for some $F'$ and $e'$} \ \\
      We now do a case analysis of \ruleispending{\Sigma}{\Gamma}{e'}{\tau'}{\tau'''}{\tau''}

      \begin{enumerate}

      \item \progressruleused{\ruleKPrimeFPrimeispendingbinoplname}{$e' = \binop{op}{\cdot}{e_2}$} \ \\

        \begin{tabbing}
        \ruleKPrimeFPrimestepswapbinopconclusion{} \\
          \` \ruleref{\rulestepswapbinopname} \\
        \end{tabbing}

      \item \progressruleused{\ruleKPrimeFPrimeispendingbinoprname}{$e' = \binop{op}{v'}{\cdot}$} \ \\

        \begin{tabbing}
        \ruleevalbinop{\mu}{op}{v}{v'}{\mu'}{t}
          \` Lemma \ref{binopprogress} \\
        \ruleKPrimeFPrimesteppopbinopconclusion{}
          \` \ruleref{\rulesteppopbinopname} \\
        \end{tabbing}

      \item \progressruleused{\ruleKPrimeFPrimeispendingmonopname}{$e' = \monop{op}{\cdot}$} \ \\

        \begin{tabbing}
        \ruleevalmonopdefaultconclusion{}
          \` Lemma \ref{monopprogress} \\
        \ruleKPrimeFPrimesteppopmonopconclusion{}
          \` \ruleref{\rulesteppopmonopname} \\
        \end{tabbing}

      \item \progressruleused{\ruleKPrimeFPrimeispendingcallfname}{$e' = \call{\cdot}{e_a}$} \ \\

        \begin{tabbing}
        \ruleKPrimeFPrimesteppushcallargsconclusion{} \\
          \` \ruleref{\rulesteppushcallargsname} \\
        \end{tabbing}

      \item \progressruleused{\ruleKPrimeFPrimeispendingtuplename}{$e' = \tupleexp{v_1,\ldots,v_{i-1},\cdot,e_{i+1},\ldots,e_n}$} \ \\
        Case analysis on the structure of \tupleexp{v_1,\ldots,v_{i-1},\cdot,e_{i+1},\ldots,e_n}
        \begin{itemize}
        \item $n = i$
          \begin{tabbing}
          \ruleKPrimeFPrimesteplasttupleelemconclusion{} \\
            \` \ruleref{\rulesteplasttupleelemname} \\
          \end{tabbing}

        \item $n > i$
          \begin{tabbing}
          \ruleKPrimeFPrimestepnexttupleelemconclusion{} \\
            \` \ruleref{\rulestepnexttupleelemname} \\
          \end{tabbing}

        \end{itemize}

      \item \progressruleused{\ruleKPrimeFPrimeispendingifname}{$e' = \ifexp{\cdot}{e_t}{e_f}$} \ \\
        Clearly $\tau' = \bool$.

        \begin{tabbing}
        \hspace{2em} \= \kill
        $v = \true$ or $v = \false$
          \` Lemma \ref{cfl} \\
        If $v = \true$ then \+ \\
          \ruleKPrimeFPrimesteppopiftrueconclusion{} \\
            \` \ruleref{\rulesteppopiftruename} \- \\
        Otherwise $v = \false$ so \+ \\
          \ruleKPrimeFPrimesteppopiffalseconclusion{} \\
            \` \ruleref{\rulesteppopiffalsename} \- \\
        \end{tabbing}

      \item \progressruleused{\ruleKPrimeFPrimeispendingassignname}{$e' = \assign{x}{\cdot}$} \ \\

        \begin{tabbing}
        \hspace{2em} \= \kill
        We know that either $F'$ has the form \frameexp{\framevar{F''}{x}{\tau'}}{\ldots} or it has the form \\
            \frameexp{\frameval{F''}{x}{\tau_x}{v'}}{\ldots}
          \` Lemma \ref{stackvarshape} \\
        In the former case, \\
          \prgstate{\mu}{\stackframe{K'}{\frameexp{\frameexp{\framevar{F''}{x}{\tau'}}{\ldots}}{\assign{x}{\cdot}}}}{\lhd}{v}
            $\rightarrow$ \+ \\
          \prgstate{\mu}{\stackframe{K'}{\frameexp{\frameval{F''}{x}{\tau_x}{v}}{\ldots}}}{\lhd}{\nop} \-
            \` \ruleref{\rulesteppopassignfirstname} \\
        In the latter case, \\
          \prgstate{\mu}{\stackframe{K'}{\frameexp{\frameexp{\frameval{F''}{x}{\tau'}{v'}}{\ldots}}{\assign{x}{\cdot}}}}{\lhd}{v}
            $\rightarrow$ \+ \\
          \prgstate{\mu}{\stackframe{K'}{\frameexp{\frameval{F''}{x}{\tau_x}{v}}{\ldots}}}{\lhd}{\nop} \-
            \` \ruleref{\rulesteppopassignname} \\
        \end{tabbing}

      \item \progressruleused{\ruleKPrimeFPrimeispendingreturnname}{$e' = \return{e''}$} \ \\

        \begin{tabbing}
        \ruleKPrimeFPrimesteppopretconclusion{}
          \` \ruleref{\rulesteppopretname} \\
        \end{tabbing}

      \item \progressruleused{\ruleKPrimeFPrimeispendingallocarrayname}{$e' = \allocarrayexp{\tau_a}{\cdot}$} \ \\
        \begin{tabbing}
        \hspace{2em} \= \kill
        $v = \intliteral{n}$
          \` Lemma \ref{cfl} \\
        If $n \ge 0$ then \+ \\
          \ruleKPrimeFPrimesteppopallocarraypremiseb{}
            \` Lemma \ref{allocarrayprogress} \\
          \ruleKPrimeFPrimesteppopallocarrayconclusion{} \\
            \` \ruleref{\rulesteppopallocarrayname} \- \\
        Otherwise, \ruleKPrimeFPrimesteppopallocarrayerrconclusion{} \\
          \` \ruleref{\rulesteppopallocarrayerrname} \\
        \end{tabbing}

      \end{enumerate}

    \end{enumerate}

  \item \progressruleused{\ruleprogressmodifiedtaucheckexpbinopname}{$e = \binop{op}{e_1}{e_2}$} \ \\

    \begin{tabbing}
    $\bowtie = \rhd$
      \` Inversion on \ruledirok{\bowtie}{\binop{op}{e_1}{e_2}} \\
    \ruleKPrimesteppushbinopconclusion{} \\
      \` \ruleref{\rulesteppushbinopname} \\
    \end{tabbing}

  \item \progressruleused{\ruleprogressmodifiedtaucheckexpmonopname}{$e = \monop{op}{e'}$} \ \\

    \begin{tabbing}
    $\bowtie = \rhd$
      \` Inversion on \ruledirok{\bowtie}{\monop{op}{e_a}} \\
    \ruleKPrimesteppushmonopconclusion{}
      \` \ruleref{\rulesteppushmonopname} \\
    \end{tabbing}

  \item \progressruleused{\ruleprogressmodifiedtaucheckexptuplename}{$e = \tupleexp{e_1, \ldots, e_n}$} \ \\

    We do a case analysis on \ruleprogresstupledirokdefaultconclusion:
    \begin{enumerate}
    \item \progressruleused{\ruleprogresstupledirokpushingname}{$\bowtie = \rhd$} \ \\

      \begin{tabbing}
      \hspace{2em} \= \kill
      If $n = 0$ then \+ \\
        \ruleKPrimesteppushemptytupleconclusion{}
          \` \ruleref{\rulesteppushemptytuplename} \- \\
      Otherwise \+
        \ruleKPrimesteppushtupleelemconclusion{} \\
          \` \ruleref{\rulesteppushtupleelemname} \- \\
      \end{tabbing}

    \item \progressruleused{\ruleprogresstupledirokreturningname}{$\bowtie = \lhd$ and $e = v$} \ \\

      Does not occur since there is no derivation of \rulecheckval{\Sigma}{v}{\tuple{\tau_1,\ldots,\tau_n}}.

    \item \progressruleused{\ruleprogresstuplediroktuplename}{$\bowtie = \lhd$ and $e = \tupleexp{v_1,\ldots,v_n}$} \ \\
      We do a case analysis on the derivation of \ruleprogresstuplecheckframetypedefaultconclusion{}

      \begin{enumerate}

      \item \progressruleused{\ruleprogresstuplecheckframetypeframeexpnoretname}{$F' = \frameexp{F''}{e'}$} \ \\
        \begin{tabbing}
        \hspace{3em} \= \hspace{3em} \= \kill
        There are three possible derivations of \\
          \` \ruleispending{\Sigma}{\Gamma}{e'}{\tupleexp{\tau_1,\ldots,\tau_n}}{\tau'''}{\tau''}. \\
        Case \rulefmt{\ruleispendingreturnname}: \+ \\
          $e' = \return{\cdot}$ and \ruletausmall{\tuple{\tau_1,\ldots,\tau_n}} \\
            \` Inversion on \ruleispending{\Sigma}{\Gamma}{e'}{\tupleexp{\tau_1,\ldots,\tau_n}}{\tau'''}{\tau''}. \\
          $n = 0$
            \` Inversion on \ruletausmall{\tuple{\tau_1,\ldots,\tau_n}} \\
          \rulestep{\mu}{K'}{\frameexp{F'}{\return{\cdot}}}{\lhd}{\tupleexp{}}
                   {\mu}{K'}{\lhd}{\tupleexp{}}
            \` \ruleref{\rulesteppopretname} \- \\
        Case \rulefmt{\ruleispendingmonopname} where $op = \opign$: \+ \\
          $e' = \monop{\opign, \cdot}$ and $\opign : \tuple{\tau_1,\ldots,\tau_n} \rightarrow \commandty$ \\
            \` Inversion on \ruleispending{\Sigma}{\Gamma}{e'}{\tupleexp{\tau_1,\ldots,\tau_n}}{\tau'''}{\tau''}. \\
          \ruletausmall{\tuple{\tau_1,\ldots,\tau_n}} \\
            \` Inversion on $\opign : \tuple{\tau_1,\ldots,\tau_n} \rightarrow \commandty$ \\
          $n = 0$
            \` Inversion on \ruletausmall{\tuple{\tau_1,\ldots,\tau_n}} \\
          \ruleevalmonop{\mu}{\opign}{\tupleexp{}}{t}
            \` Lemma \ref{monopprogress} \\
          \ruleKPrimeFPrimesteppopmonopconclusion{} \\
            \` \ruleref{\rulesteppopmonopname} \- \\
        Otherwise \rulefmt{\ruleispendingcallaname} was used: \+ \\
          $e' = \call{\address{a}, \cdot{}}$ and
          \rulecheckval{\Sigma}{\address{a}}{\pointer{(\function{\tuple{\tau_1, \ldots, \tau_n}}{\tau'''})}} \\
            \` Inversion on \ruleispending{\Sigma}{\Gamma}{e'}{\tupleexp{\tau_1,\ldots,\tau_n}}{\tau'''}{\tau''}. \\
          $\Sigma(a) = \function{\tuple{\tau_1, \ldots, \tau_n}}{\tau'''}$ \\
            \` Inversion on \rulecheckval{\Sigma}{\address{a}}{\pointer{\function{\tuple{\tau_1, \ldots, \tau_n}}{\tau'''}}} \\
          $\mu(a) = v$ and \rulecheckval{\Sigma}{v}{\function{\tuple{\tau_1, \ldots, \tau_n}}{\tau'''}} \` $\mu : \Sigma$ \\
          $v = \func{x_1, \ldots, x_n, e_b}$ \` Lemma \ref{cfl} \\
          \prgstate{\mu}{K'}{\frameexp{F'}{\call{\address{a},\cdot}}}{\lhd}{\tupleexp{v_1,\ldots,v_n}} $\rightarrow$ \+ \\
            \prgstate{\mu}{\stackframe{\stackframe{K'}{F'}}{\frameval{\frameexp{\frameval{\cdot}{x_1}{\tau_1}{v_1}}{\ldots}}{x_n}{\tau_n}{v_n}}}{\rhd}{e_b} \- \\
            \` \ruleref{\rulestepfinalizecallname} \\
        \\
        \end{tabbing}
      \item \progressruleused{\ruleprogresstuplecheckframetypeframeexpretname}{$F' = \frameexp{F''}{e'}$} \ \\
        Progress holds via the same reasoning as in the previous case though only the \rulefmt{\ruleispendingreturnname} case occurs since there are no derivations of \ruledoesreturn{e'} for the other cases.

      \end{enumerate}

    \end{enumerate}

  \item \progressruleused{\ruleprogressmodifiedtaucheckexpcallname}{$e = \call{e_f, e}$} \ \\

    \begin{tabbing}
    $\bowtie{} = \rhd$
      \` Inversion on \ruledirok{\bowtie}{\call{e_f, e_a}} \\
    \ruleKPrimesteppushcallfnconclusion{}
      \` \ruleref{\rulesteppushcallfnname} \\
    \end{tabbing}

  \item \progressruleused{\ruleprogressmodifiedtaucheckexpifname}{$e = \ifexp{e_c}{e_t}{e_f}$} \ \\

    \begin{tabbing}
    $\bowtie{} = \rhd$
      \` Inversion on \ruledirok{\bowtie}{\ifexp{e_c}{e_t}{e_f}} \\
    \ruleKPrimesteppushifconclusion
      \` \ruleref{\rulesteppushifname} \\
    \end{tabbing}

  \item \progressruleused{\ruleprogressmodifiedtaucheckexpdeclname}{$e = decl{x}{\tau_x}{e'}$} \ \\

    \begin{tabbing}
    $\bowtie{} = \rhd$
      \` Inversion on \ruledirok{\bowtie}{\decl{x}{\tau_x}{e'}} \\
    \ruleprogressmodifiedtausteppushdeclconclusion
      \` \ruleref{\rulesteppushdeclname} \\
    \end{tabbing}

  \item \progressruleused{\ruleprogressmodifiedtaucheckexpassignname}{$e = \assign{x}{e'}$} \ \\

    \begin{tabbing}
    $\bowtie{} = \rhd$
      \` Inversion on \ruledirok{\bowtie}{\assign{x}{e'}} \\
    \ruleKPrimesteppushassignconclusion
      \` \ruleref{\rulesteppushassignname} \\
    \end{tabbing}

  \item \progressruleused{\ruleprogressmodifiedtaucheckexpreturnname}{$e = \return{e'}$} \ \\

    \begin{tabbing}
    $\bowtie{} = \rhd$
      \` Inversion on \ruledirok{\bowtie}{\return{e'}} \\
    \ruleKPrimesteppushretconclusion
      \` \ruleref{\rulesteppushretname} \\
    \end{tabbing}

  \item \progressruleused{\ruleprogressmodifiedtaucheckexploopname}{$e = \loopstm{e_c}{e}$} \ \\

    \begin{tabbing}
    $\bowtie{} = \rhd$
      \` Inversion on \ruledirok{\bowtie}{\loopstm{e_c}{e'}} \\
    \ruleKPrimesteploopconclusion \\
      \` \ruleref{\rulesteploopname} \\
    \end{tabbing}

  \item \progressruleused{\ruleprogressmodifiedtaucheckexpbreakname}{$e = \breakstm$} \ \\

    \begin{tabbing}
    $\bowtie{} = \rhd$
      \` Inversion on \ruledirok{\bowtie}{\breakstm} \\
    \ruleKPrimestepbreakconclusion
      \` \ruleref{\rulestepbreakname} \\
    \end{tabbing}

  \item \progressruleused{\ruleprogressmodifiedtaucheckexpcontinuename}{$e = \continue$} \ \\

    \begin{tabbing}
    $\bowtie{} = \rhd$
     \` Inversion on \ruledirok{\bowtie}{\continue} \\
    \ruleKPrimestepcontinueconclusion
      \` \ruleref{\rulestepcontinuename} \\
    \end{tabbing}

  \item \progressruleused{\ruleprogressmodifiedtaucheckexpallocname}{$e = \allocexp{\tau_a}$} \ \\

    \begin{tabbing}
    $\bowtie{} = \rhd$
     \` Inversion on \ruledirok{\bowtie}{\allocexp{\tau_a}} \\
    \ruleallocval{\mu}{a}{\tau_a}{\mu'}
      \` Lemma \ref{allocprogress} \\
    \ruleKPrimestepallocconclusion
      \` \ruleref{\rulestepallocname} \\
    \end{tabbing}

  \item \progressruleused{\ruleprogressmodifiedtaucheckexpallocarrayname}{$e = \allocarrayexp{\tau_a}{e'}$} \ \\

    \begin{tabbing}
    $\bowtie{} = \rhd$
     \` Inversion on \ruledirok{\bowtie}{\allocarrayexp{\tau_a}{e'}} \\
    \rulestep{\mu}{K'}{F}{\rhd}{\allocarrayexp{\tau_a}{e'}}{\mu}{\stackframe{K'}{\frameexp{F}{\allocarrayexp{\tau_a}{\cdot}}}}{\rhd}{e'} \\
      \` \ruleref{\rulesteppushallocarrayname} \\
    \end{tabbing}

  \end{enumerate} % e : \tau'

\item Case \rulefmt{\ruleKPrimecheckstatereturnsname} was used \ \\
  By inversion on \rulereturns{\Sigma}{\Gamma}{e}{\tau''} we know \ruledoesreturndefaultconclusion{}
  so by case analysis, $e$ is one of the following:
  \begin{itemize}
  \item \ruledoesreturnbinoplhsconclusion{}
  \item \ruledoesreturnbinoprhsconclusion{}
  \item \ruledoesreturnifconclusion{}
  \item \ruledoesreturndeclconclusion{}
  \item \ruledoesreturnreturnconclusion{}
  \end{itemize}

  These cases are handled exactly the same as if the rule used in the derivation was
  \rulefmt{\ruleKPrimecheckstatenormalname}.

\item \progressruleused{\ruleKPrimecheckstateloopbrkname}{$K = \stackframe{K'}{F}$} \ \\

  We now do case analysis on the structure of $F$:

  \begin{description}
  \item[$F=\framevar{F'}{x}{\tau_x}$] \ \\
    \begin{tabbing}
    \ruleKPrimeFPrimestepbreakvarconclusion{}
      \` \ruleref{\rulestepbreakvarname} \\
    \end{tabbing}
  \item[$F=\frameval{F'}{x}{\tau_x}{v}$] \ \\
    \begin{tabbing}
    \ruleKPrimeFPrimestepbreakvalconclusion{}
      \` \ruleref{\rulestepbreakvalname} \\
    \end{tabbing}
  \item[$F=\frameloop{F'}{e_c}{e}$] \ \\
    \begin{tabbing}
    \ruleKPrimeFPrimestepbreakloopconclusion{}
      \` \ruleref{\rulestepbreakloopname} \\
    \end{tabbing}
  \item[$F=\frameexp{F'}{e}$] \ \\
    \begin{tabbing}
    \ruleKPrimeFPrimestepbreakexpconclusion{}
      \` \ruleref{\rulestepbreakexpname} \\
    \end{tabbing}
  \end{description}

\item \progressruleused{\ruleKPrimecheckstateloopcontname}{$K = \stackframe{K'}{F}$} \ \\

  We now do case analysis on the structure of $F$:

  \begin{description}
  \item[$F=\framevar{F'}{x}{\tau_x}$] \ \\
    \begin{tabbing}
    \ruleKPrimeFPrimestepcontinuevarconclusion{}
      \` \ruleref{\rulestepcontinuevarname} \\
    \end{tabbing}
  \item[$F=\frameval{F'}{x}{\tau_x}{v}$] \ \\
    \begin{tabbing}
    \ruleKPrimeFPrimestepcontinuevalconclusion{} \\
      \` \ruleref{\rulestepcontinuevalname} \\
    \end{tabbing}
  \item[$F=\frameloop{F'}{e_c}{e}$] \ \\
    \begin{tabbing}
    \ruleKPrimeFPrimestepcontinueloopconclusion{} \\
      \` \ruleref{\rulestepcontinueloopname} \\
    \end{tabbing}
  \item[$F=\frameexp{F'}{e}$] \ \\
    \begin{tabbing}
    \ruleKPrimeFPrimestepcontinueexpconclusion{}
      \` \ruleref{\rulestepcontinueexpname} \\
    \end{tabbing}
  \end{description}

\end{enumerate} % K d t : \tau

\subsection{Preservation}
\label{preservationproof}

Proof by induction on the derivation of \rulestepdefaultconclusion. The cases where $\Sigma' \not= \Sigma$ are noted explicitly.

\begin{description}

\item[\rulestepval] \ \\
\newcommand{\presvalunwind}{\rulecheckstate{\Sigma}{\stackframe{K}{F}}{\tau}{\rhd}{v}}
\begin{tabbing}
\rulegetframecontextdefaultconclusion{} \` Inversion on \presvalunwind \\
\rulecheckexp{\Sigma}{\Gamma}{v}{\tau'} \` Inversion on \presvalunwind \\
\rulegetframeassigneddefaultconclusion  \` Inversion on \presvalunwind \\
\rulecheckassignvalueconclusion{}       \` Inversion on \presvalunwind \\
\rulegetloopcontextdefaultconclusion    \` Inversion on \presvalunwind \\
\rulecheckloopvalueconclusion           \` Inversion on \presvalunwind \\
\rulecheckframetype{\Sigma}{A}{F}{\tau'}{\tau'''} \` Inversion on \presvalunwind \\
\rulecheckstack{\Sigma}{K}{\tau''}{\tau}\` Inversion on \presvalunwind \\
\impliescmdreturnsimple{v}              \` Inversion on \presvalunwind \\
\ruledirok{\lhd}{v}                     \` \ruleref{\ruledirokreturningname} \\
\rulecheckstate{\Sigma}{\stackframe{K}{F}}{\tau}{\lhd}{v} \` \ruleref{\rulecheckstatenormalname} \\
\end{tabbing}

\item[\rulestepvar] \ \\
\newcommand{\varframe}{\frameexp{\frameval{F}{x}{\tau_x}{v}}{\ldots}}
\newcommand{\presvarunwind}{\rulecheckstate{\Sigma}{\stackframe{K}{F}}{\tau}{\rhd}{x}}
\begin{tabbing}
\rulegetframecontext{\Sigma}{\varframe}{\Gamma} \` Inversion on \presvarunwind \\
\rulecheckexp{\Sigma}{\Gamma}{x}{\tau'} \` Inversion on \presvarunwind \\
\rulegetframeassigned{\varframe}{A}     \` Inversion on \presvarunwind \\
\rulecheckassignvarconclusion{}         \` Inversion on \presvarunwind \\
\rulegetloopcontext{\varframe}{L}       \` Inversion on \presvarunwind \\
\rulecheckframetype{\Sigma}{A}{\varframe}{\tau'}{\tau''} \` Inversion on \presvarunwind \\
$\Gamma = \varctx{\Gamma'}{x}{\tau'}$   \` Inversion on \rulecheckexp{\Sigma}{\Gamma}{x}{\tau'} \\
\rulecheckexp{\Sigma}{\Gamma}{v}{\tau'} \` Lemma \ref{stackvarshape} \\
\rulecheckloopvalueconclusion           \` \ruleref{\rulecheckloopvaluename} \\
\rulecheckstack{\Sigma}{K}{\tau''}{\tau}\` Inversion on \presvarunwind \\
If $\tau' = \commandty$, $v = \nop$ 
  \` Lemma \ref{cfl} \\
\impliescmdreturnsimple{v} \\
  \` \ruleonlyreturns{\Sigma}{\Gamma}{\nop}{\tau''} by rule \rulefmt{\ruleonlyreturnsnopname} \\
\ruledirok{\lhd}{v}                     \` \ruleref{\ruledirokreturningname} \\
\rulecheckstate{\Sigma}{\stackframe{K}{\varframe}}{\tau}{\lhd}{v} \` \ruleref{\rulecheckstatenormalname} \\
\end{tabbing}

\item[\rulesteppushbinop] \ \\
  \newcommand{\presbinopunwind}{\rulecheckstate{\Sigma}{\stackframe{K}{F}}{\tau}{\rhd}{\binop{op}{e_1}{e_2}}}
  \newcommand{\presbinopF}{\frameexp{F}{\binop{op}{\cdot}{e_2}}}

  \begin{tabbing}
  \hspace{3em} \= \hspace{3em} \= \kill
  \rulegetframecontextdefaultconclusion{}
    \` Inversion on \presbinopunwind \\
  \rulegetframeassigneddefaultconclusion
    \` Inversion on \presbinopunwind \\
  \rulecheckassignbinopconclusion{}
    \` Inversion on \presbinopunwind \\
  \rulegetloopcontextdefaultconclusion
    \` Inversion on \presbinopunwind \\
  \rulecheckloopbinopconclusion
    \` Inversion on \presbinopunwind \\
  \rulecheckstack{\Sigma}{K}{\tau''}{\tau}
    \` Inversion on \presbinopunwind \\
  % Unpack exp
  \rulecheckexp{\Sigma}{\Gamma}{\binop{op}{e_1}{e_2}}{\tau'}
    \` Inversion on \presbinopunwind \\
  \rulecheckbinop{op}{\tau_1}{\tau_2}{\tau'} and \rulecheckexp{\Sigma}{\Gamma}{e_1}{\tau_1} and \rulecheckexp{\Sigma}{\Gamma}{e_2}{\tau_2} \\
    \` Inversion on \rulecheckexp{\Sigma}{\Gamma}{\binop{op}{e_1}{e_2}}{\tau'} \\
  % Unpack assign
  \rulecheckassign{\Gamma}{A}{e_1}{A'} and \rulecheckassign{\Gamma}{A'}{e_2}{A''}
    \` Inversion on \rulecheckassignbinopconclusion \\
  \rulecheckassign{\Gamma}{A'}{\cdot}{A'}
    \` \ruleref{\rulecheckassignholename} \\
  \rulecheckassign{\Gamma}{A'}{\binop{op}{\cdot}{e_2}}{A''}
    \` \ruleref{\rulecheckassignbinopname} \\
  % Unpack loop
  \rulecheckloop{L}{e_1} and \rulecheckloop{L}{e_2}
    \` Inversion on \rulegetloopcontext{L}{\binop{op}{e_1}{e_2}} \\
  \rulecheckloop{L}{\cdot}
    \` \ruleref{\rulecheckloopvarname} \\
  \rulecheckloop{L}{\binop{op}{\cdot}{e_2}}
    \` \ruleref{\rulecheckloopbinopname} \\
  % show inner frame
  If the rule used for \presbinopunwind{} was {\tt \rulecheckstatenormalname} then \+ \\
    \rulecheckframetype{\Sigma}{A''}{F}{\tau'}{\tau'''}
      \` Inversion on \presbinopunwind \\
    \impliescmdreturnsimple{\binop{op}{e_1}{e_2}} \\
      \` Inversion on \presbinopunwind \\
    \impliesab{\tau' = \commandty}{\ruleonlyreturns{\Sigma}{\Gamma}{e_1}{\tau''}} and
    \impliesab{\tau' = \commandty}{\ruleonlyreturns{\Sigma}{\Gamma}{e_2}{\tau''}} \\
      \` Inversion on \impliescmdreturnsimple{\binop{op}{e_1}{e_2}} \\
    If $\tau' = \commandty$ then $op = \opseq, \tau_1 = \commandty, $ and $\tau_2 = \commandty$ \\
      \` Inversion on \rulecheckbinop{op}{\tau_1}{\tau_2}{\tau'} \\
    $\tau' = \commandty \Leftrightarrow \tau_1 = \commandty \Leftrightarrow \tau_2 = \commandty \Leftrightarrow op = \opseq$ \\
      \` By case analysis of the rules for \rulecheckbinop{op}{\tau_1}{\tau_2}{\tau'} \\
    \pushtabs
    Thus \= \impliesab{\tau_1 = \commandty}{\ruleonlyreturns{\Sigma}{\Gamma}{e_1}{\tau''}} and \+ \\
            \impliesab{\tau_2 = \commandty}{\ruleonlyreturns{\Sigma}{\Gamma}{e_2}{\tau''}} \- \\
    \poptabs
    % show pending
    \ruleispending{\Sigma}{\Gamma}{\binop{op}{\cdot}{e_2}}{\tau_1}{\tau'}{\tau''}
      \` \ruleref{\ruleispendingbinoplname} \\
    \rulecheckframetype{\Sigma}{A'}{\presbinopF}{\tau_1}{\tau''}
      \` \ruleref{\rulecheckframetypeframeexpnoretname} \\
    \rulegetframecontext{\Sigma}{\presbinopF}{\Gamma}
      \` \ruleref{\rulegetframecontextexpname} \\
    \rulegetframeassigned{\presbinopF}{A}
      \` \ruleref{\rulegetframeassignedexpname} \\
    \rulegetloopcontext{\presbinopF}{L}
      \` \ruleref{\rulegetloopcontextexpname} \\
    % show direction ok
    \ruledirok{\rhd}{e_1}
      \` \ruleref{\ruledirokpushingname} \\
    % DONE!
    \rulecheckstate{\Sigma}{\stackframe{K}{\presbinopF}}{\tau}{\rhd}{e_1}
      \` \ruleref{\rulecheckstatenormalname} \- \\
  Otherwise the {\tt \rulecheckstatereturnsname} rule was used. \+ \\
    \rulereturns{\Sigma}{\Gamma}{\binop{op}{e_1}{e_2}}{\tau''} \\
      \` Inversion on \presbinopunwind{} \\
    \ruleonlyreturns{\Sigma}{\Gamma}{\binop{op}{e_1}{e_2}}{\tau''} and \ruledoesreturn{\binop{op}{e_1}{e_2}} and \\
      \hspace{1em} \ruletausmall{\tau''} \\
      \` Inversion on \rulereturns{\Sigma}{\Gamma}{\binop{op}{e_1}{e_2}}{\tau''} \\
    $L = \notinloop$
      \` Inversion on \presbinopunwind{} \\
    $\tau' = \commandty$
      \` Inversion on \presbinopunwind{} \\
    $\tau' = \commandty \Leftrightarrow \tau_1 = \commandty \Leftrightarrow \tau_2 = \commandty \Leftrightarrow op = \opseq$ \\
      \` By case analysis of the rules for \rulecheckbinop{op}{\tau_1}{\tau_2}{\tau'} \\
    Via case analysis on \ruledoesreturn{\binop{op}{e_1}{e_2}}, there are two possibilities. \\
    Case \ruledoesreturn{e_1} \+ \\
      \ruleonlyreturns{\Sigma}{\Gamma}{e_1}{\tau''} \\
        \` Inversion on \ruleonlyreturns{\Sigma}{\Gamma}{\binop{op}{e_1}{e_2}}{\tau''} \\
      \rulereturns{\Sigma}{\Gamma}{e_1}{\tau''}
        \` \ruleref{\rulereturnsdefaultname} \\
      \rulegetframecontext{\Sigma}{\presbinopF}{\Gamma}
        \` \ruleref{\rulegetframecontextexpname} \\
      \rulegetframeassigned{\presbinopF}{A}
        \` \ruleref{\rulegetframeassignedexpname} \\
      \rulegetloopcontext{\presbinopF}{\notinloop} \\
        \` \ruleref{\rulegetloopcontextexpname} \\
      \ruledirok{\rhd}{e_1}
        \` \ruleref{\ruledirokpushingname} \\
      \rulecheckstate{\Sigma}{\stackframe{K}{\presbinopF}}{\tau}{\rhd}{e_1}
        \` \ruleref{\rulecheckstatereturnsname} \- \\
    Case \ruledoesreturn{e_2} \+ \\
      \ruleonlyreturns{\Sigma}{\Gamma}{e_2}{\tau''} \\
        \` Inversion on \ruleonlyreturns{\Sigma}{\Gamma}{\binop{op}{e_1}{e_2}}{\tau''} \\
      \impliescmdreturn{\tau_2}{\Sigma}{\Gamma}{e_2}{\tau''}
        \` Weakening \\
      % show pending
      \ruleispending{\Sigma}{\Gamma}{\binop{op}{\cdot}{e_2}}{\tau_1}{\tau'}{\tau''}
        \` \ruleref{\ruleispendingbinoplname} \\
      \ruledoesreturn{\binop{op}{\cdot}{e_2}}
        \` \ruleref{\ruledoesreturnbinoprhsname} \\
      \rulereturns{\Sigma}{\Gamma}{\binop{op}{\cdot}{e_2}}{\tau''}
        \` \ruleref{\rulereturnsdefaultname} \\
      \rulecheckframetype{\Sigma}{A'}{\presbinopF}{\tau_1}{\tau''} \\
        \` \ruleref{\rulecheckframetypeframeexpretname} \\
      \rulegetframecontext{\Sigma}{\presbinopF}{\Gamma}
        \` \ruleref{\rulegetframecontextexpname} \\
      \rulegetframeassigned{\presbinopF}{A}
        \` \ruleref{\rulegetframeassignedexpname} \\
      \rulegetloopcontext{\presbinopF}{L}
        \` \ruleref{\rulegetloopcontextexpname} \\
      % show direction ok
      \ruledirok{\rhd}{e_1}
        \` \ruleref{\ruledirokpushingname} \\
      % DONE!
      \rulecheckstate{\Sigma}{\stackframe{K}{\presbinopF}}{\tau}{\rhd}{e_1}
        \` \ruleref{\rulecheckstatenormalname} \- \\
  \end{tabbing}

\item[\rulestepswapbinop] \ \\
  \newcommand{\swapbinopstackpre}{\frameexp{F}{\binop{op}{\cdot}{e_2}}}
  \newcommand{\swapbinopstack}{\frameexp{F}{\binop{op}{v}{\cdot}}}
  \begin{tabbing}
  \hspace{3em} \= \hspace{3em} \= \kill
  \rulegetframecontext{\Sigma}{\swapbinopstackpre}{\Gamma}
    \` Inversion on \rulecheckstate{\Sigma}{\stackframe{K}{\swapbinopstackpre}}{\tau}{\lhd}{v} \\
  \rulecheckexp{\Sigma}{\Gamma}{v}{\tau_1}
    \` Inversion on \rulecheckstate{\Sigma}{\stackframe{K}{\swapbinopstackpre}}{\tau}{\lhd}{v} \\
  \rulegetframeassigned{\swapbinopstackpre}{A}
    \` Inversion on \rulecheckstate{\Sigma}{\stackframe{K}{\swapbinopstackpre}}{\tau}{\lhd}{v} \\
  \rulecheckassign{\Gamma}{A}{v}{A}
    \` Inversion on \rulecheckstate{\Sigma}{\stackframe{K}{\swapbinopstackpre}}{\tau}{\lhd}{v} \\
  \rulegetloopcontext{\swapbinopstackpre}{L}
    \` Inversion on \rulecheckstate{\Sigma}{\stackframe{K}{\swapbinopstackpre}}{\tau}{\lhd}{v} \\
  \rulecheckloop{L}{v}
    \` Inversion on \rulecheckstate{\Sigma}{\stackframe{K}{\swapbinopstackpre}}{\tau}{\lhd}{v} \\
  \rulecheckframetype{\Sigma}{A}{\swapbinopstackpre}{\tau_1}{\tau''}
    \` Inversion on \rulecheckstate{\Sigma}{\stackframe{K}{\swapbinopstackpre}}{\tau}{\lhd}{v} \\
  \rulecheckstack{\Sigma}{K}{\tau''}{\tau}
    \` Inversion on \rulecheckstate{\Sigma}{\stackframe{K}{\swapbinopstackpre}}{\tau}{\lhd}{v} \\
  \\
  \rulegetframecontext{\Sigma}{F}{\Gamma}
    \` Inversion on \rulecheckframetype{\Sigma}{A}{\swapbinopstackpre}{\tau_1}{\tau''} \\
  \rulecheckassign{\Gamma}{A}{\binop{op}{\cdot}{e_2}}{A'}
    \` Inversion on \rulecheckframetype{\Sigma}{A}{\swapbinopstackpre}{\tau_1}{\tau''} \\
  \ruleispending{\Sigma}{\Gamma}{\binop{op}{\cdot}{e_2}}{\tau_1}{\tau'}{\tau''} \\
    \` Inversion on \rulecheckframetype{\Sigma}{A}{\swapbinopstackpre}{\tau_1}{\tau''} \\
  \rulecheckbinop{op}{\tau_1}{\tau_2}{\tau'}
    \` Inversion on \rulecheckframetype{\Sigma}{A}{\swapbinopstackpre}{\tau_1}{\tau''} \\
  \impliescmdreturn{\tau_2}{\Sigma}{\Gamma}{e_2}{\tau''} \\
    \` Inversion on \ruleispending{\Sigma}{\Gamma}{\binop{op}{\cdot}{e_2}}{\tau_1}{\tau'}{\tau''} \\
  \rulecheckexp{\Sigma}{\Gamma}{e_2}{\tau_2}
    \` Inversion on \ruleispending{\Sigma}{\Gamma}{\binop{op}{\cdot}{e_2}}{\tau_1}{\tau'}{\tau''} \\
  \rulecheckassign{\Gamma}{A}{e_2}{A'}
    \` Inversion on \rulecheckassign{\Gamma}{A}{\binop{op}{\cdot}{e_2}}{A} \\
  \\
  There are two possible rules used for the derivation of \rulecheckframetype{\Sigma}{A}{\swapbinopstackpre}{\tau_1}{\tau''} \\
  Case \rulefmt{\rulecheckframetypeframeexpnoretname} \+ \\
    \rulecheckframetype{\Sigma}{A'}{F}{\tau'}{\tau''}
      \` Inversion on \rulecheckframetype{\Sigma}{A}{\swapbinopstackpre}{\tau_1}{\tau''} \\
    \rulecheckassign{\Gamma}{A'}{\cdot}{A'}
      \` \ruleref{\rulecheckassignholename} \\
    \rulecheckassign{\Gamma}{A'}{v}{A'}
      \` \ruleref{\rulecheckassignvaluename} \\
    \rulecheckassign{\Gamma}{A'}{\binop{op}{v}{\cdot}}{A'}
      \` \ruleref{\rulecheckassignbinopname} \\
    \rulegetloopcontext{F}{L}
      \` Inversion on \rulecheckframetype{\Sigma}{A}{\swapbinopstackpre}{\tau_1}{\tau''} \\
    \rulecheckloop{L}{\cdot}
      \` \ruleref{\rulecheckloopvarname} \\
    \rulecheckloop{L}{\binop{op}{v}{\cdot}}
      \` \ruleref{\rulecheckloopbinopname} \\
    \rulecheckval{\Sigma}{v}{\tau_1}
      \` Inversion on \rulecheckexp{\Sigma}{\Gamma}{v}{\tau_1} \\
    \ruleispending{\Sigma}{\Gamma}{\binop{op}{v}{\cdot}}{\tau_2}{\tau'}{\tau''}
      \` \ruleref{\ruleispendingbinoprname} \\
    \rulecheckframetype{\Sigma}{A'}{\swapbinopstack}{\tau_2}{\tau''}
      \` \ruleref{\rulecheckframetypeframeexpnoretname} \\
    \\
    \rulegetframecontext{\Sigma}{\swapbinopstack}{\Gamma}
      \` \ruleref{\rulegetframecontextexpname} \\
    \rulegetframeassigned{F}{A}
      \` Inversion on \rulegetframeassigned{\swapbinopstackpre}{A} \\
    \rulegetframeassigned{\swapbinopstack}{A}
      \` \ruleref{\rulegetframeassignedexpname} \\
    \rulegetloopcontext{\swapbinopstack}{L}
      \` \ruleref{\rulegetloopcontextexpname} \\
    \rulecheckloop{L}{\binop{op}{\cdot}{e_2}}
      \` Inversion on \rulecheckframetype{\Sigma}{A}{\swapbinopstackpre}{\tau_1}{\tau''} \\
    \rulecheckloop{L}{e_2}
      \` Inversion on \rulecheckloop{L}{\binop{op}{\cdot}{e_2}} \\
    \ruledirok{\rhd}{e_2}
      \` \ruleref{\ruledirokpushingname} \\
    \rulecheckstate{\Sigma}{\stackframe{K}{\swapbinopstack}}{\tau}{\rhd}{e_2}
      \` \ruleref{\rulecheckstatenormalname} \- \\
  Otherwise \rulefmt{\rulecheckframetypeframeexpretname} was used \+ \\
    \ruledoesreturn{\binop{op}{\cdot}{e_2}}
      \` Inversion on \rulecheckframetype{\Sigma}{A}{\swapbinopstackpre}{\tau_1}{\tau''} \\
    \rulegetloopcontext{F}{\notinloop}
      \` Inversion on \rulecheckframetype{\Sigma}{A}{\swapbinopstackpre}{\tau_1}{\tau''} \\
    \rulecheckloop{\notinloop}{\binop{op}{\cdot}{e_2}} \\
      \` Inversion on \rulecheckframetype{\Sigma}{A}{\swapbinopstackpre}{\tau_1}{\tau''} \\
    $\tau' = \commandty$
      \` Inversion on \rulecheckframetype{\Sigma}{A}{\swapbinopstackpre}{\tau_1}{\tau''} \\
    $\tau' = \commandty \Leftrightarrow \tau_2 = \commandty \Leftrightarrow op = \opseq$ \\
    \rulegetframecontext{\Sigma}{\swapbinopstack}{\Gamma}
      \` \ruleref{\rulegetframecontextexpname} \\
    \rulegetframeassigned{\swapbinopstack}{A}
      \` \ruleref{\rulegetframeassignedexpname} \\
    \rulegetloopcontext{\swapbinopstack}{\notinloop}
      \` \ruleref{\rulegetloopcontextexpname} \\
    \rulecheckloop{\notinloop}{e_2}
      \` Inversion on \rulecheckloop{\notinloop}{\binop{op}{\cdot}{e_2}} \\
    \ruledirok{\rhd}{e_2}
      \` \ruleref{\ruledirokpushingname} \\
    \ruleonlyreturns{\Sigma}{\Gamma}{e_2}{\tau''}
      \` Modus ponens \\
    \rulereturns{\Sigma}{\Gamma}{e_2}{\tau''}
      \` \ruleref{\rulereturnsonlyname} \\
    \rulecheckstate{\Sigma}{\stackframe{K}{\swapbinopstack}}{\tau}{\rhd}{e_2}
      \` \ruleref{\rulecheckstatereturnsname} \\
  \end{tabbing}


\item[\rulesteppopbinop] \ \\
\newcommand{\popbinopstack}{\frameexp{F}{\binop{op}{v_1}{\cdot}}}
\newcommand{\presbinoppop}{\rulecheckstate{\Sigma}{\stackframe{K}{\popbinopstack}}{\tau}{\rhd}{v_2}}
\begin{tabbing}
\rulegetframecontext{\Sigma}{\popbinopstack}{\Gamma}
  \` Inversion on \presbinoppop \\
\rulecheckexp{\Sigma}{\Gamma}{v_2}{\tau_2}
  \` Inversion on \presbinoppop \\
\rulegetframeassigned{\popbinopstack}{A}
  \` Inversion on \presbinoppop \\
\rulecheckassign{\Gamma}{A}{v_2}{A}
  \` Inversion on \presbinoppop \\
\rulegetloopcontext{\binop{op}{v_1}{\cdot}}{L}
  \` Inversion on \presbinoppop \\
\rulecheckstack{\Sigma}{K}{\tau''}{\tau}
  \` Inversion on \presbinoppop \\
\rulecheckframetype{\Sigma}{A}{\popbinopstack}{\tau_2}{\tau'}
  \` Inversion on \presbinoppop \\
\\
\rulegetframecontext{\Sigma}{F}{\Gamma}
  \` Inversion on \rulecheckframetype{\Sigma}{A}{\popbinopstack}{\tau_2}{\tau'} \\
\rulegetframeassigned{F}{A}
  \` Inversion on \rulecheckframetype{\Sigma}{A}{\popbinopstack}{\tau_2}{\tau'} \\
\rulecheckassign{\Gamma}{A}{\binop{op}{v}{\cdot}}{A}
  \` Inversion on \rulecheckframetype{\Sigma}{A}{\popbinopstack}{\tau_2}{\tau'} \\
\rulegetloopcontext{F}{L}
  \` Inversion on \rulecheckframetype{\Sigma}{A}{\popbinopstack}{\tau_2}{\tau'} \\
\rulecheckframetype{\Sigma}{A}{F}{\tau'}{\tau''}
  \` Inversion on \rulecheckframetype{\Sigma}{A}{\popbinopstack}{\tau_2}{\tau'} \\
\ruleispending{\Sigma}{\Gamma}{\binop{op}{v_1}{\cdot}}{\tau_2}{\tau'}{\tau''} \\
  \` Inversion on \rulecheckframetype{\Sigma}{A}{\popbinopstack}{\tau_2}{\tau'} \\
\rulecheckexp{\Sigma}{\Gamma}{v}{\tau_1} and \rulecheckbinop{op}{\tau_1}{\tau_2}{\tau'} \\
  \` Inversion on \ruleispending{\Sigma}{\Gamma}{\binop{op}{v_1}{\cdot}}{\tau_2}{\tau'}{\tau''} \\
\ruleevalbinopdefaultconclusion{}
  \` Inversion on \rulesteppopbinopconclusion{} \\
$\mu' : \Sigma$ and either $t = \exn{}$ or else $t = v'$ and \rulecheckval{\Sigma}{v'}{\tau'}
  \` Lemma \ref{binoppres} \\
If $t = \exn{}$ then \rulecheckstateexnconclusion{}
  \` \ruleref{\rulecheckstateexnname} \\
Otherwise, $t = v'$ and \rulecheckval{\Sigma}{v'}{\tau'} \\
\rulecheckexp{\Sigma}{\Gamma}{v'}{\tau'}
  \` \ruleref{\rulecheckexpvaluename} \\
\rulecheckassign{\Gamma}{A}{v'}{A}
  \` \ruleref{\rulecheckassignvaluename} \\
\rulecheckloop{L}{v'}
  \` \ruleref{\rulecheckloopvaluename} \\
\ruledirokreturningconclusion{}
  \` \ruleref{\ruledirokreturningname} \\
If $\tau' = \commandty$ then $v' = \nop$
  \` Lemma \ref{cfl} \\
\impliescmdreturn{\tau'}{\Sigma}{\Gamma}{v'}{\tau''}
  \` \ruleonlyreturns{\Sigma}{\Gamma}{\nop}{\tau''} \\
\rulecheckstate{\Sigma}{\stackframe{K}{F}}{\tau}{\lhd}{v'}
  \` \ruleref{\rulecheckstatenormalname} \\

\end{tabbing}

\item[\rulesteppushmonop] \ \\
  \newcommand{\monopstack}{\frameexp{F}{\monop{op}{\cdot}}}
  \begin{tabbing}
  \rulegetframecontext{\Sigma}{F}{\Gamma}
    \` Inversion on \rulecheckstate{\Sigma}{\stackframe{K}{F}}{\tau}{\rhd}{\monop{op}{e}} \\
  \rulecheckexp{\Sigma}{\Gamma}{\monop{op}{e}}{\tau'''}
    \` Inversion on \rulecheckstate{\Sigma}{\stackframe{K}{F}}{\tau}{\rhd}{\monop{op}{e}} \\
  \rulegetframeassigned{F}{A}
    \` Inversion on \rulecheckstate{\Sigma}{\stackframe{K}{F}}{\tau}{\rhd}{\monop{op}{e}} \\
  \rulecheckassign{\Gamma}{A}{\monop{op}{e}}{A}
    \` Inversion on \rulecheckstate{\Sigma}{\stackframe{K}{F}}{\tau}{\rhd}{\monop{op}{e}} \\
  \rulegetloopcontext{F}{L}
    \` Inversion on \rulecheckstate{\Sigma}{\stackframe{K}{F}}{\tau}{\rhd}{\monop{op}{e}} \\
  \rulecheckloop{L}{\monop{op}{e}}
    \` Inversion on \rulecheckstate{\Sigma}{\stackframe{K}{F}}{\tau}{\rhd}{\monop{op}{e}} \\
  \rulecheckframetype{\Sigma}{A}{F}{\tau'''}{\tau''}
    \` Inversion on \rulecheckstate{\Sigma}{\stackframe{K}{F}}{\tau}{\rhd}{\monop{op}{e}} \\
  \rulecheckstack{\Sigma}{K}{\tau''}{\tau}
    \` Inversion on \rulecheckstate{\Sigma}{\stackframe{K}{F}}{\tau}{\rhd}{\monop{op}{e}} \\
  \\
  \rulecheckassign{\Gamma}{A}{\cdot}{A}
    \` \ruleref{\rulecheckassignholename} \\
  \rulecheckassign{\Gamma}{A}{\monop{op}{\cdot}}{A}
    \` \ruleref{\rulecheckassignmonopname} \\
  \rulecheckloop{L}{\cdot}
    \` \ruleref{\rulecheckloopvarname} \\
  \rulecheckloop{L}{\monop{op}{\cdot}}
    \` \ruleref{\rulecheckloopmonopname} \\
  \rulecheckmonop{op}{\tau'}{\tau'''}
    \` Inversion on \rulecheckexp{\Sigma}{\Gamma}{\monop{op}{e}}{\tau'''} \\
  \ruleispending{\Sigma}{\Gamma}{\monop{op}{\cdot}}{\tau'}{\tau'''}{\tau''}
    \` \ruleref{\ruleispendingmonopname} \\
  \rulecheckframetype{\Sigma}{A}{\monopstack}{\tau'}{\tau'''}
    \` \ruleref{\rulecheckframetypeframeexpnoretname} \\
  \\
  \rulegetframecontext{\Sigma}{\monopstack}{\Gamma}
    \` \ruleref{\rulegetframecontextexpname} \\
  \rulecheckexp{\Sigma}{\Gamma}{e}{\tau'}
    \` Inversion on \rulecheckexp{\Sigma}{\Gamma}{\monop{op}{e}}{\tau'''} \\
  \rulegetframeassigned{\monopstack}{A}
    \` \ruleref{\rulegetframeassignedexpname} \\
  \rulecheckassign{\Gamma}{A}{e}{A}
    \` Inversion on \rulecheckassign{\Gamma}{A}{\monop{op}{e}}{A} \\
  \rulecheckloop{L}{e}
    \` Inversion on \rulecheckloop{L}{\monop{op}{e}} \\
  \ruledirok{\rhd}{e}
    \` \ruleref{\ruledirokpushingname} \\
  $\tau' \not= \commandty$
    \` Inversion on \rulecheckmonop{op}{\tau'}{\tau'''} \\
  \impliescmdreturnsimple{\tau'}
    \` Vacuously true \\
  \rulecheckstate{\Sigma}{\stackframe{K}{\monopstack}}{\tau}{\rhd}{e}
    \` \ruleref{\rulecheckstatenormalname} \\

  \end{tabbing}

\item[\rulesteppopmonop] \ \\
  \begin{tabbing}
  \hspace{3em} \= \hspace{3em} \= \kill
  \rulegetframecontext{\Sigma}{\monopstack}{\Gamma}
    \` Inversion on \rulecheckstate{\Sigma}{\stackframe{K}{\monopstack}}{\tau}{\lhd}{v} \\
  \rulecheckexp{\Sigma}{\Gamma}{v}{\tau'}
    \` Inversion on \rulecheckstate{\Sigma}{\stackframe{K}{\monopstack}}{\tau}{\lhd}{v} \\
  \rulegetframeassigned{\monopstack}{A}
    \` Inversion on \rulecheckstate{\Sigma}{\stackframe{K}{\monopstack}}{\tau}{\lhd}{v} \\
  \rulecheckassign{\Gamma}{A}{v}{A}
    \` Inversion on \rulecheckstate{\Sigma}{\stackframe{K}{\monopstack}}{\tau}{\lhd}{v} \\
  \rulecheckframetype{\Sigma}{A}{\monopstack}{\tau'}{\tau''}
    \` Inversion on \rulecheckstate{\Sigma}{\stackframe{K}{\monopstack}}{\tau}{\lhd}{v} \\
  \rulecheckstack{\Sigma}{K}{\tau''}{\tau}
    \` Inversion on \rulecheckstate{\Sigma}{\stackframe{K}{\monopstack}}{\tau}{\lhd}{v} \\
  \\
  \rulegetframecontext{\Sigma}{F}{\Gamma}
    \` Inversion on \rulegetframecontext{\Sigma}{\monopstack}{\Gamma} \\
  \rulecheckassign{\Gamma}{A}{\monop{op}{\cdot}}{A}
    \` Inversion on \rulecheckframetype{\Sigma}{A}{\monopstack}{\tau'}{\tau''} \\
  \ruleispending{\Sigma}{\Gamma}{\monop{op}{\cdot}}{\tau'}{\tau'''}{\tau''} \\
    \` Inversion on \rulecheckframetype{\Sigma}{A}{\monopstack}{\tau'}{\tau''} \\
  \rulecheckframetype{\Sigma}{A}{F}{\tau'''}{\tau''}
    \` Inversion on \rulecheckframetype{\Sigma}{A}{\monopstack}{\tau'}{\tau''} \\
  \rulegetloopcontext{F}{L}
    \` Inversion on \rulecheckframetype{\Sigma}{A}{\monopstack}{\tau'}{\tau''} \\
  \rulecheckmonop{op}{\tau'}{\tau'''}
    \` Inversion on \ruleispending{\Sigma}{\Gamma}{\monop{op}{\cdot}}{\tau'}{\tau'''}{\tau''} \\
  Either $t = \exn{}$ or $t = v'$ where \rulecheckval{\Sigma}{v'}{\tau'''}
    \` Lemma \ref{monoppres} \\
  If $t = \exn{}$ \+ \\
    Trivially \rulecheckstateexnconclusion{} \- \\
  Otherwise $t = v'$ where \rulecheckval{\Sigma}{v'}{\tau'''} \+ \\
    \rulecheckexp{\Sigma}{\Gamma}{v'}{\tau'''}
      \` \ruleref{\rulecheckexpvaluename} \\
    \rulegetframeassigned{F}{A}
      \` Inversion on \rulegetframeassigned{\monopstack}{A} \\
    \rulecheckassign{\Gamma}{A}{v'}{A}
      \` \ruleref{\rulecheckassignvaluename} \\
    \rulecheckloop{L}{v'}
      \` \ruleref{\rulecheckloopvaluename} \\
    \ruledirok{\lhd}{v}
      \` \ruleref{\ruledirokreturningname} \\
    If $\tau''' = \commandty$ then $v' = \nop$
      \` Lemma \ref{cfl} \\
    \impliescmdreturn{\tau'''}{\Sigma}{\Gamma}{v}{\tau''}
      \` \ruleonlyreturns{\Sigma}{\Gamma}{\nop}{\tau''} \\
    \rulecheckstate{\Sigma}{\stackframe{K}{F}}{\tau}{\lhd}{v'}
      \` \ruleref{\rulecheckstatenormalname} \- \\
  \end{tabbing}

\item[\rulesteppushcallfn] \ \\

  \newcommand{\pushcallfnstack}{\frameexp{F}{\call{\cdot,e}}}
  \begin{tabbing}
  \rulegetframecontext{\Sigma}{F}{\Gamma}
    \` Inversion on \rulecheckstate{\Sigma}{\stackframe{K}{F}}{\tau}{\rhd}{\call{e_f,e}} \\
  \rulecheckexp{\Sigma}{\Gamma}{\call{e_f,e}}{\tau_r}
    \` Inversion on \rulecheckstate{\Sigma}{\stackframe{K}{F}}{\tau}{\rhd}{\call{e_f,e}} \\
  \rulegetframeassigned{F}{A}
    \` Inversion on \rulecheckstate{\Sigma}{\stackframe{K}{F}}{\tau}{\rhd}{\call{e_f,e}} \\
  \rulecheckassign{\Gamma}{A}{\call{e_f,e}}{A}
    \` Inversion on \rulecheckstate{\Sigma}{\stackframe{K}{F}}{\tau}{\rhd}{\call{e_f,e}} \\
  \rulegetloopcontext{F}{L}
    \` Inversion on \rulecheckstate{\Sigma}{\stackframe{K}{F}}{\tau}{\rhd}{\call{e_f,e}} \\
  \rulecheckloop{L}{\call{e_f,e}}
    \` Inversion on \rulecheckstate{\Sigma}{\stackframe{K}{F}}{\tau}{\rhd}{\call{e_f,e}} \\
  \rulecheckframetype{\Sigma}{A}{F}{\tau_r}{\tau''}
    \` Inversion on \rulecheckstate{\Sigma}{\stackframe{K}{F}}{\tau}{\rhd}{\call{e_f,e}} \\
  \rulecheckstack{\Sigma}{K}{\tau''}{\tau}
    \` Inversion on \rulecheckstate{\Sigma}{\stackframe{K}{F}}{\tau}{\rhd}{\call{e_f,e}} \\
  \\
  \rulecheckassign{\Gamma}{A}{e}{A}
    \` Inversion on \rulecheckassign{\Gamma}{A}{\call{e_f,e}}{A} \\
  \rulecheckassign{\Gamma}{A}{\cdot}{A}
    \` \ruleref{\rulecheckassignholename} \\
  \rulecheckassign{\Gamma}{A}{\call{\cdot,e}}{A}
    \` \ruleref{\rulecheckassigncallname} \\

  \rulecheckloop{L}{e}
    \` Inversion on \rulecheckloop{L}{\call{e_f,e}} \\
  \rulecheckloop{L}{\cdot}
    \` \ruleref{\rulecheckloopvarname} \\
  \rulecheckloop{L}{\call{\cdot,e}}
    \` \ruleref{\rulecheckloopcallname} \\

  \rulecheckexp{\Sigma}{\Gamma}{e}{\tuple{\tau_1, \ldots, \tau_n}}
    \` Inversion on \rulecheckexp{\Sigma}{\Gamma}{\call{e_f,e}}{\tau_r} \\
  \ruleispending{\Sigma}{\Gamma}{\call{\cdot,e}}{\pointer{(\function{\tuple{\tau_1, \ldots, \tau_n}}{\tau_r})}}{\tau_r}{\tau''} \\
    \` \ruleref{\ruleispendingcallfname} \\

  \rulecheckframetype{\Sigma}{A}{\pushcallfnstack}{\pointer{(\function{\tuple{\tau_1, \ldots, \tau_n}}{\tau_r})}}{\tau''} \\
    \` \ruleref{\rulecheckframetypeframeexpnoretname} \\
  \\
  \rulegetframecontext{\Sigma}{\pushcallfnstack}{\Gamma}
    \` \ruleref{\rulegetframecontextexpname} \\
  \rulecheckexp{\Sigma}{\Gamma}{e_f}{\pointer{(\function{\tuple{\tau_1, \ldots, \tau_n}}{\tau_r})}}
    \` Inversion on \rulecheckexp{\Sigma}{\Gamma}{\call{e_f,e}}{\tau_r} \\
  \rulegetframeassigned{\pushcallfnstack}{A}
    \` \ruleref{\rulegetframeassignedexpname} \\
  \rulecheckassign{\Gamma}{A}{e_f}{A}
    \` Inversion on \rulecheckassign{\Gamma}{A}{\call{e_f,e}}{A} \\
  \rulegetloopcontext{\pushcallfnstack}{L}
    \` \ruleref{\rulegetloopcontextexpname} \\
  \rulecheckloop{L}{e_f}
    \` Inversion on \rulecheckloop{L}{\call{e_f}{e}} \\
  \ruledirok{\rhd}{e_f}
    \` \ruleref{\ruledirokpushingname} \\
  \impliescmdreturn{\pointer{(\function{\tuple{\tau_1, \ldots, \tau_n}}{\tau_r})}}{\Sigma}{\Gamma}{e_f}{\tau''}
    \` Vacuously true \\
  \rulecheckstate{\Sigma}{\stackframe{K}{\pushcallfnstack}}{\tau}{\rhd}{e_f}
    \` \ruleref{\rulecheckstatenormalname} \\
  \end{tabbing}

\item[\rulesteppushcallargs] \ \\

  \newcommand{\pushargsstack}{\frameexp{F}{\call{\cdot,e}}}
  \newcommand{\pushargsstackpost}{\frameexp{F}{\call{v,\cdot}}}
  \begin{tabbing}
  \rulegetframecontext{\Sigma}{\pushargsstack}{\Gamma}
    \` Inversion on \rulecheckstate{\Sigma}{\stackframe{K}{\pushargsstack}}{\tau}{\lhd}{v} \\
  \rulegetframeassigned{\pushargsstack}{A}
    \` Inversion on \rulecheckstate{\Sigma}{\stackframe{K}{\pushargsstack}}{\tau}{\lhd}{v} \\
  \rulecheckassign{\Gamma}{A}{v}{A}
    \` Inversion on \rulecheckstate{\Sigma}{\stackframe{K}{\pushargsstack}}{\tau}{\lhd}{v} \\
  \rulecheckframetype{\Sigma}{A}{\pushargsstack}{\pointer{(\function{\tau_a}{\tau_r})}}{\tau_r}
    \` Inversion on \rulecheckstate{\Sigma}{\stackframe{K}{\pushargsstack}}{\tau}{\lhd}{v} \\
  \rulecheckstack{\Sigma}{K}{\tau''}{\tau}
    \` Inversion on \rulecheckstate{\Sigma}{\stackframe{K}{\pushargsstack}}{\tau}{\lhd}{v} \\
  \rulegetframecontext{\Sigma}{F}{\Gamma}
    \` Inversion on \rulecheckframetype{\Sigma}{A}{\pushargsstack}{\pointer{(\function{\tau_a}{\tau_r})}}{\tau_r} \\
  \ruleispending{\Sigma}{\Gamma}{\call{\cdot,e}}{\pointer{(\function{\tau_a}{\tau_r})}}{\tau_r}{\tau''} \\
    \` Inversion on \rulecheckframetype{\Sigma}{A}{\pushargsstack}{\pointer{(\function{\tau_a}{\tau_r})}}{\tau_r} \\
  \rulecheckexp{\Sigma}{\Gamma}{v}{\pointer{(\function{\tau_a}{\tau_r})}}
    \` Inversion on \rulecheckstate{\Sigma}{\stackframe{K}{\pushargsstack}}{\tau}{\lhd}{v} \\
  $v = \address{a}$ where $\Sigma(a) = \function{\tau_a}{\tau_r}$
    \` Lemma \ref{cfl} \\
  \rulecheckexp{\Sigma}{\Gamma}{\address{a}}{\pointer{(\function{\tau_a}{\tau_r})}}
    \` \ruleref{\rulecheckexpvarname} \\
  \ruleispending{\Sigma}{\Gamma}{\call{\address{a}, \cdot}}{\tau_a}{\tau_r}{\tau''}
    \` \ruleref{\ruleispendingcallaname} \\
  \rulegetloopcontext{F}{L}
    \` Inversion on \rulecheckframetype{\Sigma}{A}{\pushargsstack}{\pointer{(\function{\tau_a}{\tau_r})}}{\tau_r} \\
  \rulecheckloop{L}{\call{v,\cdot}}
    \` \ruleref{\rulecheckloopcallname} \\
  \rulecheckassign{\Gamma}{A}{v}{A}
    \` \ruleref{\rulecheckassignvaluename} \\
  \rulecheckassign{\Gamma}{A}{\cdot}{A}
    \` \ruleref{\rulecheckassignholename} \\
  \rulecheckassign{\Gamma}{A}{\call{v,\cdot}}{A}
    \` \ruleref{\rulecheckassigncallname} \\

  \rulecheckframetype{\Sigma}{A}{\pushargsstackpost}{\tau_a}{\tau''}
    \` \ruleref{\rulecheckframetypeframeexpnoretname} \\

  \rulegetframecontext{\Sigma}{\pushargsstackpost}{\Gamma}
    \` \ruleref{\rulegetframecontextexpname} \\
  \rulecheckexp{\Sigma}{\Gamma}{e}{\tau_a}
    \` Inversion on \ruleispending{\Sigma}{\Gamma}{\call{\cdot,e}}{\pointer{(\function{\tau_a}{\tau_r})}}{\tau_r}{\tau''} \\
  \rulegetframeassigned{F}{A}
    \` Inversion on \rulegetframeassigned{\pushargsstack}{A} \\
  \rulegetframeassigned{\pushargsstackpost}{A}
    \` \ruleref{\rulegetframeassignedexpname} \\
  \rulecheckassign{\Gamma}{A}{\call{\cdot,e}}{A}
    \` Inversion on \rulecheckframetype{\Sigma}{A}{\pushargsstack}{\pointer{(\function{\tau_a}{\tau_r})}}{\tau_r} \\
  \rulecheckassign{\Gamma}{A}{e}{A}
    \` Inversion on \rulecheckassign{\Gamma}{A}{\call{\cdot,e}}{A} \\
  \rulegetloopcontext{\pushargsstackpost}{L}
    \` \ruleref{\rulegetloopcontextexpname} \\
  \rulecheckloop{L}{\call{\cdot,e}}
    \` Inversion on \rulecheckframetype{\Sigma}{A}{\pushargsstack}{\pointer{(\function{\tau_a}{\tau_r})}}{\tau_r} \\
  \rulecheckloop{L}{e}
    \` Inversion on \rulecheckloop{L}{\call{\cdot,e}} \\
  \ruledirok{\rhd}{e}
    \` \ruleref{\ruledirokpushingname} \\

  $\mu(a) = v_f$ where \rulecheckval{\Sigma}{v_f}{\function{\tau_a}{\tau_r}}
    \` $\mu : \Sigma$ \\
  $v_f = \func{x_1, \ldots, x_n, e'}$
    \` Lemma \ref{cfl} \\
  $\tau_a = \tuple{\tau_1, \ldots, \tau_n}$
    \` Inversion on \rulecheckval{\Sigma}{v_f}{\function{\tau_a}{\tau_r}} \\
  \impliescmdreturn{\tau_a}{\Sigma}{\Gamma}{e}{\tau''}
    \` Vacuously true \\

  \rulecheckstate{\Sigma}{\stackframe{K}{\pushargsstackpost}}{\tau}{\rhd}{e}
    \` \ruleref{\rulecheckstatenormalname} \\
%  \rulecheckexp{\Sigma}{\Gamma}{v}{\pointer{(\function{\tau_a}{\tau_r})}}
%    \` Inversion on \rulecheckstate{\Sigma}{\stackframe{K}{\pushargsstack}}{\tau}{\lhd}{v} \\
  \end{tabbing}

\item[\rulestepfinalizecall] \ \\
\newcommand{\newframe}{\frameval{\frameexp{\frameval{\cdot}{x_1}{\tau_1}{v_1}}{\ldots}}{x_n}{\tau_n}{v_n}}
\newcommand{\presfcallunwind}{\rulecheckstate{\Sigma}{\stackframe{K}{\frameexp{F}{\call{\address{a}, \cdot}}}}{\tau}{\lhd}{\tupleexp{v_1, \ldots, v_n}}}

\begin{tabbing}
\hspace{3em} \= \hspace{3em} \= \kill
\rulecheckexp{\Sigma}{\Gamma}{\tupleexp{v_1, \ldots, v_n}}{\tuple{\tau_1, \ldots, \tau_n}} \\
  \` Inversion on \presfcallunwind \\
\rulegetframeassigned{\frameexp{F}{\call{\address{a}, \cdot}}}{A} \\
  \` Inversion on \presfcallunwind \\
\rulecheckassign{\Gamma}{A}{\tupleexp{v_1, \ldots, v_n}}{A} \` Inversion on \presfcallunwind \\
\rulecheckstack{\Sigma}{K}{\tau''}{\tau}\` Inversion on \presfcallunwind \\
% Show call frame
\rulecheckframetype{\Sigma}{A}{\frameexp{F}{\call{\address{a}, \cdot}}}{\tuple{\tau_1, \ldots, \tau_n}}{\tau''} \\
  \` Inversion on \presfcallunwind \\
% Show call doesn't assign
\rulecheckassign{\Gamma}{A}{\call{\address{a}, \cdot}}{A} \\
  \` Inversion on \rulecheckframetype{\Sigma}{A}{\frameexp{F}{\call{\address{a}, \cdot}}}{\tuple{\tau_1, \ldots, \tau_n}}{\tau''} \\
% Unpack pending
\ruleispending{\Sigma}{\Gamma}{\call{\address{a}, \cdot}}{\tuple{\tau_1, \ldots, \tau_n}}{\tau_r}{\tau''} \\
\` Inversion on \rulecheckframetype{\Sigma}{A}{\frameexp{F}{\call{\address{a}, \cdot}}}{\tuple{\tau_1, \ldots, \tau_n}}{\tau''} \\
% Show type of call
\rulecheckexp{\Sigma}{\Gamma}{\address{a}}{\pointer{\function{\tuple{\tau_1, \ldots, \tau_n}}{\tau_r}}} \\
\` Inversion on \ruleispending{\Sigma}{\Gamma}{\call{\address{a}, \cdot}}{\tuple{\tau_1, \ldots, \tau_n}}{\tau_r}{\tau''} \\
% show function value is well typed
\rulecheckval{\Sigma}{\func{x_1, \ldots, x_n, e}}{\function{\tuple{\tau_1, \ldots \tau_n}}{\tau_r}} \\
  \` Inversion on \rulecheckexp{\Sigma}{\Gamma}{\address{a}}{\pointer{\function{\tuple{\tau_1, \ldots, \tau_n}}{\tau_r}}} \\
% Show F
\rulecheckframetype{\Sigma}{A}{F}{\tau_r}{\tau''} \` Inversion on \rulecheckframetype{\Sigma}{A}{\frameexp{F}{\call{\address{a}, \cdot}}}{\tuple{\tau_1, \ldots, \tau_n}}{\tau''} \\
% Show assigned(F) = A
\rulegetframeassigned{F}{A}  \` Inversion on \rulegetframeassigned{\frameexp{F}{\call{\address{a}, \cdot}}}{A} \\
% Show K;F
\rulecheckstack{\Sigma}{\stackframe{K}{F}}{\tau_r}{\tau} \` \ruleref{\rulecheckstacknonemptyname} \\
% Show context in new frame
\rulegetframecontext{\Sigma}{\newframe}{\Gamma'} \` Lemma \ref{constructctx} \\
% Show assigned in new frame
\rulegetframeassigned{\newframe}{A'} \` Lemma \ref{constructA} \\
% Show well typed body
\rulecheckexp{\Sigma}{\Gamma}{e}{\commandty}
  \` Inversion on \rulecheckval{\Sigma}{\func{x_1, \ldots, x_n, e}}{\function{\tuple{\tau_1, \ldots \tau_n}}{\tau_r}} \\
% Show body assign
\rulecheckassigndefaultconclusion{}
  \` Inversion on \rulecheckval{\Sigma}{\func{x_1, \ldots, x_n, e}}{\function{\tuple{\tau_1, \ldots \tau_n}}{\tau_r}} \\
% Show loopnest
\rulegetloopcontext{F}{\notinloop} \` By construction \\
% Show loop check
\rulecheckloop{\notinloop}{e}
  \` Inversion on \rulecheckval{\Sigma}{\func{x_1, \ldots, x_n, e}}{\function{\tuple{\tau_1, \ldots \tau_n}}{\tau_r}} \\
% Show returns
\rulereturns{\Sigma}{\Gamma}{e}{\tau_r}
  \` Inversion on \rulecheckval{\Sigma}{\func{x_1, \ldots, x_n, e}}{\function{\tuple{\tau_1, \ldots \tau_n}}{\tau_r}} \\
% show direction
\ruledirok{\rhd}{e} \` \ruleref{\ruledirokpushingname} \\
% Done!
\rulecheckstate{\Sigma}{\stackframe{\stackframe{K}{F}}{\newframe}}{\tau}{\rhd}{e} \` \ruleref{\rulecheckstatereturnsname} \\
\end{tabbing}

\item[\rulestepcallnull] \ \\
  \begin{tabbing}
  \rulecheckstate{\Sigma}{\stackframe{K}{F}}{\tau}{\lhd}{\exn{}}
    \` \ruleref{\rulecheckstateexnname} \\
  \end{tabbing}

\item[\rulesteppushemptytuple{}] \ \\

  \begin{tabbing}
  \rulegetframecontext{\Sigma}{F}{\Gamma}
    \` Inversion on \rulecheckstate{\Sigma}{\stackframe{K}{F}}{\tau}{\rhd}{\tupleexp{}} \\
  \rulecheckexp{\Sigma}{\Gamma}{\tupleexp{}}{\tuple{}}
    \` Inversion on \rulecheckstate{\Sigma}{\stackframe{K}{F}}{\tau}{\rhd}{\tupleexp{}} \\
  \rulegetframeassigned{F}{A}
    \` Inversion on \rulecheckstate{\Sigma}{\stackframe{K}{F}}{\tau}{\rhd}{\tupleexp{}} \\
  \rulecheckassign{\Gamma}{A}{\tupleexp{}}{A}
    \` Inversion on \rulecheckstate{\Sigma}{\stackframe{K}{F}}{\tau}{\rhd}{\tupleexp{}} \\
  \rulecheckframetype{\Sigma}{A}{F}{\tuple{}}{\tau''}
    \` Inversion on \rulecheckstate{\Sigma}{\stackframe{K}{F}}{\tau}{\rhd}{\tupleexp{}} \\
  \rulecheckstack{\Sigma}{K}{\tau''}{\tau}
    \` Inversion on \rulecheckstate{\Sigma}{\stackframe{K}{F}}{\tau}{\rhd}{\tupleexp{}} \\
  \rulegetloopcontext{F}{L}
    \` Inversion on \rulecheckstate{\Sigma}{\stackframe{K}{F}}{\tau}{\rhd}{\tupleexp{}} \\
  \rulecheckloop{L}{\tupleexp{}}
    \` Inversion on \rulecheckstate{\Sigma}{\stackframe{K}{F}}{\tau}{\rhd}{\tupleexp{}} \\
  \impliescmdreturn{\tuple{}}{\Sigma}{\Gamma}{\tupleexp{}}{\tau''}
    \` Vacuously true \\
  \ruledirok{\lhd}{\tupleexp{}}
    \` \ruleref{\rulediroktuplename} \\
  \rulecheckstate{\Sigma}{\stackframe{K}{F}}{\tau}{\lhd}{\tupleexp{}}
    \` \ruleref{\rulecheckstatenormalname} \\
  \end{tabbing}

\item[\rulesteppushtupleelem] \ \\
  \newcommand{\pushtuplestack}{\frameexp{F}{\tupleexp{\cdot,\ldots}}}
  \begin{tabbing}
  \rulegetframecontext{\Sigma}{F}{\Gamma}
    \` Inversion on \rulecheckstate{\Sigma}{\stackframe{K}{F}}{\tau}{\rhd}{\tupleexp{e_1,\ldots}} \\
  \rulecheckexp{\Sigma}{\Gamma}{\tupleexp{e_1,\ldots}}{\tuple{\tau_1,\ldots}}
    \` Inversion on \rulecheckstate{\Sigma}{\stackframe{K}{F}}{\tau}{\rhd}{\tupleexp{e_1,\ldots}} \\
  \rulegetframeassigned{F}{A}
    \` Inversion on \rulecheckstate{\Sigma}{\stackframe{K}{F}}{\tau}{\rhd}{\tupleexp{e_1,\ldots}} \\
  \rulecheckassign{\Gamma}{A}{\tupleexp{e_1,\ldots}}{A}
    \` Inversion on \rulecheckstate{\Sigma}{\stackframe{K}{F}}{\tau}{\rhd}{\tupleexp{e_1,\ldots}} \\
  \rulegetloopcontext{F}{L}
    \` Inversion on \rulecheckstate{\Sigma}{\stackframe{K}{F}}{\tau}{\rhd}{\tupleexp{e_1,\ldots}} \\
  \rulecheckloop{L}{\tupleexp{e_1,\ldots}}
    \` Inversion on \rulecheckstate{\Sigma}{\stackframe{K}{F}}{\tau}{\rhd}{\tupleexp{e_1,\ldots}} \\
  \rulecheckframetype{\Sigma}{A}{F}{\tuple{\tau_1,\ldots}}{\tau''}
    \` Inversion on \rulecheckstate{\Sigma}{\stackframe{K}{F}}{\tau}{\rhd}{\tupleexp{e_1,\ldots}} \\
  \rulecheckstack{\Sigma}{K}{\tau''}{\tau}
    \` Inversion on \rulecheckstate{\Sigma}{\stackframe{K}{F}}{\tau}{\rhd}{\tupleexp{e_1,\ldots}} \\
  \\
  \rulecheckexp{\Sigma}{\Gamma}{e_1}{\tau_1} \ldots
    \` Inversion on \rulecheckexp{\Sigma}{\Gamma}{\tupleexp{e_1,\ldots}}{\tuple{\tau_1,\ldots}} \\
  \rulecheckassign{\Gamma}{A}{e_1}{A} \ldots
    \` Inversion on \rulecheckassign{\Gamma}{A}{\tupleexp{e_1,\ldots}}{A} \\
  \rulecheckloop{L}{e_1} \ldots
    \` Inversion on \rulecheckloop{L}{\tupleexp{e_1,\ldots}} \\
  \ruletausmall{\tau_1} \ldots
    \` Inversion on \rulecheckexp{\Sigma}{\Gamma}{\tupleexp{e_1,\ldots}}{\tuple{\tau_1,\ldots}} \\
  \\
  \rulecheckassign{\Gamma}{A}{\cdot}{A}
    \` \ruleref{\rulecheckassignholename} \\
  \rulecheckassign{\Gamma}{A}{\tupleexp{\cdot,\ldots}}{A}
    \` \ruleref{\rulecheckassigntuplename} \\
  \rulecheckloop{L}{\cdot}
    \` \ruleref{\rulecheckloopvarname} \\
  \rulecheckloop{L}{\tupleexp{\cdot,\ldots}}
    \` \ruleref{\rulechecklooptuplename} \\
  \ruleispending{\Sigma}{\Gamma}{\tupleexp{\cdot,\ldots}}{\tau_1}{\tuple{\tau_1,\ldots}}{\tau''}
    \` \ruleref{\ruleispendingtuplename} \\
  \rulecheckframetype{\Sigma}{A}{\pushtuplestack}{\tau_1}{\tau''}
    \` \ruleref{\rulecheckframetypeframeexpnoretname} \\
  \\
  \rulegetframecontext{\Sigma}{\pushtuplestack}{\Gamma}
    \` \ruleref{\rulegetframecontextexpname} \\
  \rulegetframeassigned{\pushtuplestack}{A}
    \` \ruleref{\rulegetframeassignedexpname} \\
  \rulegetloopcontext{\pushtuplestack}{L}
    \` \ruleref{\rulegetloopcontextexpname} \\
  \ruledirok{\rhd}{e_1}
    \` \ruleref{\ruledirokpushingname} \\
  $\tau_1 \not= \commandty$
    \` No possible derivation of \ruletausmall{\commandty} \\
  \impliescmdreturn{\tau_1}{\Sigma}{\Gamma}{e_1}{\tau''}
    \` Vacuously true \\
  \rulecheckstate{\Sigma}{\stackframe{K}{\pushtuplestack}}{\tau}{\rhd}{e_1}
    \` \ruleref{\rulecheckstatenormalname} \\
  \end{tabbing}


\item[\rulestepnexttupleelem] \ \\
  \newcommand{\nexttuplestackpre}{\frameexp{F}{\tupleexp{\ldots,\cdot,e_i,\ldots}}}
  \newcommand{\nexttuplestack}{\frameexp{F}{\tupleexp{\ldots,v,\cdot,\ldots}}}
  \begin{tabbing}
  \rulegetframecontext{\Sigma}{\nexttuplestackpre}{\Gamma}
    \` Inversion on \rulecheckstate{\Sigma}{\stackframe{K}{\nexttuplestackpre}}{\tau}{\lhd}{v} \\
  \rulecheckexp{\Sigma}{\Gamma}{v}{\tau_{i-1}}
    \` Inversion on \rulecheckstate{\Sigma}{\stackframe{K}{\nexttuplestackpre}}{\tau}{\lhd}{v} \\
  \rulegetframeassigned{\nexttuplestackpre}{A}
    \` Inversion on \rulecheckstate{\Sigma}{\stackframe{K}{\nexttuplestackpre}}{\tau}{\lhd}{v} \\
  \rulecheckassign{\Gamma}{A}{v}{A}
    \` Inversion on \rulecheckstate{\Sigma}{\stackframe{K}{\nexttuplestackpre}}{\tau}{\lhd}{v} \\
  \rulegetloopcontext{\nexttuplestackpre}{L}
    \` Inversion on \rulecheckstate{\Sigma}{\stackframe{K}{\nexttuplestackpre}}{\tau}{\lhd}{v} \\
  \rulecheckloop{L}{v}
    \` Inversion on \rulecheckstate{\Sigma}{\stackframe{K}{\nexttuplestackpre}}{\tau}{\lhd}{v} \\
  \rulecheckframetype{\Sigma}{A}{\nexttuplestackpre}{\tau_{i-1}}{\tau''}
    \` Inversion on \rulecheckstate{\Sigma}{\stackframe{K}{\nexttuplestackpre}}{\tau}{\lhd}{v} \\
  \rulecheckstack{\Sigma}{K}{\tau''}{\tau}
    \` Inversion on \rulecheckstate{\Sigma}{\stackframe{K}{\nexttuplestackpre}}{\tau}{\lhd}{v} \\
  \\
  \rulecheckassign{\Gamma}{A}{\tupleexp{\ldots,\cdot,e_i,\ldots}}{A}
    \` Inversion \rulecheckframetype{\Sigma}{A}{\nexttuplestackpre}{\tau_{i-1}}{\tau''} \\
  \rulegetframecontext{\Sigma}{F}{\Gamma}
    \` Inversion \rulecheckframetype{\Sigma}{A}{\nexttuplestackpre}{\tau_{i-1}}{\tau''} \\
  \rulegetloopcontext{F}{L}
    \` Inversion \rulecheckframetype{\Sigma}{A}{\nexttuplestackpre}{\tau_{i-1}}{\tau''} \\
  \rulecheckloop{L}{\tupleexp{\ldots, \cdot, e_i, \ldots}}
    \` Inversion \rulecheckframetype{\Sigma}{A}{\nexttuplestackpre}{\tau_{i-1}}{\tau''} \\
  \rulecheckframetype{\Sigma}{A}{F}{\tuple{\tau_1,\ldots,\tau_n}}{\tau''}
    \` Inversion \rulecheckframetype{\Sigma}{A}{\nexttuplestackpre}{\tau_{i-1}}{\tau''} \\
  \ldots{} \rulecheckassign{\Gamma}{A}{\cdot}{A} \rulecheckassign{\Gamma}{A}{e_i}{A} \ldots
    \` Inversion on \rulecheckassign{\Gamma}{A}{\tupleexp{\ldots,\cdot,e_i,\ldots}}{A} \\
  \ruleispending{\Sigma}{\Gamma}{\tupleexp{\ldots,\cdot,e_i,\ldots}}{\tau_{i-1}}{\tuple{\tau_1,\ldots,\tau_n}}{\tau''} \\
    \` Inversion \rulecheckframetype{\Sigma}{A}{\nexttuplestackpre}{\tau_{i-1}}{\tau''} \\
  \ldots{} \rulecheckexp{\Sigma}{\Gamma}{e_i}{\tau_i} \\
    \` Inversion on \ruleispending{\Sigma}{\Gamma}{\tupleexp{\ldots,\cdot,e_i,\ldots}}{\tau_{i-1}}{\tuple{\tau_1,\ldots,\tau_n}}{\tau''} \\
  \ruletausmall{\tau_1} \ldots \ruletausmall{\tau_n} \\
    \` Inversion on \ruleispending{\Sigma}{\Gamma}{\tupleexp{\ldots,\cdot,e_i,\ldots}}{\tau_{i-1}}{\tuple{\tau_1,\ldots,\tau_n}}{\tau''} \\
  \ldots{} \rulecheckloop{L}{\cdot} \rulecheckloop{L}{e_i} \ldots
    \` Inversion on \rulecheckloop{L}{\tupleexp{\ldots, \cdot, e_i, \ldots}} \\
  \\
  \rulecheckassign{\Gamma}{A}{\tupleexp{\ldots,v,\cdot,\ldots}}{A}
    \` \ruleref{\rulecheckassigntuplename} \\
  \rulecheckloop{L}{\tupleexp{\ldots,v,\cdot,\ldots}}
    \` \ruleref{\rulechecklooptuplename} \\
  \ruleispending{\Sigma}{\Gamma}{\tupleexp{\ldots,v,\cdot,\ldots}}{\tau_i}{\tuple{\tau_1,\ldots,\tau_n}}{\tau''}
    \` \ruleref{\ruleispendingtuplename} \\
  \rulecheckframetype{\Sigma}{A}{\nexttuplestack}{\tau_i}{\tau''}
    \` \ruleref{\rulecheckframetypeframeexpnoretname} \\
  \\
  \rulegetframecontext{\Sigma}{\nexttuplestack}{\Gamma}
    \` \ruleref{\rulegetframecontextexpname} \\
  \rulegetframeassigned{\nexttuplestack}{A}
    \` \ruleref{\rulegetframeassignedexpname} \\
  \rulegetloopcontext{\nexttuplestack}{L}
    \` \ruleref{\rulegetloopcontextexpname} \\
  \ruledirok{\rhd}{e_i}
    \` \ruleref{\ruledirokpushingname} \\
  $\tau_i \not= \commandty$
    \` No possible derivation of \ruletausmall{\commandty} \\
  \impliescmdreturn{\tau_i}{\Sigma}{\Gamma}{e_i}{\tau''}
    \` Vacuously true \\
  \rulecheckstate{\Sigma}{\stackframe{K}{\nexttuplestack}}{\tau}{\rhd}{e_i}
    \` \ruleref{\rulecheckstatenormalname} \\
  \end{tabbing}

\item[\rulesteplasttupleelem] \ \\
  \newcommand{\lasttuplestack}{\frameexp{F}{\tupleexp{\ldots,\cdot}}}
  \begin{tabbing}
  \rulegetframecontext{\Sigma}{\lasttuplestack}{\Gamma}
    \` Inversion on \rulecheckstate{\Sigma}{\stackframe{K}{\lasttuplestack}}{\tau}{\lhd}{v} \\
  \rulecheckexp{\Sigma}{\Gamma}{v}{\tau_n}
    \` Inversion on \rulecheckstate{\Sigma}{\stackframe{K}{\lasttuplestack}}{\tau}{\lhd}{v} \\
  \rulegetframeassigned{\lasttuplestack}{A}
    \` Inversion on \rulecheckstate{\Sigma}{\stackframe{K}{\lasttuplestack}}{\tau}{\lhd}{v} \\
  \rulecheckassign{\Gamma}{A}{v}{A}
    \` Inversion on \rulecheckstate{\Sigma}{\stackframe{K}{\lasttuplestack}}{\tau}{\lhd}{v} \\
  \rulegetloopcontext{\lasttuplestack}{L}
    \` Inversion on \rulecheckstate{\Sigma}{\stackframe{K}{\lasttuplestack}}{\tau}{\lhd}{v} \\
  \rulecheckloop{L}{v}
    \` Inversion on \rulecheckstate{\Sigma}{\stackframe{K}{\lasttuplestack}}{\tau}{\lhd}{v} \\
  \rulecheckframetype{\Sigma}{A}{\lasttuplestack}{\tau_n}{\tau''}
    \` Inversion on \rulecheckstate{\Sigma}{\stackframe{K}{\lasttuplestack}}{\tau}{\lhd}{v} \\
  \rulecheckstack{\Sigma}{K}{\tau''}{\tau}
    \` Inversion on \rulecheckstate{\Sigma}{\stackframe{K}{\lasttuplestack}}{\tau}{\lhd}{v} \\
  \\
  \rulegetframecontext{\Sigma}{F}{\Gamma}
    \` Inversion on \rulegetframecontext{\Sigma}{\lasttuplestack}{\Gamma} \\
  \rulecheckassign{\Gamma}{A}{\tupleexp{\ldots,\cdot}}{A}
    \` Inversion on \rulecheckframetype{\Sigma}{A}{\lasttuplestack}{\tau_n}{\tau''} \\
  \rulegetloopcontext{F}{L}
    \` Inversion on \rulecheckframetype{\Sigma}{A}{\lasttuplestack}{\tau_n}{\tau''} \\
  \rulecheckloop{L}{\tupleexp{\ldots,\cdot}}
    \` Inversion on \rulecheckframetype{\Sigma}{A}{\lasttuplestack}{\tau_n}{\tau''} \\
  \ruleispending{\Sigma}{\Gamma}{\tupleexp{\ldots,\cdot}}{\tau_n}{\tuple{\tau_1,\ldots,\tau_n}}{\tau''} \\
    \` Inversion on \rulecheckframetype{\Sigma}{A}{\lasttuplestack}{\tau_n}{\tau''} \\
  \rulecheckframetype{\Sigma}{A}{F}{\tuple{\tau_1,\ldots,\tau_n}}{\tau''}
    \` Inversion on \rulecheckframetype{\Sigma}{A}{\lasttuplestack}{\tau_n}{\tau''} \\
  \\
  \ruletausmall{\tau_1} \ldots \ruletausmall{\tau_n} \\
    \` Inversion on \ruleispending{\Sigma}{\Gamma}{\tupleexp{\ldots,\cdot}}{\tau_n}{\tuple{\tau_1,\ldots,\tau_n}}{\tau''} \\
  $\ldots = { v_i }$ where \rulecheckval{\Sigma}{v_i}{\tau_i} \\
    \` Inversion on \ruleispending{\Sigma}{\Gamma}{\tupleexp{\ldots,\cdot}}{\tau_n}{\tuple{\tau_1,\ldots,\tau_n}}{\tau''} \\
  \rulegetframeassigned{F}{A}
    \` Inversion on \rulegetframeassigned{\lasttuplestack}{A} \\
  \rulegetloopcontext{F}{L}
    \` Inversion on \rulegetloopcontext{\lasttuplestack}{L} \\
  \ldots \rulecheckexp{\Sigma}{\Gamma}{v_i}{\tau_i} \ldots
    \` \ruleref{\rulecheckexpvaluename} \\
  \rulecheckexp{\Sigma}{\Gamma}{\tupleexp{\ldots,v}}{\tuple{\tau_1,\ldots,\tau_n}}
    \` \ruleref{\rulecheckexptuplename} \\
  \ldots \rulecheckloop{L}{v_i} \ldots
    \` \ruleref{\rulecheckloopvaluename} \\
  \rulecheckloop{L}{\tupleexp{\ldots,v}}
    \` \ruleref{\rulechecklooptuplename} \\
  \ruledirok{\lhd}{\tuple{\ldots,v}}
    \` \ruleref{\rulediroktuplename} \\
  \impliescmdreturn{\tuple{\tau_1,\ldots,\tau_n}}{\Sigma}{\Gamma}{\tuple{\ldots,v}}{\tau''}
    \` Vacuously true \\
  \rulecheckstate{\Sigma}{\stackframe{K}{F}}{\tau}{\lhd}{\tuple{\ldots,v}}
    \` \ruleref{\rulecheckstatenormalname} \\
  \end{tabbing}

\item[\rulestepalloc] \ \\
  Note: some trivial steps omitted here.
  \begin{tabbing}
  % Show type
  \rulecheckexp{\Sigma}{\Gamma}{\allocexp{\tau_a}}{\pointer{\tau}}
    \` Inversion on \rulecheckstate{\Sigma}{\stackframe{K}{F}}{\tau}{\rhd}{\allocexp{\tau_a}} \\
  % Show can alloc
  \rulecanalloc{\tau_a}
    \` Inversion on \rulecheckexp{\Sigma}{\Gamma}{\allocexp{\tau_a}}{\pointer{\tau}} \\
  % Apply lemma
  $\exists \mu' . \ruleallocval{\mu}{a}{\tau_a}{\mu'}$ and $\mu' : \Sigma'$ and $\Sigma \le \Sigma'$ and $\Sigma'(a) = \tau_a$.
    \` Lemma \ref{allocprogress} \\
  % Show stack is well formed in new memory signature
  \rulecheckstate{\Sigma'}{\stackframe{K}{F}}{\tau}{\rhd}{\allocexp{\tau_a}} \` Lemma \ref{allocsafety} \\
  % Extract new gamma
  \rulegetframecontext{\Sigma'}{F}{\Gamma'} \`
    \` Inversion on \rulecheckstate{\Sigma'}{\stackframe{K}{F}}{\tau}{\rhd}{\allocexp{\tau_a}} \\
  % Extract assign changes
  \rulecheckassign{\Gamma}{A}{\allocexp{\tau_a}}{A}
    \` Inversion on \rulecheckstate{\Sigma'}{\stackframe{K}{F}}{\tau}{\rhd}{\allocexp{\tau_a}} \\
  % Extract well typed stack
  \rulecheckframetype{\Sigma'}{A}{F}{\pointer{\tau_a}}{\tau''}
    \` Inversion on \rulecheckstate{\Sigma'}{\stackframe{K}{F}}{\tau}{\rhd}{\allocexp{\tau_a}} \\
  % extract rest of stack
  \rulecheckstack{\Sigma'}{K}{\tau''}{\tau}
    \` Inversion on \rulecheckstate{\Sigma'}{\stackframe{K}{F}}{\tau}{\rhd}{\allocexp{\tau_a}} \\
  % Show loop
  \rulecheckloop{L}{\address{a}} \` \ruleref{\rulecheckloopvaluename} \\
  % Show assign changes
  \rulecheckassign{\Gamma}{A}{\address{a}}{A} \` \ruleref{\rulecheckassignvaluename} \\
  % Show dir ok
  \ruledirok{\lhd}{\address{a}} \` \ruleref{\ruledirokreturningname} \\
  % Show value is well typed
  \rulecheckval{\Sigma'}{\address{a}}{\pointer{\tau_a}} \` \ruleref{\rulecheckvaladdressname} \\
  % Show value is well typed as exp
  \rulecheckexp{\Sigma'}{\Gamma'}{\address{a}}{\pointer{\tau_a}} \` \ruleref{\rulecheckexpvaluename} \\
  % Show the return safety
  \impliescmdreturn{\pointer{\tau_a}}{\Sigma}{\Gamma}{\address{a}}{\tau''}
    \` Vacuously true \\
  % done!
  \rulecheckstate{\Sigma'}{\stackframe{K}{F}}{\tau}{\lhd}{\address{a}} \` \ruleref{\rulecheckstatenormalname} \\
  \end{tabbing}

\item[\rulesteppushallocarray] \ \\
  \newcommand{\pushallocarraystack}{\frameexp{F}{\allocarrayexp{\tau_a}{\cdot}}}
  \begin{tabbing}
  \rulegetframecontext{\Sigma}{F}{\Gamma}
    \` Inversion on \rulecheckstate{\Sigma}{\stackframe{K}{F}}{\tau}{\lhd}{\allocarrayexp{\tau_a}{e}} \\
  \rulecheckexp{\Sigma}{\Gamma}{\allocarrayexp{\tau_a}{e}}{\arraytype{\tau_a}}
    \` Inversion on \rulecheckstate{\Sigma}{\stackframe{K}{F}}{\tau}{\lhd}{\allocarrayexp{\tau_a}{e}} \\
  \rulegetframeassigned{F}{A}
    \` Inversion on \rulecheckstate{\Sigma}{\stackframe{K}{F}}{\tau}{\lhd}{\allocarrayexp{\tau_a}{e}} \\
  \rulecheckassign{\Gamma}{A}{\allocarrayexp{\tau_a}{e}}{A}
    \` Inversion on \rulecheckstate{\Sigma}{\stackframe{K}{F}}{\tau}{\lhd}{\allocarrayexp{\tau_a}{e}} \\
  \rulecheckframetype{\Sigma}{A}{F}{\arraytype{\tau_a}}{\tau''}
    \` Inversion on \rulecheckstate{\Sigma}{\stackframe{K}{F}}{\tau}{\lhd}{\allocarrayexp{\tau_a}{e}} \\
  \rulecheckstack{\Sigma}{K}{\tau''}{\tau}
    \` Inversion on \rulecheckstate{\Sigma}{\stackframe{K}{F}}{\tau}{\lhd}{\allocarrayexp{\tau_a}{e}} \\
  \rulegetloopcontext{F}{L}
    \` Inversion on \rulecheckstate{\Sigma}{\stackframe{K}{F}}{\tau}{\lhd}{\allocarrayexp{\tau_a}{e}} \\
  \rulecheckloop{L}{\allocarrayexp{\tau_a}{e}}
    \` Inversion on \rulecheckstate{\Sigma}{\stackframe{K}{F}}{\tau}{\lhd}{\allocarrayexp{\tau_a}{e}} \\
  \\
  \rulecheckexp{\Sigma}{\Gamma}{e}{\integer} and \rulecanalloc{\tau_a}
    \` Inversion on \rulecheckexp{\Sigma}{\Gamma}{\allocarrayexp{\tau_a}{e}}{\arraytype{\tau}} \\
  \rulecheckassign{\Gamma}{A}{\cdot}{A}
    \` \ruleref{\rulecheckassignholename} \\
  \rulecheckassign{\Gamma}{A}{\allocarrayexp{\tau_a}{\cdot}}{A}
    \` \ruleref{\rulecheckassignallocarrayname} \\
  \rulecheckloop{L}{\cdot}
    \` \ruleref{\rulecheckloopvarname} \\
  \rulecheckloop{L}{\allocarrayexp{\tau_a}{\cdot}}
    \` \ruleref{\rulecheckloopallocarrayname} \\
  \ruleispending{\Sigma}{\Gamma}{\allocarrayexp{\tau_a}{\cdot}}{\integer}{\bool}{\tau''}
    \` \ruleref{\ruleispendingallocarrayname} \\
  \rulecheckframetype{\Sigma}{A}{\pushallocarraystack}{\integer}{\tau''}
    \` \ruleref{\rulecheckframetypeframeexpnoretname} \\
  \\
  \rulegetframecontext{\Sigma}{\pushallocarraystack}{\Gamma}
    \` \ruleref{\rulegetframecontextexpname} \\
  \rulegetframeassigned{\pushallocarraystack}{A}
    \` \ruleref{\rulegetframeassignedexpname} \\
  \rulecheckassign{\Gamma}{A}{e}{A}
    \` Inversion on \rulecheckassign{\Gamma}{A}{\allocarrayexp{\tau_a}{e}}{A} \\
  \rulecheckloop{L}{e}
    \` Inversion on \rulecheckloop{L}{\allocarrayexp{\tau_a}{e}} \\
  \rulegetloopcontext{\pushallocarraystack}{L}
    \` \ruleref{\rulegetloopcontextexpname} \\
  \ruledirok{\rhd}{e}
    \` \ruleref{\ruledirokpushingname} \\
  \impliescmdreturn{\integer}{\Sigma}{\Gamma}{e}{\tau''}
    \` Vacuously true \\
  \rulecheckstate{\Sigma}{\stackframe{K}{\pushallocarraystack}}{\tau}{\rhd}{e}
    \` \ruleref{\rulecheckstatenormalname} \\
 \end{tabbing}

\item[\rulesteppopallocarray] \ \\
  \begin{tabbing}
  \rulegetframeassigned{\pushallocarraystack}{A} \\
    \` Inversion on \rulecheckstate{\Sigma}{\stackframe{K}{\pushallocarraystack}}{\tau}{\lhd}{\intliteral{n}} \\
  \rulecheckassign{\Gamma}{A}{\intliteral{n}}{A}
    \` Inversion on \rulecheckstate{\Sigma}{\stackframe{K}{\pushallocarraystack}}{\tau}{\lhd}{\intliteral{n}} \\
  \rulecheckframetype{\Sigma'}{A}{\pushallocarraystack}{\integer}{\tau''} \\
    \` Inversion on \rulecheckstate{\Sigma}{\stackframe{K}{\pushallocarraystack}}{\tau}{\lhd}{\intliteral{n}} \\
  \rulecheckstack{\Sigma}{K}{\tau''}{\tau}
    \` Inversion on \rulecheckstate{\Sigma}{\stackframe{K}{\pushallocarraystack}}{\tau}{\lhd}{\intliteral{n}} \\
  \\
  \rulegetframecontext{\Sigma}{F}{\Gamma}
    \` Inversion on \rulecheckframetype{\Sigma}{A}{\pushallocarraystack}{\integer}{\tau''} \\
  \ruleispending{\Sigma}{\Gamma}{\allocarrayexp{\tau_a}{\cdot}}{\integer}{\arraytype{\tau_a}}{\tau''} \\
    \` Inversion on \rulecheckframetype{\Sigma}{A}{\pushallocarraystack}{\integer}{\tau''} \\
  \rulecanalloc{\tau_a}
    \` Inversion on \rulecheckframetype{\Sigma}{A}{\pushallocarraystack}{\integer}{\tau''} \\
  \\
  $\exists \Sigma'.\mu' : \Sigma'$ and $\Sigma \le \Sigma'$ and \rulecheckval{\Sigma'}{\arrayval{a}{n}}{\arraytype{\tau_a}}
    \` Lemma \ref{allocarraypres} \\
  \rulecheckstate{\Sigma'}{\stackframe{K}{\pushallocarraystack}}{\tau}{\lhd}{\intliteral{n}}
    \` Lemma \ref{allocsafety} \\
  \rulegetframeassigned{\pushallocarraystack}{A} \\
    \` Inversion on \rulecheckstate{\Sigma'}{\stackframe{K}{\pushallocarraystack}}{\tau}{\lhd}{\intliteral{n}} \\
  \rulecheckassign{\Gamma}{A}{\intliteral{n}}{A}
    \` Inversion on \rulecheckstate{\Sigma'}{\stackframe{K}{\pushallocarraystack}}{\tau}{\lhd}{\intliteral{n}} \\
  \rulecheckframetype{\Sigma'}{A}{\pushallocarraystack}{\integer}{\arraytype{\tau_a}} \\
    \` Inversion on \rulecheckstate{\Sigma'}{\stackframe{K}{\pushallocarraystack}}{\tau}{\lhd}{\intliteral{n}} \\
  \rulecheckstack{\Sigma'}{K}{\tau''}{\tau}
    \` Inversion on \rulecheckstate{\Sigma'}{\stackframe{K}{\pushallocarraystack}}{\tau}{\lhd}{\intliteral{n}} \\
  \\
  \rulegetframecontext{\Sigma'}{F}{\Gamma}
    \` Inversion on \rulecheckframetype{\Sigma'}{A}{\pushallocarraystack}{\integer}{\tau''} \\
  \rulecheckassign{\Gamma}{A}{\allocarrayexp{\tau_a}{\cdot}}{A}
    \` Inversion on \rulecheckframetype{\Sigma'}{A}{\pushallocarraystack}{\integer}{\tau''} \\
  \ruleispending{\Sigma'}{\Gamma}{\allocarrayexp{\tau_a}{\cdot}}{\integer}{\arraytype{\tau_a}}{\tau''} \\
    \` Inversion on \rulecheckframetype{\Sigma'}{A}{\pushallocarraystack}{\integer}{\tau''} \\
  \rulegetloopcontext{F}{L}
    \` Inversion on \rulecheckframetype{\Sigma'}{A}{\pushallocarraystack}{\integer}{\tau''} \\
  \rulecheckframetype{\Sigma'}{\Gamma}{F}{\arraytype{\tau_a}}{\tau''}
    \` Inversion on \rulecheckframetype{\Sigma'}{A}{\pushallocarraystack}{\integer}{\tau''} \\
  \\
  \rulegetframeassigned{F}{A}
    \` Inversion on \rulegetframeassigned{\pushallocarraystack}{A} \\
  \rulecheckloop{L}{\arrayval{a}{n}}
    \` \ruleref{\rulecheckloopvaluename} \\
  \rulecheckassign{\Gamma}{A}{\arrayval{a}{n}}{A}
    \` \ruleref{\rulecheckassignvaluename} \\
  \ruledirok{\lhd}{\arrayval{a}{n}}
    \` \ruleref{\ruledirokreturningname} \\
  \impliescmdreturn{\arraytype{\tau_a}}{\Sigma'}{\Gamma}{\arrayval{a}{n}}{\tau''}
    \` Vacuously true \\
  \rulecheckstate{\Sigma'}{\stackframe{K}{F}}{\tau}{\lhd}{\arrayval{a}{n}}
    \` \ruleref{\rulecheckstatenormalname} \\

  \end{tabbing}

\item[\rulesteppopallocarrayerr] \ \\
  \begin{tabbing}
  \rulecheckstate{\Sigma}{\stackframe{K}{F}}{\tau}{\lhd}{\exn{}}
    \` \ruleref{\rulecheckstateexnname} \\
  \end{tabbing}

\item[\rulesteppushif] \ \\
  \newcommand{\ifstack}{\frameexp{F}{\ifexp{\cdot}{e_t}{e_f}}}
  \begin{tabbing}
  \hspace{3em} \= \hspace{3em} \= \\
  \rulegetframecontext{\Sigma}{F}{\Gamma}
    \` Inversion on \rulecheckstate{\Sigma}{\stackframe{K}{F}}{\tau}{\rhd}{\ifexp{e_c}{e_t}{e_f}} \\
  \rulecheckexp{\Sigma}{\Gamma}{\ifexp{e_c}{e_t}{e_f}}{\tau'}
    \` Inversion on \rulecheckstate{\Sigma}{\stackframe{K}{F}}{\tau}{\rhd}{\ifexp{e_c}{e_t}{e_f}} \\
  \rulegetframeassigned{F}{A}
    \` Inversion on \rulecheckstate{\Sigma}{\stackframe{K}{F}}{\tau}{\rhd}{\ifexp{e_c}{e_t}{e_f}} \\
  \rulecheckassign{\Gamma}{A}{\ifexp{e_c}{e_t}{e_f}}{A'}
    \` Inversion on \rulecheckstate{\Sigma}{\stackframe{K}{F}}{\tau}{\rhd}{\ifexp{e_c}{e_t}{e_f}} \\
  \rulecheckstack{\Sigma}{K}{\tau''}{\tau}
    \` Inversion on \rulecheckstate{\Sigma}{\stackframe{K}{F}}{\tau}{\rhd}{\ifexp{e_c}{e_t}{e_f}} \\
  \\
  \rulecheckexp{\Sigma}{\Gamma}{e_c}{\bool} and \rulecheckexp{\Sigma}{\Gamma}{e_t}{\tau'} and \rulecheckexp{\Sigma}{\Gamma}{e_f}{\tau'} \\
    \` Inversion on \rulecheckexp{\Sigma}{\Gamma}{\ifexp{e_c}{e_t}{e_f}}{\tau'} \\
  \rulecheckassign{\Gamma}{A}{e_c}{A} and \rulecheckassign{\Gamma}{A}{e_t}{A_t} and \rulecheckassign{\Gamma}{A}{e_f}{A_f} where $A_t \cap A_f = A'$ \\
    \` Inversion on \rulecheckassign{\Gamma}{A}{\ifexp{e_c}{e_t}{e_f}}{A'} \\
  \rulecheckassign{\Gamma}{A}{\cdot}{A}
    \` \ruleref{\rulecheckassignholename} \\
  \rulecheckassign{\Gamma}{A}{\ifexp{\cdot}{e_t}{e_f}}{A'}
    \` \ruleref{\rulecheckassignifname} \\
  \rulegetloopcontext{F}{L}
    \` Inversion on \rulecheckstate{\Sigma}{\stackframe{K}{F}}{\tau}{\rhd}{\ifexp{e_c}{e_t}{e_f}} \\
  \rulecheckloop{L}{\ifexp{e_c}{e_t}{e_f}}
    \` Inversion on \rulecheckstate{\Sigma}{\stackframe{K}{F}}{\tau}{\rhd}{\ifexp{e_c}{e_t}{e_f}} \\
  \\
  There are two possible rules used for the derivation of
    \rulecheckstate{\Sigma}{\stackframe{K}{F}}{\tau}{\rhd}{\ifexp{e_c}{e_t}{e_f}} \\
  Case \rulefmt{\rulecheckstatenormalname} \+ \\
    \rulecheckframetype{\Sigma}{A'}{F}{\tau'}{\tau''}
      \` Inversion on \rulecheckstate{\Sigma}{\stackframe{K}{F}}{\tau}{\rhd}{\ifexp{e_c}{e_t}{e_f}} \\
    \impliescmdreturn{\tau'}{\Sigma}{\Gamma}{\ifexp{e_c}{e_t}{e_f}}{\tau''} \\
      \` Inversion on \rulecheckstate{\Sigma}{\stackframe{K}{F}}{\tau}{\rhd}{\ifexp{e_c}{e_t}{e_f}} \\
    \rulecheckloop{L}{e_t} and \rulecheckloop{L}{e_f} and \rulecheckloop{L}{e_c}
      \` Inversion on \rulecheckloop{L}{\ifexp{e_c}{e_t}{e_f}} \\
    \rulecheckloop{L}{\cdot}
      \` \ruleref{\rulecheckloopvarname} \\
    \rulecheckloop{L}{\ifexp{\cdot}{e_t}{e_f}}
      \` \ruleref{\rulecheckloopifname} \\
    \impliescmdreturnsimple{e_t} and \impliescmdreturnsimple{e_f} \\
      \` Inversion on \impliescmdreturn{\tau'}{\Sigma}{\Gamma}{\ifexp{e_c}{e_t}{e_f}}{\tau''} \\
    \ruleispending{\Sigma}{\Gamma}{\ifexp{\cdot}{e_t}{e_f}}{\bool}{\tau'}{\tau''}
      \` \ruleref{\ruleispendingifname} \\
    \rulecheckframetype{\Sigma}{A}{\ifstack}{\bool}{\tau''}
      \` \ruleref{\rulecheckframetypeframeexpnoretname} \- \\
  Otherwise \rulefmt{\rulecheckstatereturnsname} was used: \+ \\
    \rulereturns{\Sigma}{\Gamma}{\ifexp{e_c}{e_t}{e_f}}{\tau''}
      \` Inversion on \rulecheckstate{\Sigma}{\stackframe{K}{F}}{\tau}{\rhd}{\ifexp{e_c}{e_t}{e_f}} \\
    $L = \notinloop$
      \` Inversion on \rulecheckstate{\Sigma}{\stackframe{K}{F}}{\tau}{\rhd}{\ifexp{e_c}{e_t}{e_f}} \\
    $\tau' = \commandty$
      \` Inversion on \rulecheckstate{\Sigma}{\stackframe{K}{F}}{\tau}{\rhd}{\ifexp{e_c}{e_t}{e_f}} \\
    \ruleonlyreturns{\Sigma}{\Gamma}{\ifexp{e_c}{e_t}{e_f}}{\tau''} \\
      \` Inversion on \rulereturns{\Sigma}{\Gamma}{\ifexp{e_c}{e_t}{e_f}}{\tau''} \\
    \ruleonlyreturns{\Sigma}{\Gamma}{e_t}{\tau''} and \ruleonlyreturns{\Sigma}{\Gamma}{e_f}{\tau''} \\
      \` Inversion on \ruleonlyreturns{\Sigma}{\Gamma}{\ifexp{e_c}{e_t}{e_f}}{\tau''} \\
    \ruledoesreturn{\ifexp{e_c}{e_t}{e_f}}
      \` Inversion on \rulereturns{\Sigma}{\Gamma}{\ifexp{e_c}{e_t}{e_f}}{\tau''} \\
    \ruledoesreturn{e_t} and \ruledoesreturn{e_f}
      \` Inversion on \ruledoesreturn{\ifexp{e_c}{e_t}{e_f}} \\
    \ruledoesreturn{\ifexp{\cdot}{e_t}{e_f}}
      \` \ruleref{\ruledoesreturnifname} \\
    \ruleispending{\Sigma}{\Gamma}{\ifexp{\cdot}{e_t}{e_f}}{\bool}{\commandty}{\tau''}
      \` \ruleref{\ruleispendingifname} \\
    \rulecheckframetype{\Sigma}{A}{\ifstack}{\bool}{\tau''}
      \` \ruleref{\rulecheckframetypeframeexpretname} \- \\
  \rulegetframecontext{\Sigma}{\ifstack}{\Gamma}
    \` \ruleref{\rulegetframecontextexpname} \\
  \rulegetframeassigned{\ifstack}{A}
    \` \ruleref{\rulegetframeassignedexpname} \\
  \rulegetloopcontext{\ifstack}{L}
    \` \ruleref{\rulegetloopcontextexpname} \\
  \rulecheckloop{L}{e_c}
    \` Inversion on \rulecheckloop{L}{\ifexp{e_c}{e_t}{e_f}} \\
  \ruledirok{\rhd}{e_c}
    \` \ruleref{\ruledirokpushingname} \\
  \impliescmdreturn{\bool}{\Sigma}{\Gamma}{e_c}{\tau''}
    \` Vacuously true \\
  \rulecheckstate{\Sigma}{\stackframe{K}{\ifstack}}{\tau}{\rhd}{e_c}
    \` \ruleref{\rulecheckstatenormalname} \\

  \end{tabbing}

\item[\rulesteppopiftrue] \ \\
  \begin{tabbing}
  \hspace{3em} \= \hspace{3em} \= \\
  \rulegetframecontext{\Sigma}{\ifstack}{\Gamma}
    \` Inversion on \rulecheckstate{\Sigma}{\stackframe{K}{\ifstack}}{\tau}{\lhd}{v} \\
  \rulecheckexp{\Sigma}{\Gamma}{\true}{\bool}
    \` Inversion on \rulecheckstate{\Sigma}{\stackframe{K}{\ifstack}}{\tau}{\lhd}{v} \\
  \rulegetframeassigned{\ifstack}{A}
    \` Inversion on \rulecheckstate{\Sigma}{\stackframe{K}{\ifstack}}{\tau}{\lhd}{v} \\
  \rulecheckassign{\Gamma}{A}{\true}{A}
    \` Inversion on \rulecheckstate{\Sigma}{\stackframe{K}{\ifstack}}{\tau}{\lhd}{v} \\
  \rulecheckframetype{\Sigma}{A}{\ifstack}{\bool}{\tau''}
    \` Inversion on \rulecheckstate{\Sigma}{\stackframe{K}{\ifstack}}{\tau}{\lhd}{v} \\
  \rulecheckstack{\Sigma}{K}{\tau''}{\tau}
    \` Inversion on \rulecheckstate{\Sigma}{\stackframe{K}{\ifstack}}{\tau}{\lhd}{v} \\
  \\
  \rulegetframecontext{\Sigma}{F}{\Gamma}
    \` Inversion on \rulegetframecontext{\Sigma}{\ifstack}{\Gamma} \\
  \rulegetframeassigned{F}{A}
    \` Inversion on \rulegetframeassigned{\ifstack}{A} \\
  \ruleispending{\Sigma}{\Gamma}{\ifexp{\cdot}{e_t}{e_f}}{\bool}{\tau'}{\tau''} \\
    \` Inversion on \rulecheckframetype{\Sigma}{A}{\ifstack}{\bool}{\tau''} \\
  \rulecheckexp{\Sigma}{\Gamma}{e_t}{\tau'}
    \` Inversion on \ruleispending{\Sigma}{\Gamma}{\ifexp{\cdot}{e_t}{e_f}}{\bool}{\tau'}{\tau''} \\
  \impliescmdreturn{\tau'}{\Sigma}{\Gamma}{e_t}{\tau''} \\
    \` Inversion on \ruleispending{\Sigma}{\Gamma}{\ifexp{\cdot}{e_t}{e_f}}{\bool}{\tau'}{\tau''} \\
  \rulecheckassign{\Gamma}{A}{\ifexp{\cdot}{e_t}{e_f}}{A'}
    \` Inversion on \rulecheckframetype{\Sigma}{A}{\ifstack}{\bool}{\tau''} \\
  \rulecheckassign{\Gamma}{A}{e_t}{A_t} and \rulecheckassign{\Gamma}{A}{e_f}{A_f} where $A_t \cap A_f = A'$ \\
    \` Inversion on \rulecheckassign{\Gamma}{A}{\ifexp{\cdot}{e_t}{e_f}}{A'} \\
  There are two possible rules used for the derivation of \rulecheckframetype{\Sigma}{A}{\ifstack}{\bool}{\tau''} \\
  Case \rulefmt{\rulecheckframetypeframeexpnoretname} \+ \\
    \rulegetloopcontext{F}{L}
      \` Inversion on \rulecheckframetype{\Sigma}{A}{\ifstack}{\bool}{\tau''} \\
    \rulecheckloop{L}{\ifexp{\cdot}{e_t}{e_f}}
      \` Inversion on \rulecheckframetype{\Sigma}{A}{\ifstack}{\bool}{\tau''} \\
    \rulecheckframetype{\Sigma}{A'}{F}{\tau'}{\tau''}
      \` Inversion on \rulecheckframetype{\Sigma}{A}{\ifstack}{\bool}{\tau''} \\
    \rulecheckframetype{\Sigma}{A_t}{F}{\tau'}{\tau''}
      \` Lemma \ref{extendAframetype} \\
    \rulecheckloop{L}{e_t}
      \` Inversion on \rulecheckloop{L}{\ifexp{\cdot}{e_t}{e_f}} \\
    \ruledirok{\rhd}{e_t}
      \` \ruleref{\ruledirokpushingname} \\
    \rulecheckstate{\Sigma}{\stackframe{K}{F}}{\tau}{\rhd}{e_t}
      \` \ruleref{\rulecheckstatenormalname} \- \\
  Otherwise \rulefmt{\rulecheckframetypeframeexpretname} \+ \\
    \rulegetloopcontext{F}{\notinloop}
      \` Inversion on \rulecheckframetype{\Sigma}{A}{\ifstack}{\bool}{\tau''} \\
    \rulecheckloop{\notinloop}{\ifexp{\cdot}{e_t}{e_f}}
      \` Inversion on \rulecheckframetype{\Sigma}{A}{\ifstack}{\bool}{\tau''} \\
    \ruledoesreturn{\ifexp{\cdot}{e_t}{e_f}}
      \` Inversion on \rulecheckframetype{\Sigma}{A}{\ifstack}{\bool}{\tau''} \\
    $\tau' = \commandty$
      \` Inversion on \rulecheckframetype{\Sigma}{A}{\ifstack}{\bool}{\tau''} \\
    \ruledoesreturn{e_t}
      \` Inversion on \ruledoesreturn{\ifexp{\cdot}{e_t}{e_f}} \\
    \rulereturns{\Sigma}{\Gamma}{e_t}{\tau''}
      \` \ruleref{\rulereturnsonlyname} \\
    \rulecheckloop{\notinloop}{e_t}
      \` Inversion on \rulecheckloop{\notinloop}{\ifexp{\cdot}{e_t}{e_f}} \\
    \ruledirok{\rhd}{e_t}
      \` \ruleref{\ruledirokpushingname} \\
    \rulecheckstate{\Sigma}{\stackframe{K}{F}}{\tau}{\rhd}{e_t}
      \` \ruleref{\rulecheckstatereturnsname} \\

  \end{tabbing}

\item[\rulesteppopiffalse] \ \\
  \begin{tabbing}
  \hspace{3em} \= \hspace{3em} \= \\
  \rulegetframecontext{\Sigma}{\ifstack}{\Gamma}
    \` Inversion on \rulecheckstate{\Sigma}{\stackframe{K}{\ifstack}}{\tau}{\lhd}{v} \\
  \rulecheckexp{\Sigma}{\Gamma}{\false}{\bool}
    \` Inversion on \rulecheckstate{\Sigma}{\stackframe{K}{\ifstack}}{\tau}{\lhd}{v} \\
  \rulegetframeassigned{\ifstack}{A}
    \` Inversion on \rulecheckstate{\Sigma}{\stackframe{K}{\ifstack}}{\tau}{\lhd}{v} \\
  \rulecheckassign{\Gamma}{A}{\false}{A}
    \` Inversion on \rulecheckstate{\Sigma}{\stackframe{K}{\ifstack}}{\tau}{\lhd}{v} \\
  \rulecheckframetype{\Sigma}{A}{\ifstack}{\bool}{\tau''}
    \` Inversion on \rulecheckstate{\Sigma}{\stackframe{K}{\ifstack}}{\tau}{\lhd}{v} \\
  \rulecheckstack{\Sigma}{K}{\tau''}{\tau}
    \` Inversion on \rulecheckstate{\Sigma}{\stackframe{K}{\ifstack}}{\tau}{\lhd}{v} \\
  \\
  \rulegetframecontext{\Sigma}{F}{\Gamma}
    \` Inversion on \rulegetframecontext{\Sigma}{\ifstack}{\Gamma} \\
  \rulegetframeassigned{F}{A}
    \` Inversion on \rulegetframeassigned{\ifstack}{A} \\
  \ruleispending{\Sigma}{\Gamma}{\ifexp{\cdot}{e_t}{e_f}}{\bool}{\tau'}{\tau''} \\
    \` Inversion on \rulecheckframetype{\Sigma}{A}{\ifstack}{\bool}{\tau''} \\
  \rulecheckexp{\Sigma}{\Gamma}{e_f}{\tau'}
    \` Inversion on \ruleispending{\Sigma}{\Gamma}{\ifexp{\cdot}{e_t}{e_f}}{\bool}{\tau'}{\tau''} \\
  \impliescmdreturn{\tau'}{\Sigma}{\Gamma}{e_f}{\tau''} \\
    \` Inversion on \ruleispending{\Sigma}{\Gamma}{\ifexp{\cdot}{e_t}{e_f}}{\bool}{\tau'}{\tau''} \\
  \rulecheckassign{\Gamma}{A}{\ifexp{\cdot}{e_t}{e_f}}{A'}
    \` Inversion on \rulecheckframetype{\Sigma}{A}{\ifstack}{\bool}{\tau''} \\
  \rulecheckassign{\Gamma}{A}{e_t}{A_t} and \rulecheckassign{\Gamma}{A}{e_f}{A_f} where $A_t \cap A_f = A'$ \\
    \` Inversion on \rulecheckassign{\Gamma}{A}{\ifexp{\cdot}{e_t}{e_f}}{A'} \\
  There are two possible rules used for the derivation of \rulecheckframetype{\Sigma}{A}{\ifstack}{\bool}{\tau''} \\
  Case \rulefmt{\rulecheckframetypeframeexpnoretname} \+ \\
    \rulegetloopcontext{F}{L}
      \` Inversion on \rulecheckframetype{\Sigma}{A}{\ifstack}{\bool}{\tau''} \\
    \rulecheckloop{L}{\ifexp{\cdot}{e_t}{e_f}}
      \` Inversion on \rulecheckframetype{\Sigma}{A}{\ifstack}{\bool}{\tau''} \\
    \rulecheckframetype{\Sigma}{A'}{F}{\tau'}{\tau''}
      \` Inversion on \rulecheckframetype{\Sigma}{A}{\ifstack}{\bool}{\tau''} \\
    \rulecheckframetype{\Sigma}{A_f}{F}{\tau'}{\tau''}
      \` Lemma \ref{extendAframetype} \\
    \rulecheckloop{L}{e_f}
      \` Inversion on \rulecheckloop{L}{\ifexp{\cdot}{e_t}{e_f}} \\
    \ruledirok{\rhd}{e_f}
      \` \ruleref{\ruledirokpushingname} \\
    \rulecheckstate{\Sigma}{\stackframe{K}{F}}{\tau}{\rhd}{e_f}
      \` \ruleref{\rulecheckstatenormalname} \- \\
  Otherwise \rulefmt{\rulecheckframetypeframeexpretname} \+ \\
    \rulegetloopcontext{F}{\notinloop}
      \` Inversion on \rulecheckframetype{\Sigma}{A}{\ifstack}{\bool}{\tau''} \\
    \rulecheckloop{\notinloop}{\ifexp{\cdot}{e_t}{e_f}}
      \` Inversion on \rulecheckframetype{\Sigma}{A}{\ifstack}{\bool}{\tau''} \\
    \ruledoesreturn{\ifexp{\cdot}{e_t}{e_f}}
      \` Inversion on \rulecheckframetype{\Sigma}{A}{\ifstack}{\bool}{\tau''} \\
    $\tau' = \commandty$
      \` Inversion on \rulecheckframetype{\Sigma}{A}{\ifstack}{\bool}{\tau''} \\
    \ruledoesreturn{e_f}
      \` Inversion on \ruledoesreturn{\ifexp{\cdot}{e_t}{e_f}} \\
    \rulereturns{\Sigma}{\Gamma}{e_f}{\tau''}
      \` \ruleref{\rulereturnsonlyname} \\
    \rulecheckloop{\notinloop}{e_f}
      \` Inversion on \rulecheckloop{\notinloop}{\ifexp{\cdot}{e_t}{e_f}} \\
    \ruledirok{\rhd}{e_f}
      \` \ruleref{\ruledirokpushingname} \\
    \rulecheckstate{\Sigma}{\stackframe{K}{F}}{\tau}{\rhd}{e_f}
      \` \ruleref{\rulecheckstatereturnsname} \\

  \end{tabbing}

\item[\rulesteppushdecl] \ \\
  \newcommand{\pushdeclstack}{\framevar{F}{x}{\tau_x}}
  \begin{tabbing}
  \hspace{3em} \= \hspace{3em} \= \\
  \rulegetframecontext{\Sigma}{F}{\Gamma}
    \` Inversion on \rulecheckstate{\Sigma}{\stackframe{K}{F}}{\tau}{\rhd}{\decl{x}{\tau_x}{e}} \\
  \rulecheckexp{\Sigma}{\Gamma}{\decl{x}{\tau_x}{e}}{\commandty}
    \` Inversion on \rulecheckstate{\Sigma}{\stackframe{K}{F}}{\tau}{\rhd}{\decl{x}{\tau_x}{e}} \\
  \rulegetframeassigned{F}{A}
    \` Inversion on \rulecheckstate{\Sigma}{\stackframe{K}{F}}{\tau}{\rhd}{\decl{x}{\tau_x}{e}} \\
  \rulecheckassign{\Gamma}{A}{\decl{x}{\tau_x}{e}}{A' - \{x\}}
    \` Inversion on \rulecheckstate{\Sigma}{\stackframe{K}{F}}{\tau}{\rhd}{\decl{x}{\tau_x}{e}} \\
  \rulecheckstack{\Sigma}{K}{\tau''}{\tau}
    \` Inversion on \rulecheckstate{\Sigma}{\stackframe{K}{F}}{\tau}{\rhd}{\decl{x}{\tau_x}{e}} \\
  $x \not\in A$ and \rulecheckassign{\varctx{\Gamma}{x}{\tau_x}}{A}{e}{A'}
    \` Inversion on \rulecheckassign{\Gamma}{A}{\decl{x}{\tau}{e}}{A' - \{x\}} \\
  \rulecheckexp{\Sigma}{\varctx{\Gamma}{x}{\tau_x}}{e}{\commandty}
    \` Inversion on \rulecheckexp{\Sigma}{\Gamma}{\decl{x}{\tau_x}{e}}{\commandty} \\
  There are two possible rules used for the derivation of
     \rulecheckstate{\Sigma}{\stackframe{K}{F}}{\tau}{\rhd}{\decl{x}{\tau_x}{e}} \\
  Case \rulefmt{\rulecheckstatenormalname}: \+ \\
    \rulecheckframetype{\Sigma}{A' - \{x\}}{F}{\commandty}{\tau''}
      \` Inversion on \rulecheckstate{\Sigma}{\stackframe{K}{F}}{\tau}{\rhd}{\decl{x}{\tau_x}{e}} \\
    \rulegetloopcontext{F}{L}
      \` Inversion on \rulecheckstate{\Sigma}{\stackframe{K}{F}}{\tau}{\rhd}{\decl{x}{\tau_x}{e}} \\
    \rulecheckloop{L}{\decl{x}{\tau_x}{e}}
      \` Inversion on \rulecheckstate{\Sigma}{\stackframe{K}{F}}{\tau}{\rhd}{\decl{x}{\tau_x}{e}} \\
    \impliescmdreturn{\commandty}{\Sigma}{\Gamma}{\decl{x}{\tau}{e}}{\tau''} \\
      \` Inversion on \rulecheckstate{\Sigma}{\stackframe{K}{F}}{\tau}{\rhd}{\decl{x}{\tau_x}{e}} \\
    \rulecheckframetype{\Sigma}{A'}{\pushdeclstack}{\commandty}{\tau''}
      \` \ruleref{\rulecheckframetypeframevarname} \\
    \rulegetframecontext{\Sigma}{\pushdeclstack}{\varctx{\Gamma}{x}{\tau_x}}
      \` \ruleref{\rulegetframecontextvarname} \\
    \rulegetframeassigned{\pushdeclstack}{A}
      \` \ruleref{\rulegetframeassignedvarname} \\
    \rulegetloopcontext{\pushdeclstack}{L}
      \` \ruleref{\rulegetloopcontextdeclname} \\
    \rulecheckloop{L}{e}
      \` Inversion on \rulecheckloop{L}{\decl{x}{\tau_x}{e}} \\
    \ruledirok{\rhd}{e}
      \` \ruleref{\ruledirokpushingname} \\
    \ruleonlyreturns{\Sigma}{\varctx{\Gamma}{x}{\tau_x}}{e}{\tau''}
      \` Inversion on \ruleonlyreturns{\Sigma}{\Gamma}{\decl{x}{\tau_x}{e}}{\tau''} \\
    \impliescmdreturn{\commandty}{\Sigma}{\varctx{\Gamma}{x}{\tau_x}}{e}{\tau''}
      \` Weakening \\
    \rulecheckstate{\Sigma}{\stackframe{K}{\pushdeclstack}}{\tau}{\rhd}{e}
      \` \ruleref{\rulecheckstatenormalname} \- \\
  Otherwise \rulefmt{\rulecheckstatereturnsname} was used: \+ \\
    \rulegetloopcontext{F}{\notinloop}
      \` Inversion on \rulecheckstate{\Sigma}{\stackframe{K}{F}}{\tau}{\rhd}{\decl{x}{\tau_x}{e}} \\
    \rulecheckloop{\notinloop}{\decl{x}{\tau_x}{e}}
      \` Inversion on \rulecheckstate{\Sigma}{\stackframe{K}{F}}{\tau}{\rhd}{\decl{x}{\tau_x}{e}} \\
    \rulereturns{\Sigma}{\Gamma}{\decl{x}{\tau}{e}}{\tau''}
      \` Inversion on \rulecheckstate{\Sigma}{\stackframe{K}{F}}{\tau}{\rhd}{\decl{x}{\tau_x}{e}} \\
    \ruledoesreturn{\decl{x}{\tau_x}{e}}
      \` Inversion on \rulereturns{\Sigma}{\Gamma}{\decl{x}{\tau}{e}}{\tau''} \\
    \ruledoesreturn{e}
      \` Inversion on \ruledoesreturn{\decl{x}{\tau_x}{e}} \\
    \ruletausmall{\tau''}
      \` Inversion on \rulereturns{\Sigma}{\Gamma}{\decl{x}{\tau}{e}}{\tau''} \\
    \ruleonlyreturns{\Sigma}{\Gamma}{\decl{x}{\tau_x}{e}}{\tau''} \\
      \` Inversion on \rulereturns{\Sigma}{\Gamma}{\decl{x}{\tau}{e}}{\tau''} \\
    \ruleonlyreturns{\Sigma}{\varctx{\Gamma}{x}{\tau_x}}{e}{\tau''}
      \` Inversion on \ruleonlyreturns{\Sigma}{\Gamma}{\decl{x}{\tau_x}{e}}{\tau''} \\
    \rulereturns{\Sigma}{\varctx{\Gamma}{x}{\tau_x}}{e}{\tau''}
      \` \ruleref{\rulereturnsonlyname} \\
    \rulegetloopcontext{\pushdeclstack}{\notinloop}
      \` \ruleref{\rulegetloopcontextdeclname} \\
    \rulecheckloop{\notinloop}{e}
      \` Inversion on \rulecheckloop{\notinloop}{\decl{x}{\tau_x}{e}} \\
    \ruledirok{\rhd}{e}
      \` \ruleref{\ruledirokpushingname} \\
    \rulegetframecontext{\Sigma}{\pushdeclstack}{\varctx{\Gamma}{x}{\tau_x}}
      \` \ruleref{\rulegetframecontextvarname} \\
    \rulegetframeassigned{\pushdeclstack}{A}
      \` \ruleref{\rulegetframeassignedvarname} \\
    \rulecheckstate{\Sigma}{\stackframe{K}{\pushdeclstack}}{\tau}{\rhd}{e}
      \` \ruleref{\rulecheckstatereturnsname} \\
  \end{tabbing}

\item[\rulesteppopdecl] \ \\
  \newcommand{\popdeclstack}{\framevar{F}{x}{\tau_x}}
  \begin{tabbing}
  \rulegetframeassigned{\popdeclstack}{A}
    \` Inversion on \rulecheckstate{\Sigma}{\stackframe{K}{\popdeclstack}}{\tau}{\lhd}{\nop} \\
  \rulecheckassign{\Gamma}{A}{\nop}{A}
    \` Inversion on \rulecheckstate{\Sigma}{\stackframe{K}{\popdeclstack}}{\tau}{\lhd}{\nop} \\
  \rulecheckframetype{\Sigma}{A}{\popdeclstack}{\commandty}{\tau''}
    \` Inversion on \rulecheckstate{\Sigma}{\stackframe{K}{\popdeclstack}}{\tau}{\lhd}{\nop} \\
  \rulegetloopcontext{\popdeclstack}{L}
    \` Inversion on \rulecheckstate{\Sigma}{\stackframe{K}{\popdeclstack}}{\tau}{\lhd}{\nop} \\
  \rulecheckstack{\Sigma}{K}{\tau''}{\tau}
    \` Inversion on \rulecheckstate{\Sigma}{\stackframe{K}{\popdeclstack}}{\tau}{\lhd}{\nop} \\
  \rulegetframeassigned{F}{A} and $x \not\in A$
    \` Inversion on \rulegetframeassigned{\popdeclstack}{A} \\
  \rulecheckframetype{\Sigma}{A - \{x\}}{F}{\commandty}{\tau''}
    \` Inversion on \rulecheckframetype{\Sigma}{A}{\popdeclstack}{\commandty}{\tau''} \\
  \rulecheckframetype{\Sigma}{A}{F}{\commandty}{\tau''}
    \` $x \not\in A \Rightarrow A - \{x\} = A$ \\
  \rulegetframecontext{\Sigma}{F}{\Gamma} for some $\Gamma$
    \` By construction \\
  \rulecheckvalnopconclusion{}
    \` \ruleref{\rulecheckvalnopname} \\
  \rulecheckexp{\Sigma}{\Gamma}{\nop}{\commandty}
    \` \ruleref{\rulecheckexpvaluename} \\
  \rulegetloopcontext{F}{L}
    \` Inversion on \rulegetloopcontext{\popdeclstack}{L} \\
  \ruleonlyreturns{\Sigma}{\Gamma}{\nop}{\tau''}
    \` \ruleref{\ruleonlyreturnsnopname} \\
  \impliescmdreturn{\commandty}{\Sigma}{\Gamma}{\nop}{\tau''}
    \` Weakening \\
  \rulecheckstate{\Sigma}{\stackframe{K}{F}}{\tau}{\lhd}{\nop}
    \` \ruleref{\rulecheckstatenormalname} \\
  \end{tabbing}

\item[\rulesteppushassign] \ \\
  \newcommand{\pushassignstack}{\frameexp{F}{\assign{x}{\cdot}}}
  \begin{tabbing}
  \rulegetframecontext{\Sigma}{F}{\Gamma}
    \` Inversion on \rulecheckstate{\Sigma}{\stackframe{K}{F}}{\tau}{\rhd}{\assign{x}{e}} \\
  \rulecheckexp{\Sigma}{\Gamma}{\assign{x}{e}}{\commandty}
    \` Inversion on \rulecheckstate{\Sigma}{\stackframe{K}{F}}{\tau}{\rhd}{\assign{x}{e}} \\
  \rulegetframeassigned{F}{A}
    \` Inversion on \rulecheckstate{\Sigma}{\stackframe{K}{F}}{\tau}{\rhd}{\assign{x}{e}} \\
  \rulecheckassign{\Gamma}{A}{\assign{x}{e}}{A \cup \{x\}}
    \` Inversion on \rulecheckstate{\Sigma}{\stackframe{K}{F}}{\tau}{\rhd}{\assign{x}{e}} \\
  \rulegetloopcontext{F}{L}
    \` Inversion on \rulecheckstate{\Sigma}{\stackframe{K}{F}}{\tau}{\rhd}{\assign{x}{e}} \\
  \rulecheckloop{L}{\assign{x}{e}}
    \` Inversion on \rulecheckstate{\Sigma}{\stackframe{K}{F}}{\tau}{\rhd}{\assign{x}{e}} \\
  \rulecheckframetype{\Sigma}{A \cup \{x\}}{F}{\commandty}{\tau''}
    \` Inversion on \rulecheckstate{\Sigma}{\stackframe{K}{F}}{\tau}{\rhd}{\assign{x}{e}} \\
  \rulecheckstack{\Sigma}{K}{\tau''}{\tau}
    \` Inversion on \rulecheckstate{\Sigma}{\stackframe{K}{F}}{\tau}{\rhd}{\assign{x}{e}} \\
  $\Gamma = \varctx{\Gamma'}{x}{\tau'}$ and \rulecheckexp{\Sigma}{\Gamma}{e}{\tau'} and \ruletausmall{\tau'}
    \` Inversion on \rulecheckexp{\Sigma}{\Gamma}{\assign{x}{e}}{\commandty} \\
  \rulecheckassign{\Gamma}{A}{e}{A}
    \` Inversion on \rulecheckassign{\Gamma}{A}{\assign{x}{e}}{A \cup \{x\}} \\
  \rulecheckloop{L}{e}
    \` Inversion on \rulecheckloop{L}{\assign{x}{e}} \\
  \rulecheckassign{\Gamma}{A}{\cdot}{A}
    \` \ruleref{\rulecheckassignholename} \\
  \rulecheckassign{\Gamma}{A}{\assign{x}{\cdot}}{A \cup \{x \}}
    \` \ruleref{\rulecheckassignassignname} \\
  \rulecheckloop{L}{\cdot}
    \` \ruleref{\rulecheckloopvarname} \\
  \rulecheckloop{L}{\assign{x}{\cdot}}
    \` \ruleref{\rulecheckloopassignname} \\
  \rulecheckexp{\Sigma}{\Gamma}{x}{\tau'}
    \` \ruleref{\rulecheckexpvarname} \\
  \ruleispending{\Sigma}{\Gamma}{\assign{x}{\cdot}}{\tau'}{\commandty}{\tau''}
    \` \ruleref{\ruleispendingassignname} \\
  \rulecheckframetype{\Sigma}{A}{\pushassignstack}{\tau'}{\tau''}
    \` \ruleref{\rulecheckframetypeframeexpnoretname} \\

  \rulegetframecontext{\Sigma}{\pushassignstack}{\Gamma}
    \` \ruleref{\rulegetframecontextexpname} \\
  \rulegetframeassigned{\pushassignstack}{A}
    \` \ruleref{\rulegetframeassignedexpname} \\
  \rulegetloopcontext{\pushassignstack}{L}
    \` \ruleref{\rulegetloopcontextexpname} \\
  \ruledirok{\rhd}{e}
    \` \ruleref{\ruledirokpushingname} \\

  Since \ruletausmall{\tau''} by inversion $\tau'' \not= \commandty$. \\
  \impliescmdreturn{\tau''}{\Sigma}{\Gamma}{e}{\tau''}
    \` Vacuously true \\

  \rulecheckstate{\Sigma}{\stackframe{K}{\pushassignstack}}{\tau}{\rhd}{e}
    \` \ruleref{\rulecheckstatenormalname} \\

  \end{tabbing}

\item[\rulesteppopassign]
  \newcommand{\popassignstackmini}{\frameexp{\frameval{F}{x}{\tau_x}{v}}{\ldots}}
  \newcommand{\popassignstackminipost}{\frameexp{\frameval{F}{x}{\tau_x}{v'}}{\ldots}}
  \newcommand{\popassignstack}{\frameexp{\popassignstackmini}{\assign{x}{\cdot}}}
  \begin{tabbing}
  \hspace{3em} \= \hspace{3em} \= \\
  \rulegetframecontext{\Sigma}{\popassignstack}{\Gamma} \\
    \` Inversion on \rulecheckstate{\Sigma}{\stackframe{K}{\popassignstack}}{\tau}{\lhd}{v'} \\
  \rulecheckexp{\Sigma}{\Gamma}{v'}{\tau'}
    \` Inversion on \rulecheckstate{\Sigma}{\stackframe{K}{\popassignstack}}{\tau}{\lhd}{v'} \\
  \rulegetframeassigned{\popassignstack}{A} \\
    \` Inversion on \rulecheckstate{\Sigma}{\popassignstack}{\tau}{\lhd}{v'} \\
  \rulecheckassign{\Gamma}{A}{v'}{A}
    \` Inversion on \rulecheckstate{\Sigma}{\popassignstack}{\tau}{\lhd}{v'} \\
  \rulecheckframetype{\Sigma}{A}{\popassignstack}{\tau'}{\tau''} \\
    \` Inversion on \rulecheckstate{\Sigma}{\popassignstack}{\tau}{\lhd}{v'} \\
  \rulecheckstack{\Sigma}{K}{\tau''}{\tau}
    \` Inversion on \rulecheckstate{\Sigma}{\popassignstack}{\tau}{\lhd}{v'} \\
  \ruleispending{\Sigma}{\Gamma}{\assign{x}{\cdot}}{\tau'}{\commandty}{\tau''} \\
    \` Inversion on \rulecheckframetype{\Sigma}{A}{\popassignstack}{\tau'}{\tau''} \\
  \rulecheckexp{\Sigma}{\Gamma}{x}{\tau'}
    \` Inversion on \ruleispending{\Sigma}{\Gamma}{\assign{x}{\cdot}}{\tau'}{\commandty}{\tau''} \\
  \rulecheckexp{\Sigma}{\varctx{\Gamma'}{x}{\tau'}}{x}{\tau'} where $\Gamma = \varctx{\Gamma'}{x}{\tau'}$
    \` Inversion on \rulecheckexp{\Sigma}{\Gamma}{x}{\tau'} \\
  \rulecheckval{\Sigma}{v'}{\tau'}
    \` Inversion on \rulecheckexp{\Sigma}{\Gamma}{v'}{\tau'} \\
  \rulecheckval{\Sigma}{v}{\tau'} and $\tau' = \tau_x$
    \` Lemma \ref{stackvarshape} \\
  \rulegetframeassigned{\popassignstackminipost}{A}
    \` Lemma \ref{assignassigned} \\
  \rulecheckframetype{\Sigma}{A}{\popassignstackmini}{\commandty}{\tau''} \\
    \` Inversion on \rulecheckframetype{\Sigma}{A}{\popassignstack}{\tau'}{\tau''} \\
  \rulecheckframetype{\Sigma}{A}{\popassignstackminipost}{\commandty}{\tau''}
    \` Lemma \ref{assignframetype} \\
  \rulegetframecontext{\Sigma}{\popassignstackminipost}{\Gamma''} for some $\Gamma''$
    \` By construction \\
  \rulecheckexp{\Sigma}{\Gamma''}{\nop}{\commandty}
    \` \ruleref{\rulecheckexpvaluename} \\
  \rulecheckassign{\Gamma}{A}{\nop}{A}
    \` \ruleref{\rulecheckassignvarname} \\
  \rulegetloopcontext{\popassignstackminipost}{L} for some $L$
    \` By construction \\
  \rulecheckloop{L}{\nop}
    \` \ruleref{\rulecheckloopvaluename} \\
  \ruledirok{\lhd}{\nop}
    \` \ruleref{\ruledirokreturningname} \\
  \impliescmdreturn{\commandty}{\Sigma}{\Gamma}{\nop}{\tau''}
    \` Weakening of \ruleref{\ruleonlyreturnsnopname} \\

  \rulecheckstate{\Sigma}{\stackframe{K}{\popassignstackminipost}}{\tau}{\lhd}{\nop}
    \` \ruleref{\rulecheckstatenormalname} \\
  \end{tabbing}

\item[\rulesteppopassignfirst] \ \\
  \newcommand{\popassignfirststackmini}{\frameexp{\framevar{F}{x}{\tau}}{\ldots}}
  \newcommand{\popassignfirststackminipost}{\frameexp{\frameval{F}{x}{\tau}{v'}}{\ldots}}
  \newcommand{\popassignfirststack}{\frameexp{\popassignfirststackmini}{\assign{x}{\cdot}}}
  \begin{tabbing}
  \hspace{3em} \= \hspace{3em} \= \\
  \rulegetframecontext{\Sigma}{\popassignfirststack}{\Gamma} \\
    \` Inversion on \rulecheckstate{\Sigma}{\stackframe{K}{\popassignfirststack}}{\tau}{\lhd}{v'} \\
  \rulecheckexp{\Sigma}{\Gamma}{v'}{\tau'}
    \` Inversion on \rulecheckstate{\Sigma}{\stackframe{K}{\popassignfirststack}}{\tau}{\lhd}{v'} \\
  \rulegetframeassigned{\popassignfirststack}{A} \\
    \` Inversion on \rulecheckstate{\Sigma}{\popassignfirststack}{\tau}{\lhd}{v'} \\
  \rulecheckassign{\Gamma}{A}{v'}{A}
    \` Inversion on \rulecheckstate{\Sigma}{\popassignfirststack}{\tau}{\lhd}{v'} \\
  \rulecheckframetype{\Sigma}{A}{\popassignfirststack}{\tau'}{\tau''} \\
    \` Inversion on \rulecheckstate{\Sigma}{\popassignfirststack}{\tau}{\lhd}{v'} \\
  \rulecheckstack{\Sigma}{K}{\tau''}{\tau}
    \` Inversion on \rulecheckstate{\Sigma}{\popassignstack}{\tau}{\lhd}{v'} \\
  \ruleispending{\Sigma}{\Gamma}{\assign{x}{\cdot}}{\tau'}{\commandty}{\tau''} \\
    \` Inversion on \rulecheckframetype{\Sigma}{A}{\popassignfirststack}{\tau'}{\tau''} \\
  \rulecheckexp{\Sigma}{\Gamma}{x}{\tau'}
    \` Inversion on \ruleispending{\Sigma}{\Gamma}{\assign{x}{\cdot}}{\tau'}{\commandty}{\tau''} \\
  \rulecheckexp{\Sigma}{\varctx{\Gamma'}{x}{\tau'}}{x}{\tau'} where $\Gamma = \varctx{\Gamma'}{x}{\tau'}$
    \` Inversion on \rulecheckexp{\Sigma}{\Gamma}{x}{\tau'} \\
  \rulecheckval{\Sigma}{v'}{\tau'}
    \` Inversion on \rulecheckexp{\Sigma}{\Gamma}{v'}{\tau'} \\
  \rulecheckval{\Sigma}{v'}{\tau_x}
    \` Lemma \ref{stackvarshape} \\
  \rulegetframeassigned{\popassignfirststackminipost}{A \cup \{x\}}
    \` Lemma \ref{assignfirstassigned} \\
  \rulecheckframetype{\Sigma}{A \cup \{x\}}{\popassignfirststackmini}{\commandty}{\tau''} \\
    \` Inversion on \rulecheckframetype{\Sigma}{A}{\popassignfirststack}{\tau'}{\tau''} \\
  \rulecheckframetype{\Sigma}{A \cup \{x\}}{\popassignfirststackminipost}{\commandty}{\tau''}
    \` Lemma \ref{assignfirstframetype} \\
  \rulegetframecontext{\Sigma}{\popassignfirststackminipost}{\Gamma''} for some $\Gamma''$
    \` By construction \\
  \rulecheckexp{\Sigma}{\Gamma''}{\nop}{\commandty}
    \` \ruleref{\rulecheckexpvaluename} \\
  \rulecheckassign{\Gamma}{A \cup \{x\}}{\nop}{A \cup \{x\}}
    \` \ruleref{\rulecheckassignvarname} \\
  \rulegetloopcontext{\popassignfirststackminipost}{L} for some $L$
    \` By construction \\
  \rulecheckloop{L}{\nop}
    \` \ruleref{\rulecheckloopvaluename} \\
  \ruledirok{\lhd}{\nop}
    \` \ruleref{\ruledirokreturningname} \\

  \impliescmdreturn{\commandty}{\Sigma}{\Gamma}{\nop}{\tau''}
    \` Weakening of \ruleref{\ruleonlyreturnsnopname} \\

  \rulecheckstate{\Sigma}{\stackframe{K}{\popassignfirststackminipost}}{\tau}{\lhd}{\nop}
    \` \ruleref{\rulecheckstatenormalname} \\
  \end{tabbing}

\item[\rulesteppopassigned] \ \\
  \newcommand{\popassignedstack}{\frameval{F}{x}{\tau_x}{v}}
  \begin{tabbing}
  \rulegetframeassigned{\popassignedstack}{A}
    \` Inversion on \rulecheckstate{\Sigma}{\stackframe{K}{\popassignedstack}}{\tau}{\lhd}{\nop} \\
  \rulecheckassign{\Gamma}{A}{\nop}{A}
    \` Inversion on \rulecheckstate{\Sigma}{\stackframe{K}{\popassignedstack}}{\tau}{\lhd}{\nop} \\
  \rulecheckframetype{\Sigma}{A}{\popassignedstack}{\commandty}{\tau''}
    \` Inversion on \rulecheckstate{\Sigma}{\stackframe{K}{\popassignedstack}}{\tau}{\lhd}{\nop} \\
  \rulecheckframetype{\Sigma}{A - \{x\}}{F}{\commandty}{\tau''}
    \` Inversion on \rulecheckframetype{\Sigma}{A}{\popassignedstack}{\commandty}{\tau''} \\
  \rulegetframeassigned{F}{A'} where $A' \cup \{x\} = A$ and $x \not\in A'$ \\
    \` Inversion on \rulegetframeassigned{\popassignedstack}{A} \\
  $A - \{x\} = A'$
    \` $A' \cup \{x\} = A$ and $x \not\in A$ \\
  \rulecheckassign{\Gamma}{A'}{\nop}{A'}
    \` \ruleref{\rulecheckassignvaluename} \\

  \rulegetframecontext{\Sigma}{\popassignedstack}{\Gamma}
    \` Inversion on \rulecheckframetype{\Sigma}{A}{\popassignedstack}{\commandty}{\tau''} \\
  \rulegetframecontext{\Sigma}{F}{\Gamma}
    \` Inversion on \rulegetframecontext{\Sigma}{\popassignedstack}{\Gamma} \\

  \rulecheckvalnopconclusion{}
    \` \ruleref{\rulecheckvalnopname} \\
  \rulecheckexp{\Sigma}{\Gamma}{\nop}{\commandty}
    \` \ruleref{\rulecheckexpvaluename} \\

  \rulegetloopcontext{\popassignedstack}{L}
    \` Inversion on \rulecheckframetype{\Sigma}{A}{\popassignedstack}{\commandty}{\tau''} \\
  \rulegetloopcontext{F}{L}
    \` Inversion on \rulegetloopcontext{\popassignedstack}{L} \\

  \rulecheckloop{L}{\nop}
    \` \ruleref{\rulecheckloopvaluename} \\

  \rulecheckstack{\Sigma}{K}{\tau''}{\tau}
    \` Inversion on \rulecheckframetype{\Sigma}{A}{\popassignedstack}{\commandty}{\tau''} \\

  \ruledirok{\lhd}{\nop}
    \` Inversion on \rulecheckframetype{\Sigma}{A}{\popassignedstack}{\commandty}{\tau''} \\

  \impliescmdreturn{\commandty}{\Sigma}{\Gamma}{\nop}{\tau''}
    \` Inversion on \rulecheckframetype{\Sigma}{A}{\popassignedstack}{\commandty}{\tau''} \\

  \rulecheckstate{\Sigma}{\stackframe{K}{F}}{\tau}{\lhd}{\nop}
    \` \ruleref{\rulecheckstatenormalname} \\

  \end{tabbing}

\item[\rulesteppushret] \ \\
  \newcommand{\pushretstack}{\frameexp{F}{\return{\cdot}}}
  \begin{tabbing}
  \hspace{3em} \= \hspace{3em} \= \kill
  \rulegetframecontext{\Sigma}{F}{\Gamma}
    \` Inversion on \rulecheckstate{\Sigma}{\stackframe{K}{F}}{\tau}{\rhd}{\return{e}} \\
  \rulecheckexp{\Sigma}{\Gamma}{\return{e}}{\commandty}
    \` Inversion on \rulecheckstate{\Sigma}{\stackframe{K}{F}}{\tau}{\rhd}{\return{e}} \\
  \rulegetframeassigned{F}{A}
    \` Inversion on \rulecheckstate{\Sigma}{\stackframe{K}{F}}{\tau}{\rhd}{\return{e}} \\
  \rulecheckassignreturnconclusion{} \\
    \` Inversion on \rulecheckstate{\Sigma}{\stackframe{K}{F}}{\tau}{\rhd}{\return{e}} \\
  \rulecheckstack{\Sigma}{K}{\tau''}{\tau}
    \` Inversion on \rulecheckstate{\Sigma}{\stackframe{K}{F}}{\tau}{\rhd}{\return{e}} \\
  There are two possible rules used for the derivation of
    \rulecheckstate{\Sigma}{\stackframe{K}{F}}{\tau}{\rhd}{\return{e}} \\
  Case \rulefmt{\rulecheckstatenormalname}: \+ \\
    \rulecheckframetype{\Sigma}{A}{F}{\commandty}{\tau''}
      \` Inversion on \rulecheckstate{\Sigma}{\stackframe{K}{F}}{\tau}{\rhd}{\return{e}} \\
    \impliescmdreturn{\commandty}{\Sigma}{\Gamma}{\return{e}}{\tau''} \\
      \` Inversion on \rulecheckstate{\Sigma}{\stackframe{K}{F}}{\tau}{\rhd}{\return{e}} \\
    \ruleonlyreturns{\Sigma}{\Gamma}{\return{e}}{\tau''}
      \` Modus ponens \\
    \rulecheckexp{\Sigma}{\Gamma}{e}{\tau''} and \ruletausmall{\tau''}
      \` Inversion on \ruleonlyreturns{\Sigma}{\Gamma}{\return{e}}{\tau''} \\
    \rulegetloopcontext{F}{L}
      \` Inversion on \rulecheckstate{\Sigma}{\stackframe{K}{F}}{\tau}{\rhd}{\return{e}} \\
    \rulecheckloop{L}{\return{e}}
      \` Inversion on \rulecheckstate{\Sigma}{\stackframe{K}{F}}{\tau}{\rhd}{\return{e}} \\
    % Now start constructing the stack
    \ruledoesreturn{\return{\cdot}}
      \` \ruleref{\ruledoesreturnreturnname} \\
    \rulecheckassign{\Gamma}{A}{\cdot}{A}
      \` \ruleref{\rulecheckassignholename} \\
    \rulecheckassign{\Gamma}{A}{\return{\cdot}}{A}
      \` \ruleref{\rulecheckassignreturnname} \\
    \rulecheckloop{L}{\return{\cdot}}
      \` \ruleref{\rulecheckloopreturnname} \\
    \ruleispending{\Sigma}{\Gamma}{\return{\cdot}}{\tau''}{\commandty}{\tau''}
      \` \ruleref{\ruleispendingreturnname} \\
    \rulecheckframetype{\Sigma}{A}{\pushretstack}{\tau''}{\tau''}
      \` \ruleref{\rulecheckframetypeframeexpnoretname} \\
    % Now show safe w/top
    \rulegetframecontext{\Sigma}{\pushretstack}{\Gamma}
      \` \ruleref{\rulegetframecontextexpname} \\
    \rulegetframeassigned{\pushretstack}{A}
      \` \ruleref{\rulegetframeassignedexpname} \\
    \rulecheckassign{\Gamma}{A}{e}{A}
      \` Inversion on \rulecheckassign{\Gamma}{A}{\return{e}}{A} \\
    \rulegetloopcontext{\pushretstack}{L}
      \` \ruleref{\rulegetloopcontextexpname} \\
    \rulecheckloop{L}{e}
      \` Inversion on \rulecheckloop{L}{\return{e}} \\
    \ruledirok{\rhd}{e}
      \` \ruleref{\ruledirokpushingname} \\
    Since \ruletausmall{\tau''} by inversion $\tau'' \not= \commandty$. \\
    \impliescmdreturn{\tau''}{\Sigma}{\Gamma}{e}{\tau''}
      \` Vacuously true \\
    \rulecheckstate{\Sigma}{\stackframe{K}{\pushretstack}}{\tau}{\rhd}{e}
      \` \ruleref{\rulecheckstatenormalname} \- \\
  Otherwise \rulefmt{\rulecheckstatereturnsname} \+ was used \\
    \rulegetloopcontext{F}{\notinloop}
      \` Inversion on \rulecheckstate{\Sigma}{\stackframe{K}{F}}{\tau}{\rhd}{\return{e}} \\
    \rulecheckloop{\notinloop}{\return{e}}
      \` Inversion on \rulecheckstate{\Sigma}{\stackframe{K}{F}}{\tau}{\rhd}{\return{e}} \\

    \rulecheckexp{\Sigma}{\Gamma}{e}{\tau''}
      \` Inversion on \rulecheckexp{\Sigma}{\Gamma}{e}{\tau''} \\
    \ruledoesreturn{\return{\cdot}}
      \` \ruleref{\ruledoesreturnreturnname} \\

    \rulecheckassign{\Gamma}{A}{\cdot}{A}
      \` \ruleref{\rulecheckassignholename} \\
    \rulecheckassign{\Gamma}{A}{\return{\cdot}}{A}
      \` \ruleref{\rulecheckassignreturnname} \\
    \rulecheckloop{\notinloop}{\return{\cdot}}
      \` \ruleref{\rulecheckloopreturnname} \\
    \ruleispending{\Sigma}{\Gamma}{\return{\cdot}}{\tau''}{\commandty}{\tau''}
      \` \ruleref{\ruleispendingreturnname} \\
    \rulecheckframetype{\Sigma}{A}{\pushretstack}{\tau''}{\tau''}
      \` \ruleref{\rulecheckframetypeframeexpretname} \\

    \rulegetframecontext{\Sigma}{\pushretstack}{\Gamma}
      \` \ruleref{\rulegetframecontextexpname} \\
    \rulegetframeassigned{\pushretstack}{A}
      \` \ruleref{\rulegetframeassignedexpname} \\
    \rulecheckassign{\Gamma}{A}{e}{A}
      \` Inversion on \rulecheckassign{\Gamma}{A}{\return{e}}{A} \\
    \rulegetloopcontext{\pushretstack}{\notinloop}
      \` \ruleref{\rulegetloopcontextexpname} \\
    \rulecheckloop{\notinloop}{e}
      \` Inversion on \rulecheckloop{\notinloop}{\return{e}} \\
    \ruledirok{\rhd}{e}
      \` \ruleref{\ruledirokpushingname} \\
    Since \ruletausmall{\tau''} by inversion $\tau'' \not= \commandty$. \\
    \impliescmdreturn{\tau''}{\Sigma}{\Gamma}{e}{\tau''}
      \` Vacuously true \\
    \rulecheckstate{\Sigma}{\stackframe{K}{\pushretstack}}{\tau}{\rhd}{e}
      \` \ruleref{\rulecheckstatenormalname} \- \\
  \end{tabbing}

\item[\rulesteppopret]
  \newcommand{\retstack}{\stackframe{K}{\frameexp{F}{\return{\cdot}}}}
  \begin{tabbing}
  \hspace{3em} \= \hspace{3em} \= \\
  \rulecheckexp{\Sigma}{\Gamma}{e}{\tau'}
    \` Inversion on \rulecheckstate{\Sigma}{\retstack}{\tau}{\lhd}{e}\\
  \rulecheckframetype{\Sigma}{A}{\retstack}{\tau'}{\tau''}
    \` Inversion on \rulecheckstate{\Sigma}{\retstack}{\tau}{\lhd}{e}\\
  \ruledirok{\lhd}{e}
    \` Inversion on \rulecheckstate{\Sigma}{\retstack}{\tau}{\lhd}{e}\\
  \ruleispending{\Sigma}{\Gamma}{\return{\cdot}}{\tau'}{\commandty}{\tau''} \\
    \` Inversion on \rulecheckframetype{\Sigma}{A}{\retstack}{\tau'}{\tau''} \\
  $\tau' = \tau''$ and \ruletausmall{\tau''}
    \` Inversion on \ruleispending{\Sigma}{\Gamma}{\return{\cdot}}{\tau'}{\commandty}{\tau''} \\
  \rulecheckstack{\Sigma}{K}{\tau''}{\tau}
    \` Inversion on \rulecheckexp{\Sigma}{\Gamma}{e}{\tau'} \\
  Case analysis on the derivation of \rulecheckstack{\Sigma}{K}{\tau''}{\tau} gives two possible cases. \\
  If the rule {\tt \rulecheckstackemptyname} was used then \+ \\
    $K = \emptystack{}$ and $\tau = \tau''$
      \` Inversion on \rulecheckstack{\Sigma}{K}{\tau''}{\tau} \\
    Via case analysis on \ruledirok{\lhd}{e}: either $e = $ some $v$ or $e = \tupleexp{}$. \\
    If $e = v$ then \+ \\
      \rulecheckexp{\Sigma}{\cdot}{v}{\tau}
        \` \ruleref{\rulecheckexpvaluename} \- \\
    Otherwise $e = \tupleexp{}$ \+ \\
      $\tau' = \tupleexp{}$
        \` Inversion on \rulecheckexp{\Sigma}{\Gamma}{e}{\tau'} \\
      \rulecheckexp{\Sigma}{\cdot}{\tupleexp{}}{\tau}
        \` \ruleref{\rulecheckexptuplename} \- \\
    \rulecheckstateemptyconclusion{}
      \` \ruleref{\rulecheckstateemptyname} \- \\
  Otherwise the rule {\tt \rulecheckstacknonemptyname} was used. \+ \\
    $K = K';F'$ and \rulegetframeassigned{F'}{A'} and \rulecheckframetype{\Sigma}{A'}{F'}{\tau''}{\tau'''} \\
      \hspace{1em} and \rulecheckstack{\Sigma}{K'}{\tau'''}{\tau} 
      \` Inversion on \rulecheckstack{\Sigma}{K}{\tau''}{\tau} \\
    \rulegetframecontext{\Sigma}{F'}{\Gamma'} \` By construction \\
    \rulegetloopcontext{F'}{L} \` By construction \\
    Via case analysis on \ruledirok{\lhd}{e}: either $e = $ some $v$ or $e = \tupleexp{}$. \\
    If $e = v$ then \+ \\
      \rulecheckexp{\Sigma}{\Gamma'}{v}{\tau'}
        \` \ruleref{\rulecheckexpvaluename} \\
      \rulecheckassign{\Gamma}{A'}{v}{A'}
        \` \ruleref{\rulecheckassignvaluename} \\
      \rulecheckloop{L}{v}
        \` \ruleref{\rulecheckloopvaluename} \- \\
    Otherwise $e = \tupleexp{}$ \+ \\
      $\tau' = \tupleexp{}$
        \` Inversion on \rulecheckexp{\Sigma}{\Gamma}{e}{\tau'} \\
      \rulecheckexp{\Sigma}{\Gamma'}{\tupleexp{}}{\tau'}
        \` \ruleref{\rulecheckexptuplename} \\
      \rulecheckassign{\Gamma}{A'}{\tupleexp{}}{A'}
        \` \ruleref{\rulecheckassigntuplename} \\
      \rulecheckloop{L}{\tupleexp{}}
        \` \ruleref{\rulechecklooptuplename} \- \\
    $\tau'' \not= \commandty$ \` No derivation of \ruletausmall{\commandty} \\
    \impliescmdreturn{\tau''}{\Sigma}{\Gamma}{e}{\tau'''}
      \` Vacuously true \\
    \rulecheckstate{\Sigma}{\stackframe{K'}{F'}}{\tau}{\lhd}{e} \` \ruleref{\rulecheckstatenormalname} \\
  \end{tabbing}

\item[\rulesteploop] \ \\
  \newcommand{\looppushstack}{\frameloop{F}{e_c}{e}}
  \begin{tabbing}
  \rulegetframecontextdefaultconclusion{}
    \` Inversion on \rulecheckstate{\Sigma}{\stackframe{K}{F}}{\tau}{\rhd}{\loopstm{e_c}{e}} \\
  \rulecheckexp{\Sigma}{\Gamma}{\loopstm{e_c}{e}}{\commandty}
    \` Inversion on \rulecheckstate{\Sigma}{\stackframe{K}{F}}{\tau}{\rhd}{\loopstm{e_c}{e}} \\
  \rulegetframeassigned{F}{A}
    \` Inversion on \rulecheckstate{\Sigma}{\stackframe{K}{F}}{\tau}{\rhd}{\loopstm{e_c}{e}} \\
  \rulecheckassignloopconclusion{}
    \` Inversion on \rulecheckstate{\Sigma}{\stackframe{K}{F}}{\tau}{\rhd}{\loopstm{e_c}{e}} \\
  \rulegetloopcontextdefaultconclusion{}
    \` Inversion on \rulecheckstate{\Sigma}{\stackframe{K}{F}}{\tau}{\rhd}{\loopstm{e_c}{e}} \\
  \rulechecklooploopconclusion{}
    \` Inversion on \rulecheckstate{\Sigma}{\stackframe{K}{F}}{\tau}{\rhd}{\loopstm{e_c}{e}} \\
  \rulecheckframetype{\Sigma}{A}{F}{\commandty}{\tau''}
    \` Inversion on \rulecheckstate{\Sigma}{\stackframe{K}{F}}{\tau}{\rhd}{\loopstm{e_c}{e}} \\
  \rulecheckstack{\Sigma}{K}{\tau''}{\tau}
    \` Inversion on \rulecheckstate{\Sigma}{\stackframe{K}{F}}{\tau}{\rhd}{\loopstm{e_c}{e}} \\
  \impliescmdreturn{\commandty}{\Sigma}{\Gamma}{\loopstm{e_c}{e}}{\tau''} \\
    \` Inversion on \rulecheckstate{\Sigma}{\stackframe{K}{F}}{\tau}{\rhd}{\loopstm{e_c}{e}} \\
  \rulecheckframetype{\Sigma}{A}{\looppushstack}{\commandty}{\tau''}
    \` \ruleref{\rulecheckframetypeframeexpnoretname} \\
  \\
  % Inner frame complete. Now show top is ok

  \rulegetframecontext{\Sigma}{\looppushstack}{\Gamma}
    \` \ruleref{\rulegetframecontextloopname} \\

  \rulecheckexp{\Sigma}{\Gamma}{e_c}{\bool}
    \` Inversion on \rulecheckexp{\Sigma}{\Gamma}{\loopstm{e_c}{e}}{\commandty} \\
  \rulecheckexp{\Sigma}{\Gamma}{e}{\commandty}
    \` Inversion on \rulecheckexp{\Sigma}{\Gamma}{\loopstm{e_c}{e}}{\commandty} \\
  \rulecheckexp{\Sigma}{\Gamma}{\breakstm}{\commandty}
    \` \ruleref{\rulecheckexpbreakname} \\
  \rulecheckexp{\Sigma}{\Gamma}{\ifexp{e_c}{e}{\breakstm}}{\commandty}
    \` \ruleref{\rulecheckexpifname} \\
  \\

  \rulegetframeassigned{\looppushstack}{A}
    \` \ruleref{\rulegetframeassignedloopname} \\

  \rulecheckassign{\Gamma}{A}{e_c}{A}
    \` Inversion on \rulecheckassignloopconclusion{} \\
  \rulecheckassign{\Gamma}{A}{e}{A'}
    \` Inversion on \rulecheckassignloopconclusion{} \\
  \rulecheckassign{\Gamma}{A}{\breakstm}{A}
    \` \ruleref{\rulecheckassignbreakname} \\
  $A \cap A' = A$
    \` Lemma \ref{assignexpand} \\
  \rulecheckassign{\Gamma}{A}{\ifexp{e_c}{e}{\breakstm}}{A}
    \` \ruleref{\rulecheckassignifname} \\
  \\

  \rulegetloopcontext{\looppushstack}{\inloop}
    \` \ruleref{\rulegetloopcontextloopname} \\

  \rulecheckloop{L}{e_c}
    \` Inversion on \rulechecklooploopconclusion{} \\
  \rulecheckloop{\inloop}{e_c}
    \` Lemma \ref{loopsub} \\
  \rulecheckloop{\inloop}{e}
    \` Inversion on \rulechecklooploopconclusion{} \\
  \rulecheckloop{\inloop}{\breakstm}
    \` \ruleref{\rulecheckloopbreakname} \\
  \rulecheckloop{\inloop}{\ifexp{e_c}{e}{\breakstm}}
    \` \ruleref{\rulecheckloopifname} \\
  \\

  \ruledirok{\rhd}{\ifexp{e_c}{e}{\breakstm}}
    \` \ruleref{\ruledirokpushingname} \\

  \ruleonlyreturns{\Sigma}{\Gamma}{\loopstm{e_c}{e}}{\tau''} \\
    \` Modus ponens with \impliescmdreturn{\commandty}{\Sigma}{\Gamma}{\loopstm{e_c}{e}}{\tau''} \\
  \ruleonlyreturns{\Sigma}{\Gamma}{e}{\tau''}
    \` Inversion on \ruleonlyreturns{\Sigma}{\Gamma}{\loopstm{e_c}{e}}{\tau''} \\
  \ruleonlyreturns{\Sigma}{\Gamma}{\breakstm}{\tau''}
    \` \ruleref{\ruleonlyreturnsbreakname} \\
  \ruleonlyreturns{\Sigma}{\Gamma}{\ifexp{e_c}{e}{\breakstm}}{\tau''}
    \` \ruleref{\ruleonlyreturnsifname} \\
  \impliescmdreturn{\commandty}{\Sigma}{\Gamma}{\ifexp{e_c}{e}{\breakstm}}{\tau''}
    \` Trivial weakening \\

  \rulecheckstate{\Sigma}{\stackframe{K}{\looppushstack}}{\tau}{\rhd}{\ifexp{e_c}{e}{\breakstm}}
    \` \ruleref{\rulecheckstatenormalname} \\

  \end{tabbing}

\item[\rulesteplooppop] \ \\
  \newcommand{\looppopstack}{\frameloop{F}{e_c}{e}}
  \begin{tabbing}
  \rulegetloopcontextloopconclusion{}
    \` \ruleref{\rulegetloopcontextloopname} \\
  \rulegetinnerlooploopconclusion{}
    \` \ruleref{\rulegetinnerlooploopname} \\
  \rulegetframeassigned{\looppopstack}{A}
    \` Inversion on \rulecheckstate{\Sigma}{\stackframe{K}{\looppopstack}}{\tau}{\lhd}{\nop} \\
  \rulecheckassignnopconclusion{}
    \` Inversion on \rulecheckstate{\Sigma}{\stackframe{K}{\looppopstack}}{\tau}{\lhd}{\nop} \\
  \rulegetframecontext{\Sigma}{\looppopstack}{\Gamma}
    \` Inversion on \rulecheckstate{\Sigma}{\stackframe{K}{\looppopstack}}{\tau}{\lhd}{\nop} \\
  \rulecheckexp{\Sigma}{\Gamma}{\nop}{\commandty}
    \` Inversion on \rulecheckstate{\Sigma}{\stackframe{K}{\looppopstack}}{\tau}{\lhd}{\nop} \\
  \rulecheckframetype{\Sigma}{A}{\looppopstack}{\commandty}{\tau''}
    \` Inversion on \rulecheckstate{\Sigma}{\stackframe{K}{\looppopstack}}{\tau}{\lhd}{\nop} \\
  \rulecheckstack{\Sigma}{K}{\tau''}{\tau}
    \` Inversion on \rulecheckstate{\Sigma}{\stackframe{K}{\looppopstack}}{\tau}{\lhd}{\nop} \\
  \rulecheckstate{\Sigma}{\stackframe{K}{\looppopstack}}{\tau}{\lhd}{\continue}
    \` \ruleref{\rulecheckstateloopcontname} \\
  \end{tabbing}

\item[\rulestepbreak] \ \\

  \begin{tabbing}
  \rulecheckloop{\inloop}{\breakstm}
    \` Inversion on \rulecheckstate{\Sigma}{\stackframe{K}{F}}{\tau}{\rhd}{\breakstm} \\
  \rulegetloopcontext{F}{\inloop}
    \` Inversion on \rulecheckstate{\Sigma}{\stackframe{K}{F}}{\tau}{\rhd}{\breakstm} \\
  \rulegetframeassigned{F}{A}
    \` Inversion on \rulecheckstate{\Sigma}{\stackframe{K}{F}}{\tau}{\rhd}{\breakstm} \\
  \rulecheckassignbreakconclusion{}
    \` Inversion on \rulecheckstate{\Sigma}{\stackframe{K}{F}}{\tau}{\rhd}{\breakstm} \\
  \rulegetframecontext{\Sigma}{F}{\Gamma}
    \` Inversion on \rulecheckstate{\Sigma}{\stackframe{K}{F}}{\tau}{\rhd}{\breakstm} \\
  \rulecheckexp{\Sigma}{\Gamma}{\breakstm}{\tau'}
    \` Inversion on \rulecheckstate{\Sigma}{\stackframe{K}{F}}{\tau}{\rhd}{\breakstm} \\
  $\tau' = \commandty$
    \` Inversion on \rulecheckexp{\Sigma}{\Gamma}{\breakstm}{\tau'} \\
  \rulecheckframetype{\Sigma}{A}{F}{\commandty}{\tau''}
    \` Inversion on \rulecheckstate{\Sigma}{\stackframe{K}{F}}{\tau}{\rhd}{\breakstm} \\
  \rulecheckstack{\Sigma}{K}{\tau''}{\tau}
    \` Inversion on \rulecheckstate{\Sigma}{\stackframe{K}{F}}{\tau}{\rhd}{\breakstm} \\
  \rulegetinnerloop{F}{\frameloop{F'}{e_c}{e}} and \rulegetframeassigned{F'}{A'} and \\
  \rulecheckframetype{\Sigma}{A'}{F'}{\commandty}{\tau''}
    \` Lemma \ref{innerloopcheck} \\
  \rulecheckframetype{\Sigma}{A'}{\frameloop{F'}{e_c}{e}}{\commandty}{\tau''}
    \` \ruleref{\rulecheckframetypeframeloopname} \\
  \rulecheckstate{\Sigma}{\stackframe{K}{F}}{\tau}{\lhd}{\breakstm}
    \` \ruleref{\rulecheckstateloopbrkname} \\
  \end{tabbing}

\item[\rulestepbreakval] \ \\
  \newcommand{\breakvalstack}{\frameval{F}{x}{\tau_x}{v}}
  \begin{tabbing}
  \rulegetframeassigned{F'}{A}
    \` Inversion on \rulecheckstate{\Sigma}{\stackframe{K}{\breakvalstack}}{\tau}{\lhd}{\break} \\
  \rulecheckframetype{\Sigma}{A}{F'}{\commandty}{\tau''}
    \` Inversion on \rulecheckstate{\Sigma}{\stackframe{K}{\breakvalstack}}{\tau}{\lhd}{\break} \\
  \rulecheckstack{\Sigma}{K}{\tau''}{\tau}
    \` Inversion on \rulecheckstate{\Sigma}{\stackframe{K}{\breakvalstack}}{\tau}{\lhd}{\break} \\
  \rulegetloopcontext{\breakvalstack}{\inloop}
    \` Inversion on \rulecheckstate{\Sigma}{\stackframe{K}{\breakvalstack}}{\tau}{\lhd}{\break} \\
  \rulegetinnerloop{\breakvalstack}{F'}
    \` Inversion on \rulecheckstate{\Sigma}{\stackframe{K}{\breakvalstack}}{\tau}{\lhd}{\break} \\
  \rulegetloopcontext{F}{\inloop}
    \` Inversion on \rulegetloopcontext{\breakvalstack}{\inloop} \\
  \rulegetinnerloop{F}{F'}
    \` Inversion on \rulegetinnerloop{\breakvalstack}{F'} \\
  \rulecheckstate{\Sigma}{\stackframe{K}{F}}{\tau}{\lhd}{\breakstm}
    \` \ruleref{\rulecheckstateloopbrkname} \\
  \end{tabbing}

\item[\rulestepbreakvar] \ \\
  \newcommand{\breakstmvarstack}{\framevar{F}{x}{\tau_x}}
  \begin{tabbing}
  \rulegetframeassigned{F'}{A}
    \` Inversion on \rulecheckstate{\Sigma}{\stackframe{K}{\breakstmvarstack}}{\tau}{\lhd}{\breakstm} \\
  \rulecheckframetype{\Sigma}{A}{F'}{\commandty}{\tau''}
    \` Inversion on \rulecheckstate{\Sigma}{\stackframe{K}{\breakstmvarstack}}{\tau}{\lhd}{\breakstm} \\
  \rulecheckstack{\Sigma}{K}{\tau''}{\tau}
    \` Inversion on \rulecheckstate{\Sigma}{\stackframe{K}{\breakstmvarstack}}{\tau}{\lhd}{\breakstm} \\
  \rulegetloopcontext{\breakstmvarstack}{\inloop}
    \` Inversion on \rulecheckstate{\Sigma}{\stackframe{K}{\breakstmvarstack}}{\tau}{\lhd}{\breakstm} \\
  \rulegetinnerloop{\breakstmvarstack}{F'}
    \` Inversion on \rulecheckstate{\Sigma}{\stackframe{K}{\breakstmvarstack}}{\tau}{\lhd}{\breakstm} \\
  \rulegetloopcontext{F}{\inloop}
    \` Inversion on \rulegetloopcontext{\breakstmvarstack}{\inloop} \\
  \rulegetinnerloop{F}{F'}
    \` Inversion on \rulegetinnerloop{\breakstmvarstack}{F'} \\
  \rulecheckstate{\Sigma}{\stackframe{K}{F}}{\tau}{\lhd}{\breakstm}
    \` \ruleref{\rulecheckstateloopbrkname} \\
  \end{tabbing}

\item[\rulestepbreakexp] \ \\
  \newcommand{\breakstmexpstack}{\frameexp{F}{e}}
  \begin{tabbing}
  \rulegetframeassigned{F'}{A}
    \` Inversion on \rulecheckstate{\Sigma}{\stackframe{K}{\breakstmexpstack}}{\tau}{\lhd}{\breakstm} \\
  \rulecheckframetype{\Sigma}{A}{F'}{\commandty}{\tau''}
    \` Inversion on \rulecheckstate{\Sigma}{\stackframe{K}{\breakstmexpstack}}{\tau}{\lhd}{\breakstm} \\
  \rulecheckstack{\Sigma}{K}{\tau''}{\tau}
    \` Inversion on \rulecheckstate{\Sigma}{\stackframe{K}{\breakstmexpstack}}{\tau}{\lhd}{\breakstm} \\
  \rulegetloopcontext{\breakstmexpstack}{\inloop}
    \` Inversion on \rulecheckstate{\Sigma}{\stackframe{K}{\breakstmexpstack}}{\tau}{\lhd}{\breakstm} \\
  \rulegetinnerloop{\breakstmexpstack}{F'}
    \` Inversion on \rulecheckstate{\Sigma}{\stackframe{K}{\breakstmexpstack}}{\tau}{\lhd}{\breakstm} \\
  \rulegetloopcontext{F}{\inloop}
    \` Inversion on \rulegetloopcontext{\breakstmexpstack}{\inloop} \\
  \rulegetinnerloop{F}{F'}
    \` Inversion on \rulegetinnerloop{\breakstmexpstack}{F'} \\
  \rulecheckstate{\Sigma}{\stackframe{K}{F}}{\tau}{\lhd}{\breakstm}
    \` \ruleref{\rulecheckstateloopbrkname} \\
  \end{tabbing}

\item[\rulestepbreakloop] \ \\
  \newcommand{\breakloopstack}{\frameloop{F}{e_c}{e}}
  \begin{tabbing}
  \rulegetinnerloop{\breakloopstack}{F'}
    \` Inversion on \rulecheckstate{\Sigma}{\stackframe{K}{\breakloopstack}}{\tau}{\lhd}{\breakstm} \\
  $F' = \breakloopstack$
    \` Inversion on \rulegetinnerloop{\breakloopstack}{F'} \\
  \rulecheckframetype{\Sigma}{A}{\breakloopstack}{\commandty}{\tau''} \\
    \` Inversion on \rulecheckstate{\Sigma}{\stackframe{K}{\breakloopstack}}{\tau}{\lhd}{\breakstm} \\
  \rulegetframeassigned{F}{A}
    \` Inversion on \rulecheckframetype{\Sigma}{A}{\breakloopstack}{\commandty}{\tau''} \\
  \rulecheckframetype{\Sigma}{A}{F}{\commandty}{\tau''}
    \` Inversion on \rulecheckframetype{\Sigma}{A}{\breakloopstack}{\commandty}{\tau''} \\
  \rulecheckstack{\Sigma}{K}{\tau''}{\tau}
    \` Inversion on \rulecheckstate{\Sigma}{\stackframe{K}{\breakloopstack}}{\tau}{\lhd}{\breakstm} \\
  \rulegetframecontextdefaultconclusion{} for some $\Gamma$
    \` By construction \\
  \rulecheckvalnopconclusion{}
    \` \ruleref{\rulecheckvalnopname} \\
  \rulecheckexp{\Sigma}{\Gamma}{\nop}{\commandty}
    \` \ruleref{\rulecheckexpvaluename} \\
  \rulecheckassign{\Gamma}{A}{\nop}{A}
    \` \ruleref{\rulecheckassignvaluename} \\
  \rulegetloopcontext{F}{L} for some $L$
    \` By construction \\
  \rulecheckloop{L}{\nop}
    \` \ruleref{\rulecheckloopvaluename} \\
  \ruledirok{\lhd}{\nop}
    \` \ruleref{\ruledirokreturningname} \\
  \ruleonlyreturns{\Sigma}{\Gamma}{\nop}{\tau''}
    \` \ruleref{\ruleonlyreturnsnopname} \\
  \impliescmdreturn{\commandty}{\Sigma}{\Gamma}{\nop}{\tau''}
    \` Always true \\
  \rulecheckstate{\Sigma}{\stackframe{K}{F}}{\tau}{\lhd}{\nop}
    \` \ruleref{\rulecheckstatenormalname} \\
  \end{tabbing}

\item[\rulestepcontinue] \ \\

  \begin{tabbing}
  \rulecheckloop{\inloop}{\continue}
    \` Inversion on \rulecheckstate{\Sigma}{\stackframe{K}{F}}{\tau}{\rhd}{\continue} \\
  \rulegetloopcontext{F}{\inloop}
    \` Inversion on \rulecheckstate{\Sigma}{\stackframe{K}{F}}{\tau}{\rhd}{\continue} \\
  \rulegetframeassigned{F}{A}
    \` Inversion on \rulecheckstate{\Sigma}{\stackframe{K}{F}}{\tau}{\rhd}{\continue} \\
  \rulecheckassigncontinueconclusion{} \\
    \` Inversion on \rulecheckstate{\Sigma}{\stackframe{K}{F}}{\tau}{\rhd}{\continue} \\
  \rulegetframecontext{\Sigma}{F}{\Gamma}
    \` Inversion on \rulecheckstate{\Sigma}{\stackframe{K}{F}}{\tau}{\rhd}{\continue} \\
  \rulecheckexp{\Sigma}{\Gamma}{\continue}{\tau'}
    \` Inversion on \rulecheckstate{\Sigma}{\stackframe{K}{F}}{\tau}{\rhd}{\continue} \\
  $\tau' = \commandty$
    \` Inversion on \rulecheckexp{\Sigma}{\Gamma}{\continue}{\tau'} \\
  \rulecheckframetype{\Sigma}{A}{F}{\commandty}{\tau''}
    \` Inversion on \rulecheckstate{\Sigma}{\stackframe{K}{F}}{\tau}{\rhd}{\continue} \\
  \rulecheckstack{\Sigma}{K}{\tau''}{\tau}
    \` Inversion on \rulecheckstate{\Sigma}{\stackframe{K}{F}}{\tau}{\rhd}{\continue} \\
  \rulegetinnerloop{F}{\frameloop{F'}{e_c}{e}} and \rulegetframeassigned{F'}{A'} and \\
  \rulecheckframetype{\Sigma}{A'}{F'}{\commandty}{\tau''}
    \` Lemma \ref{innerloopcheck} \\
  \rulecheckframetype{\Sigma}{A'}{\frameloop{F'}{e_c}{e}}{\commandty}{\tau''}
    \` \ruleref{\rulecheckframetypeframeloopname} \\
  \rulecheckstate{\Sigma}{\stackframe{K}{F}}{\tau}{\lhd}{\continue}
    \` \ruleref{\rulecheckstateloopcontname} \\
  \end{tabbing}

\item[\rulestepcontinueval] \ \\
  \newcommand{\continuevalstack}{\frameval{F}{x}{\tau_x}{v}}
  \begin{tabbing}
  \rulegetframeassigned{F'}{A}
    \` Inversion on \rulecheckstate{\Sigma}{\stackframe{K}{\continuevalstack}}{\tau}{\lhd}{\continue} \\
  \rulecheckframetype{\Sigma}{A}{F'}{\commandty}{\tau''}
    \` Inversion on \rulecheckstate{\Sigma}{\stackframe{K}{\continuevalstack}}{\tau}{\lhd}{\continue} \\
  \rulecheckstack{\Sigma}{K}{\tau''}{\tau}
    \` Inversion on \rulecheckstate{\Sigma}{\stackframe{K}{\continuevalstack}}{\tau}{\lhd}{\continue} \\
  \rulegetloopcontext{\continuevalstack}{\inloop}
    \` Inversion on \rulecheckstate{\Sigma}{\stackframe{K}{\continuevalstack}}{\tau}{\lhd}{\continue} \\
  \rulegetinnerloop{\continuevalstack}{F'}
    \` Inversion on \rulecheckstate{\Sigma}{\stackframe{K}{\continuevalstack}}{\tau}{\lhd}{\continue} \\
  \rulegetloopcontext{F}{\inloop}
    \` Inversion on \rulegetloopcontext{\continuevalstack}{\inloop} \\
  \rulegetinnerloop{F}{F'}
    \` Inversion on \rulegetinnerloop{\continuevalstack}{F'} \\
  \rulecheckstate{\Sigma}{\stackframe{K}{F}}{\tau}{\lhd}{\continue}
    \` \ruleref{\rulecheckstateloopcontname} \\
  \end{tabbing}

\item[\rulestepcontinuevar] \ \\
  \newcommand{\continuevarstack}{\framevar{F}{x}{\tau_x}}
  \begin{tabbing}
  \rulegetframeassigned{F'}{A}
    \` Inversion on \rulecheckstate{\Sigma}{\stackframe{K}{\continuevarstack}}{\tau}{\lhd}{\continue} \\
  \rulecheckframetype{\Sigma}{A}{F'}{\commandty}{\tau''}
    \` Inversion on \rulecheckstate{\Sigma}{\stackframe{K}{\continuevarstack}}{\tau}{\lhd}{\continue} \\
  \rulecheckstack{\Sigma}{K}{\tau''}{\tau}
    \` Inversion on \rulecheckstate{\Sigma}{\stackframe{K}{\continuevarstack}}{\tau}{\lhd}{\continue} \\
  \rulegetloopcontext{\continuevarstack}{\inloop}
    \` Inversion on \rulecheckstate{\Sigma}{\stackframe{K}{\continuevarstack}}{\tau}{\lhd}{\continue} \\
  \rulegetinnerloop{\continuevarstack}{F'}
    \` Inversion on \rulecheckstate{\Sigma}{\stackframe{K}{\continuevarstack}}{\tau}{\lhd}{\continue} \\
  \rulegetloopcontext{F}{\inloop}
    \` Inversion on \rulegetloopcontext{\continuevarstack}{\inloop} \\
  \rulegetinnerloop{F}{F'}
    \` Inversion on \rulegetinnerloop{\continuevarstack}{F'} \\
  \rulecheckstate{\Sigma}{\stackframe{K}{F}}{\tau}{\lhd}{\continue}
    \` \ruleref{\rulecheckstateloopcontname} \\
  \end{tabbing}

\item[\rulestepcontinueexp] \ \\
  \newcommand{\continueexpstack}{\frameexp{F}{e}}
  \begin{tabbing}
  \rulegetframeassigned{F'}{A}
    \` Inversion on \rulecheckstate{\Sigma}{\stackframe{K}{\continueexpstack}}{\tau}{\lhd}{\continue} \\
  \rulecheckframetype{\Sigma}{A}{F'}{\commandty}{\tau''}
    \` Inversion on \rulecheckstate{\Sigma}{\stackframe{K}{\continueexpstack}}{\tau}{\lhd}{\continue} \\
  \rulecheckstack{\Sigma}{K}{\tau''}{\tau}
    \` Inversion on \rulecheckstate{\Sigma}{\stackframe{K}{\continueexpstack}}{\tau}{\lhd}{\continue} \\
  \rulegetloopcontext{\continueexpstack}{\inloop}
    \` Inversion on \rulecheckstate{\Sigma}{\stackframe{K}{\continueexpstack}}{\tau}{\lhd}{\continue} \\
  \rulegetinnerloop{\continueexpstack}{F'}
    \` Inversion on \rulecheckstate{\Sigma}{\stackframe{K}{\continueexpstack}}{\tau}{\lhd}{\continue} \\
  \rulegetloopcontext{F}{\inloop}
    \` Inversion on \rulegetloopcontext{\continueexpstack}{\inloop} \\
  \rulegetinnerloop{F}{F'}
    \` Inversion on \rulegetinnerloop{\continueexpstack}{F'} \\
  \rulecheckstate{\Sigma}{\stackframe{K}{F}}{\tau}{\lhd}{\continue}
    \` \ruleref{\rulecheckstateloopcontname} \\
  \end{tabbing}

\item[\rulestepcontinueloop] \ \\
  \newcommand{\continueloopstack}{\frameloop{F}{e_c}{e}}
  \begin{tabbing}
  \rulegetinnerloop{\continueloopstack}{F'} \\
    \` Inversion on \rulecheckstate{\Sigma}{\stackframe{K}{\continueloopstack}}{\tau}{\lhd}{\continue} \\
  $F' = \continueloopstack$
    \` Inversion on \rulegetinnerloop{\continueloopstack}{F'} \\
  \rulecheckframetype{\Sigma}{A}{\continueloopstack}{\commandty}{\tau''} \\
    \` Inversion on \rulecheckstate{\Sigma}{\stackframe{K}{\continueloopstack}}{\tau}{\lhd}{\continue} \\
  \rulecheckframetype{\Sigma}{A}{F}{\commandty}{\tau''} \\
    \` Inversion on \rulecheckframetype{\Sigma}{A}{\continueloopstack}{\commandty}{\tau''} \\
  \rulegetframeassigned{\continueloopstack}{A}
    \` Inversion on \rulecheckframetype{\Sigma}{A}{\continueloopstack}{\commandty}{\tau''} \\
  \rulegetframeassigned{F}{A}
    \` Inversion on \rulegetframeassigned{\continueloopstack}{A} \\
  \rulecheckframetype{\Sigma}{A}{F}{\commandty}{\tau''}
    \` Inversion on \rulecheckframetype{\Sigma}{A}{\continueloopstack}{\commandty}{\tau''} \\
  \rulecheckstack{\Sigma}{K}{\tau''}{\tau}
    \` Inversion on \rulecheckstate{\Sigma}{\stackframe{K}{\continueloopstack}}{\tau}{\lhd}{\continue} \\
  \rulegetframecontextdefaultconclusion{}
    \` Inversion on \rulecheckframetype{\Sigma}{A}{\continueloopstack}{\commandty}{\tau''} \\
  \rulecheckexploopconclusion{}
    \` Inversion on \rulecheckframetype{\Sigma}{A}{\continueloopstack}{\commandty}{\tau''} \\
  \rulecheckassign{\Gamma}{A'}{\loopstm{e_c}{e}}{A'}
    \` Inversion on \rulecheckframetype{\Sigma}{A}{\continueloopstack}{\commandty}{\tau''} \\
  \rulegetloopcontext{F}{L}
    \` Inversion on \rulecheckframetype{\Sigma}{A}{\continueloopstack}{\commandty}{\tau''} \\
  \rulechecklooploopconclusion{}
    \` Inversion on \rulecheckframetype{\Sigma}{A}{\continueloopstack}{\commandty}{\tau''} \\
  \ruledirok{\rhd}{\loopstm{e_c}{e}}
    \` \ruleref{\ruledirokpushingname} \\
  \ruleonlyreturns{\Sigma}{\Gamma}{\loopstm{e_c}{e}}{\tau''}
    \` Inversion on \rulecheckframetype{\Sigma}{A}{\continueloopstack}{\commandty}{\tau''} \\
  \impliescmdreturn{\commandty}{\Sigma}{\Gamma}{\loopstm{e_c}{e}}{\tau''}
    \` Always true \\
  \rulecheckstate{\Sigma}{\stackframe{K}{F}}{\tau}{\rhd}{\loopstm{e_c}{e}}
    \` \ruleref{\rulecheckstatenormalname} \\
  \end{tabbing}

\item[\rulestepexnprop] \ \\

  \begin{tabbing}
  \rulecheckstate{\Sigma}{\cdot}{\tau}{\lhd}{\exn{}}
    \` \ruleref{\rulecheckstateexnname} \\
  \end{tabbing}

\end{description}

\end{appendices}

\begin{appendices}
\chapter{Standard Libraries}

These are the standard libraries as of this publication.

\newcommand{\describefunction}[2]{
\begin{tabbing}
\hspace{3em} \= \\
\> {\tt\small \lstinline|#1| } \\
\end{tabbing}
#2
}

\newcommand{\beginstruct}{
\begin{tabbing}
\hspace{3em} \= \hspace{9em} \= \\
}

\newcommand{\structfield}[2]{
\> {\tt\small \lstinline|#1|} \> #2 \\
}

\newcommand{\structend}{
\end{tabbing}
}

\section{Input/Output}

\subsection{\tt conio}

The {\tt conio} library contains functions for performing basic
console input and output.
\describefunction{void print(string s)}{
Prints {\tt s} to standard output.
}

\describefunction{void println(string s)}{
Prints {\tt s} to standard output along with a trailing newline \langtext{\\n}.
}
\describefunction{void printint(int i)}{
Prints the integer {\tt i} to standard output.
}
\describefunction{void printbool(bool b)}{
Prints the boolean {\tt b} to standard output.
}
\describefunction{void printchar(char c)}{
Prints the char c to standard output.
}
\describefunction{string readline()}{
Reads a sequence of characters from standard input followed by a newline
(either \langtext{\\n} or \langtext{\\r\\n}) and returns the sequence as a
string.  The trailing newline is not returned.
}

\describefunction{void error(string s)}{
Prints s to standard error and aborts the program.
}

\subsection{\tt file}

The {\tt file} library contains functions for reading lines out of files.
File handles are represented by the {\tt file\_t} type.  The handle contains
an internal position which ranges from 0 to the logical size of the file in
bytes.  File handles should be closed when they are no longer needed.  The
program must close them explicitly---garbage collection of a file handle
will not close it.

\describefunction{file\_t file\_read(string path)}{
Creates a handle for reading from the file given by the specified path. If
the file cannot be opened for reading, the program aborts.
}

\describefunction{void file\_close(file\_t f)}{
Releases any resources associated with the file handle. This function should
not be invoked twice on the same handle.
}

\describefunction{bool file\_eof(file\_t f)}{
Returns true if the internal position of the handle is the size of the file.
}

\describefunction{string file\_readline(file\_t f)}{
Reads a sequence of characters from the given file followed by a newline
(either \langtext{\\n} or \langtext{\\r\\n}) and returns the sequence as a
string, advancing the handle's internal position by the number of
characters in the returned string plus the trailing newline sequence.  The
trailing newline is not returned.
}

\subsection{\tt args}

The {\tt args} library provides functions for basic argument parsing. There are
several functions that set up the description of the argument schema and then a
single function (\langtext{args\_parse}) which performs the parsing.

\describefunction{void args\_flag(string name, bool *ptr)}{
Describes a simple boolean flag. If \langtext{name} is present on the
command line, \langtext{args\_parse} sets \langtext{*ptr} to
\langtext{true}.
}
\describefunction{void args\_int(string name, int *ptr)}{
Describes a switch expecting an integer of the form accepted by
\langtext{parse\_int} with \langtext{base = 10}. If \langtext{name} is
present on the command line, \langtext{args\_parse} sets \langtext{*ptr}
to the value parsed from the argument following \langtext{name}. If the
value could not be parsed, it is not set.
}
\describefunction{void args\_string(string name, string *ptr)}{
Describes a switch expecting some additional argument. If \langtext{name}
is present on the command line, \langtext{args\_parse} sets \langtext{*ptr}
to the argument following \langtext{name}.
}

The \langtext{args\_parse} function returns a pointer to a
\langtext{struct args}, which has the following members:
\beginstruct
    \structfield{int argc}{The count of unparsed arguments}
    \structfield{string[] argv}{An array containing the unparsed arguments}
\structend
By invariant, the length of the array \langtext{argv} is always
\langtext{argc}.

\describefunction{struct args *args\_parse()}{
Attempts to parse the command line arguments given to the program by the
operating system according to the argument schema described by calls to
the functions above.  Arguments that indicate a switch consume the next
argument.  Arguments that are not matched to switches or flags are
considered positional arguments and are returned in a pointer to an
\texttt{args} struct.  The result \texttt{args} struct does not contain
the name of the program itself.
}

\section{Data manipulation}

\subsection{\tt parse}

The {\tt parse} library provides two functions to parse integers and
booleans, returning pointers to the resulting data to indicate the
possibility of failure..

% These functions return pointers to integers
% and \langtext{struct parsed\_int}. \\\langtext{struct parsed\_bool} is has the
% following members:
% \beginstruct
% \structfield{bool parsed}{Indicates if the string was successfully parsed}
% \structfield{bool value}{If the {\tt parsed} field is true, holds the parsed
% value}
% \structend
% \langtext{struct parsed\_int} has the same members except that \langtext{value}
% is of type \langtext{int}.

\describefunction{bool *parse\_bool(string s)}{
Attempts to parse {\tt s} into a value of type {\tt bool}. Accepts {\tt
"true"} and {\tt "false"} as valid strings, and returns \langtext{NULL}
if {\tt s} is neither.
}
\describefunction{int *parse\_int(string s, int b)}{
Attempts to parse {\tt s} as a number written in base {\tt b}. Supported bases
range from \langtext{2} to \langtext{36}, with the letters {\tt A} through
{\tt Z} representing the digits above {\tt 9} in bases greater than {\tt 10}.
% include 8, 10 and 16. If {\tt b} is 0, the base of the number is inferred from
% the leading digits. \langtext{0x} indicates that the base is 16, otherwise
% \langtext{0} indicates base 8 and any other digit indicates base 10.
Returns \langtext{NULL} if {\tt s} cannot be completely parsed to an
\langtext{int}, or if its value would be too large to be represented as an
\langtext{int}.
}

\subsection{\tt string}

The {\tt string} library contains a few basic routines for working with strings
and characters.

\describefunction{int string\_length(string s)}{
Returns the number of characters in {\tt s}.
}
\describefunction{char string\_charat(string s, int n)}{
Returns the {\tt n}th character in {\tt s}. If {\tt n} is less than zero or
greater than the length of the string, the program aborts.
}
\describefunction{string string\_join(string a, string b)}{
Returns a string containing the contents of \langtext{b} appended to the
contents of \langtext{a}.  The result string has a length equal to the sum
of the lengths of \langtext{a} and \langtext{b}.
}

% force pagebreak -wjl
\newpage
\describefunction{string string\_sub(string s, int start, int end)}{
Returns the substring composed of the characters of {\tt s} beginning at index
given by {\tt start} and continuing up to but not including the index given
by end.
If \langtext{end == start}, the empty string is returned.
If \langtext{start} and \langtext{end} do not represent valid substring
indices, the program aborts.
}
\describefunction{bool string\_equal(string a, string b)}{
Returns \langtext{true} if the contents of \langtext{a} and \langtext{b} are
equal and \langtext{false} otherwise.
}
\describefunction{int string\_compare(string a, string b)}{
Compares \langtext{a} and \langtext{b} lexicographically. If \langtext{a} comes
before \langtext{b}, then the return value is \langtext{-1}. If
\langtext{string\_equal(a,b)} is \langtext{true}, the return value is
\langtext{0}.  Otherwise \langtext{a} comes after \langtext{b} and the
return value is \langtext{1}.
}

\describefunction{string string\_frombool(bool b)}{
Returns a canonical representation of \langtext{b} as a string. The
returned value will always be parsed by \langtext{parse\_bool} into a value
equal to {\tt b}.
}
\describefunction{string string\_fromint(int i)}{
Returns a canonical representation of {\tt i} as a string. The returned value
will always be parsed by \langtext{parse\_int} into a value equal to {\tt i}.
}
\describefunction{string string\_fromchar(char c)}{
Returns a string of length one containing the character {\tt c}.
}

\describefunction{string string\_tolower(string s)}{
Returns a string containing the same character sequence as {\tt s} but with
each uppercase character replaced by its lowercase version.
}

\describefunction{char[] string\_to\_chararray(string s)}{
Returns the characters of {\tt s}.  The length of the result array is
at least one more than the length of {\tt s}, and the end of the string is
indicated by a \langtext{'\\0'} character.
}

\describefunction{string string\_from\_chararray(char[] A)}{
Returns a string containing the characters from {\tt A} up to a terminating
\langtext{'\\0'} character.  If {\tt A} does not contain a \langtext{'\\0'}
character, the program will abort.
}

\describefunction{int char\_ord(char c)}{
Returns an integer representing the ASCII encoding of {\tt c}.
}

\describefunction{char char\_chr(int n)}{
Decodes {\tt n} as a 7-bit ASCII character and returns the result.
If {\tt n} cannot be decoded as 7-bit ASCII, the program aborts.
}

\section{Images}

\subsection{\tt img}

The {\tt img} library defines a type for two dimensional images represented as
pixels with 4 color channels---alpha, red, green and blue---packed into one {\tt
int}. It defines an image type {\tt image\_t}. Images must be explicitly
destroyed when they are no longer needed with the {\tt image\_destroy} function.

\describefunction{image\_t image\_create(int width, int height)}{
Creates an image with the given width and height. The default pixel color is
transparent black. {\tt width} and {\tt height} must be positive.
}

\describefunction{image\_t image\_clone(image\_t image)}{
Creates a copy of the image.
}

\describefunction{void image\_destroy(image\_t image)}{
Releases any internal resources associated with {\tt image}. The array returned
by a previous {\tt image\_data} call will remain valid however any subsequent calls using {\tt
image} will cause the program to abort.
}

\describefunction{image\_t image\_subimage(image\_t image, int x, int y, int w, int h)}{
Creates a partial copy of {\tt image} using the rectangle as the source
coordinates in {\tt image}. Any parts of the given rectangle that are not contained in
{\tt image} are treated as transparent black.
}

\describefunction{image\_t image\_load(string path)}{
Loads an image from the file given by {\tt path} and converts it if need be to
an ARGB image. If the file cannot be found, the program aborts.
}

\describefunction{void image\_save(image\_t image, string path)}{
Saves {\tt image} to the file given by {\tt path}. If the file cannot be
written, the program aborts.
}

\describefunction{int image\_width(image\_t image)}{
Returns the width in pixels of {\tt image}.
}

\describefunction{int image\_height(image\_t image)}{
Returns the height in pixels of {\tt image}.
}

\describefunction{int[] image\_data(image\_t image)}{
Returns an array of pixels representing the image. The pixels are given line by
line so a pixel at (x,y) would be located at \langtext{y*image\_width(image) +
x}. Any writes to the array will be reflected in calls to {\tt image\_save},
{\tt image\_clone} and {\tt image\_subimage}. The channels are encoded as
\langtext{0xAARRGGBB}.
}


\end{appendices}

\begin{appendices}
\chapter{Code Listing}

\section{Sample \tt c0defs.h}
\label{trans}
\lstinputlisting[language=C,
                 morekeywords={bool,string},
                 basicstyle=\small\tt]{c0defs.h}

\section{\langname{} runtime interface}
\label{c0runtimeinterface}
\lstinputlisting[language=C,
                 morekeywords={bool,string},
                 basicstyle=\small\tt]{c0runtime.h}

\end{appendices}


\backmatter

%\renewcommand{\baselinestretch}{1.0}\normalsize

% By default \bibsection is \chapter*, but we really want this to show
% up in the table of contents and pdf bookmarks.
\renewcommand{\bibsection}{\chapter{\bibname}}
%\newcommand{\bibpreamble}{This text goes between the ``Bibliography''
%  header and the actual list of references}
\bibliographystyle{plainnat}
\bibliography{thesis} %your bib file

\end{document}
