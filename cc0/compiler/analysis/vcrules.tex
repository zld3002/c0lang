\documentclass[11pt]{article}

\usepackage[margin=1.6in]{geometry}
\usepackage{proof}
\usepackage{amsmath,amsthm,amssymb}
\usepackage{tipa}

% font{
\usepackage{pxfonts}
%% fix sans serif
\renewcommand\sfdefault{cmss}
\DeclareMathAlphabet{\mathsf}{OT1}{cmss}{m}{n}
\SetMathAlphabet{\mathsf}{bold}{OT1}{cmss}{b}{n}
% }font

\oddsidemargin 0cm
\topmargin -2cm
\textwidth 16.5cm
\textheight 23.5cm
\setlength{\parindent}{0pt}
\setlength{\parskip}{5pt plus 1pt}

\theoremstyle{definition}
\newtheorem{task}{Task}

\newcommand\true{\;\textit{true}}
\newcommand\ddd{\raisebox{0.2em}[1.3em]{$\vdots$}}
\newcommand\com{\raisebox{0.3em}{$\ ,\ \ $}}
\newcommand\hyp[2]{\infer[#1]{#2}{}}

\newcommand\ttt{\texttt}
\renewcommand\and{\mathop{\wedge}}
\renewcommand\or{\mathop{\vee}}
\newcommand\ent{\vdash}
\newcommand\G{\Gamma}
\newcommand\D{\Delta}
\newcommand\tri{\triangleright}
\newcommand\openc{\text{\textopencorner}}
\newcommand\closec{\text{\textcorner}}
\newcommand\con[1]{\openc #1 \closec}
\newcommand\imp[2]{\G #1 \ent \ttt{#2}}
\newcommand\impt[3]{\G #1 \ent \ttt{#2} \tri #3}
\newcommand\impa[4]{#1 \ent \ttt{#2} \tri [#3;#4]}

\begin{document}


The following are the set of rules for generating verification conditions. They verify contracts, as well as check for other potential errors.

The statement $\impa{\G}{s}{\G'}{\D}$ means that given assertions $\G$, the process of verifying statement
\ttt{s}  makes new assertions $\G'$ that are known to be true after \ttt{s} is run (e.g. \ttt{x = 3}) with a list of
noncritical errors $\D$. These errors are satisfiable expressions that could result in an error, given a specific model of
the existing variables. Critical errors occur when the code can be proven to have an error on all inputs, which causes
verification to halt and return the errors.

Note that the current system makes assertions aggresively, in that it asserts whatever it can if it knows something is true,
even if that assertion is not useful or even something that can't be understood. These get filtered out when passed to Z3,
so an assertion like \ttt{A[i] = x} would (in the current implementation) not get to Z3 because Conditions wouldn't understand
how to assert \ttt{A[i]}.

Also, these rules are made with SSA in mind, since that is necessary to even make this work (otherwise we'd be asserting contradictory
things on multiple assertions, among other things). While it won't be made explicit, much of the reasoning behind certain rules will
be due to invariants upheld by SSA form. Isolation is also run on the code, so we know that anything that could have effects will
happen in an assignment as they are isolated out of their original expressions.

\section*{Statements}

\subsection*{Assignments}

\[
  \infer[]{\impa{\G}{e$_1$ = e$_2$}{\G_1 \and \G_2 \and (\ttt{e$_1$ = e$_2$})}{\D_1,\D_2}}
	  {\impa{\G \and \G_1}{e$_1$}{\G_1}{\D_1} & \impa{\G}{e$_2$}{\G_2}{\D_2}}
\]

The basic assignment form is pretty simple, with just checking the subexpressions and combining with the new assertion. However, there are a few special cases:

Assigning into an array variable will give us an assertion about its length, for use when making memory accesses.

\[
  \infer[]{\impa{\G}{A = e}{\G_1 \and (\ttt{$\backslash$length(A) = $\backslash$length(e)})}{\D_1}}
	  {\ttt{Type(A)} = \ttt{Array} & \impa{\G}{e}{\G_1}{\D_1}}
\]

A more specific case of this, to help simplify things, is the following rule.

\[
  \infer[]{\impa{\G}{A = alloc$\_$array($\_$,e)}{\G_1 \and (\ttt{$\backslash$length(A) = e})}{\D_1}}
	  {\impa{\G}{e}{\G_1}{\D_1}}
\]

When we allocate a pointer, we know that it is not NULL. We also know that it is not equal to any other existing pointers, however that knowledge
has not been implemented yet.

\[
  \infer[]{\impa{\G}{p = alloc(e)}{\G_1 \and (\ttt{p != NULL})}{\D_1}}
	  {\impa{\G}{e}{\G_1}{\D_1}}
\]

For functions, we have to check that the requires contracts hold, and afterwards we know that the ensures contracts are true for the result,
which is put in the assigned variable.

\[
  \infer[]{\impa{\G}{x = f(e$_1$,...,e$_n$)}{\G_1 \and ... \and \G_n \and \G' \and (\ttt{[x/$\backslash$result]ensures(f)})}{\D_1,..., \D_n, \D'}}
	  {\impa{\G}{e$_i$}{\G_i}{\D_i} & \impa{\G \and \G_1 \and ... \and \G_n}{requires(f,es)}{\G'}{\D'}}
\]

However if we can inline the function, then we do that instead, which is more complicated.

\subsection*{Returns}

For a \ttt{return} statement we have to make sure that the ensures statements hold for the returned expression.

\[
  \infer[]{\impa{\G}{return e}{\G_1 \and \G_2}{\D_1,\D_2}}
	  {\impa{\G}{e}{\G_1}{\D_1} & \impa{\G \and \G_1}{[e/$\backslash$result]ensures}{\G_2}{\D_2}}
\]

\subsection*{Asserts}

Asserts actually get filtered out by some part of the compiler before reaching this verification, so they are never run.
However, if they were, these rules would apply. The first rule illustrates how if the context can guarantee that e is false,
then it raises a critical error and stops. The second is when it is successful.

\[
  \infer[]{\impt{}{assert e}{ERROR}}
	  {\impa{\G}{e}{\G_1}{\D_1} & \imp{\and \G_1}{$\neg$ e}}\quad
  \infer[]{\impa{\G}{assert e}{\G_1}{\D_1}}
	  {\impa{\G}{e}{\G_1}{\D_1} & \imp{\and \G_1}{e}}
\]

This last rule only applies if neither of the above do, so when e can't be verified or contradicted.

\[
  \infer[]{\impa{\G}{assert e}{\G_1 \and \ttt{e}}{\D, \ttt{assert e}}}
	      {\impa{\G}{e}{\G_1}{\D_1}}
\]


\subsection*{Ifs}
    
We know that one of the branches must have been taken, so it must be
that either the assertions from the then branch or the assertions from the else branch are true.

\[
  \infer[]{\impa{\G}{if e then s$_1$ else s$_2$}{\G_1 \and ((\G_2 \and \ttt{e}) \or (\G_3 \and \ttt{!e}))}{\D_1,\D_2,\D_3}}
 {\impa{\G}{e}{\G_1}{\D_1} &
  \impa{\G \and \G_1 \and \ttt{e}}{s$_1$}{\G_2}{\D_2} &
  \impa{\G \and \G_1 \and \ttt{!e}}{s$_2$}{\G_3}{\D_3}}
\]
    
This is actually more of a conceptual rule, since it isn't exactly what happens (due to SSA). In reality there are phi
functions for the \ttt{if} statement. We instead keep around all assertions (both $\G_2$ and $\G_3$) and instead we 
assert the disjunction of \ttt{e} and $\neg$ \ttt{e} in relation to the phi functions.

\subsection*{Whiles}

Before verifying the loop we must check that the invariants are true. We must also check
that the invariants still hold for the changed variables at the end of the loop (which proves
the inductive quality of the invariants). We can keep around all the contexts while in the loop
since variable assignments are unique due to SSA. After the loop, only $\G_2$ is related to things
outside. Also we only know that the loop invariants are true and loop condition is false if
the loop contains no \ttt{break} statements.

\[
  \infer[]{\impa{\G}{while(e) invs s}{\G_2 \and \ttt{invs} \and \ttt{!e}}{\D_1,\D_2,\D_3,\D_4}}{
  \deduce[]{\impa{\G \and \G_1}{e}{\G_2}{\D_2}}{\impa{\G}{invs}{\G_1}{\D_1}} &
  \deduce[]{\impa{\G \and \G_1 \and \G_2 \and \G_3 \and \ttt{invs} \and \ttt{e}}{invs$_{new}$}{\G_4}{\D_4}}
           {\impa{\G \and \G_1 \and \G_2 \and \ttt{invs} \and \ttt{e}}{s}{\G_3}{\D_3}}}
\]

While not included in the rule so as to make it clearer, when asserting and checking things, we need to transform
them in accordance with the phi functions. The different uses of \ttt{invs} aren't all the same, as they need to have
the correct generations of each of the variables given their context.

\subsection*{Sequence}

\[
  \infer[]{\impa{\G}{s$_1$ ; s$_2$}{\G_1 \and \G_2}{\D_1,\D_2}}{
    \impa{\G}{s$_1$}{\G_1}{\D_1} &
    \impa{\G_1}{s$_2$}{\G_2}{\D_2}}
\]

\section*{Expressions}

\subsection*{Pointer Dereferences}

\[
  \infer[]{\impa{\G}{*e}{\G_1}{\D_1}}{\impa{\G}{e}{\G_1}{\D_1} & \imp{\and \G_1}{e != NULL}}\quad\quad
  \infer[]{\impt{}{*e}{ERROR}}{\impa{\G}{e}{\G_1}{\D_1} & \imp{\and \G_1}{e == NULL}}\quad\quad
\]

\[
  \infer[]{\impa{\G}{*e}{\G_1 \and \ttt{e != NULL}}{\D_1,\ttt{*e}}}{\impa{\G}{e}{\G_1}{\D_1}}
\]

\subsection*{Array Accesses}

\[
  \infer[]{\impt{}{A[e]}{ERROR}}{\impa{\G}{e}{\G_1}{\D_1} & \imp{\and \G_1}{$\ttt{e} < 0$}}\quad\quad
  \infer[]{\impt{}{A[e]}{ERROR}}{\impa{\G}{e}{\G_1}{\D_1} & \imp{_1}{$\ttt{e} \geq \backslash\ttt{length(A)}$}}
\]

\[
  \infer[]{\impa{\G}{A[e]}{\G_1}{\D_1}}
	  {\impa{\G}{e}{\G_1}{\D_1} & \imp{\and \G_1}{$0 \leq \ttt{e}$} & \imp{\and \G_1}{$\ttt{e} < \backslash \ttt{length(A)}$}}
\]

\[
  \infer[]{\impa{\G}{A[e]}{\G_1 \and (0 \leq \ttt{e}) \and (\ttt{e} < \backslash \ttt{length(A)})}
	      {\D_1,0 \leq \ttt{e},\ttt{e} < \backslash \ttt{length(A)}}}
	  {\impa{\G}{e}{\G_1}{\D_1}}
\]

\subsection*{Division}

\[
  \infer[]{\impt{}{$\ttt{e}_1 / \ttt{e}_2$}{ERROR}}
	  {\impa{\G}{e$_1$}{\G_1}{\D_1} & \impa{\G \and \G_1}{e$_2$}{\G_2}{\D_2} & \imp{\and \G_1 \and \G_2}{$\ttt{e}_2 == 0$}}
\]

\[
  \infer[]{\impt{}{$\ttt{e}_1 / \ttt{e}_2$}{ERROR}}
	  {\impa{\G}{e$_1$}{\G_1}{\D_1} & \impa{\G \and \G_1}{e$_2$}{\G_2}{\D_2} &
	    \imp{\and \G_1 \and \G_2}{$\ttt{e}_2 == -1$} & \imp{\and \G_1 \and \G_2}{$\ttt{e}_1 == \ttt{INT}\_\ttt{MIN}$}}
\]

\[
  \infer[]{\impa{\G}{$\ttt{e}_1 / \ttt{e}_2$}{\G_1 \and \G_2}{\D_1,\D_2}}
	  {\impa{\G}{e$_1$}{\G_1}{\D_1} & \impa{\G \and \G_1}{e$_2$}{\G_2}{\D_2} &
	    \imp{\and \G_1 \and \G_2}{$(\ttt{e}_2 \ttt{!=} 0) \and ((\ttt{e}_2 \ttt{!=} -1) \or (\ttt{e}_1 \ttt{!=} \ttt{INT}\_\ttt{MIN}))$}}
\]

And if none of the above apply,

\[
  \infer[]{\impa{\G}{$\ttt{e}_1 / \ttt{e}_2$}{\G_1 \and \G_2 \and (\ttt{e}_2 \ttt{!=} 0) \and ((\ttt{e}_2 \ttt{!=} -1) \or (\ttt{e}_1 \ttt{!=} \ttt{INT}\_\ttt{MIN}))}
					      {\D_1,\D_2,\ttt{e}_1 / \ttt{e}_2}}
	  {\impa{\G}{e$_1$}{\G_1}{\D_1} & \impa{\G \and \G_1}{e$_2$}{\G_2}{\D_2}}
\]

The above rules are the same for modulus.

\end{document}
