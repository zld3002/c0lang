One of the best ways to get students to learn is to get them excited about
projects and motivated to complete them. These interesting projects often
require supporting code such as graphics or file system access which is provided
as a library to students. The goal of the standard libraries is to provide just
enough support to students so that they can do these interesting projects. One
nice side effect is that students are briefly exposed to different fields of
computer science through these libraries and this can inform their choices of
electives later on in the program.  \footnote{The standard library is also a
good place to show students how to design a good API. The C standard library has
some poor choices in modularity and an unfortunate distaste for function names
longer than 8 characters.  Unfortunately, \langname{} does not provide any
additional features to aid modularity so the best we can do is adopt a prefix
for the different libraries}

\section{Design Principles}

The standard libraries were developed at the same time as the projects for the
course. This allowed us to tailor each library exactly to the requirements of
the assignments. The standard libraries so far just cover just a few basic
areas: images, string manipulation, and basic file/console IO. Though these are
the basics for the projects in the course, they are also sufficient for
students to explore projects beyond the scope of the course. By the end of the
course, they will know enough C (primarily from using \langname{}) to write
their own libraries.

Though allocations within the language are garbage collected, the native
libraries upon which the standard libraries are built do not use garbage
collection. As in popular modern garbage collected languages like C\#, manual
resource management is used for external resources.

\section{Design}

Carnegie Mellon provides several clusters of computers that run Linux for
students to use so our primary platform for the initial offering of the course
is these RedHat Linux-based machines. Students may have little or no experience
with Linux though and will probably prefer, at least initially, to develop on
their own machines. Since the compiler and runtimes work on at least two major
desktop platforms (OS X, Linux) and can support the third (Windows) with some
effort, this is feasible so long as the standard library also works on these
platforms.  In considering which libraries to base the standard library upon, we
tried to pick only libraries that would work with minimal hassle both now and to
support in the future.

There are several broad areas for which we would like to have libraries
support. Though some areas such as windowing are not currently needed, we must
consider which libraries to use for their implementation when considering
related areas such as graphics and images. The libraries' APIs were designed to
be as minimal as possible and we looked at libraries used in other introductory
computer science courses such as Princeton's \cite{PrincetonJavaLibs} and
Processing \cite{Processing}.

\subsection{Input/Output}

We expect that most output in the course will be for debugging purposes and not
a major focus of assignments. The console IO library (\langtext{conio}) contains
just 4 functions to handle the basic task of printing data to the screen and
the occasional user input. The following example demonstrates the basic usage:

\langsnippet{basicio.c}

The lack of a \langtext{printf}-like function adds to the verbosity but it is not a
trivial feature to add to this language. It would require the addition of C's
varargs typing to the concrete syntax and typechecker as well as the formal
semantics. The type of the function would have to be inferred from its format
string which require the compiler to understand the format string. This would
break the abstraction boundary between the compiler and libraries. An
alternative is to introduce \langtext{printf} as an intrinsic part of the
language where the complicated type checking would not need to be exposed to
programmers.

A simple file library was added to allow programs to read lines from a file as
{\tt string}s. File handles are one of the first manually managed resources
with which students interact.

\subsection{Data Manipulation}

The {\tt string} library contains basic routines for manipulating and inspecting
strings and characters. It allows strings to be joined, split, converted to and
from null-terminated character arrays, compared, generated from \langtext{int}s
and \langtext{bool}s and queried for their length.  It also allows
\langtext{char}s to be compared and converted to and from their ASCII
representation.

\subsection{Images}

The first assignment for 15-122 had students write code to manipulate
two-dimensional images. The API allows programs to load, save and manipulate
the image as a \langname{} array of integers representing the pixel data
packed as ARGB. Like file handles, images also require manual resource
management.

\subsection{Windowing}

The projects in 15-122 should require only the most basic windowing support.  It
defines a window as an abstraction of a keyboard, a mouse, and a screen - it
receives input via mouse and keyboard and displays an image. There are no child
widgets for buttons or editboxes - students can write those themselves if
desired but the text console ought to be sufficient for functionality required
the projects in the course. Furthermore, the text interface is much easier to
write automated tests for than a graphical one.

\subsection{Graphics}

For graphics support, we want to expose a minimal but powerful API. For
inspiration, we looked at Java's AWT graphics objects, HTML's {\tt <canvas>} 2D
context API, and Cairo's API. Students are able to draw and manipulate images
as well as draw shapes at arbitrary positions and scales and rotations.
Students are able to construct 2D geometric paths and perform basic queries on
them.  They can also construct transformations from primitives like
translation, rotation and scale. Paths and transforms are used with a graphics
context which is obtained from an image or a window. With a graphics context,
basic operations such as drawing/filling a path with a transformation and
clipping region allow great flexibility with a minimal API.

\section{Implementation}

Defining libraries is a rather simple process. A library named \langtext{foo}
must provide a header file \langtext{foo.h0} containing the \langname{} types
and function declarations for the library and a shared library with the same
name base name as the library (ex: \langtext{libfoo.so},
\langtext{libfoo.dylib}, \langtext{foo.dll}). The compiler is instructed to use
the library \langtext{foo} by passing \langtext{-lfoo}. It will process the
declarations and definitions in the library's header and link the executable
against the shared library.

\subsection{Native Dependencies}

Many of the basic libraries can be implemented using the standard C library
since it is available on all three major desktop platforms.

For windowing there are both many and few choices. Each platform has its own
native libraries for basic window operations but supporting separate branches
that work well on moving platforms such as OS X or diverse platforms such as
Linux is not a good long-term approach so we decided to look at cross-platform
abstractions. There are many such libraries but the popular ones are Mozilla's
XUL/Gecko library, GTK, wxWidgets, FLTK, Qt.

The size of XUL/Gecko both in terms of size and scope is simply too great for
our needs. Likewise, GTK requires quite a bit of overhead on non-Linux
platforms (particularly Windows) and does not integrate with Microsoft's
compiler and toolchain, something that students may want to do when learning C
on Windows. wxWidgets has a fairly full featured set of tools that go beyond
just windowing. It does not abstract the details of the underlying platform
that well so students may see different behavior than their grader and likewise
the graders may not be able to see problems that the students are having. FLTK
is a minimal windowing toolkit that does not attempt to look native. It
provides basic graphics support as well. Like wxWidgets, Qt provides quite a
bit of non-GUI support and though it doesn't have a native look and feel, it
comes quite close.

The choices of graphics library is closely related to that of the windowing
library. FLTK, wxWidgets and Qt provide their own graphics libraries. Out of
these three, only wxWidgets does not abstract out the details of the underlying
platform since wxWidgets relies upon the native platform for much of its
drawing. There are also dedicated graphics libraries such as Skia and Cairo
which could potentially work with whatever windowing library was chosen.

Though there are many individual libraries that would be suitable for the
implementation of the major components of the standard library, Qt has the
advantage of providing APIs for all the major requirements. Moreover, the
interaction between these separate components is already written and tested. Qt
is used by a number of major corporations and projects around the world such as
Adobe\footnote{http://qt.nokia.com/qt-in-use/story/app/adobe-photoshop-album},
KDE, Mathematica, Opera and VLC. It is thus expected that Qt will remain a
stable and supported platform upon which the standard library can be used by
students with a minimal number of bugs and support required by the course staff.
