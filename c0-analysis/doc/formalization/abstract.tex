The primary goal of a computer science course on data structures and algorithms
is to educate students on the fundamental abstract concepts and expose them to
concrete examples using programming assignments. For these assignments, the
programming language should not dominate the assignment but instead complement
it so that students spend most of their time engaged in learning the course
material. Choosing an appropriate programming language requires consideration
not just of the course material and the programs students will write but also
their prior programming experience and the position of the course in the course
sequence for their degree program.

We have developed a new programming language, \langname{}, designed for
Carnegie Mellon's new Principles of Imperative Programming course. \langname{}
is almost a subset of C, eschewing complex semantics and undefined or
unspecified behaviors in favor of simple semantics and formally specified
behavior. It provides features such as garbage collection and array bounds
checks to aid students in developing and reasoning about their programs. In
addition to the compiler and runtime, we have also developed a small set of
libraries for students to use in their assignments.

