The prior section specified the behavior of \langname{} via examples and prose
description. To make the specifications and expected behavior clear, we must
define not only the concrete syntax, but also an abstract one whose semantics
are unambiguously defined. The abstract syntax follows the concrete syntax
fairly closely but omits some features from the concrete language such as
strings, characters and annotations because they are not interesting or
essential to the formal model.

\section{Concrete syntax}

The grammar specification below does not give definitions for the terminals
\nt{identifier}, \nt{integer-constant}, \nt{character-constant} and
\nt{string-constant}. These terminals are defined as follows:

\begin{description}
\item[\nt{identifier}]
  An identifier is a sequence of letters, digits, and underscores. The first
  character must be a letter. Identifiers are case sensitive.
\item[\nt{integer-constant}]
  Integer literals come in two forms. A decimal literal is a sequence of digits
  (\langtext{0} through \langtext{9}) where the first digit is not 0.
  Hexadecimal literals begin with the two character sequence 0x and are followed
  by one or more hexadecimal digits (\langtext{0} through \langtext{9},
  \langtext{a} through \langtext{f}). Hexadecimal literals are case insensitive.
\item[\nt{character-constant}]
  A character constant is a single ASCII character or one of the permitted
  escape sequences enclosed in single quotes (\langtext{'x'}). The character
  \langtext{'} must be escaped. The following escape sequences are
  permitted:

  \begin{tabular}{ll|ll}
  \langtextish{\textbackslash{}t} & tab &
  \langtextish{\textbackslash{}b} & backspace \\
  \langtextish{\textbackslash{}r} & return &
  \langtextish{\textbackslash{}n} & newline \\
  \langtextish{\textbackslash{}'} & single quote &
  \langtextish{\textbackslash{}"} & double quote \\
  \langtextish{\textbackslash{}0} & null & \ & \ \\
  \end{tabular}
\item[\nt{string-constant}]
  A string constant (or literal) is a sequence of characters and escape
  sequences enclosed in double quotes (\langtext{"Hello"}). The set of
  permitted escape sequences for strings is the same as for characters.
\end{description}

\grammarbegin
\grmboolconst

\declprod{pointer-constant}
\proddef{\lit{NULL}}

\declprod{constant}
\proddef{\nt{bool-constant}}
\proddef{\nt{integer-constant}}
\proddef{\nt{character-constant}}
\proddef{\nt{string-constant}}
\proddef{\nt{pointer-constant}}

\declprod{variable-name}
\proddef{\nt{identifier}}

\declprod{function-name}
\proddef{\nt{identifier}}

\declprod{type-name}
\proddef{\nt{identifier}}

\declprod{field-name}
\proddef{\nt{identifier}}

\declprod{monop}
\proddef{any of \lit{* ! - \~{\ }}}

\declprod{binop}
\proddef{any of \lit{+ - * / \% \& | \^{\ } \&\& |\hspace{-1em}| => << >> < > <= >= == !=}}

\declprod{assign-op}
\proddef{any of \lit{= += -= *= /= \%= \&= |= \^{}= <<= >>=}}

\declprod{post-op}
\proddef{any of \lit{++ --}}

\grmexprcommon
\grmexprarray
\proddef{\nt{function-name}}
\proddef{\nt{call-expression}}
\proddef{\nt{expression} \lit{->} \nt{field-name}}
\proddef{\nt{expression} \lit{.} \nt{field-name}}
\proddef{\nt{expression} \lit{?} \nt{expression} \lit{:} \nt{expression}}
\proddef{\lit{alloc} \lit{(} \nt{ty} \lit{)}}
\proddef{\lit{alloc\_array} \lit{(} \nt{ty} \lit{,} \nt{expression} \lit{)}}
\proddef{\lit{(} \nt{expression} \lit{)}}

\grmcallexpr

\declprod{argument-list}
\proddef{\nt{expression}}
\proddef{\nt{argument-list} \lit{,} \nt{expression}}

\grmaebasic
\grmaeptr
\grmaearr
\grmaestr

\grmlhsbasic
\grmlhsptr

\grmdeclstatement
\grmdeclexpstatement

\grmassignstatement
\grmcallstm
\proddef{\nt{atomic-expression} \nt{post-op}}

\grminitializerstatement{}

\declprod{jump-statement}
\proddef{\lit{return} \maybe{\nt{expression}} \lit{;}}
\grmbreakstatement
\grmcontinuestatement

\grmselectionstatement

\declprod{iteration-statement}
\grmwhilestatement
\grmforstatement

\grmcompoundstatement

\declprod{statement}
\proddef{\nt{initializer} \lit{;}}
\proddef{\nt{jump-statement}}
\proddef{\nt{selection-statement}}
\proddef{\nt{iteration-statement}}
\proddef{\nt{compound-statement}}

\declprod{struct-decl}
\proddef{\lit{struct} \nt{struct-name}}

\declprod{ty}
\proddef{\nt{type-name}}
\proddef{\lit{void}}
\proddef{\lit{bool}}
\proddef{\lit{int}}
\proddef{\nt{ty}\lit{*}}
\proddef{\nt{ty}\lit{[]}}
\proddef{\nt{struct-decl}}

\declprod{struct-def}
\proddef{\nt{struct-decl} \lit{\{} \maybe{\nt{struct-fields}} \lit{\}}}

\declprod{struct-fields}
\proddef{\nt{field-decl}}
\proddef{\nt{struct-fields} \nt{field-decl}}

\declprod{field-decl}
\proddef{\nt{ty} \nt{field-name} \lit{;}}

\grmfundecl

\grmfundef

\declprod{function-param}
\proddef{\nt{ty} \nt{variable-name}}

\declprod{function-params}
\proddef{\nt{function-param}}
\proddef{\nt{function-params} \lit{,} \nt{function-param}}

\declprod{type-def}
\proddef{\lit{typedef} \nt{ty} \nt{type-name} \lit{;}}

\declprod{gdecl}
\proddef{\nt{struct-decl} \lit{;}}
\proddef{\nt{struct-def} \lit{;}}
\proddef{\nt{type-def}}
\proddef{\nt{function-def}}
\proddef{\nt{function-decl} \lit{;}}

\declprod{program}
\proddef{\nt{gdecl}}
\proddef{\nt{program} \nt{gdecl}}

\grammarend

The binary and unary operators have the following precedence and associativity in order from highest to least:\\
\begin{tabular}{l|l}
Operators & Associativity \\
\hline
\langtext{()} \langtext{[]} \langtext{->} \langtext{.} & Left \\
\langtext{!} \langtext{\~{\ }} \langtext{-} \langtext{*} & Right \\
\langtext{*} \langtext{/} \langtext{\%} & Left \\
\langtext{+} \langtext{-} & Left \\
\langtext{<<} \langtext{>>} & Left \\
\langtext{<} \langtext{<=} \langtext{>} \langtext{>=} & Left \\
\langtext{==} \langtext{!=} & Left \\
\langtext{\&} & Left \\
\langtextish{\^{}} & Left \\
\langtextish{|} & Left \\
\langtext{&&} & Left \\
\langtext{||} & Left \\
\langtext{=>} & Left \\
\langtext{?} \langtext{:} & Right \\
\langtext{=} \langtext{op=} & Right \\
\end{tabular}

There is one ambiguity in the grammar which is resolved as follows:
\begin{description}
\item[Matching \langtextish{else} branches in nested \langtextish{if} statements] \ \\
As in C, in the case of \langsnippet{ifelseambiguity.c} the \langtext{else}
branch is bound to the innermost \langtext{if} statement.
\end{description}

\subsection{Abstract syntax}

We start with the abstract syntax's types $\tau$.  There are two basic types:
\bool{} and \integer.  Given a type $\tau$ we can construct pointer and array
types $\pointer{\tau}$ and $\arraytype{\tau}$.  Function types are denoted as
\function{\tau'}{\tau} Structure types are represented as \struct{p} where $p$
describes the fields.  $p$ is an ordered, possibly empty, mapping of field
names to types. The definition of $p$ is written as follows:

\classp

Two other types are used internally in the abstract syntax - they cannot be
writte in the concrete language. The first is the command type \commandty{}
which represents the type of statements such as loops or declarations. The other
is a tuple type \tuple{\tau_1, \ldots, \tau_n} which is composed of 0 or more
types. These are used in the elaboration of the concrete \langtext{void} type
and as the type for function arguments. The full definition of $\tau$ is written
as follows:

\classtau

The heap is modeled as a mapping of addresses $a$ to values or functions. The
addressing model in the abstract syntax is more expressive than that of the
concrete syntax. In particular it allows addresses into arrays and structure
fields. A basic address is just a label that identifies some value or function
in the heap. These are obtained from evaluating \langtext{alloc}. An address may
also be formed by combining a label and an integer offset. These are used in
intermediate expressions involving arrays such as \langtext{a[i] += 9;} where a
temporary pointer needs to be created to point to the \langtext{i}th index of
\langtext{a}. Similarly, structure fields can be addressed by combining an
address with a field name. Structure fields addresses need to use an arbitrary
address instead of a label because structures can nest inside each other and can
be used as the element type of an array. Thus the definition of addresses is
writen as follows:

\classa

Values are rather straightforward. \true{} and \false{} are inhabitants of the
\bool{} type. \intliteral{n} represents an integer $n$. \address{a} is a pointer
with address $a$ with type $\pointer{\tau}$ where the value at address $a$ has
type $\tau$. Similarly, \arrayval{a}{n} is a reference to an array with base
address $a$ of length $n$. \structval{a} is a reference to a structure with base
address $a$. \func{x_1, \ldots, x_n, e} represents a function with bound
variables $x_1, \ldots x_n$ and function body $e$. Type \commandty{} has one
value \nop{} which represents a command that does nothing. There are no values
of type \tuple{\ldots} however the empty tuple \tuple{} is treated as a value
for the purposes of returning from functions.

\classv

Unlike most other languages, expressions are permitted to have both local and
non-local side effects. They encompass both statements and expressions in the
concrete language. The simplest expressions are variables $x$ and values $v$.
Binary and unary operations are represented with \binop{op}{e_1}{e_2} and
\monop{op}{e}. Tuple expressions look just like their concrete counterpart.
Function calls are represented as \call{e_f, e} where $e_f$ is an expression for
a pointer to the function to be called and $e$ is the expression for the
argument tuple. \langtext{if} statements and the conditional operator are
represented by the \ifexp{e_c}{e_t}{e_f} structure. Variable declarations
\decl{x}{\tau}{e} restrict the declared variable's scope to $e$. \assign{x}{e}
represents assignments to local variables. \return{e} represents the
\langtext{return} statement from the concrete language. \loopstm{e_c}{e}
represents a basic \langtext{while} loop with condition $e_c$ and body $e$.
\langtext{break} and \langtext{continue} are straightforward. \allocexp{\tau} is
an expression whose evaluation allocates a value of type $\tau$ on the heap.
Similarly, \allocarrayexp{\tau}{e} is an expression whose evaluation allocates
an array of type \arraytype{\tau} with length given by the value of $e$.

Noticeably missing from these expressions are memory reads/writes and sequencing
of expressions. These operations are represented as binary and unary operations.
The \opseq{} binary operator sequences two statements. The \opwrite{} binary
operator takes an address and a value and writes the value to the address in
memory (or generates an exception). Similarly, the \opread{} unary operator
takes an address and produces the value at that address (or an exception). The
unary \opfield{f} and binary \oparrayindex{} provide a means for "pointer
arithmetic" so that addresses to fields of structures and elements of arrays
can be produced. Finally, the \opign{} unary operator allows for the evaluation
of an expression for its side effects to be treated as a command.

\classe

\classop

To statically check if \breakstm{} and \continue{} are used in loops, this
simple structure is used:

\classL

When evaluating a function, it is necessary to store some information such as
pending expressions and the values of local variables. This information is
stored in function frames $F$. A function frame is an ordered list of the
following: variables that are declared but not assigned (\framevar{F}{x}{\tau}),
the values of assigned variables (\frameval{F}{x}{\tau_x}{v}), expressions that are yet
to be evaluated pending the evaluation of their subexpressions
(\frameexp{F}{e}), and a construct to indicate a pending loop
(\frameloop{F}{e_c}{e}). The latter is used for the execution of the \breakstm{}
and \continue{} expressions.

\classF

A stack $K$ is an ordered list of function frames:

\classK

The top of the stack $t$ is represented explicitly. It is either an expression
or an exception.

\classt

There is also a direction associated with the top to indicate if it is needs
potentially further evaluation ($\rhd$) or if it is ready to be used in
evaluating the rest of the call stack ($\lhd$).

\classbowtie

Evaluation involves two primary components:

\begin{itemize}

\item An abstract heap $\mu$ with signature $\Sigma$ containing values $v$ at
addresses $a$.

\item An abstract stack $K$ with top $t$ and direction $\bowtie$ yielding a value of type $\tau$.

\end{itemize}

The initial heap contains entries for the functions and globals defined by the
program and the initial stack is empty and the initial top is a call to the
"main" function of the program with no arguments.

Evaluation terminates when either a value or exception is on top and the rest of
the stack is empty.

\section{Elaboration to Abstract Syntax}

In order to formalize the language definition, we must provide an elaboration from the concrete syntax to the abstract syntax. Though mostly straightforward, there are some non-trivial portions of the elaboration:
\begin{description}
\item[Declarations]
In the concrete syntax, declarations extend to the end of the block. In the abstract syntax, there is no notion of "block." Instead declaration statements explicitly contain the scope of the declaration. When elaborating a declaration in a block, the rest of the block is first elaborated into some abstract statement $s$ and then the declaration can then wrap $s$.
\item[Short-circuit boolean operators]
The abstract syntax has no notion of the short-circuiting logical boolean operators. Both of them can be desugared into a use of the ternary operator as follows:
\begin{tabular}{lll}
\langtext{a \&\& b} & $\Rightarrow$ & \langtext{a ? b : false} \\
\langtext{a \|\| b} & $\Rightarrow$ & \langtext{a ? true : b} \\
\end{tabular}

The ternary operator \langtext{a ? b : c} elaborates to \ifexp{a}{b}{c}.
\item[Empty statements]
These elaborated to \monop{\opign}{\false}
\item[For-loops]
The \langtext{for} loop can be transformed into a \langtext{while} loop to elaborate it.
The post-body statement is inserted into the loop body at the end and at every location with a non-nested \langtext{continue}. Call this \langtext{body'}.
The loop \langtext{for (init; cond; post) body}

is rewritten as \langtext{ init; while(cond) body' }.

\item[Array indices]
\langtext{a[i]} is elaborated as \monop{\opread}{\binop{\oparrayindex}{a}{i}} where $a$ and $i$ are the elaborations of \langtext{a} and \langtext{i} respectively.

\item[Structure field projection]
Field projection comes in two forms. The first \langtext{e->f} elaborates to \monop{\opread}{\monop{\opfield{f}}{e}}. The second \langtext{e.f} is trickier to handle since the type of $e$ in a well-typed program should be a structure type, not a pointer type. By induction on the elaboration of \langtext{e} we know that the elaboration of \langtext{e} can be canonicalized the form \monop{\opread}{e'}. Intuitively this makes sense since all structures are allocated on the heap. Thus, \langtext{e.f} elaborates to \monop{\opread}{\monop{\opfield{f}}{e'}}.

\item[Compound assignment]
For compound assignment, the elaboration depends on the left hand side.
\begin{description}
\item[\langtextish{x op= e}] This elaborates to \assign{x}{\binop{op}{x}{e}}.
\item[\langtextish{*p op= e}] The left hand side elaborates to \monop{\opread}{p} where $p$ is the elaboration of \langtext{p}. Thus the elaboration becomes \\\decl{p'}{\tau*}{\binop{\opseq}{\assign{p'}{p}}{\monop{\opign}{\binop{\opwrite}{p'}{s}}}} where $p'$ is a fresh variable and $s = \binop{op}{\monop{\opread}{p'}}{e}$.
\item[Arrays and structures]
Since array index expressions and structure field projections elaborate to the same form as pointers, the procedure is the same as with pointers.
\end{description}

\item[Functions with \langtextish{void} return type]
The \langtext{void} type is elaborated to \abslang{\tuple{}} and
empty return statements are simply elaborated to \abslang{\return{\tupleexp{}}}.
Since these functions are not required to have a return statement to terminate
them, one is implicitly added to the end of the function body.

\item[Structure type elaboration]
Because the abstract syntax does not have nominal typing, structural types in
the concrete syntax are elaborated by using the name of the structure to rename
the fields of the structure. References in code to these fields must also be
renamed. Structure types without any fields must remain distinct in the
elaboration so a dummy field must be added to them prior to elaboration
otherwise \langtext{struct F \{\}} and \langtext{struct G \{\}} would both
elaborate to \struct{\cdot}. Here are some examples:

\begin{tabular}{ll}
Concrete & Abstract \\
\langtext{struct F \{\};}
  & \struct{\varctx{\cdot}{F\$\$dummy}{\bool}} \\
\langtext{struct G \{ bool dummy; \}}
  & \struct{\varctx{\cdot}{G\$dummy}{\bool}} \\
\langtext{struct Point \{ int x; int y; \}}
  & \struct{\varctx{\varctx{\cdot}{Point\$x}{\integer}}{Point\$y}{\integer}} \\
\langtext{struct Size \{ int x; int y; \}}
  & \struct{\varctx{\varctx{\cdot}{Size\$x}{\integer}}{Size\$y}{\integer}} \\
%\langtext{struct Rect \{ struct Point p; struct Size s; \}} &
\end{tabular}

Since \$ is not permitted in field names or structure names in \langname{}, it
is safe to use as a delimeter.

\item[Order of declarations for functions/types and typedefs]
The abstract syntax does not take the order of declarations into account. The
elaborator can ignore the declarations and elaborate the definitions. Typedefs
are resolved to their underlying structural type during elaboration.

\end{description}

\newcommand{\abstractstep}{
$\prgstate{\mu}{K}{\bowtie}{t} \rightarrow \prgstate{\mu'}{K'}{\bowtie'}{t'}$
}
\newcommand{\progresstheorem}{ If $\mu : \Sigma$ and
\rulecheckstatedefaultconclusion{} then either
\rulefinalstatedefaultconclusion{} or \abstractstep{} for some
$\mu'$, $K'$, $\bowtie'$, and $t'$. }

\newcommand{\preservationtheorem}{
If $\mu : \Sigma$ and \rulecheckstate{\Sigma}{K}{\tau}{\bowtie}{t} and \abstractstep{} then $\exists \Sigma'. \mu' : \Sigma'$ and \rulecheckstate{\Sigma'}{K'}{\tau}{\bowtie}{t'} and $\Sigma \le \Sigma'$.

}

\section{Proof of safety}

The proof of safety for this abstract language is given by two theorems called progress and preservation. They are defined in \ref{progressproof} and \ref{preservationproof}.

\begin{description}
\item[Progress] \ \\
\progresstheorem{}

\item[Preservation] \ \\
\preservationtheorem{}

\end{description}
