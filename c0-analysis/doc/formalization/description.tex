\langname{} is a statically typed imperative programming language. It strongly
resembles C in syntax and semantics. In designing it, we tried to stay close to
C's syntax and semantics in the hopes that students could simply include a
header file in each \langname{} program and it would be accepted by a C
compiler.  Unfortunately, some of \langname's semantics differ from those of C
and in these cases, using C's syntax would be a confusing choice for students
when it came time for them to learn C. Thus we changed the syntax for these
constructs to the syntax of modern languages with similar semantics. This made
it impossible for there to be a header that will allow \langname{} programs to
be compiled as C.  However, the transformation remains a simple operation with a
trivial {\tt sed} script or doing it by hand for small programs.

\section{Basic data types and expressions}

Expressions are evaluated in left-to-right order (that is, a post-order
traversal of the abstract syntax tree). Expressions come in many forms. The
common ones are:

\grammarbegin
\grmexprcommon
\grammarend

Expressions in \langname{} are not pure; they may have side effects such as
modifying memory or raising exceptions.

\subsection{Booleans}
\langname{} has a few basic builtin types. The simplest of these is the boolean
type. Similar to C++, Java and C\#, we write the type as \langtext{bool} and its
two values are \langtext{true} and \langtext{false}.

\grammarbegin
\grmboolconst

\declprod{constant}
\proddef{\nt{bool-constant}}

\declprod{ty}
\proddef{\lit{bool}}
\grammarend

\langsampleinline{bool b = true \|\| false \&\& !(x == y);}

The usual set of basic boolean operations are provided for logical and
(\langtext{\&\&}), logical or (\langtext{\|\|}), logical not (\langtext{!}), and
equality/inequality (\langtext{==},\langtext{!=}). As in many languages, these
operators can "short circuit" their evaluation when the value from evaluating
the left hand side determines the expression's value.

There is also a conditional expression as in C and Java: \langtext{e ? a : b}
where \langtext{e} is an expression that yields a boolean. If it evaluates to
\langtext{true} then \langtext{a} is evaluated to be the result of the ternary
expression. Otherwise \langtext{b} is evaluted to be the result of the ternary
expression.

\subsection{Integers}

\langname{} has a single numerical type \langtext{int} which has a range of
$[-2^{31},2^{31})$. It is represented as a signed 32 bit two's complement
integer. We felt that it would be more helpful to specify the exact size,
numerical range and representation of the integer type.  In addition, having a
single numerical type greatly simplifies the language and cognitive requirements
for students. It is an unfortunate limitation that the numerical type is
integral as there are a number of interesting data structures and projects that
would be much easier to represent as some non-integral type like IEEE 754
floating point or .NET's decimal type.  The problem with these data types
however is that they are only approximations of real numbers so students will
need to be made aware of their underlying implementation and properties.  These
properties are non-trivial and the cognitive overhead required for students to
understand them is not worth the time and effort when there are already many
interesting data structures within the scope of the course that work well with
integers. Furthermore, these numerical types and their representations are
covered in the subsequent introductory systems course, 15-213.

Integer literals are written as a sequence of digits, either hexadecimal digits
with an \langtext{0x} prefix as in C or decimal digits with an optional
\langtext{-} prefix to indicate negative numbers. Basic arithmetic operations
include modular addition, subtraction, negation, multiplication, division and
remainder operations (\langtext{+ - * / \%}). For the division and remainder
operations, if the divisor is 0 or additionally for
negation if the result does not fall into the range of the \langtext{int} type,
an exception is triggered which results in the termination of the program. The
division and remainder operations round towards 0.

\grammarbegin
\declprod{ty}
\proddef{\lit{int}}

\declprod{constant}
\proddef{\nt{integer-constant}}
\grammarend

\langsampleinline{int x = 0x123 + -456 * 20;}

Basic bitwise operations are also supported: bitwise and (\langtext{\&}),
bitwise or (\langtext{\|}), bitwise xor ({\tt \^{}}), bitwise inversion ({\tt \~{}}).
Arithmetic shift operations are also supported with the \langtext{<<} and
\langtext{>>} binary operators. Only the least significant 5 bits of the
shift's two's complement binary representation are used to determine the number
of bits that are shifted. These 5 bits are treated as an unsigned integer. For
example, the expression \langtext{\~19 << 2} evaluates to \langtext{-80}.

\subsection{Strings and Characters}

\langname{} provides a basic opaque type \langtext{char} for representing
characters. The exact representation is not defined however they may be
specified by a printable ASCII character enclosed in single quotes such as
\langtext{'a'} and \langtext{'9'}. To accommodate some useful but not printable
ASCII characters, the following escape sequences are also permitted:
\langtextish{\textbackslash{}t} (tab), \langtextish{\textbackslash{}r} (return),
\langtextish{\textbackslash{}b} (backspace), \langtextish{\textbackslash{}n}
(newline), \langtextish{\textbackslash{}'} (single quote),
\langtextish{\textbackslash{}"} (double quote) and
\langtextish{\textbackslash{}0} (null).

\langname{} also defines an opaque immutable string type \langtext{string} which
contains a sequence of characters. There is no direct conversion between strings
and characters however several library functions such as
\langtext{string_charat} and \langtext{string_to_chararray} provide access to
the characters. String literals are written as a sequence of ASCII characters or
permitted \langtext{char} escape sequences (except \langtextish{\textbackslash
0}) enclosed in double quotes. Strings may be compared using the library
functions \langtext{string_equal} and \langtext{string_compare}.

\langtext{string s = "Hello world";}

\grammarbegin
\declprod{ty}
\proddef{\lit{char}}
\proddef{\lit{string}}

\declprod{constant}
\proddef{\nt{character-constant}}
\proddef{\nt{string-constant}}
\grammarend

\subsection{Comparisons}

Owing to its C heritage, \langname{} provides equality and inequality operators
as \langtext{==} and \langtext{!=} respectively for expressions of the same
type. These operators are defined for the \langtext{bool}, \langtext{int} and
\langtext{char} types as well as pointer types. Pointer types are checked only
for reference equality.  Comparisons on integers are also provided with the
standard and self-explanatory concrete syntax from C (\langtext{< <= > >=}).

\section{Variables \& Statements}

For now we will consider only three kinds of statements: blocks, variable
declarations and assignments. As in C, blocks are defined as a possibly empty
sequence of statements enclosed in curly braces (\langtext{\{ \}}). They are
used for grouping related statements and determining variable scope. A block
statement is executed by executing each of its statements in sequence.

\grammarbegin
\grmcompoundstatement
\grammarend

Variables are automatically managed memory locations that hold values.
Variables are identified using a sequence of alphanumeric characters (though
the identifier must start with a letter). Variable declarations such as
\langtext{int x;} introduce a new variable that can be used by expressions and
assignments. As in C99, a variable's scope is defined as the statement
following the variable declaration to the end of the innermost block the
declaration occured in.

\grammarbegin
\grmdeclstatement
\grammarend

Variable declaration restrictions:
\begin{enumerate}

\item A variable may only be referenced by expressions and statements inside
its scope.

\item A variable may not be used in an expression if it cannot be statically
verified that it has been assigned prior to use.  That is, an assignment
statement must dominate all uses in the control flow graph of the block
containing the variable's scope. Loops are assumed to potentially never run for
the purpose of this analysis.

\item Two variables of the same name may not have overlapping scopes.

\item The variable's type must be \langtext{bool}, \langtext{int}, a pointer
type or an array type.

\end{enumerate}

Ex:

\langsnippet{assignmentsample.c}

Assignment statements are used to update values stored in variables. The variable on the left hand
side must be in scope.

\grammarbegin
\grmassignstatement

\grmaebasic

\grmlhsbasic

\grammarend

For convenience, there is also hybrid class of assignment statements that
combine a binary operator with assignment for the common case where the left
hand side of a binary operation is the left hand side of the assignment as well.
For the assignment to variables, \langtext{x op=e} is simple syntactic sugar
for \langtext{x = x op e}. Note that \langtext{ *p op= e } cannot
simply desugar to \langtext{ *p = *p op e} when the evaluation of \langtext{p}
may have side effects. For this reason, the left hand side is
evaluated before the right hand side and only once.

There is a hybrid statement that is formed by combining a variable declaration
with an assignment:

\grammarbegin
\declprod{declaration}
\grmdeclexpstatement
\grammarend

This is merely syntactic sugar: $\langtext{T x = e;} \equiv \langtext{T x; x =
e;}$

\section{Functions} Functions are the primary mechanism for abstracting
computations and reusing code. A function declares parameters which behave much
like variable declarations at the top of the function's body except that they
are guaranteed to be assigned some value upon entry to the body of the function.
A function's body is executed when the function is applied to some arguments.
The arguments are evaluated to values in the conventional left-to-right
ordering. These values are then bound to the function's parameters in the same
order and the body of the function is then executed in a new scope where only
those parameters are bound. This means that functions cannot refer to variables
declared in other functions. Thus each function can be typechecked independently
from the definitions (but not declarations) of other functions.

\grammarbegin
\grmcompoundstatement

\grmfundecl

\grmfundef
\grammarend

Function application may occur as a top-level statement (that is, for its
side-effects only) or as part of an expression, including arguments to other
function applications. As an expression, function application must evaluate to
some value. To indicate this value, a function uses the \langtext{return}
statement. The \langtext{return} statement evaluates an expression to a value
and then aborts execution of the callee's body, yielding the value as the result
of the function application in the caller. To indicate what type of value is
returned, functions declare a return type. All functions must return a value and
must only return values of the declared return type. As with checking for
assignment before use for variables, a conservative analysis is used so loops
are not considered to necessarily be run, even in the trivial cases.

\grammarbegin
\grmcallexpr

\declprod{simple-statement}
\grmcallstm
\grammarend

For example, suppose we have:

\langsnippet{simplefunction.c}

which is a function named \langtext{triarea} that returns a value of type
\langtext{int} and has two parameters of type \langtext{int}, \langtext{b} and
\langtext{h}. \langtext{triarea(3,4)} is an application of \langtext{triarea} to
the expressions \langtext{3} and \langtext{4}. To evaluate this application, the
arguments are evaluated into values which they already are, then they are bound
in a new scope to the variables \langtext{b} and \langtext{h}. The return
statement then evaluates \langtext{b*h/2} to a value (the integer represented by
\langtext{6}) which then becomes the result of the original application.

Since the parameters are bound to values, a callee cannot affect the variable
bindings in its caller's scope\footnote{It can effect a change through the heap
with more advanced types which refer to the heap such as pointers and arrays.
The heap is discussed in section ~\ref{heapsection}}.

Functions may be named with any identifier using the same rules as for
variables. Function and variable names may not conflict. Like C, a function
needs to be declared before it can be used in another function.

\langsnippet{callbyvalue.c}

\section{Conditionals}

\langname{} offers a subset of the control flow statements in C that are found in
other popular languages. The most basic one is \langtext{if} statement. The
\langtext{if} statement allows execution the execution of a statement
conditional upon some provided condition. It has two forms:

\grammarbegin
\grmselectionstatement
\grammarend

Each time the \langtext{if} statement is executed, the condition is evaluated
and the corresponding inner statement (if there is one) is executed. To
repeatedly execute a statement until a condition no longer holds, the
\langtext{while} statement can be used. It behaves just like the first form of
an \langtext{if} statement that is re-executed until the condition is false.

\grammarbegin
\declprod{iteration-statement}
\grmwhilestatement
\grammarend

\langsnippet{whilesample.c}

There are two additional statement types that are available for use within a
loop: \langtext{break} and \langtext{continue}. Whenever a \langtext{continue}
statement is executed, execution jumps to the top of the current innermost loop
as if the body were done executing. Whenever a \langtext{break} is executed
within a loop, the current innermost loop is "broken" and execution resumes at
whatever statement or condition follows that loop, as if the condition were
\langtext{false}.

\grammarbegin
\declprod{jump-statement}
\grmbreakstatement
\grmcontinuestatement
\grammarend

There is an additional form of loop, the \langtext{for} loop. It behaves much
like C's: there is an initial statement that is executed before the loop, a
condition that is checked before every iteration, much like the while loop, and
a post-loop-body statement which is executed after the body of the loop.

\grammarbegin
\grminitializerstatement{}

\declprod{iteration-statement}
\grmforstatement
\grammarend

Either of the two statements may be omitted and if the condition is omitted, it
is assumed to be \langtext{true}. The initial statement may be a declaration
but the post statement may not. To give the same behavior as C99/C++, the for
loop is wrapped in an implicit block so the scope of a declaration in the
initial statement is limited to the condition, body, and post statement.

As with \langtext{while} statements, the body is repeatedly executed while the
condition is true. After the execution of the body, the post statement is run.
The \langtext{continue} statement behaves slightly differently in a
\langtext{for} loop; it jumps to the end of the body so that the post-loop-body
statement is executed. The \langtext{break} statement behaviors exactly the
same as in the \langtext{while} loop - it does not execute the post statement.

\section{The Heap \& Pointers} \label{heapsection}

The heap is a separate persistent memory store, accessible from any function.
Access to values in the heap is done only through references. References to
values in the heap (addresses) are stored as pointer values.  These pointer
values may be copied around and passed to functions just as any other value is.
They have a type dependent upon the type value they reference.  This type is
written as \langtext{T*} where \langtext{T} is the type of the value pointed
to.

As with all local variables, pointers must be initialized before use. A pointer
may be initialized to either a new value on the heap, the address of an existing
value in the heap, or a special value called \langtext{NULL}. Attempts to
reference the heap using \langtext{NULL} will abort execution. It exists solely
to indicate an invalid reference.

New values on the heap are created with the \langtext{alloc} keyword like so:

\grammarbegin
\declprod{expression}
\proddef{\lit{alloc} \lit{(} \nt{ty} \lit{)}}
\grammarend

Because it is difficult to determine if a value in the heap is initialized
before being used, all values in the heap are initialized to type-dependent
default values as follows:

\begin{tabular}{r|l}

\langtext{bool} & \langtext{false} \\

\langtext{int} & \langtext{0} \\

\langtext{char} & \langtext{'\0'} \\

\langtext{string} & \langtext{""} \\

\langtext{T*} & \langtext{NULL} \\

\langtext{T[]} & An array of dimension 0 \\

\langtext{struct { ... } } & Each field initialized according to its type.  \\

\end{tabular}

Heap values may be read by de-referencing a pointer value. Because pointers
always point to the heap, there is no explicit de-allocation of heap memory (no
\langtext{free}, no casting of pointer types and no pointer arithmetic in the
language, a pointer value is either \langtext{NULL} or a valid address in the
heap. There is also no means to obtain a pointer to a variable. The reasons for
this are discussed in section \ref{noaddressof}.

\grammarbegin
\declprod{monop}
\proddef{\lit{*}}

\declprod{lhs}
\grmlhsptr
\grammarend

\langsnippet{heapsample.c}

\section{Arrays}

\langname{} supports variable-sized mono-typed-element arrays much like those
found in Java or C\#. Array access is checked against the bound of the array,
yielding a runtime error if the bounds are exceeded. Arrays are passed around by
reference; their values are stored in the heap.

Arrays have their element type declared as in Java: \langtext{int[] x;} This is
a departure from the syntax of C but the semantics are also different. The
semantics of \langname{} arrays are identical (modulo bounds checking) to those
of pointers of the same base type in C.

An array is created by the \langtext{alloc\_array} which takes the array's
element type and the array dimensions. If the dimensions are less than 0, a
runtime error is issued and program execution is aborted. Arrays of dimension 0
are permitted but rather useless. All elements of the new array are initialized
as described in section \ref{heapsection}.

Array access uses the familiar C syntax:

\grammarbegin
\declprod{expression}
\grmexprarray

\declprod{atomic-expression}
\grmaearr
\grammarend

\langsnippet{arraysample.c}

\section{Structures}

\langname{} includes structure types like those found in C - a nominally-typed
set of name/value pairs (fields). Because structures allow for recursive types,
there are some additional complications so that compatibility with C is
maintained. Structure types may not be declared in functions - they are instead
declared at the global scope along with functions.

\grammarbegin
\declprod{ty}
\proddef{\nt{struct-decl}}

\declprod{struct-decl}
\proddef{\lit{struct} \nt{struct-name}}

\declprod{struct-def}
\proddef{\nt{struct-decl} \lit{\{} \maybe{\nt{struct-fields}} \lit{\}}}

\declprod{struct-fields}
\proddef{\nt{field-decl}}
\proddef{\nt{struct-fields} \nt{field-decl}}

\declprod{field-decl}
\proddef{\nt{ty} \nt{field-name} \lit{;}}

\declprod{gdecl}
\proddef{\nt{struct-decl} \lit{;}}
\proddef{\nt{struct-def} \lit{;}}

\grammarend

As with functions, \langname{} makes a distinction for types between declaring
and defining.  A type must be defined in order to use it with a local variable
declaration or structure field. The primitive intrinsic types \langtext{void},
\langtext{bool}, \langtext{int}, \langtext{char}, \langtext{string} are always
defined. Any type that is defined is also considered to be declared. Pointer
and array types are considered defined if their base type is declared.

A structure may not be defined twice but it may be declared more than once.

A structure definition may reference itself.

A type must be defined before it can be used with the \langtext{alloc} or
\langtext{alloc\_array} expressions.  In keeping with the limitations of C, its
definition must be parsed by the time the compiler parses the allocation
expression.

There are no equality or assignment operators provided for structures. The
motivation for their omission is to force students to think about how their
mutable data structures ought to be copied and compared. Due to the lack of
intrinsic support for these operations, structure types may not be used for
parameters, or function return types. They are also prohibited from being used
with local variables due to the lack of an equivalent to C's address-of operator
coupled with aforementioned restrictions. As such, all structures are allocated
on the heap and pointers to structure types are how structures are intended to
be used. Though this is not promoting good practice when writing programs in C,
we deemed the additional complications from allowing safe access to structures
on the stack to be unacceptable.

\section{Typedef}

Since the name of a type may not always be the best description of its purpose,
\langname{} provides the ability to alias types by providing new names.

\grammarbegin
\declprod{ty}
\proddef{\nt{type-name}}

\declprod{type-def}
\proddef{\lit{typedef} \nt{ty} \nt{type-name} \lit{;}}

\declprod{gdecl}
\proddef{\nt{type-def}}
\grammarend

The type must be declared before it can be used with the \langtext{typedef}
construct. Any program text after the \langtext{typedef} may use the alias in
place of the full type.

\section{Reasoning}

Due to its simplicity and safety properties, \langname{} allows the programmer
to reason about their programs via annotations. These annotations are not part
of the core language; they are specified via a special syntax that is embedded
in the comments of the program much like JML \cite{Leavens-Cheon05}, D
\cite{DLang} and Eiffel \cite{ecma2006ead}. Annotations allow the programmer to
declare pre- and post- conditions for functions, loop invariants, and general
assertions.  Library functions declarations can also be annotated.

These assertions about programs are checked at compile time when possible and
at runtime when not. If they fail at runtime, an exception is raised and the
program terminates.

Annotations are designed to encourage students to reason about their code
however they can only check what the programmer has intended to check and may
even contain bugs. The following example is an implementation of binary search
which contains at least one bug that is not caught by the assertions:
\lstinputlisting[language=C,
                 morekeywords={bool,string},
                 firstline=40,
                 lastline=67,
                 basicstyle=\small\tt]{binsearch.c0}

\section{Comparison with C}

\langname{} is almost but not quite a subset of C. Though we strived to stay as
close to C as possible, the array semantics proved impossible to resolve.
However, it is still quite easy to convert a \langname{} program to C such that
the meaning of the \langname{} program is one of the possible meanings of its C
version. The conversion process requires 2 steps:

\begin{enumerate}

\item Convert array declarations by replacing all occurrences of \langtext{[]}
with \langtext{*}.

\item Insert \langtext{\#include "c0defs.h"} at the top of each

\langname{} file.  \langtext{c0defs.h} defines the macros for the
\langtext{alloc} and \langtext{alloc\_array} expressions using
\langtext{calloc} to allocate the memory and \langtext{sizeof} to determine the
bytes. It would also need to define the \langtext{bool} type as well as
\langtext{true} and \langtext{false} since those are not defined in C. Any
library headers used would also need to be converted and included. A sample
\langtext{c0defs.h} is included in the appendix.

\end{enumerate}

\subsection{Differences}

Differences between \langname{} and C fall into two categories: omissions from
C and changes from C. There are many examples of the former:

\begin{itemize}
\item No unions
\item No casting
\item No pointer arithmetic
\item No sizeof operator
\item No address-of operator (\&)
\item No storage modifiers (static, volatile, register, extern)
\item No labels, goto, switch statements or do...while loops
\item No assignments in expressions
\item No floating point datatypes or literals
\item No complex types
\item No unsigned or other integral types besides int
\item Structure types may not be declared as local variables or used as return types for functions
\item No comma separated expressions
\item No explicit memory deallocation
\item Allocation is done using types not a size in bytes.
\item No fixed size arrays
\item No stack allocated arrays
\end{itemize}

As previously noted, the array semantics are different in \langname{} than C,
matching the pointer index semantics instead.

\subsection{Union Support}

It is unfortunate that \langname{} does not contain any good constructs for
disjoint types. Though C has the \langtext{union} construct which supports them in an
unsafe way, getting a type-safe version of them while maintaining the design
goals of the language is rather difficult. The straightforward approach would be
to layout the union in memory as in C and keep a hidden tag that is updated
whenever a field in the union is written and checked when read. The major
drawback to this approach is that there is no way to inspect the tag without
attempting to read a value. One convention in C is to nest unions in a struct
and use a separate struct field to indicate the tag. Either this separate field
must be kept in sync with the hidden tag or else new syntax/semantics must be
invented to tie the tag field to the union. The former is an undue burden on the
programmer and the latter would deviate even more from C.

The issue is deeper however. In C and other languages, unions keep the tag and
data separate which leads to issues such as the classic variant record update
problem in Pascal, later encountered in Cyclone, another strongly typed
imperative C-like language, \cite{GrossmanHJM04}. The problem is that if you
allow aliases to the data stored in a union, you can break type safety because
users of those aliases won't know to typecheck their writes and reads.  In the
following example, neither function is obviously incorrect until their
definitions are used together.

\langsnippet{unionsample.c}

The real fix is to always tie the tag and the data together. In the ML family of
languages, this is accomplished with constructors which take the appropriately
typed data and add the right tag. Examining the data stored in a union requires
a case analysis of the tag which allows access to the data. Even with the
addition of constructors and case analysis, aliasing is still an issue in C's
memory model if access to the data is given by reference. Consider the following
example:

\langsnippet{cctor.c}

A solution here would to copy the value and provide that to the case body
instead of a reference. Cyclone supports this solution as well as forbidding
assignment \cite{CycloneUnions}.

\subsection{Lack of \& for local variables} \label{noaddressof}

In C, it is a common idiom to return multiple values from a function by
requiring the caller pass a pointer to a location where the callee can write
them. Since the callee typically never stores the pointer anywhere, the caller
can expect to safely pass the address of a stack variable instead of allocating
a variable on the heap. For \langname{}, we'd like to permit this in a typesafe
manner. We could do so by introducing an unescaping pointer type which cannot
be stored into the heap. Having two types of pointers is rather strange so it
would be nice to allow regular pointers to be treated as unescaping pointer but
this necessitates the introduction of subtyping which so far has been avoided.
It would also entail that most pointers in function declarations would be
rewritten to use the unescaping pointer type instead of the regular pointer type
since there is no conversion from a potentially escaping pointer to one that
does not. This puts more distance between \langname{} and C in a fairly
intrusive way.

\section{Analysis}

\langname{} strikes a balance between being too limited and too complex. Though
clearly unsuitable for writing large programs, it provides sufficient syntax and
semantics for expressing the data structures and algorithms used in the course.
In comparison to C, it removes a number of dangerous features to gain type
safety while maintaining the core syntax and semantics of the language. Though
not a subset, it is close enough so that

\begin{enumerate}

\item A single lecture would be enough to instruct students in proper usage of
C for writing the equivalent programs.

\item The meaning of a well-formed (terminates without exceptions) \langname{}
program is one of the possible meanings of its transliterated C version.

\end{enumerate}
